%-----------------------------------------------------------------
%	BASIC DOCUMENT LAYOUT
%-----------------------------------------------------------------
\documentclass[paper=a4, fontsize=12pt, twoside=semi]{scrartcl}
\usepackage[T1]{fontenc}
\usepackage[utf8]{inputenc}
\usepackage{lmodern}
\usepackage{slantsc}
\usepackage{microtype}
\usepackage[catalan]{babel}
\usepackage[fixlanguage]{babelbib}
\selectbiblanguage{catalan}

% Sectioning layout\\
\addtokomafont{sectioning}{\normalfont\scshape}
\usepackage{tocstyle}
\usetocstyle{standard}
\renewcommand*\descriptionlabel[1]{\hspace\labelsep\normalfont\bfseries{#1}}

% Empty pages
\usepackage{etoolbox}
\pretocmd{\section}{\cleardoubleevenemptypage}{}{}
\pretocmd{\part}{\cleardoubleevenemptypage\thispagestyle{empty}}{}{}
\renewcommand\partheadstartvskip{\clearpage\null\vfil}
\renewcommand\partheadmidvskip{\par\nobreak\vskip 20pt\thispagestyle{empty}}

% Paragraph indentation behaviour
\setlength{\parindent}{0pt}
\setlength{\parskip}{0.3\baselineskip plus2pt minus2pt}
\newcommand{\sk}{\medskip\noindent}

% Fancy header and footer
\usepackage{fancyhdr}
\pagestyle{fancyplain}
\fancyhead[LO]{\thepage}
\fancyhead[CO]{}
\fancyhead[RO]{\nouppercase{\mytitle}}
\fancyhead[LE]{\nouppercase{\leftmark}}
\fancyhead[CE]{}
\fancyhead[RE]{\thepage}
\fancyfoot{}
\renewcommand{\headrulewidth}{0.3pt}
\renewcommand{\footrulewidth}{0pt}
\setlength{\headheight}{13.6pt}

%-----------------------------------------------------------------
%	MATHS AND SCIENCE
%-----------------------------------------------------------------
\usepackage{amsmath,amsfonts,amsthm,amssymb}
\usepackage{xfrac}
\usepackage[a]{esvect}
\usepackage{chemformula}
\usepackage{graphicx}

\usepackage[arrowdel]{physics}
	\renewcommand{\vnabla}{\vec{\nabla}}
	% \renewcommand{\vectorbold}[1]{\boldsymbol{#1}}
	% \renewcommand{\vectorarrow}[1]{\vec{\boldsymbol{#1}}}
	% \renewcommand{\vectorunit}[1]{\hat{\boldsymbol{#1}}}
	\renewcommand{\vectorarrow}[1]{\vec{#1}}
	\renewcommand{\vectorunit}[1]{\hat{#1}}
	\renewcommand*{\grad}[1]{\vnabla #1}
	\renewcommand*{\div}[1]{\vnabla \vdot \va{#1}}
	\renewcommand*{\curl}[1]{\vnabla \cp \va{#1}}
	\let\rot\curl

% SI units
\usepackage[separate-uncertainty=true, alsoload=astro, alsoload=hep]{siunitx}
\sisetup{range-phrase = \text{--}, range-units = brackets}
\DeclareSIPrePower\quartic{4}
	\DeclareSIUnit\C{C}

% Legacy definitions
\newcommand*{\dif}{\mathrm{d}}
\newcommand*{\diff}{\mathop{}\!\mathrm{d}}
% \newcommand*{\vnabla}{\vec{\nabla}}
\DeclareMathOperator{\im}{im}
% \DeclareMathOperator{\rank}{rang}
\DeclareMathOperator{\vap}{vap}
\DeclareMathOperator{\vep}{vep}

% Smaller trig functions
\newcommand{\Sin}{\trigbraces{\operatorname{s}}}
\newcommand{\Cos}{\trigbraces{\operatorname{c}}}
\newcommand{\Tan}{\trigbraces{\operatorname{t}}}

% Operator-style notation for matrices
\newcommand*{\mat}[1]{\hat{#1}}

% Matrices in (A|B) form via [c|c] option
\makeatletter
\renewcommand*\env@matrix[1][*\c@MaxMatrixCols c]{%
  \hskip -\arraycolsep
  \let\@ifnextchar\new@ifnextchar
  \array{#1}}
\makeatother

% Shorter \mathcal and \mathbb
\newcommand*{\mc}[1]{\mathcal{#1}}
\newcommand*{\mbb}[1]{\mathbb{#1}}

% Shorter ^\ast and ^\dagger
\newcommand*{\sast}{^{\ast}{}}
\newcommand*{\sdag}{^{\dagger}{}}

%-----------------------------------------------------------------
%	OTHER PACKAGES
%-----------------------------------------------------------------
\usepackage{environ}

%Left numbered equations
\makeatletter
	\NewEnviron{Lalign}{\tagsleft@true\begin{align}\BODY\end{align}}
\makeatother

% Plots and graphics
\usepackage{pgfplots}
\usepackage{tikz}
\usepackage{color}
	\makeatletter
		\color{black}
		\let\default@color\current@color
	\makeatother

% Richer enumerate, figure, and table support
\usepackage{enumerate}
\usepackage[shortlabels]{enumitem}
\usepackage{float}
\usepackage{tabularx}
\usepackage{booktabs}
	% \setlength{\intextsep}{8pt}
\numberwithin{equation}{section}
\numberwithin{figure}{section}
\numberwithin{table}{section}

% No indentation after certain environments
\makeatletter
\newcommand*\NoIndentAfterEnv[1]{%
	\AfterEndEnvironment{#1}{\par\@afterindentfalse\@afterheading}}
\makeatother
% \NoIndentAfterEnv{thm}
\NoIndentAfterEnv{defi}
\NoIndentAfterEnv{example}
\NoIndentAfterEnv{table}

% Misc packages
\usepackage{ccicons}
\usepackage{lipsum}

%-----------------------------------------------------------------
%	THEOREMS
%-----------------------------------------------------------------
\usepackage{thmtools}

% Theroems layout
\declaretheoremstyle[
	spaceabove=6pt, spacebelow=6pt,
	headfont=\normalfont,
	notefont=\mdseries, notebraces={(}{)},
	bodyfont=\small,
	postheadspace=1em,
]{small}

\declaretheorem[style=plain,name=Teorema,qed=$\square$,numberwithin=section]{thm}
\declaretheorem[style=plain,name=Corol·lari,qed=$\square$,sibling=thm]{cor}
\declaretheorem[style=plain,name=Lema,qed=$\square$,sibling=thm]{lem}
\declaretheorem[style=definition,name=Definició,qed=$\blacksquare$,numberwithin=section]{defi}
\declaretheorem[style=definition,name=Exemple,qed=$\blacktriangle$,numberwithin=section]{example}
\declaretheorem[style=small,name=Demostració,numbered=no,qed=$\square$]{sproof}

%-----------------------------------------------------------------
%	ELA MOTHERFUCKING GEMINADA
%-----------------------------------------------------------------
\def\xgem{%
	\ifmmode
		\csname normal@char\string"\endcsname l%
	\else
		\leftllkern=0pt\rightllkern=0pt\raiselldim=0pt
		\setbox0\hbox{l}\setbox1\hbox{l\/}\setbox2\hbox{.}%
		\advance\raiselldim by \the\fontdimen5\the\font
		\advance\raiselldim by -\ht2
		\leftllkern=-.25\wd0%
		\advance\leftllkern by \wd1
		\advance\leftllkern by -\wd0
		\rightllkern=-.25\wd0%
		\advance\rightllkern by -\wd1
		\advance\rightllkern by \wd0
		\allowhyphens\discretionary{-}{}%
		{\kern\leftllkern\raise\raiselldim\hbox{.}%
			\kern\rightllkern}\allowhyphens
	\fi
}
\def\Xgem{%
	\ifmmode
		\csname normal@char\string"\endcsname L%
	\else
		\leftllkern=0pt\rightllkern=0pt\raiselldim=0pt
		\setbox0\hbox{L}\setbox1\hbox{L\/}\setbox2\hbox{.}%
		\advance\raiselldim by .5\ht0
		\advance\raiselldim by -.5\ht2
		\leftllkern=-.125\wd0%
		\advance\leftllkern by \wd1
		\advance\leftllkern by -\wd0
		\rightllkern=-\wd0%
		\divide\rightllkern by 6
		\advance\rightllkern by -\wd1
		\advance\rightllkern by \wd0
		\allowhyphens\discretionary{-}{}%
		{\kern\leftllkern\raise\raiselldim\hbox{.}%
			\kern\rightllkern}\allowhyphens
	\fi
}

\expandafter\let\expandafter\saveperiodcentered
	\csname T1\string\textperiodcentered \endcsname

\DeclareTextCommand{\textperiodcentered}{T1}[1]{%
	\ifnum\spacefactor=998
		\Xgem
	\else
		\xgem
	\fi#1}

%-----------------------------------------------------------------
%	PDF INFO AND HYPERREF
%-----------------------------------------------------------------
\usepackage{hyperref}
\usepackage{cleveref}
\hypersetup{colorlinks, citecolor=black, filecolor=black, linkcolor=black, urlcolor=black}

\newcommand*{\mytitle}{Àlgebra lineal}
\newcommand*{\myauthor}{Alfredo Hernández Cavieres}
\newcommand*{\myuni}{Universitat Autònoma de Barcelona, Departament de Física}
\newcommand*{\mydate}{\normalsize 2012-2013}

\usepackage{hyperxmp}
\hypersetup{pdfauthor={\myauthor}, pdftitle={\mytitle}}

%----------------------------------------------------------------------------------------
%    TITLE SECTION AND DOCUMENT BEGINNING
%----------------------------------------------------------------------------------------
\newcommand{\horrule}[1]{\rule{\linewidth}{#1}}
\title{
	\normalfont
	\small \scshape{\myuni} \\ [25pt]
	\horrule{0.5pt} \\[0.4cm]
	\huge \mytitle \\
	\horrule{2pt} \\[0.5cm]
}
\author{\myauthor}
\date{\mydate}

\begin{document}

\clearpage\maketitle
\thispagestyle{empty}
\addtocounter{page}{-1}

%----------------------------------------------------------------------------------------
%    LLICÈNCIA
%----------------------------------------------------------------------------------------
\section*{}\thispagestyle{empty}
\begin{centering}
	\huge \ccbyncsaeu

	\normalsize Aquesta obra està subjecta a una llicència de

	Reconeixement-NoComercial-CompartirIgual 4.0

	Internacional de Creative Commons.

\end{centering}

%----------------------------------------------------------------------------------------
%    TABLE OF CONTENTS
%----------------------------------------------------------------------------------------
\cleardoubleevenemptypage
\pdfbookmark[1]{\contentsname}{toc}
\tableofcontents

%----------------------------------------------------------------------------------------
%    SECTIONS
%----------------------------------------------------------------------------------------
% \part*{Àlgebra I}
	%----------------------------------------------------------------------------------------
%    TEORIA DE CONJUNTS
%----------------------------------------------------------------------------------------
\section{Teoria de conjunts}
\subsection{Relacions binàries $\mathcal{R}$}
$\mathcal{R}$ en $A \subseteq A^{2}, \quad (a,b) \in \mathcal{R} \subseteq A^{2} = a \mathcal{R} b$.
\subsubsection*{Propietats}
\begin{itemize}
    \item Reflexiva: $a \mathcal{R} b, \quad \forall a \in A$.
    \item Simètrica: $a \mathcal{R} b \Rightarrow b \mathcal{R} a$.
    \item Antisimètrica: $a \mathcal{R} b$ i $b \mathcal{R} a \Rightarrow a = b$.
    \item Transitiva: $a \mathcal{R} b$ i $b \mathcal{R} c \Rightarrow a \mathcal{R} c$.
\end{itemize}

\subsubsection*{Relacions d'ordre $\leq$}
\begin{itemize}
    \item Reflexiva: $a \leq b, \quad \forall a \in A$.
    \item Antisimètrica: $a \leq b$ i $b \leq a \Rightarrow a = b$.
    \item Transitiva: $a \leq b$ i $b \leq c \Rightarrow a \leq c$.
\end{itemize}

\subsubsection*{Relacions d'equivalència $\sim$}
\begin{itemize}
    \item Reflexiva: $a \sim b, \quad \forall a \in A$.
    \item Simètrica: $a \sim b \Rightarrow b \sim a$.
    \item Transitiva: $a \sim b$ i $b \sim c \Rightarrow a \sim c$.
\end{itemize}

\paragraph{Classe d'equivalència de $a$}
$\left[ a \right]$ és el conjunt d'elements $\in A$ relacionats amb $a$.
\begin{align}
    \left[ a \right] \equiv \{b \in a \mid b \sim a \} 
\end{align}

\paragraph{Conjunt quocient de $A$}
 $A / \sim$ és el conjunt de totes les classes (disjuntes, per definició) de $A$. 
\begin{align}
    a \in A \textrm{, } \left[ a \right] \subseteq A \textrm{, però } \left[ a \right] \in A / \sim
\end{align}

%----------------------------------------------------------------------------------------
\subsection{Principi d'inducció}
El principi d'inducció diu que si $S \subseteq \mathbb{N}$ tal que
\begin{enumerate}[i)]
    \item $0 \in S$ (a vegades, però, es treballa amb $S^{*} $).
    \item $n \in S \Rightarrow n+1 \in S$.
\end{enumerate}

$\Rightarrow S = \mathbb{N}$, demostrant així que una propietat $P(n)$ que és certa $\forall n \in S$, ho serà també $\forall n \in \mathbb{N}$.

%----------------------------------------------------------------------------------------
\subsection{Aplicacions}
Una aplicació de $A$ a $B$ és la tripleta $(A,B,f)$ on $f \subseteq A \times B$ que compleix:
\begin{enumerate}[i)]
    \item $\forall a \in A, \quad \exists b \in B$ tal que $(a,b) \in f$. 
    \item $(a,b),(a',b') \in f \Rightarrow b = b'$.
\end{enumerate}

Per denotar que $(A,B,f)$ és aplicació, s'escriu $f: A \to B$.

\subsubsection*{Tipus d'aplicacions}
\begin{itemize}
    \item Injectiva: $\hookrightarrow$ o $\iota$. 
        \subitem $\iota \Leftrightarrow f(a) = f(a') \Rightarrow a = a'$.
    \item Exhaustiva: $\twoheadrightarrow$ o $\pi$. 
        \subitem $\pi \Leftrightarrow \forall b \in B, \exists a \in A$ tal que $f(a) = b$.
    \item Bijectiva: $\leftrightarrow$ o $\tilde{f}$. 
        \subitem $\tilde{f} \Leftrightarrow f$ és $\iota$ i $\pi$ alhora.
\end{itemize}

\subsubsection*{Descomposició canònica d'una aplicació}
Tota aplicació $f:A \to B$ pot ésser expressada com a:
\begin{align}
    f = \iota \circ \tilde{f} \circ \pi = \iota ( \tilde{f} ( \pi (A))) \subseteq B
\end{align}

\begin{example}
\begin{align*}
    A \to B = A \twoheadrightarrow A / \sim \leftrightarrow f(A) \hookrightarrow B
\end{align*}
\end{example}

\subsubsection*{Inclusió canònica}
Sigui $B \subseteq A$,

$\iota: B \to A$ és definida per $\iota (x), \quad \forall b \in B$.

\subsubsection*{Aplicació identitat}
És un cas concret de $\iota (x)$.

$id_{A}: A \to A$ és definida per $\iota (x), \quad \forall a \in A$.

\subsubsection*{Imatge i antiimatge}
Sigui $f: A \to B$,

Imatge de $A$: $\im(f) = f(A) \equiv \{ f(x) \mid x \in A \}$

Antiimatge de $B$: $f^{-1} (B) \equiv \{ x \in A \mid f(x) \in B \}$

\subsubsection*{Composició d'aplicacions}
Siguin $f: A \to B$ i $g: B \to C$, 

$g \circ f: A \to C$, on $(g \circ f)(x) = g(f(x)), \quad \forall x \in A$.

\subsubsection*{Inversa d'una funció bijectiva}
Sigui $\tilde{f}: A \to B, \;A \neq 0$ i $B \neq 0$,

$\exists \tilde{g}: A \to B$ tal que $\tilde{g} \circ \tilde{f} = id_{A}$ i $\tilde{f} \circ \tilde{g} = id_{B}$
	%----------------------------------------------------------------------------------------
%    GRUPS
%----------------------------------------------------------------------------------------
\section{Grups}
\subsection{Grups}
Un grup és un conjunt G amb una operació $\ast$ qualsevol.
\begin{align} 
\begin{matrix}
    (G, \ast ) \textrm{, on } \quad \ast : & G \times G & \to &G \\
    & (a,b) & \mapsto & a \ast b
\end{matrix}
\end{align}

\subsubsection*{Propietats}
\begin{itemize}
    \item Associativa: $(a \ast b) \ast c = a \ast (b \ast c), \quad \forall a,b,c \in G$.
    \item Element neutre: $\exists e$ tal que $a \ast e = e \ast a = a, \quad \forall a \in G$.
    \item Element simètric: $\forall a \in G, \quad \exists a'$ tal que $a \ast a' = a' \ast a = e$. 
    \item (Commutativa): Si $a \ast b = b \ast a, \quad \forall a \in G \Rightarrow$ $G$ és un grup abelià o commutatiu.
\end{itemize}

\subsubsection*{Notacions}
\begin{itemize}
    \item Additiva: 
        \subitem $a \ast b = a+b, \;\quad e = 0$ (zero de $G$), $\quad a' = -a$ (oposat de $a$). 
    \item Multiplicativa: 
    \subitem $a \ast b = ab, \;\quad e = 1_{G}$ (unitat de $G$), $\quad a' = a^{-1}$ (invers de $a$).
\end{itemize}

%----------------------------------------------------------------------------------------
\subsection{Subgrups}
Sigui $H \subseteq G$. $H$ és un subanell de $G$ $\Leftrightarrow$
\begin{enumerate}[i)]
    \item $ab \in H, \quad \forall a,b \in H$.
    \item $a^{-1} \in H, \quad \forall a \in H$.
    \item $1_{G} \in S$.
\end{enumerate}
O equivalentment:
\begin{enumerate}[i)]
    \item $H \neq \varnothing$.
    \item $ab^{-1} \in H, \quad \forall a,b \in H$.
\end{enumerate}

%----------------------------------------------------------------------------------------
\subsection{Construcció de grups}
\begin{itemize}
    \item Producte cartesià:
        \subitem Siguin $( G_{1} , \ast)$ i $( G_{2} , \perp )$ grups,
        \subitem $G_{1} \times G_{2} = (a,b) (x,y) = (a \ast x , b \perp y)$ és grup.
    \item Intersecció de grups:
        \subitem $H_{1}, H_{2} \subseteq G \Rightarrow H_{1} \cap H_{2} \subseteq G$.
    \item Subgrup generat per un subconjunt:
        \subitem $S \subseteq G \Rightarrow \left< S \right> \equiv$ mínim subgrup que conté $S$. 
        \subitem e.g., en $(\mathbb{Z} , +), \quad \left< 3 \right> = 3 \mathbb{Z}$.
\end{itemize}

%----------------------------------------------------------------------------------------
\subsection{Grups cíclics i grups finits}
\subsubsection*{Grups cíclics}
$( G , \cdot )$ és un grup cíclic si $\exists a \in G$ tal que $\left< a \right> = G$ 
\begin{align}
    \left< a \right> \equiv \{ a^{n} \mid n \in \mathbb{Z} \} = G
\end{align}

\begin{example}
\begin{align*}
    \mathbb{Z} / n \mathbb{Z} = \left< [1] \right>
\end{align*}
\end{example}

\subsubsection*{Grups finits}
Sigui $( G , \cdot )$ un grup finit i $a \in G$,
\begin{enumerate}[i)]
    \item $\{ a_{n} \} = a, a^{2}, a^{3}, \dots , a^{m}, \dots$
    \item $\exists n \geq m, \; m,n \in \mathbb{N} $ tal que $a^{n} = a^{m} \Rightarrow a^{m} (a^{m})^{-1} = a^{n} (a^{m})^{-1} \Rightarrow a^{n-m} = 1_{G}$.
\end{enumerate}
\subsubsection*{Ordre de grups i d'elements d'un grup}
\begin{itemize}
    \item L'ordre de $G \equiv \# G$: nombre d'elements de $G$.
    \item L'ordre de $a \in G \equiv \operatorname{ord}(a) $ és el mínim $m \in \mathbb{N}$ tal que $a^{m} = 1_{G} \Rightarrow \operatorname{ord} (a) = m \Rightarrow \left< a \right> = \{a, a^{2}, a^{3}, \dots a^{m-2}, a^{m-1}, a^{m} = 1_{G} \} $. $\operatorname{ord} (a) = \text{ mínim } b \text{ tal que } \operatorname{mcm} (a,b) = \dot{n} = 1_{G}, \quad \forall a \in \mathbb{Z} / n\mathbb{Z}$.
    \item $\operatorname{ord} (a) = \# \left< a \right>$.
\end{itemize}

\subsubsection*{Teorema de Lagrange}
Sigui $( G , \cdot )$ un grup finit,
\begin{align}
\# G = n \Rightarrow a^{n} = 1_{G}, \quad \forall a \in G \Rightarrow \# \left< S \right> \leq n, \quad \forall a \in G
\end{align}

\subsubsection*{Teorema d'estructura dels grups cíclics}
Sigui $( G , \cdot )$ un grup cíclic,
\begin{align}
\# G \textrm{ infinit } \Rightarrow ( G , \cdot ) \leftrightarrow (\mathbb{Z} , + )
\end{align}
\begin{align}
\# G \textrm{ finit } \Rightarrow ( G , \cdot ) \leftrightarrow (\mathbb{Z} / n , + )
\end{align}

%----------------------------------------------------------------------------------------
\subsection{Morfismes de grups}
Siguin $G_{1}$ i $G_{2}$ grups. $f: G_{1} \to G_{2}$ és morfisme de grups o homomorfisme $\Leftrightarrow$
\begin{enumerate}[i)]
    \item $f(xy) = f(x) f(y)$.
\end{enumerate}

\subsubsection*{Tipus d'homomorfismes}
\begin{itemize}
    \item $\iota$: monomorfisme.
    \item $\pi$: epimorfisme.
    \item $\tilde{f}$: isomorfisme.
\end{itemize}
Si $f$ és isomorfisme, $f^{-1}$ és morfisme de grups.

%----------------------------------------------------------------------------------------
\subsection{Nucli ($\ker$) i imatge ($\im$) d'un grup}
Sigui $f: G \to G'$,
\begin{align}
    \ker(f) \equiv \{ x \in G \mid f(x) = 1_{G} \}
\end{align}
\begin{align}
    \im(f) \equiv \{ y \in G' \mid f(x) = y \} \equiv f(G) \subseteq G'
\end{align}
Observació:
\begin{itemize}
    \item Monomorfisme $\Leftrightarrow \ker(f) = \{ 1_{G} \}$.
    \item Epimorfisme $\Leftrightarrow \im(f) = G' $.
\end{itemize}
	%----------------------------------------------------------------------------------------
%    ANELLS
%----------------------------------------------------------------------------------------
\section{Anells}
\subsection{Anells}
Un anell és un conjunt $R$ amb dues operacions: $(R,+, \cdot)$.
\begin{align}
    \begin{matrix}
        +: & R \times R & \to & R \\
        & (a,b) & \mapsto & a + b
    \end{matrix}
\end{align}
\begin{align}
    \begin{matrix}
        \cdot: & R \times R & \to & R \\
        & (a,b) & \mapsto & ab
    \end{matrix}
\end{align}

\subsubsection*{Propietats}
\begin{itemize}
    \item $(R,+)$ és un grup abelià.
    \item Associativa del producte: $(ab)c = a(bc), \quad \forall a,b,c \in R$.
    \item Doble distributiva: $a(b+c) = ab + ac, \;(b+c)a = ba + ca, \quad \forall a,b,c \in R$.
    \item Si el producte és commutatiu$\Leftrightarrow$R és un anell commutatiu.
\end{itemize}

%----------------------------------------------------------------------------------------
\subsection{Subanells}
Sigui $S \subseteq R$. $S$ és un subanell de $R$ $\Leftrightarrow$
\begin{enumerate}[i)]
    \item $S$ és un subanell de $(R,+)$.
    \item $ab \in S, \quad \forall a,b \in S$.
    \item $1_{R} \in S$.
\end{enumerate}

%----------------------------------------------------------------------------------------
\subsection{Anell de matrius}
$M(m,n,R) = M_{m \times n} (R) \equiv$ matriu de $m$ files i $n$ columnes sobre l'anell $R$. Si la matriu és quadrada ($M(n,n,R) = M_{n \times n} (R)$), s'escriu $M(n,R) = M_{n} (R)$. Si pel context se sobreentén l'anell sobre el qual es treballa, es pot escriure $M(m,n) = M_{m \times n}$.

\subsubsection*{Operacions}
\begin{itemize}
    \item Suma (per a matrius del mateix tipus).
        \subitem $A+B = (a_{ij} + b_{ij}) \quad e=0$.
    \item Producte per escalars.
        \subitem $kA = k(a_{ij})$.
    \item Producte.
        \subitem $M(m,l) \times M(l,n) = M(m,n), \quad$ en $M(n), \; \exists e, \; e = I(n)$.
        \subitem $AB=C \Rightarrow A_{i} B = C_{i} \text{ i } AB^{j} = C^{j}$.
\end{itemize}

%----------------------------------------------------------------------------------------
\subsection{Elements invertibles d'un anell}
$a \in R$ és invertible$\Leftrightarrow \exists b$ tal que $ab = ba = 1_{R} \Rightarrow b = a^{-1}$.

\subsubsection*{Grup d'elements invertibles}
$U(R) \equiv \{ x \mid x \textrm{ és invertible} \}$.

\subsubsection*{Divisors de zero}
$a \neq 0$ i $b \neq 0$ són divisors de zero si $ab=0$ o $ba=0$.

\subsubsection*{Domini d'integritat i cos}
Un anell $R$ sense divisors de zero és un domini d'integritat. Un cos $K$ (cas particular de domini d'integritat) és un anell commutatiu no nul on tot element $a \neq 0$ és invertible.

\subsubsection*{Grup lineal de matrius}
$GL(n,K) \equiv \{ M(n,K) \mid M \textrm{ és invertible} \}$
	%----------------------------------------------------------------------------------------
%    EL COS DELS COMPLEXOS
%----------------------------------------------------------------------------------------
\section{El cos dels complexos}
\subsection{Els nombres complexos}
Els nombres complexos són expressions de la forma:
\begin{align}
    a + b \imath \in \mathbb{C}, \quad a,b \in \mathbb{R}, \quad \imath ^{2} = -1
\end{align}

\subsubsection*{Definicions}
\begin{itemize}
    \item El conjugat d'un nombre complex $z = a + b \imath$ és $a - b \imath$, i es denota per $\bar{z}$.
    \item La norma o mòdul d'un nombre complex $z = a + b \imath$ és $\sqrt{a^{2} + b^{2}}$, i es denota per $|z|$.
    \item L'argument d'un nombre complex $z = a + b \imath$ és l'angle que forma $z$ amb la recta dels reals al pla dels complexos, i es denota per $\operatorname{arg}(z) = \theta$.
\end{itemize}

\subsubsection*{Operacions}
\begin{itemize}
    \item Suma: $ (a + b \imath) + (c + d \imath) = (a + c) + (b + d) \imath$.
    \item Producte: $(a + b \imath) (c + d \imath) = (ac - bd) + (ad + bc) \imath$.
\end{itemize}
$\mathbb{C}$ és un abelià amb la suma i el producte.

%----------------------------------------------------------------------------------------
\subsection{Coordenades polars}
\begin{align}
    \text{Fórmula d'Euler: } e^{\imath \theta} = \cos \theta + \imath \sin \theta
\end{align}

Tot $z \in \mathbb{C}$ por ésser expressat com a
\begin{align}
    z = |z| e^{\imath \theta}
\end{align}
Aleshores, el producte de complexos en forma polar compleix:
\begin{enumerate}[i)]
    \item $|zw| = |z| |w|$.
    \item $\operatorname{arg}(zw) = \operatorname{arg}(z) + \operatorname{arg} (w)$.
\end{enumerate}

%----------------------------------------------------------------------------------------
\subsection{Arrels $n$-èsimes d'un complex}
Tot nombre complex té $n$ arrels $n$-èsimes.
\begin{align}
    w = z_{k}^{n} \Rightarrow z_{k} = \sqrt[n]{|w|} \exp \left[\imath \left(\frac{\theta}{n} + \frac{2 \pi k}{n} \right) \right], \quad 0 \leq k < n
\end{align}

%----------------------------------------------------------------------------------------
\subsection{Els 16 arguments bonics}
\subsubsection*{Angles de la forma $\frac{1}{4}k \pi , \; k \in \mathbb{Z}$ }
\begin{align}
    \pm \frac{\sqrt{2}}{2} \pm \frac{\sqrt{2}}{2} \imath
\end{align}
\begin{itemize}
    \item Ambdues components són $\pm \frac{\sqrt{2}}{2} R$.
\end{itemize}

\subsubsection*{Angles de la forma $\frac{1}{6}k \pi , \; k \in \mathbb{Z}$ }
\begin{align}
    \pm \frac{\sqrt{3}}{2} \pm \frac{1}{2} \imath \quad \text{o} \quad \pm \frac{1}{2} \pm \frac{\sqrt{3}}{2}
\end{align}
\begin{itemize}
    \item La component gran és $\pm \frac{\sqrt{3}}{2} R$.
    \item La component petita és $\pm \frac{1}{2} R$.
\end{itemize}

%----------------------------------------------------------------------------------------
\subsection{Conceptes trigonomètrics}
\subsubsection*{La funció arctangent}
Sigui $z = a + b \imath$, $\Rightarrow \tan \theta = \frac{a}{b}$, 
\begin{align}
    \theta =
    \begin{cases}
        \arctan \left( \frac{a}{b} \right) , & \text{si } a > 0 \\
        \frac{\pi}{2} , & \text{si } a=0, \; b>0 \\ 
        - \frac{\pi}{2} , & \text{si } a=0, \; b<0 \\
        \arctan \left( \frac{a}{b} \right) + \pi , & \text{si } a < 0
    \end{cases} 
\end{align}

\subsubsection*{Identitats trigonomètriques}
\begin{align}
\begin{gathered}
    \sin ^{2} x + \cos ^{2} x = 1 \\
    \sin (x \pm y) = \sin x \cos y \pm \cos x \sin y \\
    \cos (x \pm y) = \cos x \cos y \mp \sin x \sin y
\end{gathered}
\end{align}
	%----------------------------------------------------------------------------------------
%    ANELL DE POLINOMIS SOBRE UN COS COMMUTATIU
%----------------------------------------------------------------------------------------
\section{Anell de polinomis sobre un cos commutatiu}
\subsection{L'anell dels polinomis}
Sigui $a(x) \in K [x] = a_{n}x^{n} + a_{n-1}x^{n-1} + \dots + a_{i}x^{i} + \dots + a_{1}x + a_{0}, \; (n \in \mathbb{N}, \, a_{i} \in K)$.
\begin{itemize}
    \item $a_{i}$ són els coefincients del polinomi, essent $a_{n}$ el coeficient principal.
    \item $i$ són els graus del polinomi, essent $n$ el grau màxim del polinomi. S'expressa com a $\operatorname{gr}(a(x)) = n$.
\end{itemize}
\subsubsection*{Definicions}
\begin{itemize}
    \item Monomi: polinomi d'un sol element.
    \item Polinomi mònic: polinomi amb el coeficient principal $= 1$.
\end{itemize}

\subsubsection*{Operacions}
\begin{itemize}
    \item Suma:
        \subitem $\operatorname{gr}(a(x) + b(x)) \leq \max \{ \operatorname{gr}(a(x)), \operatorname{gr}(b(x)) \}$
    \item Producte:
        \subitem $\operatorname{gr}(a(x) b(x)) = \operatorname{gr}(a(x)) + \operatorname{gr}(b(x))$
    \item Divisió entera:
        \subitem Siguin $a(x), b(x) \in K [x], \; (b(x) \neq 0), \quad \exists$ dos úncs $q(x)$ i $r(x)$ tals que:
        \begin{enumerate}[i)]
            \item $a(x) = q(x) b(x) + r(x)$.
            \item $\operatorname{gr}(r(x)) < \operatorname{gr}(b(x))$.
        \end{enumerate}
\end{itemize}

%----------------------------------------------------------------------------------------
\subsection{Arrels d'un polinomi}
$r \in K$ és una arrel de $a(x) \Leftrightarrow a(r) = 0$

\subsubsection*{Teorema de la resta}
$u \in K$ és una arrel de $a(x) \Leftrightarrow a(x)$ és divisible, exactament, per $x-u$.
\begin{align}
    a(x) = q(x) (x-u) + a(u), \quad a(u) = r(x) = 0
\end{align}

%----------------------------------------------------------------------------------------
\subsection{Divisibilitat a un anell}
Siguin $a,b \in R$,

$D(a) \equiv \{ d \in R \mid a \in dR \}, \quad D(a,b) \equiv D(a) \cap D(b) = \{ d \in R \mid d|a \text{ i } d|b \}$

\subsubsection*{Lema d'Euclides}
\begin{align}
    a = qb + r,
    \begin{cases}
        R = \mathbb{Z} \Rightarrow 0 \leq r < |b| \\
        R = K [x] \Rightarrow \operatorname{gr} (r) < \operatorname{gr} (b)
    \end{cases}
    \Rightarrow
\end{align}

\begin{enumerate}[i)]
    \item $D (a,b) = D (b,r)$.
    \item $aR + bR = bR + rR$.
\end{enumerate}

\subsubsection*{Algoritme}
\begin{align}
\begin{gathered}
    a = b q_{0} + r_{0} \\
    b = r_{0} q_{1} + r_{1} \\
    \dots \\
    r_{n-2} = r_{n-1} q_{n} + (r_{n} = 0)
\end{gathered}
\end{align}

Per força, després d'un nombre finit $n$ d'iteracions, arribem a tenir $r_{n}=0 \Rightarrow$ (pel Lema d'Euclides) $D (a,b) = D (r_{n-1}) \Rightarrow aR + bR = r_{n-1}R \Rightarrow$
\begin{enumerate}[i)]
    \item $\operatorname{mcd} (a,b) = r_{n-1} $
    \item $\operatorname{mcm} (a,b) = \frac{ab}{\operatorname{mcd} (a,b)}$
\end{enumerate}

Tant $\operatorname{mcd}$ i $\operatorname{mcm}$ són únics i de la següent forma: 
$\begin{cases} < 0 \text{ a } R = \mathbb{Z} \\ \text{ mònic a } R = K [x] \end{cases}$

\subsubsection*{Identitat de Bezout}
\begin{align}
    aR + bR = r_{n-1}R \Rightarrow a \alpha + b \beta = r_{n-1}
\end{align}
\begin{example}
\begin{align*}
    a = b q_{0} + r_{0}, \; b = r_{0} q_{1} + r_{1}, \; r_{0} = r_{1} q_{2} + 0 \Rightarrow a \alpha + b \beta = r_{1}
\end{align*}
\end{example}
\begin{enumerate}[a)]
    \item Mètode 1.
        \subitem $ r_{1} = b - r_{0} q_{1} = b - q_{1} (a - b q_{0}) = a \underbrace{( - q_{1})}_{\alpha} + \, b \underbrace{( 1 + q_{0} q_{1})}_{\beta}$
    \item Mètode 2.
        \subitem $\begin{pmatrix} b \\ r_{0} \end{pmatrix} = \begin{pmatrix} 0 & 1 \\ 1 & -q_{0} \end{pmatrix} \begin{pmatrix} a \\ b \end{pmatrix}$, 
        \subitem $\begin{pmatrix} r_{0} \\ r_{1} \end{pmatrix} = \begin{pmatrix} 0 & 1 \\ 1 & -q_{1} \end{pmatrix} \begin{pmatrix} b \\ r_{0} \end{pmatrix} = \begin{pmatrix} 0 & 1 \\ 1 & -q_{1} \end{pmatrix} \begin{pmatrix} 0 & 1 \\ 1 & -q_{0} \end{pmatrix} \begin{pmatrix} a \\ b \end{pmatrix}$,
         \subitem $\begin{pmatrix} r_{1} \\ 0 \end{pmatrix} = \begin{pmatrix} 0 & 1 \\ 1 & -q_{2} \end{pmatrix} \begin{pmatrix} r_{0} \\ r_{1} \end{pmatrix} = \dots = \begin{pmatrix} \alpha & \beta \\ \ast & \ast \end{pmatrix} \begin{pmatrix} a \\ b \end{pmatrix}$.
\end{enumerate}

%----------------------------------------------------------------------------------------
\subsection{Factorització}
Tot $a(x) \in \mathbb{K} [x]$ és factoritzable en polinomis irreductibles $b(x) \in \mathbb{K} [x]$. La factorització d'un polinomi pot variar segons el cos $K$ sobre el qual s'estengui l'anell de polinomis.

\subsubsection*{Polinomis irreductibles}
\begin{itemize}
    \item Els polinomis irreductibles a $\mathbb{Q} [x]$ poden ésser de qualsevol grau possitiu.
    \item Els polinomis irreductibles a $\mathbb{R} [x]$ són els de garu $1$ i els de grau $2$ de la forma $a x^{2} + b x +c$ amb $\Delta < 0$.
    \item Els polinomis irreductibles a $\mathbb{C} [x]$ són els de grau $1$.
\end{itemize}

\subsubsection*{Teorema fonamental de l'àlgebra}
Tot $a(x) \in \mathbb{C} [x]$ de $\operatorname{gr} (a(x)) = n > 0$ té exactament $n$ arrels complexes $\Rightarrow a(x) = \lambda (x - \alpha_{1}) (x - \alpha_{2}) \dots (x - \alpha_{n-1}) (x - \alpha_{n})$. 
	%----------------------------------------------------------------------------------------
%    MATRIUS
%----------------------------------------------------------------------------------------
\section{Matrius}
\subsection{Transformacions elementals}
\subsubsection*{Tipus de transformacions elementals}
\begin{itemize}
    \item Intercanviar les files $i$ i $j$: $F_{i} \leftrightarrow F_{j}$.
    \item Multiplicar una fila $i$ per un escalar $\mu$: $F_{i} \mapsto \mu F_{i}$.
    \item Sumar a la fila $i$ la fila $j$ multiplicada per $\mu$: $F_{i} \mapsto F_{i} + \mu F_{j}$.
\end{itemize}
El mateix és aplicable per a columnes.

\subsubsection*{Matrius elementals}
Si apliquem una transformació elemental a la matriu identitat $I(n)$, obtenim una matriu elemental $E(n)$ que compleix:
\begin{enumerate}[i)]
    \item $E$ és invertible.
    \item $E$ és un operador universal.
\end{enumerate}

Sigui $A \in M(m,n,K), \quad EA$ coincideix amb aplicar la mateixa transformació elemental a $A$.

Si apliquem una sèrie finita de transformacions ($I \mapsto P$), aleshores:
\begin{enumerate}[i)]
    \item $P$ és invertible.
    \item $P$ és un operador universal.
\end{enumerate}

$PA$ coincideix amb aplicar les mateixes transformacions elementals a $A$ amb la propietat que no es perd l'informació que proporciona la matriu $A$: $PA = A' \Rightarrow A = P^{-1} A'$.

El mateix és aplicable per a columnes, amb la diferència que aplicar les transformacions elementals coincideix amb multiplicar per la dreta ($AQ = A'$).

%----------------------------------------------------------------------------------------
\subsection{Matriu de Gauss--Jordan}
\subsubsection*{Definicions prèvies}
\begin{itemize}
    \item El pivot d'una fila $\neq 0$ és el primer element $\neq 0$ d'esquerra a dreta.
    \item Una matriu és esglaonada si les files zero estan a la part inferior de la matriu i si els pivots estan en columnes que creixen estrictament.
\end{itemize}
Una matriu de Gauss--Jordan és una matriu esglaonada que compleix:
\begin{enumerate}[i)]
    \item Els pivots són tots $1$.
    \item A cada columna dels pivots, tots elements són $0$ excepte el pivot.
\end{enumerate}
Tota matriu $A$ pot convertir-se en una de Gauss--Jordan aplicant una sèrie adequada de transformacions elementals, i aquesta ($\operatorname{GJ} (A)$) és única.

\subsubsection*{Algoritme de Gauss--Jordan}
Fase de Gauss (esglaonar):
\begin{enumerate}[i)]
    \item Buscar col·lumnes nul·les i posar-les al principi de la matriu.
    \item Buscar un pivot $\neq 0$ i canviar files tal que el pivot de la matriu sigui $\neq 0$.
    \item Transformar la columna tal que tot el que hi ha a baix del pivot sigui $0$.
    \item Reiterar amb la resta de columnes no nul·les.
\end{enumerate}
Fase de Jordan (reduir):
\begin{enumerate}[i)]
    \item Començant pel pivot més avançat, transformar tots els pivots en $1$.
    \item Transformar tots els elements d'una columna en $0$. 
\end{enumerate}

%----------------------------------------------------------------------------------------
\subsection{Criteri d'invertibilitat}
Sigui $A \in M (n,K)$, les condicions següents són equivalents:
\begin{enumerate}[i)]
    \item $A$ és invertible.
    \item $A$ és producte de matrius elementals.
    \item $\operatorname{GJ} (A) = I(n)$.
\end{enumerate}
\subsubsection*{Algoritme d'inversa d'una matriu}
\begin{align}
\begin{gathered}
    PA=I \Rightarrow PI = A^{-1} \\
    ( A | I ) \mapsto ( I | A^{-1} )
\end{gathered}
\end{align}

%----------------------------------------------------------------------------------------
\subsection{Relació d'equivalència}
Siguin $A, B \in M(m,n,K)$, les condicions següents són equivalents:
\begin{enumerate}[i)]
    \item $A \sim B$.
    \item $ \exists P \in GL(n,K)$ tal que $PA = B$.
    \item $\operatorname{GJ} (A) = \operatorname{GJ} (B)$.
\end{enumerate}

$\Rightarrow M(m,n,K)/ \sim \, = \{ [\operatorname{GJ}_{i}] \mid \operatorname{GJ} \in M(m,n,K) \} \Rightarrow [A] = GL(n,K) A$.

%----------------------------------------------------------------------------------------
\subsection{Resolució de sistemes lineals}
Sigui un sistema $(\ast)$ de, per exemple, 5 incògnites $x$, $y$, $z$, $t$ i $u$,
\begin{itemize}
    \item Una solució de $(\ast)$ és una $5\text{-tupla} (x,y,z,t,u) \in K^{5}$ que satisfà totes les equacions.
    \item El conjunt de solucions forma una varietat lineal $L$ de $K^{5}$:
        \subitem $L \equiv \{ (x,y,z,t,u) \in K^{5} \mid \text{satisfan } (\ast) \} \subseteq K^{5}$.
    \item Diem que $(\ast)$ és una equació cartesiana de $V$.
    \begin{itemize}
        \item Sistema compatible $\Leftrightarrow V \neq \varnothing$.
        \item Sistema incompatible $\Leftrightarrow V = \varnothing$.
    \end{itemize}
    \item Resoldre el sistema vol dir trobar una equació paramètrica de $V$ que descrigui com són tots els seus punts en funció d'uns paràmetres que prenen valors lliurement. El nombre de paràmetres lliures s'anomena dimensió de V $\equiv \dim{(V)}$.
    \end{itemize}

\subsubsection*{Mètode de Gauss--Jordan}
\begin{enumerate}[i)]
    \item Calcular $\operatorname{GJ} (A|B)$ del sistema.
    \item Reinterpretar aquesta matriu en el llenguatge de les equacions.
\end{enumerate}   
\begin{align*}
\begin{pmatrix}[cccc|c]
    1 & -\sfrac{1}{2} & 0 & 0 & 2 \\
    0 & 0 & 1 & 0 & 3 \\
    0 & 0 & 0 & 1 & 2
\end{pmatrix}
\text{(Gauss--Jordan del sistema)}
\end{align*}
\begin{align*}
\begin{cases}
    x &= 2 + \frac{1}{2} y \\
    y &= y \\
    z &= 3 \\
    t &= 2
\end{cases}
\text{(Eq. paramètrica de $V$)}
\end{align*}
\begin{align*}
\begin{gathered}
    \begin{pmatrix} x \\ y \\ z \\ t \end{pmatrix}
    = \begin{pmatrix} 2 \\ 0 \\ 3 \\ 2 \end{pmatrix}
    + y \begin{pmatrix} \sfrac{1}{2} \\ 1 \\ 0 \\ 0 \end{pmatrix}
    \text{(Eq. paramètrica vectorial de $V$)} \\
    \\
    (2,0,3,2) \in V, \quad (\sfrac{1}{2},1,0,0) \text{ és un vector director de $V$}
\end{gathered}
\end{align*}
\begin{align*}
    L = \{ (2+ \frac{1}{2}y,y,3,2) \mid y \in \mathbb{R} \}
\end{align*}

%----------------------------------------------------------------------------------------
\subsection{Rang d'una matriu}
El rang d'una matriu $A$ és el nombre de files no nul·les a la forma de Gauss--Jordan de $A$, i s'escriu $\rank(A)$.

\subsubsection*{Teorema de Rouché-Capelli}
Sigui $AX = B$, amb $A \in M(m,n,K)$, $X \in M(n,1,K)$, $B \in M(m,1,K)$. Llavors, podem veure el següent
\begin{enumerate}[i)]
    \item Sistema compatible $\Leftrightarrow \rank(A) = \rank(A|B)$.
    \item $\dim{(V)} = n - \rank (A)$.
\end{enumerate}

\subsubsection*{Propietats}
\begin{itemize}
    \item $\rank(A) = \rank(A^{t})$.
    \item $\rank(PA) = \rank(A) =\rank(AQ)$.
    \item Sigui $B$ una columna de la mateixa mida que les columnes, $B$ és combinació lineal de les columnes de $A \Leftrightarrow \rank(A) = \rank(A |B)$. El mateix és aplicable per a files.
\end{itemize}
	%----------------------------------------------------------------------------------------
%    ESPAIS VECTORIALS
%----------------------------------------------------------------------------------------
\section{Espais vectorials}
\subsection{Espais vectorials}
Un espai vectorial sobre $K$ o $K$-espai vectorial és un grup abelià $(V,+)$ junt amb el producte per elements de $K$, 
\begin{align}
    \begin{matrix}
        \cdot: & K \times V & \to & V \\
        & (\lambda, u) & \mapsto & \lambda u
    \end{matrix}
\end{align}
que compleix les propietats següents:
\begin{enumerate}[i)]
    \item $\lambda (u + v) = \lambda u + \lambda v , \quad \forall \lambda \in K, \forall u , v \in V$.
    \item $(\lambda + \mu) u = \lambda u + \mu u , \quad \forall \lambda , \mu \in K, \forall u \in V$.
    \item $(\lambda \mu) u = \lambda (\mu u) , \quad \forall \lambda , \mu \in K, \forall u \in V$.
    \item $ 1 \cdot u = u, \quad \forall u \in V$.
\end{enumerate}

Els elements de $V$ es diuen vectors i els de $K$ escalars.

\subsubsection*{Exemples}
\begin{itemize}
    \item $K^{n}$ amb la suma de $n$-tuples i el producte per elements de $K$.
    \item $M(m,n,K)$ amb la suma de matrius i el producte per elements de $K$.
    \item $K [x]$ amb la suma de polinomis i el producte per elements de $K$.
\end{itemize}

%----------------------------------------------------------------------------------------
\subsection{Subespais vectorials}
Un subconjunt $W \neq \varnothing$ d'un $K$-espai vectorial $V$ és un subespai vectorial de $V \Leftrightarrow$
\begin{enumerate}[i)]
    \item $(W,+)$ és un subgrup de $(V,+)$.
    \item $\lambda \in K, \; u \in W \Rightarrow \lambda u \in W$.
\end{enumerate}

O equivalentment,
\begin{enumerate}[i)]
    \item Suma de vectors tancada: $u, v \in W \Rightarrow u + v \in W$.
    \item Magnificació tancada: $\lambda u \in W, \quad \forall \lambda \in K, \; u \in W$.
\end{enumerate}

\subsubsection*{Exemples}
\begin{itemize}
    \item El subespai trivial: $\{ 0_{V} \}$.
    \item El subespai total: $V$.
    \item Les varietats lineals $L$ de $V = K^{n}$ que passen per l'origen.
        \subitem En efecte, $L = \{ (x_{1}, \dots , x_{n}) \mid A \begin{pmatrix} x_{1}  \\ \vdots \\ x_{n}  \end{pmatrix} = \begin{pmatrix} 0 \\ \vdots \\ 0 \end{pmatrix} \} \subseteq K^{n}$ és un subespai vectorial.
\end{itemize}

%----------------------------------------------------------------------------------------
\subsection{Dependència i independència lineal}
\subsubsection*{Combinació lineal}
El vector $u$ del $K$-espai vectorial $V$ és una combinació lineal dels vectors $u_{1}, \dots, u_{n} \in V$, si $\exists \lambda_{1}, \dots, \lambda_{n} \in K$ tals que:
\begin{align}
    u = \lambda_{1} u_{1} + \dots + \lambda_{n} u_{n}.
\end{align}

Sigui $S \in V$ un subespai vectorial, $\left< S \right>$ és el conjunt de totes les combinacions lineals de $S$. $\left< S \right>$ és un subespai vectorial de $V$ i, a més, és el mínim subespai que conté $S$.

\subsubsection*{Generadors}
Si $W$ és un subespai vectorial de $V$ i $W = \left< A \right>$, llavors diem que $A$ genera $W$ o equivalentment que $A$ és un conjunt de generadors de $W$.

Si existeix un subconjunt finit $A$ de $V$ que generi $V$, llavors diem que $V$ és un espai vectorial finitament generat. 

\subsubsection*{Independència lineal}
Diem que $u_{1}, \dots , u_{m} \in V$ són una família linealment independent ($LI$) si cap vector $u_{i}$ és combinació lineal de la resta de vectors. Si no és així, diem que són una família linealment dependent ($LD$).

\subsubsection*{Criteri teòric d'independència lineal}
Siguin $u_{1}, \dots , u_{m} \in V$ i $c: K^{m} \to V$ una funció tal que $c (\lambda _{1}, \dots , \lambda _{m}) = \lambda _{1} u_{1}, \dots , \lambda _{m} u_{m}$. Les condicions següents són equivalents:
\begin{enumerate}[i)]
    \item $u_{1}, \dots , u_{m}$ són una família $LI$.
    \item $\lambda _{1} u_{1} + \dots + \lambda _{m} u_{m} = 0_{V} \Leftrightarrow \lambda _{1} = \dots = \lambda _{m} = 0$. (A la pràctica és el més útil).
    \item $c$ és una aplicació injectiva.
\end{enumerate}
\begin{align}
\begin{gathered}
    (\ast): \begin{pmatrix} u_{1} & \dots & u_{m} \end{pmatrix} \begin{pmatrix} \lambda_{1} \\ \vdots \\ \lambda_{m} \end{pmatrix} = 0_{V} \Rightarrow \\
    \Rightarrow W = \{ (\lambda _{1}, \dots , \lambda_{m}) \in K^{m} \mid \text{ satisfan } (\ast) \} \\
        \Rightarrow \begin{cases} LI: \dim{(W)} = 0 \Leftrightarrow 0_{W} \text{ és l'únic element } \in V. \\ LD: \dim{(W)} > 0 \Leftrightarrow 0_{W} \text{ no és l'únic element } \in W. \end{cases}
\end{gathered}
\end{align}

\subsubsection*{Criteri pràctic d'independència lineal}
$u_{1}, \dots , u_{m} \in V$ són una família $LI$ $\Leftrightarrow \rank \begin{pmatrix} u_{1} \\ \vdots \\ u_{m} \end{pmatrix} =m$.

%----------------------------------------------------------------------------------------
\subsection{Bases i dimensió}
Una base d'un $K$-espai vectorial $V$ és una família de vectors de $V$, $\{v_{1}, \dots , v_{n} \}$ tal que:
\begin{enumerate}[i)]
    \item $V = \left< \{v_{1}, \dots , v_{n} \} \right>$.
    \item $\{v_{1}, \dots , v_{n} \}$ són una família $LI$.
\end{enumerate}
Si una base de $V$ és finita, l'escriurem en la forma $(v_{1} , \dots , v_{n})$.

\subsubsection*{Amplació d'una família linealment independent}
Siguin $v_{1}, \dots , v_{m} \in V$ una família $LI$. Considerem un vector $v \in V$, aleshores:
\begin{align}
    \{v_{1}, \dots , v_{m}, v\} \; LI \Leftrightarrow v \notin \left< v_{1}, \dots , v_{m} \right>
\end{align}

\subsubsection*{Teorema de la base}
Sigui $V$ és un $K$-espai vectorial. Si $V = \left< v_{1}, \dots , v_{m} \right>$, aleshores qualsevol subfamília $LI$ maximal de $\{ v_{1}, \dots , v_{m} \}$ és una base de $V$.
\begin{enumerate}[i)]
    \item Totes les bases tenen el mateix nombre de vectors; aquest nombre l'anomenem $\dim (V)$.
    \item $\#$generadors $\geq \dim (V) \geq \#$ família $LI$.
    \item Si $\dim (V)= n$ i tenim $n$ vectors $u_1, \dots , u_n \in V$, aleshores, si $u_1, \dots , u_n$ són $LI$ $\Leftrightarrow$ base $\Leftrightarrow$ generadors. A $K^{n}$, a més, $\rank = n \Leftrightarrow GJ (A) = I(n)$.
    \item Qualsevol subespai $W \subseteq V$ és generat finitament i $\dim (W) \leq \dim (V)$. Així doncs, $\dim (W) = \dim (V) \Leftrightarrow W = V$.
\end{enumerate}

\subsubsection*{Completació d'una família $LI$ en base}
Siguin $e_{1}, \dots , e_{n} \in V$ una família $LI$ i $\left< v_{1}, \dots , v_{m} \right> = V$, aleshores es poden afegir a la família $e_{1}, \dots , e_{n} \in V$ vectors $v_{i}$ (convenientment escollits) tals que $e_{1}, \dots , e_{n}, \dots , v_{i} \in V$ siguin una base de $V$. 

\subsubsection*{Teorema de Steinitz}
Sigui $(v_1, \dots , v_n)$ una base d'un $K$-espai vectorial $V$ i sigun $u_1, \dots , u_m \in V$ una família de vectors $LI$. Alerhores:
\begin{enumerate}[i)]
    \item $m \leq n$.
    \item Es poden substituir $m$ vectors de la base $(v_1, \dots , v_n)$ per $u_1, \dots , u_m$ amb tal d'obtenir una nova base.
\end{enumerate}

\subsubsection*{Propietat fonamental de les bases}
$(v_{1}, \dots , v_{n})$ és una base de $V \Leftrightarrow c: K^{n} \to V$ és una aplicació bijectiva. Aleshores, $\exists$ una única $n$-tupla que multiplicada per la base doni un vector $v_{i}$ (les bases finites són conjunts ordenats).

\subsubsection*{Càlcul de bases i dimensió d'un espai vectorial $V = K^{n}$}
Mètode I:
\begin{enumerate}[i)]
    \item Sigui $W = \left< v_{1}, \dots , v_{m} \right> \subseteq V$. Aplicant la fase de Gauss, obtenim una família de vectors $LI$ maximal i, per tant, una base de $W$ i la seva dimensió.
    \item Apliquem la fase de Jordan per obtenir una base canònica de $W$.
\end{enumerate}
Mètode II (e.g.,):

$\quad F = \{ (x,y,z,t,u) \in K^{5} \mid \begin{pmatrix} 0 & 1 & 2 & 4 & 6 \\ 0 & 1 & 3 & 6 & 9 \\ 0 & 1 & 4 & 8 & 12 \end{pmatrix} \begin{pmatrix} x \\ y \\ z \\ t\\ u \end{pmatrix} = \begin{pmatrix} 0 \\ 0 \\ 0 \end{pmatrix} \} \subseteq K^{5}$.

Aleshores, 
\begin{itemize}
    \item $\dim (W) = \# \text{vectors directors} = \# \text{paràmetres lliures} = n - \rank (A)$.
    \item Una base és la família de vectors directors de l'equació paramètrica.
\end{itemize}

%----------------------------------------------------------------------------------------
\subsection{Teorema del rang}
Els següents 5 nombres coincideixen $\forall A \in M (m,n,K)$:
\begin{enumerate}[i)]
    \item $\rank (A)$.
    \item $\dim (\left< \text{files de } A \right> )$.
    \item $\text{màxim nre. files de } A \mid \text{família } LI$ 
    \item $\dim (\left< \text{columnes de } A \right> )$.
    \item $\text{màxim nre. columnes de } A \mid \text{família } LI$ 
\end{enumerate}
$\Rightarrow$ $\rank (A) = \rank (A^{t})$

%----------------------------------------------------------------------------------------
\subsection{Suma i intersecció de subespais vectorials}
Siguin $W_{1}$ i $W_{2}$ subespais vectorials d'un $K$-espai vectorial $V$.

\subsubsection*{Operacions}
\begin{itemize}
    \item Suma: $W_{1}+W_{2} = \left< W_{1} \cup W_{2} \right> \subseteq V \equiv$ mínim subespai $W'$ tal que $W_{1}, W_{2} \subseteq W' \subseteq V$.
    \item Intersecció: $W_{1} \cap W_{2} \equiv$ màxim subespai $W'$ tal que $W' \subseteq W_{1}$ i $W' \subseteq W_{3}$ alhora.
    \item Suma directa: $W_{1} \oplus W_{2} \equiv W_{1} + W_{2}$ si $W_{1} \cap W_{2} = \left\{ 0_{V} \right\}$. $W_{1} \oplus W_{2} \Leftrightarrow B_{W_{1}} \cup B_{W_{2}} = B_{W_{1} \oplus W_{2}}$.
\end{itemize}

\subsubsection*{Teorema de Grassman}
\begin{align}
    \dim (W_{1}) + \dim (W_{2}) = \dim (W_{1} + W_{2}) + \dim (W_{1} \cap W_{2})
\end{align}

\subsubsection*{Algoritme de la suma i intersecció}
Si ens proporcionen famílies de vectors que generin $W_{1}$ i $W_{2}$, podem aplicar el següent algoritme (esglaonant la matriu inicial per Gauss):
\begin{align}
    \begin{pmatrix}[c|c]
        \text{gen. } W_{1} & \text{gen. } W_{1} \\ \hline 
        \text{gen. } W_{2} & 0 
    \end{pmatrix} 
    \mapsto 
    \begin{pmatrix}[c|c]
        \text{base } W_{1} + W_{2} & \star \\ \hline 
        0 & \text{base } W_{1} \cap W_{2} \\ \hline 
        0 & 0 
    \end{pmatrix}
\end{align}

Cal observar que la matriu es construeix amb els vectors en fila.
	%----------------------------------------------------------------------------------------
%    APLICACIONS LINEALS
%----------------------------------------------------------------------------------------
\section{Aplicacions lineals}
\subsection{Aplicacions lineals}
Siguin $V$ i $V'$ $K$-espais vectorials. Una aplicació lineal de $V$ a $V'$ és una aplicació $f: V \to V'$ que compleix les propietats següents:
\begin{enumerate}[i)]
    \item $f (u + v) = f(u) + f(v) , \quad \forall u ,v \in V$.
    \item $f (\lambda u) = \lambda f(u) , \quad \forall \lambda \in K , \forall u \in V$.
\end{enumerate}
O equivalenment:
\begin{enumerate}[i)]
    \item $f (\lambda u + \mu v) = \lambda f(u) + \mu f(v) , \quad \forall \lambda , \mu \in K , \quad \forall u , v \in V$.
\end{enumerate}
En particular, $f$ envia subespais vectorials de $V$ a subespais vectorials de $V'$.
\begin{align}
    f( \left< u_{1}, \dots , u_{n} \right>) =  \left< f (u_{1}), \dots , f (u_{n}) \right>
\end{align}

%----------------------------------------------------------------------------------------
\subsection{L'espai vectorial de totes les aplicacions lineals}
\begin{align}
    \mathcal{L}(V , V') \equiv \{ f: V \to V' \mid f \text{ aplicació lineal} \}
\end{align}

%----------------------------------------------------------------------------------------
\subsection{Coordenades respecte d'una base}
Sigui $B = (u_{1}, \dots , u_{n})$ una base de $V$ i $c_{B}: K^{n} \to V$ una aplicació lineal.
\begin{align}
    c_{B}(\lambda_{1}, \dots , \lambda_{n}) \mapsto \lambda_{1} u_{1} + \dots + \lambda_{n} u_{n} = u \in V
\end{align}

Llavors, escriurem $C(u, B) = \begin{pmatrix} \lambda_{2} \\ \vdots \\ \lambda_{n} \end{pmatrix}$ per denotar que $u$ té coordenades $\lambda_{1}, \dots , \lambda_{n}$ respecte de la base $B$.

Observacions i conseqüències:
\begin{enumerate}[i)]
    \item Aquesta coordenació és una aplicació lineal bijectiva.
    \item A $K$-espai vectorial $V$ finitament generat es pot establir una aplicació bijectiva amb $K^{\dim (V)}$.
    \item Qualsevol problema lineal sobre $V$ es pot traduir en un problema numèric a $K^{\dim (V)}$, que es pot resoldre amb ordinador fàcilment.
\end{enumerate}

%----------------------------------------------------------------------------------------
\subsection{Matriu d'una aplicació lineal}
Sigui $f_{A}: V_{1} \to V_{2}$ una aplicació entre $K$-espais vectorials de dimensió finita no nuls. Siguin $B_{1} = (v_{1}, \dots , v_{n})$ una base de $V_{1}$ i $B_{2} = (u_{1}, \dots , u_{n})$ una base de $V_{2}$. Llavors:
\begin{align}
    A = M (f, B_{1}, B_{2})
\end{align}

Així doncs, 
\begin{align}
    C(f_{A} (v), B_{2}) = M (f, B_{1}, B_{2}) C(v, B_{1})
\end{align}
Observacions sobre la matriu $A$:
\begin{enumerate}[i)]
    \item Les files de $A$ són els coeficients de les formes lineals que descriuen les coordenades del vector imatge.
    \item Les columnes de $A$ són les imatges per $f_{A}$ de la base $B_{1}$ de $V_{1}$.
\end{enumerate}

\subsubsection*{Canvi de coordenades}
Siguin $B = (v_{1}, \dots , v_{n})$ i $B' = (u_{1}, \dots , u_{n})$ bases de $V$ i $u \in V$. Llavors:
\begin{align}
    C(u, B') = M (id, B, B') C(u, B)
\end{align}
Aquesta matriu, que denotarem simplement per $M(B, B')$ transforma les coordenades de $u$ respecte $B$ e coordenades del mateix vector resoecte de $B'$. Aquesta matriu s'anomena matriu del canvi de coordenades de $B$ a $B'$.

%----------------------------------------------------------------------------------------
\subsection{Nucli i imatge d'una aplicació lineal}
Sigui $f: V \to V'$ una apliació lineal.
\begin{enumerate}[i)]
    \item $f$ és monomorfisme $\Leftrightarrow \ker (f) = \{ 0_{V} \}$. $f$ envia famílies $LI$ de $V$ a famílies $LI$ de $V'$.
    \item $f$ és epimorfisme $\Leftrightarrow \im (f) = f(V) = V'$. $f$ envia generadors de $V$ a generadors de $V'$.
    \item $f$ és isomorfisme $\Leftrightarrow f$ és monomorfisme i epimorfisme alhora. $f$ envia bases de $V$ a bases de $V'$.
\end{enumerate}

\subsubsection*{Criteri a $f_{A}: K^{n} \to K^{m}$}
\begin{enumerate}[i)]
    \item Monomorfisme: $\rank (A) = n$.
    \item Epimorfisme: $\rank (A) = m$.
    \item Isomorfisme: $\rank (A) = n = m$.
\end{enumerate}

%----------------------------------------------------------------------------------------
\subsection{Problema típic}
Donada una aplicació lineal $f: V \to V'$, trobar una base i la dimensió de $\ker (f) \subseteq V'$ i $\im (f) \subseteq V'$.
\begin{example}
\begin{align*}
\begin{gathered}
    \quad f: \mathbb{R}^{5} \to \mathbb{R}^{3} \\
    f (x,y,z,t,u) = (4z + 8t + 2u , x + y + 2t + u , 3y + z + 2t + 2u)
\end{gathered}
\end{align*}
\end{example}
Procediment:
\begin{enumerate}[i)]
    \item Calculem la varietat lineal del sistema igualat a zero (nucli), tot simplificant per Gauss--Jordan el sistema. $\dim (\ker (f)) = \rank (A)$. 
    \item Transformem la matriu de Gauss--Jordan a la forma paramètrica. Una base del nucli són els vectors directors del sistema.
    \item Tenint en compte que $\dim (\ker (f)) + \dim (\im (f)) = \dim (V)$, calculem $\dim (\im (f))$.
    \item Una base de $\im (f)$ és qualsvol família $LI$ dels vectors columna de $A$ (sense simplificar per Gauss--Jordan). En concret les columnes de la matriu original que a Gauss--Jordan tenen pivot, són una base.
\end{enumerate}

% \part*{Àlgebra II}
	%----------------------------------------------------------------------------------------
%    ESPAI DUAL
%----------------------------------------------------------------------------------------
\section{Espai dual}
\subsection{Teoria}
% WIP

%----------------------------------------------------------------------------------------
\subsection{Màquina de xurros}
\begin{example}
Sigui $B=((1,2,3),(4,5,6),(7,8,9))$ una base de l'espai vectorial $V$. Llavors,
    \begin{align*}
        M(id, B, \mathcal{C}) = \begin{pmatrix} 1 & 4 & 7 \\ 2 & 5 & 8 \\ 3 & 6 & 9 \end{pmatrix}
    \end{align*}

que és la matriu de canvi de base de $B$ a $\mathcal{C}$ (la base canònica), o dit d'una altra manera, la base $B$ expressada en funció de la base $C$.

La base dual $B^{\ast}$ es calcula de la manera següent:
\begin{align*}
M(id, B^{\ast}, \mathcal{C}^{\ast}) = M(id, \mathcal{C}, B)^{t} = (M(id, B, \mathcal{C})^{t})^{-1}
\end{align*}
\end{example}

\begin{example}
Es consideren $B = (v_{1}, v_{2})$ una base de l'espai vectorial $V$ i $B^{\ast} = (\varphi_{1}, \varphi_{2})$ la seva base dual. Sigui $B' = (v_{1}+2 v_{2}, 3 v_{1} + 4 v_{2})$ una altra base de la qual volem trobar la seva base dual. Per començar, fem:
    \begin{align*}
        M(id, B', B) = \begin{pmatrix} 1 & 2 \\ 3 & 4 \end{pmatrix}
    \end{align*}
Llavors,
\begin{align*}
M(id, B'^{\ast}, B^{\ast}) = M(id, B, B')^{t} = (M(id, B', B)^{t})^{-1} = \begin{pmatrix} -2 & 1.5 \\ 1 & -0.5 \end{pmatrix}
\end{align*}
Així doncs, $B'^{\ast} = (-2 \varphi_{1} + 1.5 \varphi_{2}, \varphi_{1} - 0.5 \varphi_{2})$.
\end{example}

	%----------------------------------------------------------------------------------------
%    PERMUTACIONS
%----------------------------------------------------------------------------------------
\section{Permutacions}
\subsection{Permutacions}
Una permutació de $n$ elements és una aplicació bijectiva $\sigma : \left\{ 1 , \dots , n \right\} \to \left\{ 1 , \dots , n \right\}$. S'escriu $\sigma$ posant els valors que pren en tots els nombres $\left( 1 , \dots , n \right)$.
\begin{align}
    \begin{pmatrix} 1 & 2 & \dots & n-1 & n \\ \sigma (1) & \sigma (2) & \dots & \sigma (n-1) & \sigma (n) \end{pmatrix}
\end{align}
El conjunt de les permutacions de $n$ elments s'anomena $S_{n}$. N'hi ha $n!$ permutacions possibles:
\begin{align}
\begin{aligned}
    & S_{1} = \left\{ I \right\} \quad (1!) \\
    & S_{2} = \left\{ I , \begin{pmatrix} 1 & 2 \\ 2 & 1 \end{pmatrix} \right\} \quad (2!) \\
    & S_{3} = \left\{ I , \begin{pmatrix} 1 & 2 & 3 \\ 1 & 3 & 2 \end{pmatrix} , \dots , \begin{pmatrix} 1 & 2 & 3 \\ 3 & 2 & 1 \end{pmatrix} \right\} \quad (3!) \\
    & \dots
\end{aligned}
\end{align}
 A $S_{n}$ hi ha una operació (composició): $\sigma \circ \tau : S_{n} \times S_{n} \to S_{n}$.
\begin{example}
\begin{align*}
    \sigma = \begin{pmatrix} 1 & 2 & 3 \\ 2 & 1 & 3 \end{pmatrix}, \, \tau = \begin{pmatrix} 1 & 2 & 3 \\ 3 & 2 & 1 \end{pmatrix} \Rightarrow \sigma \circ \tau = \begin{pmatrix} 1 & 2 & 3 \\ 3 & 1 & 2 \end{pmatrix}
\end{align*}
\end{example}

\subsubsection*{Propietats}
\begin{itemize}
    \item Té propietat associativa.
    \item Té element neutre: $e \equiv I$.
    \item $\forall \sigma, \, \exists \sigma^{-1}$.
\end{itemize}

\subsubsection*{Inversa d'una permutació}
Sigui $\sigma = \tau_{1} \dots \tau_{n}$, on $\tau_{i} = (x_{1} \dots x_{n})$ són cicles disjunts. 
\begin{align*}
\begin{gathered}
    \tau_{i}: x_{1} \mapsto x_{2}, \quad \dots , \quad x_{n-1} \mapsto x_{n}, \\
    \tau_{i}^{-1}: x_{n} \mapsto x_{n-1}, \quad \dots , \quad x_{2} \mapsto x_{1}
\end{gathered}
\end{align*}

Així doncs, per calcular $\sigma^{-1}$ escribim els cicles $\tau_{i}$ del revés:
\begin{align}
\begin{gathered}
    \sigma = \begin{pmatrix}1 & 2 & 3 & 4 & 5 \\ 2 & 5 & 4 & 3 & 1\end{pmatrix},\\
    \sigma^{-1} = \begin{pmatrix}2 & 5 & 4 & 3 & 1\\ 1 & 2 & 3 & 4 & 5 \end{pmatrix} =\begin{pmatrix}1 & 2 & 3 & 4 & 5 \\ 5 & 1 & 4 & 3 & 2\end{pmatrix}
\end{gathered}
\end{align}
\begin{align}
    \sigma = (1 \, 2 \, 5) (3 \, 4), \quad \sigma^{-1} = (5 \, 2 \, 1) (4 \, 3) = (1 \, 5 \, 2) (3 \, 4)
\end{align}

%----------------------------------------------------------------------------------------
\subsection{Cicles de permutacions}
Un cicle de longitud $k$ a $S_{n} = \left\{ \text{permutacions de $n$ elements} \right\}$ és una $\sigma$ que mou $k$ elements $n_{1} , n_{2} , \dots , n_{k-1} , n_{k}$: $\sigma (n_{1}) = n_{2} , \dots , \sigma (n_{i}) = n_{i+1} , \dots , \sigma (n_{k}) = n_{1}$.
\begin{example}
\begin{align*}
    \begin{pmatrix} 1 & 2 & 3 & 4 & 5 \\ 3 & 1 & 4 & 2 &5 \end{pmatrix} \in S_{5} \equiv (1 \, 3 \, 4 \, 2) \text{ és un cicle de longitud $4$}.
\end{align*}
\end{example}

\subsubsection*{Descomposició en cicles disjunts}
Qualsevol permutació es pot escriure com a un producte (composició) de cicles en què intervenen elements diferents.
\begin{example}
\begin{align*}
\begin{gathered}
    \begin{pmatrix} 1 & 2 & 3 & 4 & 5 & 6 \\ 1 & 4 & 6 & 5 & 2 & 3 \end{pmatrix} \in S_{6} = (2 \, 4 \, 5) (3 \, 6) = (3 \, 6)  (2 \, 4 \, 5)\\
    \text{Quan dues permutacions mouen elements diferents, commuten.}
\end{gathered}
\end{align*}
\end{example}
Un cicle de longitud $n$ qualsevol es pot escriure com a producte de $n-1$ transposicions (cicles de longitud $2$):
\begin{align}
    (n_{1} \, n_{2} \, \dots \, n_{k-1} \, n_{n}) = (n_{1} \, n_{2}) \dots (n_{k-1} \, n_{k})
\end{align}

\begin{example}
\begin{align*}
    (1 \, 3 \, 4 \, 2) = (1 \, 3) (3 \, 4) (4 \, 2)
\end{align*}
\end{example}
$\Rightarrow$ Qualsevol permutació $\sigma$ es pot expressar com a un producte de transposicions no únic.

%----------------------------------------------------------------------------------------
\subsection{Ordre d'una permutació}
L'ordre d'una permutació $\sigma$ és el mínim enter positiu $m$ tal que $\sigma^{m}$ sigui la identitat.

Sigui $\sigma$ una permutació de cicles disjunts. Com que els cicles disjunts commuten, 
\begin{align}
    \sigma = \sigma_{1} \sigma_{2} \dots \sigma_{n} \Rightarrow \sigma^{m} = \sigma_{1}^{m} \sigma_{2}^{m} \dots \sigma_{n}^{m} 
\end{align}

Així doncs, és fàcil veure que l'ordre d'una permutació és el mínim comú múltiple de les longituds dels cicles que formen la descomposició de la permutació en cicles disjunts. 
\begin{example}
\begin{align*}
\begin{gathered}
    \sigma =\begin{pmatrix} 1 & 2 & 3 & 4 & 5 \\ 2 & 1 & 5 & 3 & 4 \end{pmatrix} = (1 \, 2) (3 \, 5 \, 4) \\
    \text{L'ordre de $\sigma$} = \operatorname{mcm} (2 , 3) = 6 \Rightarrow \sigma^{6} = (1 \, 2)^{6} (3 \, 5 \, 4)^{6} = I 
\end{gathered}
\end{align*}
\end{example}

%----------------------------------------------------------------------------------------
\subsection{Signe d'una permutació}
El signe o paritat d'una permutació $\sigma$ es defineix com a:
\begin{align}
    \operatorname{sgn} (\sigma) = (-1)^{m}, \quad \text{on } m \text{ és el nombre de transposicions.}
\end{align}
Per a una $\sigma$ donada, la paritat és independent de la tria de transposicions. 
	%----------------------------------------------------------------------------------------
%    DETERMINANTS
%----------------------------------------------------------------------------------------
\section{Determinants}
\subsection{Determinants}
Un determinant és una aplicació $\det : M(n,K) \to K$ que compleix les propietats següents:
\begin{enumerate}[i)]
    \item Depèn linealment de les columnes.
        \begin{enumerate}
            \item $\det (C_{1}, \dots , C_{i}+C_{i'}, \dots , C_{n}) =  \det (C_{1}, \dots , C_{i}, \dots , C_{n}) \\ + \det (C_{1}, \dots , C_{i'}, \dots , C_{n})$.
            \item $\det (C_{1}, \dots , \alpha C_{i}, \dots , C_{n}) = \alpha \det (C_{1}, \dots , C_{i}, \dots , C_{n}), \quad \forall \alpha \in K$.
        \end{enumerate}
    \item $\forall A \in M(n,K)$ amb dues columnes iguals $\Rightarrow \det(A) = 0$.
    \item $\det (I(n)) = 1$.
\end{enumerate}

\subsubsection*{Teorema}
\begin{enumerate}[i)]
    \item $\det (C_{1}, \dots , C_{i}, \dots , C_{j} , \dots , C_{n}) = - \det (C_{1}, \dots , C_{j}, \dots , C_{i} , \dots , C_{n})$.
    \item $\det (C_{1}, \dots , C_{i}, \dots , C_{j} , \dots , C_{n}) = \det (C_{1}, \dots , C_{i}+ \mu C_{j}, \dots , C_{j} , \dots , C_{n})$, $\forall \mu \in K$.
    \item $\det(AB) = \det(A) \det(B)$.
    \item $\det(A^{t}) = \det (A)$.
    \item $U = M_{\text{triangular superior}} \Rightarrow \det(U) = \prod \, \text{elements de la diagonal} \Rightarrow \det(A) = \det(GJ (A)) = \prod \, \text{elements de la diagonal}$.
    \item $\det(A) \neq 0 \Leftrightarrow A$ és invertible.
    \item Totes les propietats són equivalents per a files.
\end{enumerate}

%----------------------------------------------------------------------------------------
\subsection{Càlcul d'un determinant}
\subsubsection*{Forma permutacional}
\begin{align}
    \begin{aligned}
        \begin{vmatrix} a & b\\ c & d\end{vmatrix} &= \begin{vmatrix} a+0 & b \\ 0+c & d \end{vmatrix} = \begin{vmatrix} 0 & b \\ c & d \end{vmatrix} + \begin{vmatrix} a & b \\ 0 & d \end{vmatrix} = \begin{vmatrix} a & b \\ 0 & 0 \end{vmatrix} + \begin{vmatrix} a & 0 \\ 0 & d \end{vmatrix} + \begin{vmatrix} 0 & b \\ c & 0 \end{vmatrix} + \begin{vmatrix} 0 & 0 \\ c & d \end{vmatrix} \\
    &= ab \begin{vmatrix} 1 & 1 \\ 0 & 0 \end{vmatrix} + ad \begin{vmatrix} 1 & 0 \\ 0 & 1 \end{vmatrix} + bc \begin{vmatrix} 0 & 1 \\ 1 & 0 \end{vmatrix} + cd \begin{vmatrix} 0 & 0 \\ 1 & 1 \end{vmatrix} \\
    &= ab (0) + ad (1) +bc (-1) + cd (0) = ad - bc
    \end{aligned}
\end{align}

Sigui $A \in M(n,K)$. $\det (A)$ és la suma dels productes de $n$ (2 en aquest cas) coeficients de columnes diferents pel determinant d'una matriu amb $n$ uns en $n$ columnes i la resta zeros.
\begin{align}
    \det (A) = \sum\limits_{\sigma \in S_{n}} \operatorname{sgn} (\sigma) \, a_{\sigma (1) 1} a_{\sigma (2) 2} \dots a_{\sigma (n) n}
\end{align}

Aquesta fórmula és important, ja que demostra que $\exists \det (A)$ i també que $\det (A) = \det (A)^{t}$, però a la pràctica és poc eficient.

\subsubsection*{Desenvolupament per files/columnes}
\begin{align}
\begin{gathered}
    \det (A) = \sum\limits_{i=1}^n a_{ij} C_{ij}, \quad C_{ij} \equiv (-1)^{i+j} M_{ij}, \\ M_{ij}\equiv \text{ matriu adjunta a } a_{ij}
\end{gathered}
\end{align}
\begin{example}
\begin{align*}
    \begin{vmatrix} 3 & 2 & 1 \\ 4 & 7 & 8 \\ 0 & 1 & 2 \end{vmatrix} = 3 \begin{vmatrix} 7 & 8 \\ 1 & 2 \end{vmatrix} - 4 \begin{vmatrix} 2 & 1 \\ 1 & 2 \end{vmatrix} + 0 \begin{vmatrix} 2 & 1 \\ 7 & 8 \end{vmatrix} = 18 - 12 = 6
\end{align*}
\end{example}

Aquest mètode a la pràctica és també poc eficient, ja que quan es treballa amb matrius amb incògnites el càlcul es complica força.

\subsubsection*{Esglaonar}
Com que $\det (U) = \prod \, \text{elements de la diagonal}$, es tracta de diagonalitzar la matriu aplicant canvis elementals. Un cop diagonalitzada, el càlcul de $\det(U)$ és trivial.
\begin{example}
\begin{align*}
    \begin{vmatrix} b & b & b \\ c & c & b \\ d & c & b \end{vmatrix} = \begin{vmatrix} b & c & d \\ 0 & b-c & b-c \\ 0 & 0 & c-d \end{vmatrix} = b (b - c) (c - d)
\end{align*}
\end{example}

Encara que aquest mètode és força eficient, el millor mètode és esglaonar quan convingui per fer zeros a files/columnes i després desenvolupar per files/columnes, estalviant molts càlculs.

%----------------------------------------------------------------------------------------
\subsection{Fórmula de la matriu inversa}
Hem dit que $\det(A) = \sum\limits_{i=1}^n a_{ij} C_{ij}$. Si $j \neq i$, $a_{i1} A_{j1} + \dots + a_{ij} A_{ji} + \dots + a_{in} A_{jn} = \delta_{ij} \det (A)$.

Així doncs,
\begin{align}
    \text{Si } \det(A) \neq 0 \Rightarrow A^{-1} = \frac{\operatorname{Adj}(A)^{t}}{\det(A)}
\end{align}
%----------------------------------------------------------------------------------------
\subsection{Interpretació geomètrica}
Es pot donar una interpretació geomètrica al valor del determinant d'una matriu $A \in M(n)$ amb entrades reals: el valor absolut del determinant ens dóna el factor pel qual una àrea o volum (o qualsevol anàleg de dimensió major) està multiplicat sota la transformació lineal associada, mentre que el signe indica si la transformació manté l'orientació.

En altres paraules: $|\det(A)| \equiv$ volum del paral·lelepípede $n$-dimensional que formen els vectors fila/columna de $A$.

	%----------------------------------------------------------------------------------------
%    DIAGONALITZACIÓ
%----------------------------------------------------------------------------------------
\section{Diagonalització}
\subsection{Conceptes bàsics}
Diem que una matriu $A \in M(n, K)$ és diagonalitzable si $\exists P \in M(n, K)$ invertible tal que $P^{-1} A P \in D(n, K)$, és a dir
\begin{align}
    P^{-1} A P = \begin{pmatrix} \lambda_{1} & & 0 \\ & \ddots & \\ 0 & & \lambda_{n} \end{pmatrix}
\end{align}
per a certs $\lambda_{1}, \dots , \lambda_{n} \in K$.

Un endomorfisme $f$ de $V$ és diagonalitzable si existeix una base $B$ de $V$ tal que la matriu $A$ de $f$ en aquesta base sigui diagonal.

Sigui $f: V \to V$ un enfomorfisme. Suposem que $B = (v_{1}, \dots , v_{n})$ és una base de $V$ tal que $M(f, B)$ és la matriu diagonal $\begin{pmatrix} \lambda_{1} & & 0 \\ & \ddots & \\ 0 & & \lambda_{n} \end{pmatrix}$. Això vol dir que, per a $i = 1, \dots , n$
\begin{align}
f(v_{i}) = \lambda_{1} v_{i}
\end{align}
és a dir $B$ és una base tal que tots els seus elements satisfan que la seva imatge per $f$ és un múltiple d'ell mateix. Aquest tipus de vectors es diuen vectors propis de $f$.

Sigui $A = M(f, B)$ i $B' = (u_{1}, \dots , u_{n})$. Llavors $A$ diagonalitza en $A' = P^{-1} A P = M(f, B')$, i en particular $P = \begin{pmatrix} u_{1} & \dots & u_{n} \end{pmatrix}$.

%----------------------------------------------------------------------------------------
\subsection{Vectors i valors propis}
Sigui $f$ un endomorfisme de $V$. Un valor propi ($\vap$) de $f$ és un escalar $\lambda \in K$ tal que $\exists u \in V \backslash \{0_{V}\}$ que compleix:
\begin{align}
    f(u) = \lambda u
\end{align}
Un vector propi ($\vep$) de $f$ és un vector no nul $v \in V$ tal que 
\begin{align}
    f(v) = \lambda v \quad \text{per a algun } \lambda \in K
\end{align}

En aquest cas diem que $v$ és un $\vep$ de $f$ amb $\vap$ $\lambda$.
\begin{example}
\begin{align*}
    \begin{pmatrix} 2 & 2 \\ 1 & 3 \end{pmatrix} \begin{pmatrix} 1 \\ 1 \end{pmatrix} = \begin{pmatrix} 4 \\ 4 \end{pmatrix} = 4 \begin{pmatrix} 1 \\ 1 \end{pmatrix}
\end{align*}
$\Rightarrow u = (1,1)$ és un vector propi de $f$ amb valor propi $\lambda = 4$.
\end{example}

\subsubsection*{Subespai propi d'un $\vap$}
Si $\lambda$ és un $\vap$ de $f$, es diu que $\ker (f - \lambda id)$ és el subespai propi de $f$ de $\vap$ $\lambda$.
\\ \\
Sigui $f: V \to V$ un endomorfisme. Per a un vector $v \in V$ les afirmacions següents són equivalents.
\begin{enumerate}[i)]
    \item $v$ és un $\vep$ de $f$.
    \item $\exists \lambda \in K$ tal que $v$ és un vector no nul de $\ker (f - \lambda id)$.
\end{enumerate}
Sigui $\lambda$ un $\vap$ de $f$ de $V$. Llavors:
\begin{align}
    \left< \vep \text{ de } \vap \: \lambda \right> \equiv \ker (f - \lambda id) \backslash \{0_{V}\}
\end{align}

%----------------------------------------------------------------------------------------
\subsection{Polinomi característic d'un endomorfisme}
Sigui $A \in M(n,K)$. Definim el polinomi característic de $A$ com:
\begin{align}
    p_{A} (x) \equiv \det (A - x I(n))
\end{align}
Sigui $f$ un endomorfisme de $V$ i sigui $\lambda \in K$. Llavors:
\begin{align}
    \lambda \text{ és } \vap \text{ de } f \Leftrightarrow p_{f} (\lambda) = 0
\end{align}
Sigui $f$ un endomorfisme de $V$. Suposem que $p_{f} (x) = (x - \lambda)^{m} q(x)$, $q(\lambda) \neq 0$ i $m \geq 1 \equiv$ multiplicitat de l'arrel. Llavors:
\begin{align}
    1 \leq \dim (\ker (f - \lambda id) \leq m
\end{align}

\subsubsection*{Teorema de diagonalització}
Un endomorfisme $f$ de $V$ és diagonalitzable si i només si $\exists \lambda_{1} , \dots , \lambda_{r} \in K$ diferents i $n_{1}, \dots , n_{r} \in \mathbb{N}$ tals que
\begin{align}
\begin{gathered}
    p_{f} (x) = (-1)^{n} (x - \lambda_{1})^{m_{1}} \dots (x - \lambda_{k})^{m_{k}} \\
    \text{i} \quad \dim (\ker (f - \lambda_{i} id)) = m_{i}, \quad \forall i \in \{ 1, \dots , k \}
\end{gathered}
\end{align}

En particular, tot endomorfisme $f$ de $V$ tal que
\begin{align}
    p_{f} (x) = (x - \lambda_{1}) \dots (x - \lambda_{n})
\end{align}

amb $\lambda_{1}, \dots , \lambda_{n} \in K$ diferents, és diagonalitzable.
\\ \\
\begin{example}
Sigui $g: \mathbb{R}^{2} \to \mathbb{R}^{2}$ l'endomorfisme definit per $g(x,y) = (x-y, x+y)$. La matriu de $g$ respecte de la base canònica de $\mathbb{R}^{2}$ és
\begin{align*}
    A = \begin{pmatrix} 1 & -1 \\ -1 & 1 \end{pmatrix}
\end{align*}
El polinomi característic de $g$ és 
\begin{align*}
    p_{A}(x) = (x-2) x
\end{align*}

Per tant, $g$ diagonalitza amb $\vap$ $\lambda_{1} = 0$ i $\lambda_{2} = 2$. $(1,1)$ és un $\vep$ de $\vap$ 0 i $(1, -1)$ és un $\vep$ de $\vap$ 2. Llavors:
\begin{align*}
P = \begin{pmatrix} 1 & 1 \\ -1 & 1 \end{pmatrix} \Rightarrow P^{-1} = \begin{pmatrix} \sfrac{1}{2} & \sfrac{1}{2} \\ -\sfrac{1}{2} & \sfrac{1}{2} \end{pmatrix}
\end{align*}

i 
\begin{align*}
P^{-1} A P = \begin{pmatrix} 2 & 0 \\ 0 & 0 \end{pmatrix}
\end{align*}
\end{example}
%----------------------------------------------------------------------------------------
\subsection{Aplicacions de la diagonalització}
\subsubsection*{Potències de matrius/endomorfismes}
Si $A \in M(n,K)$ és diagonalitzable, $\exists P \in GL(n,K)$ tal que
\begin{align}
    D = P^{-1}AP \quad
\end{align}

Llavors, es compleix que
\begin{align}
    A^{n} = P D^{n} P^{-1} \quad
\end{align}

\subsubsection*{Recurrència i dinàmica}
\begin{example}
    WIP: problema 20 de la primera entrega d'Àlgebra II
\end{example}
	%----------------------------------------------------------------------------------------
%    FORMES BILINEALS
%----------------------------------------------------------------------------------------
\section{Formes bilineals}
\subsection{Formes bilineals}
Sigui $K$ un cos, i $V$ un $K$-espai vectorial. Una aplicació
\begin{align}
    \left< - \mid - \right>: V \times V \to K
\end{align}
es diu que és una forma bilineal sobre $V$ si es compleixen les propietats següents:
\begin{enumerate}[i)]
    \item $\left< \lambda u + \mu u' \mid v \right> = \lambda \left< u \mid v \right> + \mu \left< u' \mid v \right>, \quad \forall \lambda , \mu \in K , \, u , u' , v \in V$.
    \item $\left< u \mid \lambda v + \mu v' \right> = \lambda \left< u \mid v \right> + \mu \left< u \mid v' \right>, \quad \forall \lambda , \mu \in K , \, u , v , v' \in V$.
\end{enumerate}
La forma bilineal $\left< - \mid - \right>$ es diu que és simètrica si $\left< u \mid v \right> = \left< v \mid u \right>, \quad \forall u , v \in V$.

\subsubsection*{Producte escalar estàndard}
El producte estàndard de $\mathbb{R}^{n}$ és l'aplicació $\left< - \mid - \right>: \mathbb{R}^{n} \times \mathbb{R}^{n} \to \mathbb{R}$ definida per
\begin{align}
    \left< (x_{1}, \dots , x_{n}) \mid (y_{1}, \dots , y_{n}) \right> = x_{1} y_{1} + \dots + x_{n} y_{n}
\end{align}
El producte escalar estàndard és una forma bilineal simètrica.

%----------------------------------------------------------------------------------------
\subsection{Matriu associada a una forma bilineal}
Sigui $\left< - \mid - \right>: V \times V \to K$ una forma bilineal, i sigui $B = ( v_{1}, \dots , v_{n} )$ una base de $V$. Considerem la matriu $A = (a_{ij}) \in M(n,K)$, definida per 
\begin{align}
    a_{ij} = \left< v_{i} \mid v_{j} \right>
\end{align}

La matriu $A$ és diu que és la matriu associada a la forma bilineal $\left< - \mid - \right>$ en la base $B$ i la denotem per
\begin{align}
    A = M(\left< - \mid - \right>, B)
\end{align}
Amb aquesta matriu podem calcular la forma bilineal sobre qualsevol parell de vectors:
\begin{align}
    \left< u \mid v \right> = u^{t} A v \equiv C(u, B)^{t} A C(v, B)
\end{align}
\\
Sigui $B = (v_{1}, \dots , v_{n})$ una base de $V$. Una forma lineal $\left< - \mid - \right>$ sobre $V$ és simètrica $\Leftrightarrow M(\left< - \mid - \right>, B)$ és una matriu simètrica.

Un exemple n'és el producte escalar estàndard, on $\left< u \mid v \right> = u^{t} I(n) v$.

\subsubsection*{Fórmula del canvi de base}
\begin{align}
    M(\left< - \mid - \right>, B_{2}) = M(B_{2}, B_{1})^{t} M(\left< - \mid - \right>, B_{1}) M(B_{2}, B_{1})
\end{align}
%----------------------------------------------------------------------------------------
\subsection{L'espai vectorial de les formes bilineals}
Considerem el conjunt
\begin{align}
    \operatorname{Bil} (V) \equiv \{ \left< - \mid - \right>: V \times V \to K \mid \left< - \mid - \right> \text{ és una forma bilineal} \}.
\end{align}
Sobre aquest conjunt, definim una suma i un producte per escalars de $K$ de la manera següent:
\begin{align}
    \begin{matrix} 
        \left< - \mid - \right>_{1} + \left< - \mid - \right>_{2}: & V \times V & \to & K \\
        & (v, w) & \mapsto & \left< v \mid w \right>_{1} + \left< v \mid w \right>_{2}
    \end{matrix}
\end{align}
\begin{align}
    \begin{matrix} 
        \lambda \left< - \mid - \right>: & V \times V & \to & K \\
        & (v, w) & \mapsto & \lambda \left< v \mid w \right>
    \end{matrix}
\end{align}
Aquest conjunt té estructura de $K$-espai vectorial.

La dimensió de $\operatorname{Bil} (V)$ és $n^{2}$ perquè és un $K$-espai vectorial isomorf a $M(n,K)$.

\subsubsection*{Conjunt de les formes bilineals simètriques}
El conjunt de les formes bilineals simètriques és un subespai vectorial de $\operatorname{Bil} (V)$:
\begin{align}
    \operatorname{SBil} (V) \equiv \{ \left< - \mid - \right>: V \times V \to K \mid \left< - \mid - \right> \text{ és simètrica} \}. 
\end{align}
La dimensió de $\operatorname{SBil} (V)$ és $\frac{n(n+1)}{2}$ perquè és un subespai vectorial isomorf al subespai vectorial de les matrius simètriques de $M(n,K)$.

%----------------------------------------------------------------------------------------
\subsection{Bases ortogonals}
Sigui $\left< - \mid - \right>: V \times V \to \mathbb{R}$ una forma bilineal simètrica. Es diu que $u, v \in V$ són ortogonals si $\left< u \mid v \right> = 0 \Rightarrow u \perp v$.

Un vector no nul $v \in V$ és isòtrop si és ortogonal a si mateix, és a dir, si $\left< v \mid v \right> = 0$.

Sigui $B = (v_{1}, \dots , v_{n})$ una base de $V$. Es diu que $B$ és una base ortogonal respecte de $\left< - \mid - \right>$ si $\left< v_{i} \mid v_{j} \right> = 0, \quad \forall i \neq j$.

Cal observar que la matriu associada a la forma bilineal simètrica en una base ortogonal és una matriu diagonal, és a dir, $M(\left< - \mid - \right>, B) \in D(n)$.

\subsubsection*{Teorema}
Sigui $\left< - \mid - \right>: V \times V \to \mathbb{R}$ una forma bilineal simètrica. Llavors $V$ té una base ortogonal respecte $\left< - \mid - \right>$.

\subsubsection*{Complement ortogonal}
Sigui $W$ un $\operatorname{SBil} (V)$ i $\left< u_{1}, \dots , u_{r} \right> = W$, llavors, definim el subespai ortogonal a $W$ com:
\begin{align}
    W^{\perp} \equiv \{ v \in V \mid u \perp v, \quad \forall u \in W \}
\end{align}

El complement ortogonal compleix les següents propietats:
\begin{enumerate}[i)]
    \item $W \oplus W^{\perp} = V$.
    \item $\dim (W) + \dim (W^{\perp}) = \dim (V)$.
\end{enumerate}

\subsubsection*{Mètode de Gram--Schmidt d'ortogonalització}
Sigui $B = (v_{1}, v_{2}, v_{3} \dots , v_{k})$ una base de $V$ i $\left< - \mid - \right>$ una forma bilineal simètrica. Llavors, per trobar uns vectors $u_{1}, u_{2}, u_{3}, \dots u_{k}$ que formin una base $B'$ ortogonal respecte de $\left< - \mid - \right>$, apliquem el mètode de Gram--Schmidt.

Definim l'operador projector com:
\begin{align}
    \operatorname{proj}_{u} (v) \equiv \frac{\left< u \mid v \right>}{\left< u \mid u \right>} u
\end{align}

El mètode de Gram--Schmidt va de la manera següent:
\begin{align*}
    u_{1} & = v_{1}, \\
    u_{2} & = v_{2} - \operatorname{proj}_{u_{1}} (v_{2}), \\
    u_{3} & = v_{3} - \operatorname{proj}_{u_{1}} (v_{3}) - \operatorname{proj}_{u_{2}} (v_{3}), \\
    & \, \, \, \vdots
\end{align*}
\begin{align}
    u_{k} = v_{k} - \sum\limits_{j=1}^{k-1} \operatorname{proj}_{u_{j}} (v_{k})
\end{align}

Cal notar que per què aquest mètode es pugui aplicar, $\left< v_{1} \mid v_{1} \right> \neq 0$, és a dir, si el element $a_{1,1}$ de $M(\left< - \mid - \right>, B)$ és zero, haurem de permutar files abans de fer Gram--Schmidt.

\subsubsection*{Definició d'una forma bilineal}
Sigui $\left< - \mid - \right>: V \times V \to \mathbb{R}$ una forma bilineal simètrica i $v \in V$.
\begin{enumerate}[i)]
    \item $\left< - \mid - \right>$ és definida positiva $\Leftrightarrow \left< v \mid v \right> > 0  \Leftrightarrow M( \left< - \mid - \right> , B ) = I(n)$.
    \item $\left< - \mid - \right>$ és definida negativa $\Leftrightarrow \left< v \mid v \right> < 0 \Leftrightarrow M( \left< - \mid - \right> , B ) = -I(n)$.
    \item $\left< - \mid - \right>$ és semidefinida positiva $\Leftrightarrow \left< v \mid v \right> \geq 0$.
    \item $\left< - \mid - \right>$ és semidefinida negativa $\Leftrightarrow \left< v \mid v \right> \leq 0$.
\end{enumerate}

Les formes bilineals definides positives s'anomenen productes escalars.

\subsubsection*{Teorema de Sylvester}
Sigui $\left< - \mid - \right>: V \times V \to \mathbb{R}$ una forma bilineal simètrica. Llavors $V$ té una base ortogonal respecte $\left< - \mid - \right>$ tal que la matriu associada respecte d'aquesta base és de la forma
\begin{align}
M(\left< - \mid - \right>, B) = 
\begin{pmatrix}
    1 & & & & & & & & \\
    & \dots^{r_{+}} & & & & & & & \\
    & & 1 & & & & & & \\
    & & & -1 & & & & & \\
    & & & & \dots^{r_{-}} & & & & \\
    & & & & & -1 & & & \\
    & & & & & & 0 & & \\
    & & & & & & & \dots^{r_{0}} & \\
    & & & & & & & & 0
\end{pmatrix}
\end{align}

on $r_{+}$, $r_{-}$ i $r_{0}$ no depenen de la base $B$. És a dir, $\forall M(\left< - \mid - \right>, B) \in M(n, \mathbb{R})$, $r_{+} + r_{-} +r_{0} = n$.

La terna $(r_{+}, r_{-}, r_{0})$ s'en diu signatura de la forma bilineal, i és un invariant.

%----------------------------------------------------------------------------------------
\subsection{Productes escalars}
Aquesta secció la dedicarem a l'estudi dels $\mathbb{R}$-espai vectorials $V$ de dimensió finita amb un producte escalar. Aquests espais vectorials també es diuen espai euclidians.
Algunes propietats importants del producte escalar són:
\begin{itemize}
    \item En una base ortogonal $B$, tots els elements de la diagonal de la matriu $M(\left< - \mid - \right>, B)$ són $> 0$.
    \item $\exists$ una base $B$ tal que $M(\left< - \mid - \right>, B) = I(n)$.
    \item $\forall v \in V, \quad \left< v \mid v \right> \geq 0$ i $\left< v \mid v \right> \Leftrightarrow v = 0$.
\end{itemize}

\subsubsection*{Desigualtat de Cauchy-Schwartz}
Sigui $\left< - \mid - \right>$ un producte escalar sobre un $\mathbb{R}$-espai vectorial $V$. Llavors:
\begin{align}
    \left< u \mid v \right>^{2} \leq \left< u \mid u \right> \left< v \mid v \right>, \quad \forall u, v \in V
\end{align}

\subsubsection*{Norma sobre $V$}
Sigui $V$ un $\mathbb{R}$-espai vectorial. Una norma sobre $V$ és una aplicació:
\begin{align}
    \begin{matrix} \|~\|: & V & \to & \mathbb{R} \\ & u & \mapsto & \| u \| \end{matrix}
\end{align}
que compleix les següents propietats:
\begin{enumerate}[i)]
    \item $\| u \| \geq 0, \quad \forall u \in V$.
    \item $\| u \| = 0 \Leftrightarrow u = 0$.
    \item $\| \lambda u \| = | \lambda | \| u \|, \quad \forall \lambda \in \mathbb{R}, \forall u \in V$
    \item $\| u + v \| \leq \| u \| + \| v \|,  \quad \forall u, v \in V$ (desigualtat triangular).
\end{enumerate}

\subsubsection*{Norma associada a $\left< - \mid - \right>$}
Sigui $\left< - \mid - \right>$ un producte escalar sobre un $\mathbb{R}$-espai vectorial $V$. Llavors l'aplicació
\begin{align}
    \begin{matrix} \|~\|_{\left< - \mid - \right>} : & V & \to & \mathbb{R} \\ & u & \mapsto & \| u \|_{\left< - \mid - \right>} \end{matrix}
\end{align}
\begin{align}
    \text{on} \quad \| u \|_{\left< - \mid - \right>} = \sqrt{\left< u \mid u \right>}
\end{align}
és una norma, que es diu norma associada a $\left< - \mid - \right>$.

%----------------------------------------------------------------------------------------
\subsection{Bases ortonormals}
Una base ortonormal és una base ortogonal $B$ que compleix que $\| u \| = 1, \forall u \in B$. Cal notar que $M(\left< - \mid - \right>, B) = I(n)$.

Sigui $\left< - \mid - \right>$ un producte escalar sobre un $\mathbb{R}$-espai vectorial $V$ de dimensió finita. Llavors $V$ té una base ortonormal respecte $\left< - \mid - \right>$.

Per calcular una base ortonormal $B'$ es pot calcular una base ortogonal $B$ pel mètode de Gramm--Schmidt i dividir els vectors de $B$ per la seva norma:
\begin{align}
    B = ( u_{1}, \dots , u_{n} ) \Rightarrow B' = \left( \frac{u_{1}}{\| u_{1} \|} , \dots , \frac{u_{n}}{\| u_{n} \|} \right)
\end{align}

%----------------------------------------------------------------------------------------
\subsection{El Teorema Espectral}
Tota matriu simètrica de $M(n,\mathbb{R})$ diagonalitza en una base ortonormal respecte el producte escalar estàndard de $\mathbb{R}^{n}$.

\subsubsection*{Enfomorfisme autoadjunt}
Es diu que $f$ és un endomorfisme autoadjunt si i només si:
\begin{align}
    \left< f(u) \mid v \right> = \left< u \mid f(v) \right>
\end{align}

Sigui $\left< - \mid - \right>$ un producte escalar sobre un $\mathbb{R}$-espai vectorial $V$ de dimensió finita $n>0$. Sigui $f: V \to V$ un endomorfisme autoadjunt de $V$. Llavors $\exists \lambda_{1}, \dots , \lambda_{n} \in \mathbb{R}$ tals que:
\begin{align}
    P_{f}(x) = (-1)^{n} (x - \lambda_{1}) \dots (x - \lambda_{n})
\end{align}

\subsubsection*{Teorema espectral}
Sigui $\left< - \mid - \right>$ un producte escalar sobre un $\mathbb{R}$-espai vectorial $V$ de dimensió finita $n>0$. Sigui $f: V \to V$ un endomorfisme autoadjunt de $V$. Lavors $V$ té una base ortonormal de vectors propis de $f$. En particular, l'endomorfisme $f$ diagonalitza.
\begin{align}
    M(f, B, B) = M (\left< - \mid - \right>, B) \Rightarrow \left< u \mid v \right> = \left< f(u) \mid v \right>
\end{align}

Observació: si bé el mètode de Gramm--Schmidt d'ortonormalització és invàlid per a matrius no definides positives, sí que ho és el d'ortogonalització, ja que no treballa amb normes. De manera que es pot ortogonalitzar qualsevol forma bilineal sense haver d'utilitzar el teorema espectral, tot i que pot arribar a ser un procés més llarg.
	%----------------------------------------------------------------------------------------
%    GEOMETRIA EUCLIDIANA
%----------------------------------------------------------------------------------------
\section{Geometria euclidiana}
\subsection{Espai afí euclidià i varietats lineals}
Un espai vectorial euclidià és un $\mathbb{R}$-espai vectorial junt amb un producte escalar.

Un espai afí euclidià és un conjunt $E \neq \varnothing$ junt amb un espai vectorial euclidià $(V, \left< - \mid - \right>)$ i una aplicació
\begin{align}
    \begin{matrix} 
        \alpha: & E \times E & \to & V \\
        & (P, Q) & \mapsto & \vv{PQ}
    \end{matrix}
\end{align}
que compleix:
\begin{enumerate}[i)]
    \item Per a cada $P \in E$, l'aplicació
        \begin{align*}
            \begin{matrix} 
                \alpha_{P}: & E \times E & \to & V \\
                & Q & \mapsto & \vv{PQ}
            \end{matrix}
        \end{align*}
        \subitem és bijectiva.
    \item $\vv{PQ} + \vv{QR} = \vv{PR}, \quad \forall P, Q, R \in E$.
\end{enumerate}
Els elements de $E$ es diuen punts en l'espai, $V$ és l'espai vectorial associat a $E$. Definim la dimensió de l'espai afí euclidià $E$ com
\begin{align}
    \dim E \equiv \dim V
\end{align}

\subsubsection*{Varietats lineals}
Diem que un subconjunt $L$ de $E$ és una varietat lineal de $E$ si
\begin{enumerate}[i)]
    \item $\alpha (L \times L)$ és un subespai vectorial de $V$.
    \item $\forall P \in L$, 
        \begin{align*}
            \begin{matrix} 
                \alpha_{P}: & L \times L & \to & \alpha (L \times L) \\
                & Q & \mapsto & \vv{PQ}
            \end{matrix}
        \end{align*}
        \subitem és bijectiva.
\end{enumerate}
\begin{example}
Siguin $V = \mathbb{R}^{n}$ i $E = \mathbb{R}^{n}$. Sigui
\begin{align*}
    \alpha (u,v) = u-v
\end{align*}
Llavors $E$ amb $\alpha$ s'anomena espai afí euclidià estàndard.

Les varietats lineals d'aquest espai són de la forma $P+W$, on $P \in V$ i $W$ és un subespai vectorial de $V$, és a dir
\begin{align*}
    P + W = \left\{ P + w \mid w \in W \right\}
\end{align*}
Aquí $W$ és la direcció de $P + W$, i diem que $P + W$ és la varietat lineal que passa pel punt $P$ amb direcció $W$.
\end{example}
\begin{example}
Siguin $E = \mathbb{R}^{2}$, $L = \left\{ (x,y) \mid y - x = 1 \right\}$, i $P = (x,y)$ i $Q = (x',y')$ punts de $L$. Llavors:
\begin{align*}
\begin{split}
    \alpha \left( L \times L ) \right) & = \left\{ \vv{PQ} \mid y - x = 1, y' - x' = 1 \right\} = \\
    & = \left\{ (x'-x, y'-y) \mid y - x = 1, y' - x' = 1 \right\} = \left< (1, 1) \right> \\
    \Rightarrow L & = P + \left< (1, 1) \right>
\end{split}
\end{align*}
\end{example}

\subsubsection*{Intersecció de varietats lineals}
Dues varietats lineals $P_{1} + W_{1}$, $P_{2} + W_{2}$, d'un espai afí euclidià es tallen $\Leftrightarrow \vv{P_{1} P_{2}} \in W_{1} + W_{2}$.

Si $P_{1} + W_{1}$, $P_{2} + W_{2}$ són varietats lineals paral·leles, llavors o bé no es tallen o bé una està continguda dins l'altra (i si són de la mateixa dimensió, llavors són coincidents).

La intersecció de dues varietats lineals d'un espai afí euclidià és o bé buida o bé una varietat lineal.

%----------------------------------------------------------------------------------------
\subsection{Coordenades}
Un sistema de coordenades cartesianes ortonormal d'un espai afí euclidià $E$ de dimensió finita $n$ és
\begin{align}
    \mathcal{B} = \{ P, (v_{1}, \dots , v_{n}) \}
\end{align}

on $P$ és l'origen de coordenades i $(v_{1}, \dots , v_{n})$ és una base ortonormal de l'espai vectorial associat $V$; $\left< v_{1} \right>, \dots , \left< v_{n} \right>$ són els eixos de coordenades del sistema.

Les coordenades d'un punt $X \in E$ respecte de $\mathcal{B}$ són per definició les coordenades del vector $\vv{PX}$ respecte de la base $(v_{1}, \dots , v_{n})$. Escriurem $C(X, \mathcal{B}) = C(\vv{PX}, (v_{1}, \dots , v_{n}))$.

\subsubsection*{Fórmula del canvi de coordenades}
Siguin $\mathcal{B} = \{ P, (v_{1}, \dots , v_{n}) \}$, $\mathcal{B}' = \{ Q, (u_{1}, \dots , u_{n}) \}$ dos sistemes de coordenades ortonormals d'un espai afí euclidià $E$. Llavors:
\begin{align}
    C(X, \mathcal{B}) = C(Q, \mathcal{B}) + M(B', B) C(X, \mathcal{B}')
\end{align}

%----------------------------------------------------------------------------------------
\subsection{Distància i perpendicularitat}
Definim la distància en un espai afí euclidià $E$ com
\begin{align}
    \begin{matrix}
        d: & E \times E & \to & \mathbb{R} \\
        & (P, Q) & \mapsto & \| \vv{PQ} \|
    \end{matrix}
\end{align}
Observem que $d$ compleix les propietats següents:
\begin{enumerate}[i)]
    \item $d(P, Q) \geq 0, \quad \forall P, Q \in E$.
    \item $d(P, Q) = 0 \Leftrightarrow P = Q$.
    \item $d(P, Q) = d(Q, P), \quad \forall P, Q \in E$.
    \item $d(P, Q) + d(Q, R) \geq d(P, R), \quad P, Q, R \in E$ (desigualtat triangular).
\end{enumerate}

\subsubsection*{Teorema de Pitàgores}
Sigui $E$ un espai afí euclidià. Siguin $P, Q, R \in E$ tals que $\left< \vv{PQ} \mid \vv{PR} \right> = 0$. Llavors:
\begin{align}
    d(P,Q)^{2} + d(P, R)^{2} = d(Q, R)^{2}
\end{align}

\subsubsection*{Complement ortogonal}
Sigui $W$ un subespai vectorial d'un espai vectorial euclidià, llavors, definim el subespai ortogonal a $W$ com:
\begin{align}
    W^{\perp} \equiv \{ v \in V \mid u \perp v, \quad \forall u \in W \}
\end{align}

El complement ortogonal compleix les següents propietats:
\begin{enumerate}[i)]
    \item $W \oplus W^{\perp} = V$.
    \item $\dim (W) + \dim (W^{\perp}) = \dim (V)$.
\end{enumerate}

Diem que dues varietats lineals $P_{1}+W_{1}$, $P_{2}+W_{2}$ en un espai afí euclidià són ortogonals si $W_{1} \subseteq W_{2}^{\perp}$ o $W_{1}^{\perp} \subseteq W_{2}$.

\subsubsection*{Projecció ortogonal}
Sigui $E$ un espai afí euclidià de dimensió finita. Sigui $V$ l'espai vectorial associat i $\left< - \mid - \right>$ el seu producte escalar. Sigui $L$ una varietat lineal amb direcció $W$. Definim la projecció ortogonal de $E$ sobre $L$ per
\begin{align}
    \begin{matrix} \pi_{L}: & E & \to & L \\ & Q & \to & P+w_{1} \end{matrix}
\end{align}

on $P$ és un punt fixat de $L$ i $\vv{PQ} = w_{1} + w_{2}$ amb $w_{1} \in W$ i $w_{2} \in W^{\perp}$.
\\
Observem que $\vv{\pi_{L}(Q)Q} = w_{2} \in W^{\perp}$ i que $d(\pi_{L}(Q),Q)$ és mínima.

%----------------------------------------------------------------------------------------
\subsection{Distància entre dues varietats lineals}
Siguin $L_{1} = P_{1} + W_{1}$, $L_{2} = P_{2} + W_{2}$ dues varietats lineals d'un espai afí euclidià $E$ de dimensió finita. Definim la distància entre $L_{1}$ i $L_{2}$ com
\begin{align}
    d(L_{1}, L_{2}) = \inf \{d(P, Q) \mid P \in L_{1}, Q \in L_{2} \}
\end{align}

o, equivalentment,
\begin{align}
    d(L_{1}, L_{2}) = \inf \{ \| \vv{P_{1} P_{2}} + w_{1} + w_{2} \| \mid w_{1} \in W_{1}, w_{2} \in W_{2} \}
\end{align}

Observem que si $L_{1}$ i $L_{2}$ es tallen, $d(L_{1},L_{2}) = 0$.

\subsubsection*{Mètode 1}
Siguin $L_{1} = P_{1} + W_{1}$, $L_{2} = P_{2} + W_{2}$ dues varietats lineals d'un espai afí euclidià $E$ de dimensió finita. Sigui $\pi_{L}: E \to L$ la projecció ortogonal de $E$ sobre $L = P_{1} + (W_{1} + W_{2})$. Llavors:
\begin{align}
    d(L_{1}, L_{2}) = d(P_{2}, \pi_{L} (P_{2}))
\end{align}

i es compleix:
\begin{align}
    \pi_{L} (P_{2}) = P_{1} + \sum\limits_{i = 1}^{\dim (W_{1} + W_{2})} \left< \vv{P_{1} P_{2}} \mid v_{i} \right> v_{i}
\end{align}

on $(v_{1}, \dots , v_{i} , \dots , v_{n})$ és una base ortogonal de $W_{1} + W_{2}$.
\\  \\
Llavors, $\vv{P_{2} \pi_{L} (P_{2})} \subset W_{3} \equiv$ direcció de la varietat lineal $L_{3}$ tal que $L_{3} \perp L_{1}$ i $L_{3} \perp L_{2}$, que anomenarem perpendicular comuna entre $L_{1}$ i $L_{2}$.

\subsubsection*{Mètode 2}
Siguin $L_{1} = P_{1} + W_{1}$, $L_{2} = P_{2} + W_{2}$ dues varietats lineals d'un espai afí euclidià $E$ de dimensió finita. Llavors per calcular $d(L_{1}, L_{2})$ considerarem el següent:
\begin{enumerate}[i)]
    \item Calculem $(W_{1}+W_{2})^{\perp}$. Aquest subespai vectorial és ortogonal a $L = P_{1} + (W_{1} + W_{2})$.
    \item $E = (W_{1}+W_{2}) \oplus (W_{1}+W_{2})^{\perp}$.
    \item $\Rightarrow \vv{P_{1}P_{2}} = u + v, \quad u \in (W_{1}+W_{2})$, $v \in (W_{1}+W_{2})^{\perp}$. 
    \item $\Rightarrow d(L_{1}, L_{2}) = \| v \|$.
\end{enumerate}
\bigskip
Per calcular, en canvi, els dos punts $X_{1} \in L_{1}$, $X_{2} \in L_{2}$ tals que $d(X_{1}, X_{2}) = d(L_{1}, L_{2})$ haurem de fer les següents consideracions:
\begin{enumerate}[i)]
    \item $X_{1} = P_{1} + u_{1}, \quad  u_{1} \in W_{1}$.
    \item $X_{2} = P_{2} - u_{2}, \quad  u_{2} \in W_{2}$.
    \item $\vv{P_{1}P_{2}} = u_{1} + u_{2} + v, \quad v \in (W_{1}+W_{2})^{\perp}$.
    \item $u_{1} + u_{2} = u, \quad u \in (W_{1}+W_{2})$.
\end{enumerate}

%----------------------------------------------------------------------------------------
\subsection{Isometries i desplaçaments}
\subsubsection*{Isometries}
Una aplicació $f: E_{1} \to E_{2}$ entre dos espais afins euclidians es diu isometria si
\begin{align}
    d(f(P), f(Q)) = d(P, Q), \quad \forall P, Q \in E_{1}
\end{align}
Observem que $f$ és injectiva.

Siguin $E_{1}$, $E_{2}$ espais afins euclidians amb espais vectorials associats $V_{1}$, $V_{2}$ i productes escalars $\left< - \mid - \right>_{1}$, $\left< - \mid - \right>_{2}$ respectivament. Sigui $f: E_{1} \to E_{2}$ una isometria. Llavors l'aplicació
\begin{align}
    \tilde{f}: V_{1} \to V_{2}
\end{align}

definida per $\tilde{f}(\vv{PQ}) = \vv{f(P) f(Q)}$ està ben definida, és lineal i
\begin{align}
    \left< \tilde{f}(u_{1}) \mid \tilde{f}(u_{2}) \right>_{2} = \left< u_{1} \mid u_{2} \right>_{1}, \quad \forall u_{1}, u_{2} \in V_{1}
\end{align}

i es diu aplicació lineal associada a $f$.
\\
A més, si $g: V_{1} \to V_{2}$ és una aplicació lineal que conserva el producte escalar, $P_{1} \in E_{1}$ i $P_{2} \in E_{2}$, llavors l'aplicació
\begin{align}
    \begin{matrix} \tilde{g}: & E_{1} & \to & E_{2} \\ & P_{1} + u_{1} & \mapsto & P_{2} + g(u_{1}) \end{matrix}
\end{align}

és una isometria i $\tilde{\tilde{g}} = g$.

\subsubsection*{Desplaçaments}
Sigui $E$ un espai afí euclidià de dimensió finita. Un desplaçament en $E$ és una isometria de $E$ en ell mateix.

Considerem l'espai afí euclidià (estàndard) $\mathbb{R}^{n}$. Sigui
\begin{align}
    f: \mathbb{R}^{n} \to \mathbb{R}^{n}
\end{align}
un desplaçament. Llavors:
\begin{align}
    f \begin{pmatrix} x_{1} \\ \vdots \\ x_{n} \end{pmatrix} = \begin{pmatrix} P_{1} \\ \vdots \\ P_{n} \end{pmatrix} + A \begin{pmatrix} x_{1} \\ \vdots \\ x_{n} \end{pmatrix}
\end{align}

i $(P_{1}, \dots, P_{n})$ i $A \in M(n, \mathbb{R})$ són fixats per $f$.

%----------------------------------------------------------------------------------------
\end{document}

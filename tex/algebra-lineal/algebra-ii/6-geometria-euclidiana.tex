%----------------------------------------------------------------------------------------
%    GEOMETRIA EUCLIDIANA
%----------------------------------------------------------------------------------------
\section{Geometria euclidiana}
\subsection{Espai afí euclidià i varietats lineals}
Un espai vectorial euclidià és un $\mathbb{R}$-espai vectorial junt amb un producte escalar.

Un espai afí euclidià és un conjunt $E \neq \varnothing$ junt amb un espai vectorial euclidià $(V, \left< - \mid - \right>)$ i una aplicació
\begin{align}
    \begin{matrix} 
        \alpha: & E \times E & \to & V \\
        & (P, Q) & \mapsto & \vv{PQ}
    \end{matrix}
\end{align}
que compleix:
\begin{enumerate}[i)]
    \item Per a cada $P \in E$, l'aplicació
        \begin{align*}
            \begin{matrix} 
                \alpha_{P}: & E \times E & \to & V \\
                & Q & \mapsto & \vv{PQ}
            \end{matrix}
        \end{align*}
        \subitem és bijectiva.
    \item $\vv{PQ} + \vv{QR} = \vv{PR}, \quad \forall P, Q, R \in E$.
\end{enumerate}
Els elements de $E$ es diuen punts en l'espai, $V$ és l'espai vectorial associat a $E$. Definim la dimensió de l'espai afí euclidià $E$ com
\begin{align}
    \dim E \equiv \dim V
\end{align}

\subsubsection*{Varietats lineals}
Diem que un subconjunt $L$ de $E$ és una varietat lineal de $E$ si
\begin{enumerate}[i)]
    \item $\alpha (L \times L)$ és un subespai vectorial de $V$.
    \item $\forall P \in L$, 
        \begin{align*}
            \begin{matrix} 
                \alpha_{P}: & L \times L & \to & \alpha (L \times L) \\
                & Q & \mapsto & \vv{PQ}
            \end{matrix}
        \end{align*}
        \subitem és bijectiva.
\end{enumerate}
\begin{example}
Siguin $V = \mathbb{R}^{n}$ i $E = \mathbb{R}^{n}$. Sigui
\begin{align*}
    \alpha (u,v) = u-v
\end{align*}
Llavors $E$ amb $\alpha$ s'anomena espai afí euclidià estàndard.

Les varietats lineals d'aquest espai són de la forma $P+W$, on $P \in V$ i $W$ és un subespai vectorial de $V$, és a dir
\begin{align*}
    P + W = \left\{ P + w \mid w \in W \right\}
\end{align*}
Aquí $W$ és la direcció de $P + W$, i diem que $P + W$ és la varietat lineal que passa pel punt $P$ amb direcció $W$.
\end{example}
\begin{example}
Siguin $E = \mathbb{R}^{2}$, $L = \left\{ (x,y) \mid y - x = 1 \right\}$, i $P = (x,y)$ i $Q = (x',y')$ punts de $L$. Llavors:
\begin{align*}
\begin{split}
    \alpha \left( L \times L ) \right) & = \left\{ \vv{PQ} \mid y - x = 1, y' - x' = 1 \right\} = \\
    & = \left\{ (x'-x, y'-y) \mid y - x = 1, y' - x' = 1 \right\} = \left< (1, 1) \right> \\
    \Rightarrow L & = P + \left< (1, 1) \right>
\end{split}
\end{align*}
\end{example}

\subsubsection*{Intersecció de varietats lineals}
Dues varietats lineals $P_{1} + W_{1}$, $P_{2} + W_{2}$, d'un espai afí euclidià es tallen $\Leftrightarrow \vv{P_{1} P_{2}} \in W_{1} + W_{2}$.

Si $P_{1} + W_{1}$, $P_{2} + W_{2}$ són varietats lineals paral·leles, llavors o bé no es tallen o bé una està continguda dins l'altra (i si són de la mateixa dimensió, llavors són coincidents).

La intersecció de dues varietats lineals d'un espai afí euclidià és o bé buida o bé una varietat lineal.

%----------------------------------------------------------------------------------------
\subsection{Coordenades}
Un sistema de coordenades cartesianes ortonormal d'un espai afí euclidià $E$ de dimensió finita $n$ és
\begin{align}
    \mathcal{B} = \{ P, (v_{1}, \dots , v_{n}) \}
\end{align}

on $P$ és l'origen de coordenades i $(v_{1}, \dots , v_{n})$ és una base ortonormal de l'espai vectorial associat $V$; $\left< v_{1} \right>, \dots , \left< v_{n} \right>$ són els eixos de coordenades del sistema.

Les coordenades d'un punt $X \in E$ respecte de $\mathcal{B}$ són per definició les coordenades del vector $\vv{PX}$ respecte de la base $(v_{1}, \dots , v_{n})$. Escriurem $C(X, \mathcal{B}) = C(\vv{PX}, (v_{1}, \dots , v_{n}))$.

\subsubsection*{Fórmula del canvi de coordenades}
Siguin $\mathcal{B} = \{ P, (v_{1}, \dots , v_{n}) \}$, $\mathcal{B}' = \{ Q, (u_{1}, \dots , u_{n}) \}$ dos sistemes de coordenades ortonormals d'un espai afí euclidià $E$. Llavors:
\begin{align}
    C(X, \mathcal{B}) = C(Q, \mathcal{B}) + M(B', B) C(X, \mathcal{B}')
\end{align}

%----------------------------------------------------------------------------------------
\subsection{Distància i perpendicularitat}
Definim la distància en un espai afí euclidià $E$ com
\begin{align}
    \begin{matrix}
        d: & E \times E & \to & \mathbb{R} \\
        & (P, Q) & \mapsto & \| \vv{PQ} \|
    \end{matrix}
\end{align}
Observem que $d$ compleix les propietats següents:
\begin{enumerate}[i)]
    \item $d(P, Q) \geq 0, \quad \forall P, Q \in E$.
    \item $d(P, Q) = 0 \Leftrightarrow P = Q$.
    \item $d(P, Q) = d(Q, P), \quad \forall P, Q \in E$.
    \item $d(P, Q) + d(Q, R) \geq d(P, R), \quad P, Q, R \in E$ (desigualtat triangular).
\end{enumerate}

\subsubsection*{Teorema de Pitàgores}
Sigui $E$ un espai afí euclidià. Siguin $P, Q, R \in E$ tals que $\left< \vv{PQ} \mid \vv{PR} \right> = 0$. Llavors:
\begin{align}
    d(P,Q)^{2} + d(P, R)^{2} = d(Q, R)^{2}
\end{align}

\subsubsection*{Complement ortogonal}
Sigui $W$ un subespai vectorial d'un espai vectorial euclidià, llavors, definim el subespai ortogonal a $W$ com:
\begin{align}
    W^{\perp} \equiv \{ v \in V \mid u \perp v, \quad \forall u \in W \}
\end{align}

El complement ortogonal compleix les següents propietats:
\begin{enumerate}[i)]
    \item $W \oplus W^{\perp} = V$.
    \item $\dim (W) + \dim (W^{\perp}) = \dim (V)$.
\end{enumerate}

Diem que dues varietats lineals $P_{1}+W_{1}$, $P_{2}+W_{2}$ en un espai afí euclidià són ortogonals si $W_{1} \subseteq W_{2}^{\perp}$ o $W_{1}^{\perp} \subseteq W_{2}$.

\subsubsection*{Projecció ortogonal}
Sigui $E$ un espai afí euclidià de dimensió finita. Sigui $V$ l'espai vectorial associat i $\left< - \mid - \right>$ el seu producte escalar. Sigui $L$ una varietat lineal amb direcció $W$. Definim la projecció ortogonal de $E$ sobre $L$ per
\begin{align}
    \begin{matrix} \pi_{L}: & E & \to & L \\ & Q & \to & P+w_{1} \end{matrix}
\end{align}

on $P$ és un punt fixat de $L$ i $\vv{PQ} = w_{1} + w_{2}$ amb $w_{1} \in W$ i $w_{2} \in W^{\perp}$.
\\
Observem que $\vv{\pi_{L}(Q)Q} = w_{2} \in W^{\perp}$ i que $d(\pi_{L}(Q),Q)$ és mínima.

%----------------------------------------------------------------------------------------
\subsection{Distància entre dues varietats lineals}
Siguin $L_{1} = P_{1} + W_{1}$, $L_{2} = P_{2} + W_{2}$ dues varietats lineals d'un espai afí euclidià $E$ de dimensió finita. Definim la distància entre $L_{1}$ i $L_{2}$ com
\begin{align}
    d(L_{1}, L_{2}) = \inf \{d(P, Q) \mid P \in L_{1}, Q \in L_{2} \}
\end{align}

o, equivalentment,
\begin{align}
    d(L_{1}, L_{2}) = \inf \{ \| \vv{P_{1} P_{2}} + w_{1} + w_{2} \| \mid w_{1} \in W_{1}, w_{2} \in W_{2} \}
\end{align}

Observem que si $L_{1}$ i $L_{2}$ es tallen, $d(L_{1},L_{2}) = 0$.

\subsubsection*{Mètode 1}
Siguin $L_{1} = P_{1} + W_{1}$, $L_{2} = P_{2} + W_{2}$ dues varietats lineals d'un espai afí euclidià $E$ de dimensió finita. Sigui $\pi_{L}: E \to L$ la projecció ortogonal de $E$ sobre $L = P_{1} + (W_{1} + W_{2})$. Llavors:
\begin{align}
    d(L_{1}, L_{2}) = d(P_{2}, \pi_{L} (P_{2}))
\end{align}

i es compleix:
\begin{align}
    \pi_{L} (P_{2}) = P_{1} + \sum\limits_{i = 1}^{\dim (W_{1} + W_{2})} \left< \vv{P_{1} P_{2}} \mid v_{i} \right> v_{i}
\end{align}

on $(v_{1}, \dots , v_{i} , \dots , v_{n})$ és una base ortogonal de $W_{1} + W_{2}$.
\\  \\
Llavors, $\vv{P_{2} \pi_{L} (P_{2})} \subset W_{3} \equiv$ direcció de la varietat lineal $L_{3}$ tal que $L_{3} \perp L_{1}$ i $L_{3} \perp L_{2}$, que anomenarem perpendicular comuna entre $L_{1}$ i $L_{2}$.

\subsubsection*{Mètode 2}
Siguin $L_{1} = P_{1} + W_{1}$, $L_{2} = P_{2} + W_{2}$ dues varietats lineals d'un espai afí euclidià $E$ de dimensió finita. Llavors per calcular $d(L_{1}, L_{2})$ considerarem el següent:
\begin{enumerate}[i)]
    \item Calculem $(W_{1}+W_{2})^{\perp}$. Aquest subespai vectorial és ortogonal a $L = P_{1} + (W_{1} + W_{2})$.
    \item $E = (W_{1}+W_{2}) \oplus (W_{1}+W_{2})^{\perp}$.
    \item $\Rightarrow \vv{P_{1}P_{2}} = u + v, \quad u \in (W_{1}+W_{2})$, $v \in (W_{1}+W_{2})^{\perp}$. 
    \item $\Rightarrow d(L_{1}, L_{2}) = \| v \|$.
\end{enumerate}
\bigskip
Per calcular, en canvi, els dos punts $X_{1} \in L_{1}$, $X_{2} \in L_{2}$ tals que $d(X_{1}, X_{2}) = d(L_{1}, L_{2})$ haurem de fer les següents consideracions:
\begin{enumerate}[i)]
    \item $X_{1} = P_{1} + u_{1}, \quad  u_{1} \in W_{1}$.
    \item $X_{2} = P_{2} - u_{2}, \quad  u_{2} \in W_{2}$.
    \item $\vv{P_{1}P_{2}} = u_{1} + u_{2} + v, \quad v \in (W_{1}+W_{2})^{\perp}$.
    \item $u_{1} + u_{2} = u, \quad u \in (W_{1}+W_{2})$.
\end{enumerate}

%----------------------------------------------------------------------------------------
\subsection{Isometries i desplaçaments}
\subsubsection*{Isometries}
Una aplicació $f: E_{1} \to E_{2}$ entre dos espais afins euclidians es diu isometria si
\begin{align}
    d(f(P), f(Q)) = d(P, Q), \quad \forall P, Q \in E_{1}
\end{align}
Observem que $f$ és injectiva.

Siguin $E_{1}$, $E_{2}$ espais afins euclidians amb espais vectorials associats $V_{1}$, $V_{2}$ i productes escalars $\left< - \mid - \right>_{1}$, $\left< - \mid - \right>_{2}$ respectivament. Sigui $f: E_{1} \to E_{2}$ una isometria. Llavors l'aplicació
\begin{align}
    \tilde{f}: V_{1} \to V_{2}
\end{align}

definida per $\tilde{f}(\vv{PQ}) = \vv{f(P) f(Q)}$ està ben definida, és lineal i
\begin{align}
    \left< \tilde{f}(u_{1}) \mid \tilde{f}(u_{2}) \right>_{2} = \left< u_{1} \mid u_{2} \right>_{1}, \quad \forall u_{1}, u_{2} \in V_{1}
\end{align}

i es diu aplicació lineal associada a $f$.
\\
A més, si $g: V_{1} \to V_{2}$ és una aplicació lineal que conserva el producte escalar, $P_{1} \in E_{1}$ i $P_{2} \in E_{2}$, llavors l'aplicació
\begin{align}
    \begin{matrix} \tilde{g}: & E_{1} & \to & E_{2} \\ & P_{1} + u_{1} & \mapsto & P_{2} + g(u_{1}) \end{matrix}
\end{align}

és una isometria i $\tilde{\tilde{g}} = g$.

\subsubsection*{Desplaçaments}
Sigui $E$ un espai afí euclidià de dimensió finita. Un desplaçament en $E$ és una isometria de $E$ en ell mateix.

Considerem l'espai afí euclidià (estàndard) $\mathbb{R}^{n}$. Sigui
\begin{align}
    f: \mathbb{R}^{n} \to \mathbb{R}^{n}
\end{align}
un desplaçament. Llavors:
\begin{align}
    f \begin{pmatrix} x_{1} \\ \vdots \\ x_{n} \end{pmatrix} = \begin{pmatrix} P_{1} \\ \vdots \\ P_{n} \end{pmatrix} + A \begin{pmatrix} x_{1} \\ \vdots \\ x_{n} \end{pmatrix}
\end{align}

i $(P_{1}, \dots, P_{n})$ i $A \in M(n, \mathbb{R})$ són fixats per $f$.
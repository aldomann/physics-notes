%----------------------------------------------------------------------------------------
%    ESPAI DUAL
%----------------------------------------------------------------------------------------
\section{Espai dual}
\subsection{Teoria}
% WIP

%----------------------------------------------------------------------------------------
\subsection{Màquina de xurros}
\begin{example}
Sigui $B=((1,2,3),(4,5,6),(7,8,9))$ una base de l'espai vectorial $V$. Llavors,
    \begin{align*}
        M(id, B, \mathcal{C}) = \begin{pmatrix} 1 & 4 & 7 \\ 2 & 5 & 8 \\ 3 & 6 & 9 \end{pmatrix}
    \end{align*}

que és la matriu de canvi de base de $B$ a $\mathcal{C}$ (la base canònica), o dit d'una altra manera, la base $B$ expressada en funció de la base $C$.

La base dual $B^{\ast}$ es calcula de la manera següent:
\begin{align*}
M(id, B^{\ast}, \mathcal{C}^{\ast}) = M(id, \mathcal{C}, B)^{t} = (M(id, B, \mathcal{C})^{t})^{-1}
\end{align*}
\end{example}

\begin{example}
Es consideren $B = (v_{1}, v_{2})$ una base de l'espai vectorial $V$ i $B^{\ast} = (\varphi_{1}, \varphi_{2})$ la seva base dual. Sigui $B' = (v_{1}+2 v_{2}, 3 v_{1} + 4 v_{2})$ una altra base de la qual volem trobar la seva base dual. Per començar, fem:
    \begin{align*}
        M(id, B', B) = \begin{pmatrix} 1 & 2 \\ 3 & 4 \end{pmatrix}
    \end{align*}
Llavors,
\begin{align*}
M(id, B'^{\ast}, B^{\ast}) = M(id, B, B')^{t} = (M(id, B', B)^{t})^{-1} = \begin{pmatrix} -2 & 1.5 \\ 1 & -0.5 \end{pmatrix}
\end{align*}
Així doncs, $B'^{\ast} = (-2 \varphi_{1} + 1.5 \varphi_{2}, \varphi_{1} - 0.5 \varphi_{2})$.
\end{example}

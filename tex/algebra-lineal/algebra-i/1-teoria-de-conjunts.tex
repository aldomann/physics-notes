%----------------------------------------------------------------------------------------
%    TEORIA DE CONJUNTS
%----------------------------------------------------------------------------------------
\section{Teoria de conjunts}
\subsection{Relacions binàries $\mathcal{R}$}
$\mathcal{R}$ en $A \subseteq A^{2}, \quad (a,b) \in \mathcal{R} \subseteq A^{2} = a \mathcal{R} b$.
\subsubsection*{Propietats}
\begin{itemize}
    \item Reflexiva: $a \mathcal{R} b, \quad \forall a \in A$.
    \item Simètrica: $a \mathcal{R} b \Rightarrow b \mathcal{R} a$.
    \item Antisimètrica: $a \mathcal{R} b$ i $b \mathcal{R} a \Rightarrow a = b$.
    \item Transitiva: $a \mathcal{R} b$ i $b \mathcal{R} c \Rightarrow a \mathcal{R} c$.
\end{itemize}

\subsubsection*{Relacions d'ordre $\leq$}
\begin{itemize}
    \item Reflexiva: $a \leq b, \quad \forall a \in A$.
    \item Antisimètrica: $a \leq b$ i $b \leq a \Rightarrow a = b$.
    \item Transitiva: $a \leq b$ i $b \leq c \Rightarrow a \leq c$.
\end{itemize}

\subsubsection*{Relacions d'equivalència $\sim$}
\begin{itemize}
    \item Reflexiva: $a \sim b, \quad \forall a \in A$.
    \item Simètrica: $a \sim b \Rightarrow b \sim a$.
    \item Transitiva: $a \sim b$ i $b \sim c \Rightarrow a \sim c$.
\end{itemize}

\paragraph{Classe d'equivalència de $a$}
$\left[ a \right]$ és el conjunt d'elements $\in A$ relacionats amb $a$.
\begin{align}
    \left[ a \right] \equiv \{b \in a \mid b \sim a \} 
\end{align}

\paragraph{Conjunt quocient de $A$}
 $A / \sim$ és el conjunt de totes les classes (disjuntes, per definició) de $A$. 
\begin{align}
    a \in A \textrm{, } \left[ a \right] \subseteq A \textrm{, però } \left[ a \right] \in A / \sim
\end{align}

%----------------------------------------------------------------------------------------
\subsection{Principi d'inducció}
El principi d'inducció diu que si $S \subseteq \mathbb{N}$ tal que
\begin{enumerate}[i)]
    \item $0 \in S$ (a vegades, però, es treballa amb $S^{*} $).
    \item $n \in S \Rightarrow n+1 \in S$.
\end{enumerate}

$\Rightarrow S = \mathbb{N}$, demostrant així que una propietat $P(n)$ que és certa $\forall n \in S$, ho serà també $\forall n \in \mathbb{N}$.

%----------------------------------------------------------------------------------------
\subsection{Aplicacions}
Una aplicació de $A$ a $B$ és la tripleta $(A,B,f)$ on $f \subseteq A \times B$ que compleix:
\begin{enumerate}[i)]
    \item $\forall a \in A, \quad \exists b \in B$ tal que $(a,b) \in f$. 
    \item $(a,b),(a',b') \in f \Rightarrow b = b'$.
\end{enumerate}

Per denotar que $(A,B,f)$ és aplicació, s'escriu $f: A \to B$.

\subsubsection*{Tipus d'aplicacions}
\begin{itemize}
    \item Injectiva: $\hookrightarrow$ o $\iota$. 
        \subitem $\iota \Leftrightarrow f(a) = f(a') \Rightarrow a = a'$.
    \item Exhaustiva: $\twoheadrightarrow$ o $\pi$. 
        \subitem $\pi \Leftrightarrow \forall b \in B, \exists a \in A$ tal que $f(a) = b$.
    \item Bijectiva: $\leftrightarrow$ o $\tilde{f}$. 
        \subitem $\tilde{f} \Leftrightarrow f$ és $\iota$ i $\pi$ alhora.
\end{itemize}

\subsubsection*{Descomposició canònica d'una aplicació}
Tota aplicació $f:A \to B$ pot ésser expressada com a:
\begin{align}
    f = \iota \circ \tilde{f} \circ \pi = \iota ( \tilde{f} ( \pi (A))) \subseteq B
\end{align}

\begin{example}
\begin{align*}
    A \to B = A \twoheadrightarrow A / \sim \leftrightarrow f(A) \hookrightarrow B
\end{align*}
\end{example}

\subsubsection*{Inclusió canònica}
Sigui $B \subseteq A$,

$\iota: B \to A$ és definida per $\iota (x), \quad \forall b \in B$.

\subsubsection*{Aplicació identitat}
És un cas concret de $\iota (x)$.

$id_{A}: A \to A$ és definida per $\iota (x), \quad \forall a \in A$.

\subsubsection*{Imatge i antiimatge}
Sigui $f: A \to B$,

Imatge de $A$: $\im(f) = f(A) \equiv \{ f(x) \mid x \in A \}$

Antiimatge de $B$: $f^{-1} (B) \equiv \{ x \in A \mid f(x) \in B \}$

\subsubsection*{Composició d'aplicacions}
Siguin $f: A \to B$ i $g: B \to C$, 

$g \circ f: A \to C$, on $(g \circ f)(x) = g(f(x)), \quad \forall x \in A$.

\subsubsection*{Inversa d'una funció bijectiva}
Sigui $\tilde{f}: A \to B, \;A \neq 0$ i $B \neq 0$,

$\exists \tilde{g}: A \to B$ tal que $\tilde{g} \circ \tilde{f} = id_{A}$ i $\tilde{f} \circ \tilde{g} = id_{B}$
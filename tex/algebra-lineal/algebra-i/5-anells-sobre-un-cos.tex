%----------------------------------------------------------------------------------------
%    ANELL DE POLINOMIS SOBRE UN COS COMMUTATIU
%----------------------------------------------------------------------------------------
\section{Anell de polinomis sobre un cos commutatiu}
\subsection{L'anell dels polinomis}
Sigui $a(x) \in K [x] = a_{n}x^{n} + a_{n-1}x^{n-1} + \dots + a_{i}x^{i} + \dots + a_{1}x + a_{0}, \; (n \in \mathbb{N}, \, a_{i} \in K)$.
\begin{itemize}
    \item $a_{i}$ són els coefincients del polinomi, essent $a_{n}$ el coeficient principal.
    \item $i$ són els graus del polinomi, essent $n$ el grau màxim del polinomi. S'expressa com a $\operatorname{gr}(a(x)) = n$.
\end{itemize}
\subsubsection*{Definicions}
\begin{itemize}
    \item Monomi: polinomi d'un sol element.
    \item Polinomi mònic: polinomi amb el coeficient principal $= 1$.
\end{itemize}

\subsubsection*{Operacions}
\begin{itemize}
    \item Suma:
        \subitem $\operatorname{gr}(a(x) + b(x)) \leq \max \{ \operatorname{gr}(a(x)), \operatorname{gr}(b(x)) \}$
    \item Producte:
        \subitem $\operatorname{gr}(a(x) b(x)) = \operatorname{gr}(a(x)) + \operatorname{gr}(b(x))$
    \item Divisió entera:
        \subitem Siguin $a(x), b(x) \in K [x], \; (b(x) \neq 0), \quad \exists$ dos úncs $q(x)$ i $r(x)$ tals que:
        \begin{enumerate}[i)]
            \item $a(x) = q(x) b(x) + r(x)$.
            \item $\operatorname{gr}(r(x)) < \operatorname{gr}(b(x))$.
        \end{enumerate}
\end{itemize}

%----------------------------------------------------------------------------------------
\subsection{Arrels d'un polinomi}
$r \in K$ és una arrel de $a(x) \Leftrightarrow a(r) = 0$

\subsubsection*{Teorema de la resta}
$u \in K$ és una arrel de $a(x) \Leftrightarrow a(x)$ és divisible, exactament, per $x-u$.
\begin{align}
    a(x) = q(x) (x-u) + a(u), \quad a(u) = r(x) = 0
\end{align}

%----------------------------------------------------------------------------------------
\subsection{Divisibilitat a un anell}
Siguin $a,b \in R$,

$D(a) \equiv \{ d \in R \mid a \in dR \}, \quad D(a,b) \equiv D(a) \cap D(b) = \{ d \in R \mid d|a \text{ i } d|b \}$

\subsubsection*{Lema d'Euclides}
\begin{align}
    a = qb + r,
    \begin{cases}
        R = \mathbb{Z} \Rightarrow 0 \leq r < |b| \\
        R = K [x] \Rightarrow \operatorname{gr} (r) < \operatorname{gr} (b)
    \end{cases}
    \Rightarrow
\end{align}

\begin{enumerate}[i)]
    \item $D (a,b) = D (b,r)$.
    \item $aR + bR = bR + rR$.
\end{enumerate}

\subsubsection*{Algoritme}
\begin{align}
\begin{gathered}
    a = b q_{0} + r_{0} \\
    b = r_{0} q_{1} + r_{1} \\
    \dots \\
    r_{n-2} = r_{n-1} q_{n} + (r_{n} = 0)
\end{gathered}
\end{align}

Per força, després d'un nombre finit $n$ d'iteracions, arribem a tenir $r_{n}=0 \Rightarrow$ (pel Lema d'Euclides) $D (a,b) = D (r_{n-1}) \Rightarrow aR + bR = r_{n-1}R \Rightarrow$
\begin{enumerate}[i)]
    \item $\operatorname{mcd} (a,b) = r_{n-1} $
    \item $\operatorname{mcm} (a,b) = \frac{ab}{\operatorname{mcd} (a,b)}$
\end{enumerate}

Tant $\operatorname{mcd}$ i $\operatorname{mcm}$ són únics i de la següent forma: 
$\begin{cases} < 0 \text{ a } R = \mathbb{Z} \\ \text{ mònic a } R = K [x] \end{cases}$

\subsubsection*{Identitat de Bezout}
\begin{align}
    aR + bR = r_{n-1}R \Rightarrow a \alpha + b \beta = r_{n-1}
\end{align}
\begin{example}
\begin{align*}
    a = b q_{0} + r_{0}, \; b = r_{0} q_{1} + r_{1}, \; r_{0} = r_{1} q_{2} + 0 \Rightarrow a \alpha + b \beta = r_{1}
\end{align*}
\end{example}
\begin{enumerate}[a)]
    \item Mètode 1.
        \subitem $ r_{1} = b - r_{0} q_{1} = b - q_{1} (a - b q_{0}) = a \underbrace{( - q_{1})}_{\alpha} + \, b \underbrace{( 1 + q_{0} q_{1})}_{\beta}$
    \item Mètode 2.
        \subitem $\begin{pmatrix} b \\ r_{0} \end{pmatrix} = \begin{pmatrix} 0 & 1 \\ 1 & -q_{0} \end{pmatrix} \begin{pmatrix} a \\ b \end{pmatrix}$, 
        \subitem $\begin{pmatrix} r_{0} \\ r_{1} \end{pmatrix} = \begin{pmatrix} 0 & 1 \\ 1 & -q_{1} \end{pmatrix} \begin{pmatrix} b \\ r_{0} \end{pmatrix} = \begin{pmatrix} 0 & 1 \\ 1 & -q_{1} \end{pmatrix} \begin{pmatrix} 0 & 1 \\ 1 & -q_{0} \end{pmatrix} \begin{pmatrix} a \\ b \end{pmatrix}$,
         \subitem $\begin{pmatrix} r_{1} \\ 0 \end{pmatrix} = \begin{pmatrix} 0 & 1 \\ 1 & -q_{2} \end{pmatrix} \begin{pmatrix} r_{0} \\ r_{1} \end{pmatrix} = \dots = \begin{pmatrix} \alpha & \beta \\ \ast & \ast \end{pmatrix} \begin{pmatrix} a \\ b \end{pmatrix}$.
\end{enumerate}

%----------------------------------------------------------------------------------------
\subsection{Factorització}
Tot $a(x) \in \mathbb{K} [x]$ és factoritzable en polinomis irreductibles $b(x) \in \mathbb{K} [x]$. La factorització d'un polinomi pot variar segons el cos $K$ sobre el qual s'estengui l'anell de polinomis.

\subsubsection*{Polinomis irreductibles}
\begin{itemize}
    \item Els polinomis irreductibles a $\mathbb{Q} [x]$ poden ésser de qualsevol grau possitiu.
    \item Els polinomis irreductibles a $\mathbb{R} [x]$ són els de garu $1$ i els de grau $2$ de la forma $a x^{2} + b x +c$ amb $\Delta < 0$.
    \item Els polinomis irreductibles a $\mathbb{C} [x]$ són els de grau $1$.
\end{itemize}

\subsubsection*{Teorema fonamental de l'àlgebra}
Tot $a(x) \in \mathbb{C} [x]$ de $\operatorname{gr} (a(x)) = n > 0$ té exactament $n$ arrels complexes $\Rightarrow a(x) = \lambda (x - \alpha_{1}) (x - \alpha_{2}) \dots (x - \alpha_{n-1}) (x - \alpha_{n})$. 
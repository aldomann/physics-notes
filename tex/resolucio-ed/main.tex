%----------------------------------------------------------------------------------------
%    PACKAGES AND OTHER DOCUMENT CONFIGURATIONS
%----------------------------------------------------------------------------------------
\documentclass[paper=a4, fontsize=13pt, twoside=semi]{scrartcl}
\usepackage[T1]{fontenc}
\usepackage[utf8]{inputenc}
\usepackage[catalan]{babel}
\usepackage{lmodern}
\usepackage[fixlanguage]{babelbib}
\selectbiblanguage{catalan}

\addtokomafont{sectioning}{\normalfont\scshape}
\usepackage{tocstyle}
\usetocstyle{standard}
\renewcommand*\descriptionlabel[1]{\hspace\labelsep\normalfont\bfseries{#1}}

\usepackage{fancyhdr}
\pagestyle{fancyplain}
\fancyhead[L]{\mytitle}
\fancyhead[R]{\myauthor}
\fancyfoot[C]{\thepage}
\fancypagestyle{firststyle}
{
    \fancyhead[L]{\mytitle}
    \fancyhead[R]{\myauthor}
    \fancyfoot[C]{\thepage}
}
\renewcommand{\headrulewidth}{0.3pt}
\renewcommand{\footrulewidth}{0pt}
\setlength{\headheight}{13.6pt}

\usepackage{etoolbox}

\setlength{\parindent}{0pt}
\setlength{\parskip}{0.3\baselineskip plus2pt minus2pt}
\newcommand{\sk}{\medskip\noindent}

\usepackage{hyperref}
\hypersetup{colorlinks, citecolor=black, filecolor=black, linkcolor=black, urlcolor=black}

%----------------------------------------------------------------------------------------
%    MATHS AND ENVIRONMENTS
%----------------------------------------------------------------------------------------
\usepackage{amsmath,amsfonts,amsthm,amssymb}
\usepackage{xfrac}
\usepackage[a]{esvect}
\usepackage{pgfplots}
\usepackage{tikz} 
\usepackage{color}
    \makeatletter
        \color{black}
        \let\default@color\current@color
    \makeatother
\usepackage{thmtools}
\usepackage{environ}

\usepackage{siunitx}

\newcommand*{\dif}{\mathrm{d}}
\newcommand*{\diff}{\mathop{}\!\mathrm{d}}
\newcommand*{\vnabla}{\vec{\nabla}}

\numberwithin{equation}{section}
\numberwithin{figure}{section}
\numberwithin{table}{section}

\usepackage{enumerate}
\usepackage{booktabs}
\usepackage{float}
    % \setlength{\intextsep}{8pt}

\makeatletter
\renewcommand*\env@matrix[1][*\c@MaxMatrixCols c]{%
    \let\@ifnextchar\new@ifnextchar
    \array{#1}}
\makeatother

\usepackage{lipsum}

\makeatletter
\newcommand*\NoIndentAfterEnv[1]{%
  \AfterEndEnvironment{#1}{\par\@afterindentfalse\@afterheading}}
\makeatother
%\NoIndentAfterEnv{thm}
\NoIndentAfterEnv{defi}
\NoIndentAfterEnv{example}
\NoIndentAfterEnv{table}

%----------------------------------------------------------------------------------------
%    THEOREMS
%----------------------------------------------------------------------------------------
\declaretheorem[style=plain,name=Teorema,qed=$\square$,numberwithin=section]{thm}
\declaretheorem[style=plain,name=Corol·lari,qed=$\square$,sibling=thm]{cor}
\declaretheorem[style=plain,name=Lemma,qed=$\square$,sibling=thm]{lem}

\declaretheorem[style=definition,name=Definició,qed=$\blacksquare$,numberwithin=section]{defi}
\declaretheorem[style=definition,name=Exemple,qed=$\blacktriangle$,numberwithin=section]{example}

%----------------------------------------------------------------------------------------
%    ELA MOTHERFUCKING GEMINADA
%----------------------------------------------------------------------------------------
\def\xgem{%
   \ifmmode
     \csname normal@char\string"\endcsname l%
   \else
     \leftllkern=0pt\rightllkern=0pt\raiselldim=0pt
     \setbox0\hbox{l}\setbox1\hbox{l\/}\setbox2\hbox{.}%
     \advance\raiselldim by \the\fontdimen5\the\font
     \advance\raiselldim by -\ht2
     \leftllkern=-.25\wd0%
     \advance\leftllkern by \wd1
     \advance\leftllkern by -\wd0
     \rightllkern=-.25\wd0%
     \advance\rightllkern by -\wd1
     \advance\rightllkern by \wd0
     \allowhyphens\discretionary{-}{}%
     {\kern\leftllkern\raise\raiselldim\hbox{.}%
       \kern\rightllkern}\allowhyphens
   \fi
   }
\def\Xgem{%
   \ifmmode
     \csname normal@char\string"\endcsname L%
   \else
     \leftllkern=0pt\rightllkern=0pt\raiselldim=0pt
     \setbox0\hbox{L}\setbox1\hbox{L\/}\setbox2\hbox{.}%
     \advance\raiselldim by .5\ht0
     \advance\raiselldim by -.5\ht2
     \leftllkern=-.125\wd0%
     \advance\leftllkern by \wd1
     \advance\leftllkern by -\wd0
     \rightllkern=-\wd0%
     \divide\rightllkern by 6
     \advance\rightllkern by -\wd1
     \advance\rightllkern by \wd0
     \allowhyphens\discretionary{-}{}%
     {\kern\leftllkern\raise\raiselldim\hbox{.}%
       \kern\rightllkern}\allowhyphens
   \fi
   }

\expandafter\let\expandafter\saveperiodcentered
   \csname T1\string\textperiodcentered \endcsname

\DeclareTextCommand{\textperiodcentered}{T1}[1]{%
   \ifnum\spacefactor=998
     \Xgem
   \else
     \xgem
   \fi#1}

%----------------------------------------------------------------------------------------
%    PDF INFO
%----------------------------------------------------------------------------------------
\newcommand*{\mytitle}{Resolució d'equacions diferencials}
\newcommand*{\mysubtitle}{}
\newcommand*{\myauthor}{Alfredo Hernández Cavieres}
\newcommand*{\myuni}{Universitat Autònoma de Barcelona, Departament de Física}
\newcommand*{\mydate}{\normalsize \today}

\pdfstringdefDisableCommands{\def\and{i }}
\usepackage{hyperxmp}
\hypersetup{pdfauthor={\myauthor}, pdftitle={\mytitle}}

%----------------------------------------------------------------------------------------
%    TITLE SECTION AND DOCUMENT BEGINNING
%----------------------------------------------------------------------------------------
\begin{document}

%----------------------------------------------------------------------------------------
%    SECTIONS
%----------------------------------------------------------------------------------------
%\tableofcontents
\thispagestyle{firststyle}
%----------------------------------------------------------------------------------------
%    TEMA
%----------------------------------------------------------------------------------------
\section{\mytitle}

\subsection*{Continuïtat}
Estudi de la continuïtat de $f(\vec{x})$ en el punt $\vec{x} = \vec{a}$. 

Si volem demostrar que és contínua, es tracta d'avaluar (generalment agafant $\delta = \sqrt{x^{2} + y^{2}}$)
\begin{align}
    d(\vec{x}, \vec{a}) < \delta \Rightarrow d(f(\vec{x}), f(\vec{a})) < \varepsilon
\end{align}
Per tal de trobar alguna relació del tipus 
\begin{align}
    \delta < \varepsilon
\end{align}

Si volem, en canvi, demostrar que no és contínua, es tracta de trobar un contraexemple mitjançant alguna relació $x \leftrightarrow y$ (e.g., $y = ax + bx^{2}$) i veure que 
\begin{align}
    \lim_{\vec{x} \to \vec{a}} f(\vec{x}) \neq f(\vec{a})
\end{align}

\subsubsection*{Derivada direccional}
La derivada direccional de $f(\vec{x})$ en la direcció $\vec{u}$ en el punt $\vec{x} = \vec{a}$ es defineix com
\begin{align}
    \lim_{\vec{x} \to \vec{a}} f(\vec{x}) \equiv \lim_{\lambda \to 0^{+}} f(\vec{a} + \lambda \vec{u})
\end{align}

%----------------------------------------------------------------------------------------
\subsection*{Funcions vectorials d'una variable}
\subsubsection*{Corbes}
El paràmetre arc d'una corba $\vec{f}(u)$ es defineix com
\begin{align}
    s(u) = \int_{a}^{u} \| \vec{f}'(u) \| \diff u
\end{align}
i defineix la longitud de la corba en funció del paràmetre $u$.

\subsubsection*{Geometria}
Tríedre de Frénet:
\begin{align}
    \hat{T} = \frac{\vec{f}'(u)}{\| \vec{f}'(u) \|}; \quad \hat{B} = \frac{\vec{f}'(u) \times \vec{f}''(u)}{\| \vec{f}'(u) \times \vec{f}''(u) \|}; \quad \hat{T} \times \hat{N} = \hat{B}
\end{align}
\begin{align}
    \frac{\dif}{\dif s} \begin{pmatrix} \hat{T} \\ \hat{N} \\ \hat{B} \end{pmatrix} = \begin{pmatrix} 0 & \kappa & 0 \\ - \kappa & 0 & \tau \\ 0 & - \tau & 0 \end{pmatrix} \begin{pmatrix} \hat{T} \\ \hat{N} \\ \hat{B} \end{pmatrix}
\end{align}
Radi de corbatura:
\begin{align}
    \rho = \frac{1}{\kappa}    
\end{align}
%----------------------------------------------------------------------------------------
\subsection*{Diferenciabilitat}
\subsubsection*{Criteri de diferenciabilitat}
El criteri consisteix en demostrar pas a pas que és contínua, derivable i diferenciable (en aquest ordre). S'han de complir les tres propietats perquè sigui diferenciable.

Derivable $\Leftrightarrow \exists \frac{\partial f}{\partial x}, \frac{\partial f}{\partial y}, \dots$:
\begin{align}
    \exists \left( \frac{\partial f}{\partial x} \right)_{(x_{0}, y_{0})} \Leftrightarrow \exists \lim_{h \to 0} \frac{f(x_{0}+h, y_{0} - f(x_{0},y_{0})}{h}
\end{align}

Diferenciable:
\begin{align}
    \lim_{\vec{x} - \vec{a} \to \vec{0}} \frac{f(\vec{x}) - f(\vec{a}) - (\vnabla f)_{\vec{a}} \cdot (\vec{x} - \vec{a})}{\| \vec{x} - \vec{a} \|} = 0 \Leftrightarrow \frac{\partial f}{\partial x_{i}} \text{ contínua.}
\end{align} 

\subsubsection*{Diferenciació de superfícies}
$f(x,y,z) = f(x) + f(y) + f(z) = n$, ($n \in \mathbb{R}$) al punt $\vec{r}_{0} = (x_{0}, y_{0}, z_{0})$. Llavors $\dif f = \vnabla f \cdot \dif \vec{r} \equiv 0$.

Es veu fàcilment que $(\vnabla f)_{\vec{r}_{0}}$ dóna els coeficients de l'equació del pla. Així doncs, l'equació del pla és
\begin{align}
    (\vnabla f)_{(\vec{r}_{0})_{1}} (x - x_{0}) + (\vnabla f)_{(\vec{r}_{0})_{2}} (y - y_{0}) + (\vnabla f)_{(\vec{r}_{0})_{3}} (z - z_{0}) \equiv 0
\end{align}

%----------------------------------------------------------------------------------------
\subsection*{Taylor i punts crítics}
\subsubsection*{Polinomi de Taylor de segon ordre}
\begin{align}
    P_{2}^{(\vec{a})} = f(\vec{a}) + (\vnabla f)_{\vec{a}} \cdot (\vec{x} - \vec{a}) + \frac{1}{2!} (\vec{x} - \vec{a}) \cdot [(H^{f})_{\vec{a}} (\vec{x} - \vec{a})]
\end{align}

\subsubsection*{Punts crítics}
Són els punts $\vec{a}$ en què $(\vnabla f)_{\vec{a}} \equiv 0$. Llavors es tracta d'estudiar els hessians pel criteri de Sylvester. 

Segons la recurrència dels menors principals del vèrtex superior esquerre, podem tenir:
\begin{itemize}
    \item Mínim: $+, +, +, +, \dots$
    \item Màxim: $-, +, -, +, \dots$
    \item Punt de sella: $*, *, *, *, \dots$, on $* \neq 0$.
    \item No conclusiu: $*, \dots , 0, \dots$, llevat que coneguem els valors propis de l'hessià.
\end{itemize}

%----------------------------------------------------------------------------------------
\newpage
\subsection*{Funció implícita i multiplicadors de Lagrange}

%----------------------------------------------------------------------------------------
\subsection*{Integrals de línia}
\subsubsection*{$\vec{f}$ sobre $\mathcal{C}$}
\begin{align}
    \int_{\mathcal{C}} \vec{f}(\vec{x}) \cdot \dif \vec{x} = \int_{\mathcal{C}} \vec{f}(\vec{x}(u)) \cdot \vec{x}'(u) \diff u
\end{align}

\subsubsection*{$f$ sobre $\mathcal{C}$}
\begin{align}
    \int_{\mathcal{C}} \phi (\vec{x}) \diff s = \int_{\mathcal{C}} \phi (\vec{x}(u)) \| \vec{x}'(u) \| \diff u
\end{align}

%----------------------------------------------------------------------------------------
\subsection*{Integrals múltiples}
\begin{align}
    \iiint_{\mathcal{S}} f(x,y,z) \diff x \diff y \diff z = \int_{x_{1}}^{x_{2}} \left[ \int_{\phi_{1}(x)}^{\phi_{2}(x)} \left[ \int_{\varphi_{1}(x,y)}^{\varphi_{2}(x,y)} \diff z \right] \diff y \right] \diff x
\end{align}

\begin{thm}[Green]
\begin{align}
    \iint_{\mathcal{S}} \left( \frac{\partial Q}{\partial x} - \frac{\partial P}{\partial y} \right)  \diff x \diff y = \oint_{\mathcal{C}} (P\diff x + Q \diff y)
\end{align}
\end{thm}
%----------------------------------------------------------------------------------------
\subsection*{Integrals de superfície}
\subsubsection*{$\vec{f}$ sobre $\mathcal{S}$}
\begin{align}
    \iint_{\mathcal{S}} \vec{f}  \diff \mathcal{S} = \iint_{\mathcal{S}} \vec{f}(\vec{r}(u,v)) \cdot \left( \frac{\partial \vec{r}}{\partial u} \times \frac{\partial \vec{r}}{\partial v} \right) \diff u \diff v
\end{align}

\subsubsection*{$f$ sobre $\mathcal{S}$}
\begin{align}
    \iint_{\mathcal{S}} f  \diff \mathcal{S} = \iint_{\mathcal{S}} f(\vec{r}(u,v)) \left\| \frac{\partial \vec{r}}{\partial u} \times \frac{\partial \vec{r}}{\partial v} \right\| \diff u \diff v
\end{align}

\begin{thm}[Gauss]
	\begin{align}
    \begin{gathered}
	    \iiint_{\mathcal{V}} \left( \frac{\partial P}{\partial x} + \frac{\partial Q}{\partial y} + \frac{\partial R}{\partial z} \right) \diff x \diff y \diff z \\
		= \iint_{\mathcal{S}} \left[ P \diff y \wedge \dif z + Q \diff z \wedge \dif x + R \diff x \wedge \dif y \right]
	\end{gathered}
	\end{align}
	o bé 
	\begin{align}
	    \iiint_{\mathcal{V}} (\vnabla \cdot \vec{f}) \diff \mathcal{V} = \iint_{\mathcal{S}} \vec{f} \cdot \dif \vec{\mathcal{S}}
    \end{align}
\end{thm}

\begin{thm}[Stokes]
\begin{align}
	\begin{gathered}
		\iint_{\mathcal{S}} \left[ \left( \frac{\partial R}{\partial y} - \frac{\partial Q}{\partial z} \right) \diff y \wedge \dif z + \left( \frac{\partial P}{\partial z} - \frac{\partial R}{\partial Q} \right) \diff z \wedge \dif x + \left( \frac{\partial Q}{\partial x} - \frac{\partial P}{\partial y} \right) \diff x \wedge \dif y \right] \\
		= \oint_{\mathcal{C}} \left[ P \diff x + Q \diff y + R \diff z \right]
	\end{gathered}
	\end{align}
    o bé
    \begin{align}
    	\iint_{\mathcal{S}} (\vnabla \times \vec{f}) \cdot \dif \vec{\mathcal{S}} = \oint_{\mathcal{C}} \vec{f}(\vec{x}) \cdot \dif \vec{x}
    \end{align}
\end{thm}

%----------------------------------------------------------------------------------------
\subsection*{Parametritzacions típiques}


%----------------------------------------------------------------------------------------
\end{document}
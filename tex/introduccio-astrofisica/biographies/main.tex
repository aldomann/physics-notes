%-----------------------------------------------------------------
%	BASIC DOCUMENT LAYOUT
%-----------------------------------------------------------------
\documentclass[paper=a4, fontsize=12pt, twoside=semi]{scrartcl}
\usepackage[T1]{fontenc}
\usepackage[utf8]{inputenc}
\usepackage{lmodern}
\usepackage{slantsc}
\usepackage{microtype}
\usepackage[british]{babel}
\usepackage[fixlanguage]{babelbib}
\selectbiblanguage{british}

% Sectioning layout\\
\addtokomafont{sectioning}{\normalfont\scshape}
\usepackage{tocstyle}
\usetocstyle{standard}
\renewcommand*\descriptionlabel[1]{\hspace\labelsep\normalfont\bfseries{#1}}

% Empty pages
\usepackage{etoolbox}
\pretocmd{\section}{\cleardoubleevenemptypage}{}{}
\pretocmd{\part}{\cleardoubleevenemptypage\thispagestyle{empty}}{}{}
\renewcommand\partheadstartvskip{\clearpage\null\vfil}
\renewcommand\partheadmidvskip{\par\nobreak\vskip 20pt\thispagestyle{empty}}

% Paragraph indentation behaviour
\setlength{\parindent}{0pt}
\setlength{\parskip}{0.3\baselineskip plus2pt minus2pt}
\newcommand{\sk}{\medskip\noindent}

% Fancy header and footer
\usepackage{fancyhdr}
\pagestyle{fancyplain}
\fancyhead[LO]{\thepage}
\fancyhead[CO]{}
\fancyhead[RO]{\nouppercase{\mytitle}}
\fancyhead[LE]{\nouppercase{\leftmark}}
\fancyhead[CE]{}
\fancyhead[RE]{\thepage}
\fancyfoot{}
\renewcommand{\headrulewidth}{0.3pt}
\renewcommand{\footrulewidth}{0pt}
\setlength{\headheight}{13.6pt}

%-----------------------------------------------------------------
%	MATHS AND SCIENCE
%-----------------------------------------------------------------
\usepackage{amsmath,amsfonts,amsthm,amssymb}
\usepackage{xfrac}
\usepackage[a]{esvect}
\usepackage{chemformula}
\usepackage{graphicx}

\usepackage[arrowdel]{physics}
	\renewcommand{\vnabla}{\vec{\nabla}}
	% \renewcommand{\vectorbold}[1]{\boldsymbol{#1}}
	% \renewcommand{\vectorarrow}[1]{\vec{\boldsymbol{#1}}}
	% \renewcommand{\vectorunit}[1]{\hat{\boldsymbol{#1}}}
	\renewcommand{\vectorarrow}[1]{\vec{#1}}
	\renewcommand{\vectorunit}[1]{\hat{#1}}
	\renewcommand*{\grad}[1]{\vnabla #1}
	\renewcommand*{\div}[1]{\vnabla \vdot \va{#1}}
	\renewcommand*{\curl}[1]{\vnabla \cp \va{#1}}
	\let\rot\curl

% SI units
\usepackage[separate-uncertainty=true, alsoload=astro, alsoload=hep]{siunitx}
\sisetup{range-phrase = \text{--}, range-units = brackets}
\DeclareSIPrePower\quartic{4}
	\DeclareSIUnit\C{C}
	\DeclareSIUnit\dyn{dyn}
	\DeclareSIUnit\erg{erg}
	\DeclareSIUnit\year{y}
	\DeclareSIUnit\atm{atm}
	\DeclareSIUnit\cal{cal}
	\DeclareSIUnit\torr{torr}
	\DeclareSIUnit\au{AU}
	\DeclareSIUnit\jansky{Jy}
	\DeclareSIUnit\esu{esu}
	\DeclareSIUnit\planck{\textit{h}}
	\DeclareSIUnit\Msun{\textit{M}_{\odot}}
	\DeclareSIUnit\Lsun{\textit{L}_{\odot}}
	\DeclareSIUnit\Rsun{\textit{R}_{\odot}}

% Smaller trig functions
\newcommand{\Sin}{\trigbraces{\operatorname{s}}}
\newcommand{\Cos}{\trigbraces{\operatorname{c}}}
\newcommand{\Tan}{\trigbraces{\operatorname{t}}}

% Operator-style notation for matrices
\newcommand*{\mat}[1]{\hat{#1}}

% Matrices in (A|B) form via [c|c] option
\makeatletter
\renewcommand*\env@matrix[1][*\c@MaxMatrixCols c]{%
  \hskip -\arraycolsep
  \let\@ifnextchar\new@ifnextchar
  \array{#1}}
\makeatother

% Shorter \mathcal and \mathbb
\newcommand*{\mc}[1]{\mathcal{#1}}
\newcommand*{\mbb}[1]{\mathbb{#1}}

% Shorter ^\ast and ^\dagger
\newcommand*{\sast}{^{\ast}{}}
\newcommand*{\sdag}{^{\dagger}{}}

%-----------------------------------------------------------------
%	OTHER PACKAGES
%-----------------------------------------------------------------
\usepackage{environ}

% Left numbered equations
\makeatletter
	\NewEnviron{Lalign}{\tagsleft@true\begin{align}\BODY\end{align}}
\makeatother

% Plots and graphics
\usepackage{pgfplots}
\usepackage{tikz}
\usepackage{color}
	\makeatletter
		\color{black}
		\let\default@color\current@color
	\makeatother

% Richer enumerate, figure, and table support
\usepackage{enumerate}
\usepackage[shortlabels]{enumitem}
\usepackage{float}
\usepackage{tabularx}
\usepackage{booktabs}
	% \setlength{\intextsep}{8pt}
\numberwithin{equation}{section}
\numberwithin{figure}{section}
\numberwithin{table}{section}

% No indentation after certain environments
\makeatletter
\newcommand*\NoIndentAfterEnv[1]{%
	\AfterEndEnvironment{#1}{\par\@afterindentfalse\@afterheading}}
\makeatother
% \NoIndentAfterEnv{thm}
\NoIndentAfterEnv{defi}
\NoIndentAfterEnv{example}
\NoIndentAfterEnv{table}

% Misc packages
\usepackage{ccicons}
\usepackage{lipsum}

%-----------------------------------------------------------------
%	THEOREMS
%-----------------------------------------------------------------
\usepackage{thmtools}

% Theroems layout
\declaretheoremstyle[
	spaceabove=6pt, spacebelow=6pt,
	headfont=\normalfont,
	notefont=\mdseries, notebraces={(}{)},
	bodyfont=\small,
	postheadspace=1em,
]{small}

\declaretheorem[style=plain,name=Teorema,qed=$\square$,numberwithin=section]{thm}
\declaretheorem[style=plain,name=Corol·lari,qed=$\square$,sibling=thm]{cor}
\declaretheorem[style=plain,name=Lema,qed=$\square$,sibling=thm]{lem}
\declaretheorem[style=definition,name=Definició,qed=$\blacksquare$,numberwithin=section]{defi}
\declaretheorem[style=definition,name=Exemple,qed=$\blacktriangle$,numberwithin=section]{example}
\declaretheorem[style=small,name=Demostració,numbered=no,qed=$\square$]{sproof}

%-----------------------------------------------------------------
%	ELA MOTHERFUCKING GEMINADA
%-----------------------------------------------------------------
\def\xgem{%
	\ifmmode
		\csname normal@char\string"\endcsname l%
	\else
		\leftllkern=0pt\rightllkern=0pt\raiselldim=0pt
		\setbox0\hbox{l}\setbox1\hbox{l\/}\setbox2\hbox{.}%
		\advance\raiselldim by \the\fontdimen5\the\font
		\advance\raiselldim by -\ht2
		\leftllkern=-.25\wd0%
		\advance\leftllkern by \wd1
		\advance\leftllkern by -\wd0
		\rightllkern=-.25\wd0%
		\advance\rightllkern by -\wd1
		\advance\rightllkern by \wd0
		\allowhyphens\discretionary{-}{}%
		{\kern\leftllkern\raise\raiselldim\hbox{.}%
			\kern\rightllkern}\allowhyphens
	\fi
}
\def\Xgem{%
	\ifmmode
		\csname normal@char\string"\endcsname L%
	\else
		\leftllkern=0pt\rightllkern=0pt\raiselldim=0pt
		\setbox0\hbox{L}\setbox1\hbox{L\/}\setbox2\hbox{.}%
		\advance\raiselldim by .5\ht0
		\advance\raiselldim by -.5\ht2
		\leftllkern=-.125\wd0%
		\advance\leftllkern by \wd1
		\advance\leftllkern by -\wd0
		\rightllkern=-\wd0%
		\divide\rightllkern by 6
		\advance\rightllkern by -\wd1
		\advance\rightllkern by \wd0
		\allowhyphens\discretionary{-}{}%
		{\kern\leftllkern\raise\raiselldim\hbox{.}%
			\kern\rightllkern}\allowhyphens
	\fi
}

\expandafter\let\expandafter\saveperiodcentered
	\csname T1\string\textperiodcentered \endcsname

\DeclareTextCommand{\textperiodcentered}{T1}[1]{%
	\ifnum\spacefactor=998
		\Xgem
	\else
		\xgem
	\fi#1}

%-----------------------------------------------------------------
%	PDF INFO AND HYPERREF
%-----------------------------------------------------------------
\usepackage{hyperref}
\usepackage{cleveref}
\hypersetup{colorlinks, citecolor=black, filecolor=black, linkcolor=black, urlcolor=black}

\newcommand*{\mytitle}{Introducció a l'astrofísica}
\newcommand*{\mysubtitle}{Biographies}
\newcommand*{\myauthor}{Alfredo Hernández Cavieres}
\newcommand*{\myuni}{Universitat Autònoma de Barcelona, Departament de Física}
\newcommand*{\mydate}{\normalsize 2014-2015}

\pdfstringdefDisableCommands{\def\and{i }}

\usepackage{hyperxmp}
\hypersetup{pdfauthor={\myauthor}, pdftitle={\mytitle}}

%-----------------------------------------------------------------
%	TITLE SECTION AND DOCUMENT BEGINNING
%-----------------------------------------------------------------
\newcommand{\horrule}[1]{\rule{\linewidth}{#1}}
\title{
	\normalfont
	\small \scshape{\myuni} \\ [25pt]
	\horrule{0.5pt} \\[0.4cm]
	\huge \mytitle \\
	\Large \scshape{\mysubtitle} \\
	\horrule{2pt} \\[0.5cm]
}
\author{\myauthor}
\date{\mydate}

\begin{document}

\clearpage\maketitle
\thispagestyle{empty}
\addtocounter{page}{-1}

%-----------------------------------------------------------------
%	LICENCE
%-----------------------------------------------------------------
\section*{}\thispagestyle{empty}
\begin{centering}
	\href{http://creativecommons.org/licenses/by-nc-sa/4.0/deed.ca}{\huge \ccbyncsaeu}

	\normalsize This work is licensed under a Creative Commons

	Attribution-NonCommercial-ShareAlike 4.0

	International License.

\end{centering}

%-----------------------------------------------------------------
%	DOCUMENT BODY
%-----------------------------------------------------------------
\cleardoubleevenemptypage
\pdfbookmark[1]{\contentsname}{toc}
\tableofcontents

% \setcounter{secnumdepth}{2}
% \appendix
%-----------------------------------------------------------------
%	BIOGRAFIES
%	!TEX root = ./main.tex
%-----------------------------------------------------------------
\section{Biographies}\label{sec:bio}
% \footnote{Més informació a l'\cref{sec:bio} (pàgina \pageref{bio:}).}
% FIXME - & ' & ‘’ & "
% FIXME: dates (Ame to BrE)
% FIXME: shitty Eddington references
% FIXME: use \num for thousands?

\subsection[Walter \scshape{Adams}]{Walter Sydney Adams (1876-1956)}\label{bio:adams}
Adams was born in Antioch (now in Turkey). He was the son of missionaries working in Syria, then part of the Ottoman Empire, who returned to America in 1885. Adams graduated from Dartmouth College in 1898 and obtained his AM from the University of Chicago in 1900. After a year in Munich he began his career in astronomy as assistant to George Hale in 1901 at the Yerkes Observatory. He moved with Hale to the newly established Mount Wilson Observatory in 1904 where he served as assistant director, 1913--1923, and then as director from 1923 until his retirement in 1946.

At Mount Wilson Adams was able to use first the 60-inch (1.5-m) and from 1917 the 100-inch (2.5-m) reflecting telescopes in whose design and construction he had been closely associated. His early work was mainly concerned with solar spectroscopy, when he studied sunspots and solar rotation, but he gradually turned to stellar spectroscopy. In 1914 he showed how it was possible to distinguish between a dwarf and a giant star merely from their spectra. He also demonstrated that it was possible to determine the luminosity, i.e. intrinsic brightness, of a star from its spectrum. This led to Adams introducing the method of spectroscopic parallax whereby the luminosity deduced from a star's spectrum could be used to estimate its distance. The distance of many thousands of stars have been calculated by this method.

He is however better known for his work on the orbiting companion of Sirius, named Sirius B. Friedrich Bessel had first shown in 1844 that Sirius must have a companion and had worked out its mass as about the same as our Sun. The faint star was first observed telescopically by Alvan Clark in 1862. Adams succeeded in obtaining the spectrum of Sirius B in 1915 and found the star to be considerably hotter than the Sun. He realised that such a hot body, just eight light-years distant, could only remain invisible to the naked eye if it was very much smaller than the Sun, no bigger in fact than the Earth. In that case it must have an extremely high density, exceeding 100,000 times the density of water. Adams had thus discovered the first ‘white dwarf’ --a star that has collapsed into a highly compressed object after its nuclear fuel is exhausted.

If such an interpretation was correct then Sirius B should possess a very strong gravitational field. According to Einstein's general theory of relativity, this strong field should shift the wavelengths, of light waves emitted by it towards the red end of the spectrum. In 1924 Adams succeeded in making the difficult spectroscopic observations and did in fact detect the predicted red shift, which confirmed his own account of Sirius B and provided strong evidence for general relativity.

%-----------------------------------------------------------------
\subsection[Jocelyn \scshape{Bell Burnell}]{Jocelyn Bell Burnell (1943)}\label{bio:bell-burnell}
Susan Jocelyn Bell was born in Belfast, Northern Ireland on July 15, 1943. Her father was an architect and an avid reader. Through his books, Jocelyn was introduced to the world of astronomy. Her family and the staff of the Armagh Observatory, which was near her home in Belfast, encouraged her interest in astronomy. Jocelyn Bell's parents very strongly believed in educating women. When she failed the examination required for students wanting to pursue higher education in British schools, they sent her to a boarding school to continue her education.

In 1965, Jocelyn Bell earned a B.S. degree in physics from the University of Glasgow. Later that same year she began work on her Ph.D. at Cambridge University. It was while she was a graduate student at Cambridge, working under the direction of Antony Hewish, that Jocelyn Bell discovered pulsars.

Bell's first two years at Cambridge were spent assisting in the construction of an 81.5- megahertz radio telescope that was to be used to track quasars. The telescope went into operation in 1967. It was Jocelyn Bell's job to operate the telescope and to analyse over 120 meters of chart paper produced by the telescope every four days. After several weeks of analysis, Bell noticed some unusual markings on the chart paper. These markings were made by a radio source too fast and regular to be a quasar. Although the source's signal took up only about 2.5 centimetres of the 121.8 meters of chart paper, Jocelyn Bell recognised its importance. She had detected the first evidence of a pulsar.

In February of 1968, news of the discovery made by Jocelyn Bell was published in the journal \textit{Nature}. Further studies by groups of astronomers around the world identified the signals as coming from rapidly rotating neutron stars. These objects, first noticed by Jocelyn Bell, became known as pulsars. The term pulsar is an abbreviation for pulsating radio star or rapidly pulsating radio sources.

Jocelyn Bell received her Ph.D. in radio astronomy from Cambridge University in 1968. She married during that same year and changed her name to Burnell. Since leaving Cambridge in 1968, Dr. Bell Burnell has studied the sky in almost every region of the electromagnetic spectrum. She has received many honours and awards for her contributions to science. She presently heads the physics department at Open University in England.

%-----------------------------------------------------------------
\subsection[Hans \scshape{Bethe}]{Hans Albrecht Bethe (1906--2005)}\label{bio:bethe}
Hans Albrecht Bethe, an American physicist from Strasbourg, Germany, was educated at Frankfurt and Munich universities. He was born in 1906 and moved to the United States in 1935 to teach at Cornell University. He was the Director of the Theoretical Division at Los Alamos National Laboratory and participated at the most senior level in the World War II Manhattan Project that produced the first atomic weapons. During 1935-1938, he studied nuclear reactions and reaction cross sections. This research was useful to Bethe in more quantitatively developing Neil Bohr's theory of the compound nucleus. During the '80s and '90s he campaigned for the peaceful use of nuclear energy. He is most noted for his theories on atomic properties. In 1967, Bethe was awarded the Nobel Prize in Physics for his work in solar and stellar energy.

In 1995, at the age of 88, Bethe wrote an open letter calling on all scientists to ‘cease and desist’ from working on any aspect of nuclear weapons development and manufacture:
\begin{quote}
\textit{I am one of the few remaining such senior persons alive. Looking back at the half century since that time, I feel the most intense relief that these weapons have not been used since World War II, mixed with the horror that tens of thousands of such weapons have been built since that time one hundred times more than any of us at Los Alamos could ever have imagined.}

\textit{Today we are rightly in an era of disarmament and dismantlement of nuclear weapons. But in some countries nuclear weapons development still continues. Whether and when the various Nations of the world can agree to stop this is uncertain. But individual scientists can still influence this process by withholding their skills. Accordingly, I call on all scientists in all countries to cease and desist from work creating, developing, improving and manufacturing further nuclear weapons; and, for that matter, other weapons of potential mass destruction such as chemical and biological weapons.}
\end{quote}

%-----------------------------------------------------------------
\subsection[James \scshape{Bradley}]{James Bradley (1693--1762)}\label{bio:bradley}
The English astronomer James Bradley (1693-1762), one of the most determined and meticulous astronomers, discovered the aberration of light and the nutation of the earth axis.

James Bradley, who was the nephew of the astronomer James Pound, was born at Sherborne, Gloucestershire, in March 1693. He studied at Balliol College, Oxford, and took orders in 1719, when he was given his living at Bridstow. In the meantime he had become a skilled astronomer in the techniques of the day, under the instruction of his uncle. In 1718 he was elected a fellow of the Royal Society, and at the early age of 28 he became Savilian professor of astronomy at Oxford and so resigned from Bridstow.

Bradley lived at a time when an astronomer had to be his own technician --repairing, maintaining, and even making his own equipment. High magnifications were obtained by telescopes with lenses of great focal length, often so long that they were not fitted to tubes. In 1722 Bradley measured the diameter of Venus with a telescope over 212 feet in length.

Bradley was a friend of Samuel Molyneux, who had an observatory at Kew near London. There in 1725 Bradley systematically observed the star Draconis, hoping to discover the parallactic motion of the stars, that is, a seeming change in the positions of the stars, scattered through space, mirroring the change in the earth's position in its orbit around the sun. His observations were close to what he expected; the star described a tiny ellipse with an axis of only 40 seconds of arc. But the direction of the ellipse was wrong, and he concluded that the effect did not arise from parallactic motion. Greatly puzzled by the result, he at last realised that it was due to the finite velocity of light, owing to the velocity of the earth as it moved in an ellipse, which created an aberration of light. This was a very remarkable piece of work, all the more memorable for the fact that Bradley gave almost precisely the modern value for the constant of aberration, about 20.5 seconds (the modern value being 20.47 seconds).

Out of his work on aberration, Bradley discovered nutation, the oscillation of the earth's axis, caused by the changing direction of the gravitational pull of the moon on the equatorial bulge. He concluded that nutation must arise from the fact that the moon is sometimes above and sometimes below the ecliptic, and it should therefore have the periodicity of the lunar node, that is, approximately 18.6 years. His observations of this covered the period from 1727 to 1747, a full cycle of the motion of the moon's nodes.

At Greenwich, as astronomer royal, where he found the instruments in a poor state of repair, he obtained some fine new instruments, including an eight foot mural quadrant, with which he compiled a new catalogue of star positions. It was published posthumously and involved some 60,000 observations. F. W. Bessel's catalogue in 1818, with 3,000 star positions, was largely based on Bradley's observations. Bradley's health failed, and he retired to Chalford, Gloucestershire, where he died July 13, 1762.

%-----------------------------------------------------------------
\subsection[Annie \scshape{Cannon}]{Annie Jump Cannon (1863--1941)}\label{bio:cannon}
Annie Jump Cannon, Class of 1884, astronomer extraordinaire, towards the end of her life said:
\begin{quote}
\textit{In troubled days it is good to have something outside our planet, something fine and distant for comfort.}
\end{quote}

Cannon was born in Dover, Delaware, 11th December, 1863. She came to Wellesley College five years after it opened, partly because of the scientific training if offered. Professor of Physics and Astronomy Sarah Frances Whiting's enthusiasm for spectroscopy --recording stars-- inspired Cannon.

In 1892 Cannon did a grand tour of Europe, taking photos with a new box camera. On her return she prepared a booklet for the Blair Camera Company to use as a souvenir for the Chicago World's Fair in 1893.

After graduate study in physics at Wellesley College and Radcliffe, Cannon joined the staff of the Harvard College Observatory in 1897 working under world famous Professor Edward C. Pickering.

During her career Annie Jump Cannon discovered over 300 variable stars. Her speciality was classifying the characteristics of stars - over 350,000 of them. The results of her work appeared in \textit{The Henry Draper Catalogue} (1918-1924) and \textit{The Henry Draper Extension} (1925-1936).

Cannon was the first woman to be awarded the National Academy of Science's Draper Gold Medal (1931). In 1989 another Wellesley alumna, Martha P. Haynes '73 won the Henry Draper Medal for work primarily conducted at the Arecibo Observatory in Puerto Rico. She was only the second woman to win a Draper Medal.

%-----------------------------------------------------------------
\subsection[Subrahmanyan \scshape{Chandrasekhar}]{Subrahmanyan Chandrasekhar (1910--1995)}\label{bio:chandra}
Chandrasekhar studied at Presidency College, University of Madras in India and then at Trinity College, Cambridge England. From 1933 to1937 he worked at Cambridge, then joined the staff at the University of Chicago where he was to remain for the rest of his life.

In 1930 Chandra, as he was always called, showed that a star of a mass greater than 1.4 times that of the Sun had to end its life by collapsing into an object of enormous density unlike any object known at that time. He said \textit{one is left speculating on other possibilities}, namely objects such as black holes. For his work in this area he was awarded the Nobel Prize for Physics in 1983. He described this work in \textit{The Mathematical Theory of Black Holes} (1983).

His other books include \textit{Principles of Stellar Dynamics} (1942), \textit{Hydrodynamic and Hydromagnetic Stability} (1961), and \textit{Truth and Beauty: Aesthetics and Motivations in Science} (1987). He was awarded the Royal Medal of the Royal Society of London in 1962:
\begin{quote}
\textit{... in recognition of his distinguished researches in mathematical physics, particularly those related to the stability of convective motions in fluids with and without magnetic fields.}
\end{quote}

The Royal Society also awarded him their Copley Medal in 1984:
\begin{quote}
\textit{... in recognition of his distinguished work on theoretical physics, including stellar structure, theory of radiation, hydrodynamic stability and relativity.}
\end{quote}

From 1952 until 1971 Chandrasekhar was editor of the \textit{Astrophysical Journal}.

%-----------------------------------------------------------------
\subsection[Nicolaus \scshape{Copernicus}]{Nicolaus Copernicus (1473--1543)}\label{bio:copernicus}
Regarded as one of the central figures of the so-called Scientific Revolution, Copernicus (1473-1543) postulated a heliostatic theory in his De Revolutionibus Orbium Coelestium (1543). He did, however, maintain that planetary orbits were circular, and many believed that his system did not reflect the physical universe.

Born in Torun, Poland, in 1473, Copernicus first studied astronomy and astrology at the University of Cracow (1491-94). Through his uncle, Lukas Watzenrode (1447-1512), who later became the bishop of Varmia (Ermland), he was elected a canon of the cathedral chapter of Frombork (Frauenburg). As part of his requirement as a canon, he matriculated in 1496 in the University of Bologna to study both canon and civil law. There, he lodged with and worked as an assistant to Domenico Maria the Ferrarese of Novara (1454-1504), professor of mathematics and astrology and also the official compiler of prognostications for the university.

After briefly returning to Frombork, Copernicus studied medicine at the University of Padua (1501-3) and then moved on to the University of Ferrara where he obtained a doctorate in Canon Law (1503). He then returned to Varmia, where he was based for the rest of his life. He acted as medical adviser and secretary to his uncle at Heilsberg, and was later heavily involved with the administrative tasks in the diocese of Frombork. In 1514, the Lateran Council sought Copernicus's opinion on calendar reform. Around the same time, he began to circulate in manuscript the \textit{Commentariolus} (\textit{A Brief Description}), in which he criticised the current Ptolemaic system for not adhering to the principle of uniform circular motions and offered instead his own system in which the earth and all the other planets rotate around the sun.

By the 1530s, Copernicus's reputation as a skilled mathematician had even reached the ears of the Pope. A professor of mathematics at the University of Wittenberg, Georg Joachim Rheticus (1514-1574) who was on a tour of visiting distinguished scholars, visited Copernicus in 1539. Copernicus shared his ideas with him, and Rheticus published the \textit{Narratio Prima} (First Report on the Books of Revolution) in 1540 at Gdansk, in which he reported Copernicus' heliostatic theory in an astrological framework: the changing fortunes of the kingdom of the world, according to Rheticus, depended on the changing eccentricity of the sun. Following the favourable reception of the Narratio Prima, Rheticus persuaded Copernicus to publish a full account. This, of course, became the \textit{De Revolutionibus Orbium Coelestium} (\textit{On the Revolutions of the Heavenly Spheres}), published in March 1543 at Nuremberg. Copernicus died two months later.

Copernicus is often portrayed as a revolutionary figure who advocated a heliocentric system, overthrowing existing systems and institutions. Yet, his monumental work, the \textit{De Revolutionibus}, is far from a revolutionary manifesto for modern astronomy. Copernicus is known to have carried out many observations (though he explicitly mentions only about 27), and none seems to have been crucial for formulating his theory. The work follows closely the structure of Ptolemy's Almagest, it is based on parameters and data from Ptolemy, and his dedication to the Pope is written in a fashionable style. He does indeed provide a model of the universe in which the earth and all the other planets orbit around the sun and the earth acquired a daily rotation, but the sun itself was not quite in the centre of that universe. He established the order of planets and devised a system which accounted for the movements of planets without equants, but he was motivated by the desire to establish uniform circular motion, itself a classical ideal. Copernicus certainly believed that this was the true system of the physical universe, but this conviction was not shared widely by his contemporaries for various reasons.

%-----------------------------------------------------------------
\subsection[Heber \scshape{Curtis}]{Heber Doust Curtis (1872--1942)}\label{bio:curtis}
Heber D. Curtis eventually became famous for his role in astronomy's \textit{Great Debate} with Harlow Shapley in which Curtis argued that what astronomers called spiral nebulae were actually spiral galaxies outside our own Milky Way. Before we leap to science fame, we had simpler beginnings as a graduate student at the University of Virginia's Leander McCormick Observatory.

Curtis was born on June 27, 1872 in Muskegon, Michigan, the son of a one-armed Union veteran named Orson Blair Curtis and his wife Sarah Eliza Doust. His early education had little to do with astronomy. He attended Detroit High School and went on to the University of Michigan. He studied there for three years to receive his Bachelor of Arts degree and another year to receive his Master of Arts degree, both in classical languages. In his four years at the University of Michigan, he never stepped foot into the observatory there.

Upon graduation he returned to Detroit High School as a Latin instructor and six months later moved to Napa College, a small Methodist institution near San Francisco, where he taught Latin and Greek. He discovered astronomy as a hobby there with Napa College's small refracting telescope.

In 1895, he married Mary D. Raper and they went on to have four children. In 1896, Napa College merged with the College of the Pacific in San Jose and in the next year Curtis switched to become a professor of mathematics and astronomy. Curtis spent the summers of 1897 and 1898 at the Lick Observatory to further his astronomical studies and returned to the University of Michigan in the summer of 1899 to study celestial mechanics.

In 1900, Curtis attended the eclipse in Georgia as part of the Lick expedition. There, astronomers from other institutions, including the University of Virginia, encouraged Curtis to attend graduate school at UVa. That fall, Curtis, with his wife and two small children, moved to Charlottesville, where he studied as a Vanderbilt fellow under Ormond Stone. Curtis and his family managed to get by with only his fellowship to live
on and his wife later remarked to McCormick Observatory director Samuel Mitchell that their days at Virginia were \textit{the happiest of their lives}.

He received his Ph.D. from UVa in 1902 and the Lick Observatory immediately hired him, where he stayed for the next eighteen years. While at Lick Observatory, he composed his paper on spiral galaxies. It was the presentation of that paper, before the National Academy of Sciences in 1920, that erupted into a debate with Harlow Shapley now known as the \textit{Great Debate}. Edwin Hubble would later prove Curtis's theories correct.

In 1920, Curtis left the Lick Observatory for the University of Pittsburgh to serve as the director of the Allegheny Observatory. Unfortunately, the growing industrialization in Pittsburgh proved detrimental to astronomical observation. In 1930, Curtis returned to the University of Michigan one more time to serve as the director of its observatory.

Sadly, Curtis spent much of his last years suffering from a severe thyroid disease and he passed away in Ann Arbor on June 9, 1942. He was a member of the National Academy of Sciences and remembered as a well respected astronomer.

%-----------------------------------------------------------------
\subsection[Willem \scshape{de Sitter}]{Willem de Sitter (1872--1934)}\label{bio:de-sitter}
Willem De Sitter studied mathematics at Groningen and then joined the Groningen astronomical laboratory. He worked at the Cape Observatory in South Africa (1897-99) then, in 1908, de Sitter was appointed to the chair of astronomy at Leiden. From 1919 he was director of the Leiden Observatory.

In 1913 de Sitter produced an argument based on observations of double star systems which proved that the velocity of light was independent of the velocity of the source. It put to rest attempts which had been made up until this time to find emission theories of light which depended on the velocity of the source but were not in conflict with experimental evidence.

De Sitter corresponded with Ehrenfest in 1916, and he proposed that a four- dimensional space- time would fit in with cosmological models based on general relativity. He published a series of papers (1916-17) on the astronomical consequences of Einstein's general theory of relativity. He found solutions to Einstein's field equations in the absence of matter. This was significant since Mach had stated a principle that local inertial frames of reference were determined by the large scale distribution of mass in the universe. De Sitter asked:
\begin{quote}
\textit{If no matter exists other than the test body, does it have inertia.}
\end{quote}

De Sitter's work led directly to Eddington's 1919 expedition to measure the gravitational deflection of light rays passing near the Sun, results which, at that time, could only be obtained during an eclipse.

De Sitter, unlike Einstein, maintained that relativity actually implied that the universe was expanding, theoretical results which were later verified observationally and accepted by Einstein.

In fact Einstein had introduced the cosmological constant in 1917 to solve the problem of the universe which had troubled Newton before him, namely why does the universe not collapse under gravitational attraction. This rather arbitrary constant of integration which Einstein introduced admitting it was \textit{not justified by our actual knowledge of gravitation was later said by him to be the greatest blunder of my life}. However de Sitter wrote in 1919 that the term
\begin{quote}
\textit{... detracts from the symmetry and elegance of Einstein's original theory, one of whose chief attractions was that it explained so much without introducing any new hypothesis or empirical constant.}
\end{quote}

In 1932 Einstein and de Sitter published a joint paper with Einstein in which they proposed the Einstein-de Sitter model of the universe. This is a particularly simple solution of the field equations of general relativity for an expanding universe. They argued in this paper that there might be large amounts of matter which does not emit light and has not been detected. This matter, now called ‘dark matter’, has since been shown to exist by observing is gravitational effects. However the dark matter postulated by Einstein and de Sitter in 1932 still remains a mystery in that its nature is still unknown but is the subject of major research efforts today.

%-----------------------------------------------------------------
\subsection[Sir Arthur \scshape{Eddington}]{Sir Arthur Stanley Eddington (1882--1944)}\label{bio:eddington}
Arthur Eddington's father, Arthur Henry Eddington, taught at a Quaker training college in Lancashire before moving to Kendal to become headmaster of Stramongate School. He died of typhoid in an epidemic which swept the country in 1884 before his son was two years old. Arthur Eddington's mother, Sarah Ann Shout, came from Darlington and, like her husband, was also from a Quaker family. On Arthur Henry Eddington's death she was left to bring up Arthur and his older sister with relatively little income. The family moved to Weston-super-Mare where at first Arthur was educated at home before spending three years at a preparatory school.

In 1893 Arthur entered Brymelyn School in Weston-super-Mare which was mainly for boarders but he did not board at the school, being one of a small number of day pupils. The school provided a good education within the limited resources available to it and allowed Arthur to excel in mathematics and English literature. His progress through the school was rapid and he earned high distinction in mathematics. The level to which the school was able to take Arthur was, however, not very advanced and his good grounding in mathematics stopped short of reaching the differential and integral calculus.

In 1898 he was awarded a scholarship of £60 a year for three years by Somerset County (Weston-super-Mare is now in Avon but it was at that time in Somerset). Eddington had not reached sixteen years of age at the time, and so officially he was too young to enter university. It was a problem which was solved quickly, however, and did not cause him to delay his entry to Owens College, Manchester which he attended from 1898 to 1902. In his first year of study Eddington took general subjects before spending the next three years studying mainly physics. Although on a physics course, Eddington attended the mathematics lectures, being greatly influenced by one of his mathematics teachers, Horace Lamb. Of course the financial position of his family meant that they were not able to provide him with financial support but his outstanding academic work allowed him to win a number of highly competitive scholarships to provide enough money to let him complete his B.Sc. course with First Class Honours in 1902.

He was awarded a Natural Science scholarship of £75 a year to study at Trinity College, Cambridge, near the end of 1901. Entering Trinity in 1902 he received, in March 1903, a Mathematics Scholarship of £100 a year instead of the Natural Science scholarship. At Trinity he was taught by E T Whittaker, A N Whitehead and E W Barnes. He became Senior Wrangler in the Mathematical Tripos in 1904 and graduated with a M.A. in the following year. After graduating, he began a research project in the Cavendish Laboratory on thermionic emission but it appears not to have gone too well and he gave up the project. He began research in mathematics, also in 1905, but this was no more successful than his work in physics although he was to make use of the ideas many years later when he applied these early research ideas in mathematics to an astronomy problem.

Before the end of 1905 Eddington had made the move to astronomy with his appointment to a post at the Royal Observatory at Greenwich. Astronomy had been a topic of interest to him from an early age and he had been given a loan of a 3 inch telescope when less than 10 years old which had heightened his interest. On being appointed to fill a vacancy at the Royal Observatory he was immediately involved with a research project which had been underway since 1900 when photographic plates of Eros had been taken over the period of a year. Eddington's first task was to complete the reduction of these photographic observations to determine an accurate value for the solar parallax. Plummer writes in [19]:
\begin{quote}
\textit{He had introduced his method of analysis of two star-drifts, and his prevailing interest in statistical stellar astronomy was concentrated on the systematic motions and distribution of the stars throughout his Greenwich years.}
\end{quote}

Eddington was a Smith's prize winner for an essay on the proper motions of stars in 1907, and he was awarded a Trinity College Fellowship. George Darwin, a son of Charles Darwin and Plumian professor of astronomy at Cambridge, died in December 1912. In 1913 Eddington was appointed to fill the vacant position of Plumian Professor of Astronomy. There were in fact two chairs of astronomy at Cambridge, the other being the Lowndean chair. Originally the Plumian chair covered the experimental side of the subject while the Lowndean chair covered the theoretical side. Although this distinction had become somewhat blurred over the years the appointment of Eddington was certainly seen as an appointment in experimental astronomy. However, the holder of the Lowndean chair died towards the end of 1913 and, in 1914, Eddington became director of the Cambridge Observatory. In doing so he effectively took over responsibility for both theoretical and experimental astronomy at Cambridge. Shortly after his appointment as director of the Cambridge Observatory he was elected a Fellow of the Royal Society.

Shortly after taking up his role of leading astronomy research at Cambridge, World War I broke out. As we noted above Eddington came from a Quaker tradition and, as a conscientious objector, he avoided active war service and was able to continue his research at Cambridge during the war years of 1914-18. This was, however, not an easy time for him giving him a highly stressful period right at the beginning of his tenure of the Cambridge chair.

Eddington made important contributions to the theory of general relativity. His interest in this topic started in 1915 when he received papers by Einstein and by de Sitter which came to him via the Royal Astronomical Society. He became interested in this theory, particularly since it provided an explanation for the previously noticed, but unexplained, advance of the perihelion of Mercury. He lectured on relativity at the British Association meeting in 1916 and produced a major report on the topic for the Physical Society in 1918.

In the following year Eddington led an eclipse expedition to Principe Island in West Africa. Its aim was to verify the bending of light passing close to the sun which was predicted by relativity theory. At that time such observations of stars close to the sun in the sky could only be made during a total eclipse. He sailed from England in March 1919 and by mid-May had his instruments set up on Principe Island. The eclipse was due to occur at two o'clock in the afternoon of 29 May but that morning there was a storm with heavy rain. Eddington wrote (see for example [6]):
\begin{quote}
\textit{The rain stopped about noon and about 1.30 ... we began to get a glimpse of the sun. We had to carry out our photographs in faith. I did not see the eclipse, being too busy changing plates, except for one glance to make sure that it had begun and another half-way through to see how much cloud there was. We took sixteen photographs. They are all good of the sun, showing a very remarkable prominence; but the cloud has interfered with the star images. The last few photographs show a few images which I hope will give us what we need ...}
\end{quote}

He remained on Principe Island to develop the photographs and to try to measure the deviation in the stellar positions. The cloud made the plates of poor quality and hard to measure. On 3 June he recorded in his notebook:
\begin{quote}
\textit{... one plate I measured gave a result agreeing with Einstein.}
\end{quote}

The results from the Africa expedition provided the first confirmation of Einstein's theory that gravity will bend the path of light when it passes near a massive star. Eddington wrote, in a parody of the Rubaiyat of Omar Khayyam (see for example [6]):-

\textit{Oh leave the Wise our measures to collate
One thing at least is certain, light has weight
One thing is certain and the rest debate
Light rays, when near the Sun, do not go straight.}

Eddington lectured on relativity at Cambridge, giving a beautiful mathematical treatment of the topic. He used these lectures as a basis for his book Mathematical Theory of Relativity which was published in 1923. Einstein said that this work was:
\begin{quote}
\textit{... the finest presentation of the subject in any language.}
\end{quote}

In addition to his work in relativity theory Eddington also did important work on the internal structure of stars. He discovered the mass-luminosity relationship for stars, he calculated the abundance of hydrogen, and he produced a theory to explain the pulsation of Cepheid variable stars. His early research on this is contained in the important work The Internal Constitution of Stars (1926). Eddington had a long running argument with James Jeans over the mechanism by which energy was created in stars. He wrote, correctly of course, that as to the process of generating energy:
\begin{quote}
\textit{... probably the simplest hypothesis ... is that there may be a slow process of annihilation of matter.}
\end{quote}

Jeans, however, favoured the theory that the energy was the result of contraction. Of course this is not entirely wrong since a star when it forms will initially heat up under the energy generated by contraction before nuclear reactions begin and then provide the energy source for most of the star's life.

Among Eddington's many books were philosophical works such as The Nature of the Physical World (1928), New Pathways of Science (1935) and The Philosophy of Physical Science (1939). Eddington's rather unusual view of the importance of the history of a subject comes over in these works. He believed that familiarity with the history of a subject was a hindrance to creative research in that subject. The authors of this archive would have to register their strong disagreement with Eddington on this issue!

Eddington had a fascination with the fundamental constants of nature and produced some surprising numerical coincidences most of which were published after his death in Fundamental Theory (1946), a book prepared for publication by Whittaker. He writes in that book that his aim was to determine the relation between the sizes of different physical systems. Ronan writes [20]:
\begin{quote}
\textit{Eddington, hard-headed mathematician and down-to-earth astronomer though he might be, possessed a mystical side to his nature and the last years of his life were spent in an attempt to construct a huge relativistic synthesis of the physical universe, an edifice in which the bricks would be subatomic and astronomical evidence of the observer and the mortar the underlying mathematical relationships between them.}
\end{quote}

In [9] Kilmister delves deeply into the ideas which led Eddington to the theories he put forward in Fundamental Theory in attempting to unite quantum mechanics and general relativity. Kilmister explains how Eddington considered that epistemology is at the basis of physics, that physical laws and physical constants are the consequences of the condition of observation. It was Dirac's 1928 paper on the wave equation of the electron which had first set Eddington on the path of seeking ways to unify quantum mechanics and general relativity. Kilmister explains how Dirac's use of spinors had surprised Eddington and led him to study a generalisation of the Dirac algebra. His work on algebras which would give a symmetrical description of nature is also examined in [21].

Eddington was knighted in 1930 and received the Order of Merit in 1938. He received many other honours including gold medals from the Astronomical Society of the Pacific (1923), the Royal Astronomical Society (1924), The National Academy of Washington (1924), the French Astronomical Society (1928), and the Royal Society (1928). In addition to election to the Royal Society, he was elected to the Royal Society of Edinburgh, the Royal Irish Academy, the National Academy of Sciences, the Russian Academy of Sciences, the Prussian Academy of Sciences and many others. He was invited to give the Bakerian Lecture of the Royal Society of London in 1926 when he lectured on Diffuse matter in interstellar space.

Plummer writes in [18]:
\begin{quote}
\textit{To his splendid equipment as a mathematical physicist he owed much ... A bold imagination was coupled with an exceptional knowledge of those features which are accessible to observation. ... To launch out into unknown seas, to be venturesome even at the risk of error, Eddington felt himself called, and the reward of the pioneer came to him. ... Simplicity and modesty were his outstanding characteristics ...}
\end{quote}

Eddington's achievements are summed up in [3] as follows:
\begin{quote}
\textit{He was a gifted astronomer whose original theories and powers of mathematical analysis took his science a long way forward; he was a brilliant expositor of physics and astronomy, able to communicate the most difficult conceptions in the simplest and most fascinating language; and he was an able interpreter to philosophers of the significance of the latest scientific discoveries.}
\end{quote}


%-----------------------------------------------------------------
\subsection[Galileo \scshape{Galilei}]{Galileo Galilei (1564--1642)}\label{bio:galileo}
Galileo Galilei's father, Vincenzo Galilei (c.1520 - 1591), who described himself as a nobleman of Florence, was a professional musician. He carried out experiments on strings to support his musical theories. Galileo studied medicine at the university of Pisa, but his real interests were always in mathematics and natural philosophy. He is chiefly remembered for his work on free fall, his use of the telescope and his employment of experimentation.

After a spell teaching mathematics, first privately in Florence and then at the university of Pisa, in 1592 Galileo was appointed professor of mathematics at the university of Padua (the university of the Republic of Venice). There his duties were mainly to teach Euclid's geometry and standard (geocentric) astronomy to medical students, who would need to know some astronomy in order to make use of astrology in their medical practice. However, Galileo apparently discussed more unconventional forms of astronomy and natural philosophy in a public lecture he gave in connection with the appearance of a New Star (now known as ‘Kepler's supernova’) in 1604. In a personal letter written to Kepler (1571 - 1630) in 1598, Galileo had stated that he was a Copernican (believer in the theories of Copernicus). No public sign of this belief was to appear until many years later.

In the summer of 1609, Galileo heard about a spyglass that a Dutchman had shown in Venice. From these reports, and using his own technical skills as a mathematician and as a workman, Galileo made a series of telescopes whose optical performance was much better than that of the Dutch instrument. The astronomical discoveries he made with his telescopes were described in a short book called \textit{Sidereus Nuncius} (\textit{Sidereal Messenger}) published in Venice in May 1610. It caused a sensation. Galileo claimed to have seen mountains on the Moon, to have proved the Milky Way was made up of tiny stars, and to have seen four small bodies orbiting Jupiter. These last, with an eye on getting a job in Florence, he promptly named ‘the Medicean stars’.

It worked. Soon afterwards, Galileo became ‘Mathematician and [Natural] Philosopher’ to the Grand Duke of Tuscany. In Florence he continued his work on motion and on mechanics, and began to get involved in disputes about Copernicanism. In 1613 he discovered that, when seen in the telescope, the planet Venus showed phases like those of the Moon, and therefore must orbit the Sun not the Earth. This did not enable one to decide between the Copernican system, in which everything goes round the Sun, and the Tychonic (Tycho Brahe) one in which everything but the Earth (and Moon) goes round the Sun which in turn goes round the Earth. Most astronomers of the time in fact favoured the Tychonic system. However, Galileo showed a marked tendency to use all his discoveries as evidence for Copernicanism, and to do so with great verbal as well as mathematical skill. He seems to have made a lot of enemies by making his opponents look fools. Moreover, not all of them actually were fools.

There eventually followed some expression of interest by the Inquisition. Prima facie, Copernicanism was in contradiction with Scripture, and in 1616 Galileo was given some kind of secret, but official, warning that he was not to defend Copernicanism. Just what was said on this occasion was to become a subject for dispute when Galileo was accused of departing from this undertaking in his \textit{Dialogo sopra i due massimi sistemi del mondo} (\textit{Dialogue Concerning the Two Chief World Systems}), published in Florence in 1632. Galileo, who was not in the best of health, was summoned to Rome, found to be \textit{vehemently suspected of heresy}, and eventually condemned to house arrest, for life, at his villa at Arcetri (above Florence). He was also forbidden to publish. By the standards of the time he had got off rather lightly.

Galileo's sight was failing, but he had devoted pupils and amanuenses, and he found it possible to write up his studies on motion and the strength of materials. The book, \textit{Discorsi e dimostrazioni matematiche, intorno à due nuove scienze} (\textit{Discourses and Mathematical Demonstrations Relating to Two New Sciences}), was smuggled out of Italy and published in Leiden (in the Netherlands) in 1638.

Galileo wrote most of his later works in the vernacular, probably to distance himself from the conventional learning of university teachers. However, his books were translated into Latin for the international market, and they proved to be immensely influential.

%-----------------------------------------------------------------
\subsection[George \scshape{Gamow}]{Georgiy Antonovich Gamov (1904--1968)}\label{bio:gamow}
It was In Odessa, Russian Empire (now Ukraine) where George Gamow was born on 4th of March 1904. He had a mixture of Russian and Ukrainian blood since his parents were Russian-Ukrainian as well. His mother worked as a teacher and taught history and geography at an all-girls school in Odessa while his father taught literature and the Russian language in a local high school. Of course, it is a given that young George Gamow knew how to speak Russian but he learned how to speak French with the help of his mother and learned German from a tutor. Gamow did not learn how to speak English until he was in college but he did become fluent after that. In fact, nearly all of his first publications were written in Russian or German and only later on did he switch to writing in English for his lay audience and his technical papers.

George Gamow went to school at Novorossiya University in Odessa from 1922--1923 then moved on to the University of Leningrad in 1923--1929. He was mentored by Alexander Friedman when he was in Leningrad though later on he had to go to other advisors for his dissertation. While at Leningrad, he made friends with other theoretical physics students: Dmitri Ivanenko, Lev Landau, and Matvey Bronshtein (Matvey was a victim of the Soviet regime; he was arrested in 1937 and a year later, was executed). The three students became close and formed a group they called “The Three Musketeers.” The group met to analyse and talk more about important discoveries on quantum mechanics.

Once he graduated, he moved on to work at Gottingen where he conducted studies on quantum theory. He got his doctorate by way of his work with the atomic nucleus. After he got his Ph.D., he moved on to the University of Copenhagen and worked at the Theoretical Physics Institute from 1928 to 1931. He took a break to move to the Cambridge Cavendish Laboratory where he did some work with the notable Ernest Rutherford. While all this was happening, he still worked with the atomic nucleus and even proposed his “liquid drop” model and made some time to work with Fritz Houtermans and Robert Atkinson on stellar physics.

George Gamow was elected as a member of the Academy of Sciences of the USSR in 1931; he was just 28 years old at the time. This made him the youngest ever member in the organization’s long history. From 1931-33, Gamow got a job at the Radium Institute in Leningrad where he worked at the Physical Department headed by Vitaly Khlopin. It was during this time when Gamow, together with Lev Mysovskii and Igor Kurchatov, designed the first ever cyclotron in Europe. Gamow and Mysovskii submitted the draft of their design to the Academic Council of the Radium Institute for consideration which the council approved. It was in 1937 that the cyclotron was completed.

During the early parts of the 20th century, radioactive metals were known to have half-lives and at the same time, characteristic energies were known to come from radioactive emissions. Gamow, in 1928, was already able to solve the theory of alpha decay of an atom nucleus by way of tunneling. Of course, he didn’t do it on his own and had some help from Nikolai Cochin who handled the mathematical side. At the same time, Robert Gurney and Edward U. Condon were also able to solve the problem but the results they achieved were nowhere near as quantitative as the ones by Gamow. Some years later, the name Gamow-Sommerfeld factor was given to the probability of incoming nuclear particles tunneling their way through the Coulomb barrier and going through nuclear reactions. Aside from his work with radioactive decay, he also wrote a paper with a certain Ralph Alpher (a student of his) on Cosmogony. With his work in cosmogony and the Big Bang nucleosynthesis, he got introduced to DNA. The structure was discovered by Francis Crick, Rosalind Franklin, and James D. Watson in 1953 and Gamow attempted to solve the problem of how the four different bases found in chains of DNA could control protein synthesis from amino synthesis.

George Gamow worked for several Soviet establishments but due to increased oppression, he decided to leave Russia. He was denied permission in 1931 to attend a conference in Italy but it was also the year he married Lyubov Vokhmintseva (a Russian physicist). The first two years together as a married couple were spent trying to leave Russia whether they had permission or not.

He and his wife finally managed to move to America in 1934. He worked as a professor at George Washington University. He was also involved in several high profile projects such as presenting the chemical elements’ periodic table as a continuous tape. He also spent his last teaching years at the University of Colorado Boulder. He died in Colorado on 19 August 1968 due to liver failure.

%-----------------------------------------------------------------
\subsection[Carl \scshape{Gauss}]{Johann Carl Friedrich Gauss (1777--1855)}\label{bio:gauss}
% WIP

%-----------------------------------------------------------------
\subsection[George \scshape{Hale}]{George Ellery Hale (1868--1938)}\label{bio:hale}
George Ellery Hale was born in Chicago on 29 June 1868. A single child heir to his family's considerable fortune, Hale developed an interest in astronomy at a young age. In this he benefited from the continuing moral and financial support of his father, who over the years of his childhood and teenage years purchased him telescopes and spectrometers of increasing power. By 1891 Hale was effectively equipped with his own private solar astronomical laboratory. This unusual opportunity was not lost on Hale, who went on to become one of the foremost astronomer in the U.S. In 1890 he graduated from the Massachusetts Institute of Technology with a bachelor's degree in Physics, at which time his scientific reputation was already well established. Although he never completed a doctorate, in 1892 he was appointed professor of Astronomy at the University of Chicago (without salary for the first three years, however), and launched in a life long campaign to fund and build ever better astronomical observatories. Overworked and suffering from recurrent episodes of depression, Hale resigned as director of Mt. Wilson Observatory in 1923, and retired from the active scientific research scene in the following years, arguably at the height of his career. He died on 21 February 1938 in Pasadena, California.

Even though Hale capitalised heavily on his family's wealth and various connections in the mid-west's financial circles, his talents and energy as an organiser and fund raiser on behalf of various astronomical projects remains extraordinary (and arguably as yet unsurpassed) by any standards. Over the years he organised the founding of three world class astronomical observatories; in the 1890s he secured funding for the establishment of the University of Chicago's Yerkes Observatory, in nearby William's Bay, Wisconsin. That observatory became fully operational in 1897, and harboured for a time the largest telescope in the world. He then secured funds for the establishment of a solar observatory on Mt. Wilson in California, of which he became director in 1904, and which long remained the best solar observatory in the world, in addition to hosting for a time the world's largest night time telescope. Though he did not live to see the project taken to completion in 1948, Hale was the main force behind the construction of the 5 meter telescope on Mt. Palomar, which remained for over three decades the the world's largest optical telescope. He was also instrumental in the founding of the American Astronomical Society in 1899, and later in turning the then relatively unknown Throop Polytechnical Institute in Pasadena, into what is now the California Institute of Technology.

In 1895 Hale co-founded (and edited for nearly 30 years) \textit{The Astrophysical Journal}. This was originally envisioned as an international forum for the publication of astronomically relevant papers in the field of spectroscopy, but the journal rapidly expanded its scope to become (and remains to this day) the world's leading research Journal in the field of Astrophysics.

In the first two decades of the twentieth century, Hale and his collaborators were constantly innovating and pushing the limit of astronomical and spectroscopic instrumentation. They effectively invented specialised solar tower telescopes, and pioneered the field of spectropolarimetry. Hale's most acclaimed scientific work was his demonstration that sunspots are the seat of strong magnetic fields, and that their polarity reveals striking spatial and temporal regularities that betray the presence of a well- organised, large-scale magnetic field in the solar interior. Nearly a century after Schwabe's discovery of the 11 year sunspot cycle, Hale's work on sunspots finally put solar cycle studies on a truly physical footing.

%-----------------------------------------------------------------
\subsection[Sir John \scshape{Herschel}]{Sir John Frederick William Herschel (1792--1871)}\label{bio:herschel}
% WIP

%-----------------------------------------------------------------
\subsection[Hipparchus of Rhodes]{Hipparchus of Rhodes (c. 190--120 BCE)}\label{bio:hipparcus}
% WIP

%-----------------------------------------------------------------
\subsection[Edwin \scshape{Hubble}]{Edwin Powell Hubble (1889--1953)}\label{bio:hubble}
Edwin Hubble was a man who changed our view of the Universe. In 1929 he showed that galaxies are moving away from us with a speed proportional to their distance. The explanation is simple, but revolutionary: the Universe is expanding.

Hubble was born in Missouri in 1889. His family moved to Chicago in 1898, where at High School he was a promising, though not exceptional, pupil. He was more remarkable for his athletic ability, breaking the Illinois State high jump record. At university too he was an accomplished sportsman playing for the University of Chicago basketball team. He won a Rhodes scholarship to Oxford where he studied law. It was only some time after he returned to the US that he decided his future lay in astronomy.

In the early 1920s Hubble played a key role in establishing just what galaxies are. It was known that some spiral nebulae (fuzzy clouds of light on the night sky) contained individual stars, but there was no consensus as to whether these were relatively small collections of stars within our own galaxy, the “Milky Way’ that stretches right across the sky, or whether these could be separate galaxies, or ‘island universes’, as big as our own galaxy but much further away. In 1924 Hubble measured the distance to the Andromeda nebula, a faint patch of light with about the same apparent diameter as the moon, and showed it was about a hundred thousand times as far away as the nearest stars. It had to be a separate galaxy, comparable in size our own Milky Way but much further away.

Hubble was able to measure the distances to only a handful of other galaxies, but he realised that as a rough guide he could take their apparent brightness as an indication of their distance. The speed with which a galaxy was moving towards or away from us was relatively easy to measure due to the Doppler shift of their light. Just as a sound of a racing car becomes lower as it speeds away from us, so the light from a galaxy becomes redder. Though our ears can hear the change of pitch of the racing car engine our eyes cannot detect the tiny red-shift of the light, but with a sensitive spectrograph Hubble could determine the redshift of light from distant galaxies.

The observational data available to Hubble by 1929 was sketchy, but whether guided by inspired instinct or outrageous good fortune, he correctly divined a straight line fit between the data points showing the redshift was proportional to the distance. Since then much improved data has shown the conclusion to be a sound one. Galaxies are receding from us, and one another, as the Universe expands. Within General Relativity, the theory of gravity proposed by Albert Einstein in 1915, the inescapable conclusion was that all the galaxies, and the whole Universe, had originated in a Big Bang, thousands of millions of years in the past. And so the modern science of cosmology was born.

Hubble made his great discoveries on the best telescope in the world at that time - the 100-inch telescope on Mount Wilson in southern California. Today his name carried by the best telescope we have, not on Earth, but a satellite observatory orbiting our planet. The Hubble Space Telescope is continuing the work begun by Hubble himself to map our Universe, and producing the most remarkable images of distant galaxies ever seen, many of which are available via the World Wide Web.

%-----------------------------------------------------------------
\subsection[Sir James \scshape{Jeans}]{Sir James Hopwood Jeans (1877--1946)}\label{bio:jeans}
% WIP

%-----------------------------------------------------------------
\subsection[Henrietta \scshape{Leavittt}]{Henrietta Swan Leavitt (1868--1921)}\label{bio:leavitt}
Henrietta Leavitt was born in Cambridge, Massachusetts, the daughter of a Congregational minister. She attended Oberlin College and the Society for Collegiate Instruction of Women (later Radcliffe College). As a senior in 1892, Leavitt discovered astronomy. After graduation she took another course in it, but then spent several years at home when she suffered a serious illness that left her severely deaf. She hadn't forgotten about astronomy, though. She volunteered at the Harvard College Observatory in 1895. Seven years later she was appointed to the permanent staff (at a salary of 30 cents an hour) by director Charles Pickering. She got little chance to do theoretical work, but did become head of the photographic photometry department. This group studied photo images of stars to determine their magnitude.

During her career, Leavitt discovered more than 2,400 variable stars, about half of the known total in her day. These stars change from bright to dim and back fairly regularly. Leavitt's work with variable stars led to her most important contribution to the field: the Cepheid variable period- luminosity relationship. By intense observation of a certain class of variable star, the Cepheids, Leavitt discovered a direct correlation between the time it took a star to go from bright to dim to how bright it actually was. Knowing this relationship helped other astronomers, such as Edwin Hubble, to make their own ground-breaking discoveries.

Leavitt also developed a standard of photographic measurements that was accepted by the International Committee on Photographic Magnitudes in 1913, and called the Harvard Standard. To do this she used 299 plates from 13 telescopes and used logarithmic equations to order stars over 17 magnitudes of brightness. She continued refining and enlarging upon this work throughout her life.

Leavitt was not allowed to pursue her own topics of study, but researched what the head of the observatory assigned. Because of the prejudices of the day, she didn't have the opportunity to use her intellect to the fullest, but a colleague remembered her as \textit{possessing the best mind at the Observatory}, and a modern astronomer calls her \textit{the most brilliant woman at Harvard}. She worked at the Harvard College Observatory until her death from cancer in 1921.

%-----------------------------------------------------------------
\subsection[Charles \scshape{Messier}]{Charles Messier (1730--1817)}\label{bio:messier}
% WIP

%-----------------------------------------------------------------
\subsection[Jan \scshape{Oort}]{Jan Hendrik Oort (1900--1992)}\label{bio:oort}
Although Oort is best known for his theory on comets, he devoted most of his career to pioneering research into the structure and dynamics of the Milky Way galaxy. Using mathematical calculations, he showed that the outer regions of the galaxy actually rotated more slowly than the inner parts, a finding contrary to the prevailing theory that the galaxy rotated uniformly, as a wheel does. He also established that the Earth's solar system was not at the centre of the Milky Way, as was then believed, but at the galaxy's outer edges. In the 1950s Oort helped develop radio astronomy, which revolutionised the study of the Milky Way's structure. In his most publicised work Oort postulated the existence of what came to be known as the Oort cloud, a mass of icy objects surrounding the solar system. He proposed that comets were pieces detached from this cloud.

Oort was born on April 28, 1900, in Franeker, a small town in the Netherlands. His parents, Ruth Faber Oort and Abraham Oort, a physician, had five children. When Oort was three, his family moved from Franeker to Was-senaar. He eventually married, and he and his wife, Mieke, had three children of their own.
After graduation from the gymnasium in Leiden in 1917 Oort matriculated at the University of Groningen, where he studied under the renowned astronomer Jacobus Kapteyn, the first scientist to measure and compute the position of stars in the Milky Way. In 1920 Oort's astronomical gifts were recognised when he was awarded the Bachiene Foundation Prize for a paper he wrote about stars of the spectral types F, G, K, and M. Upon his graduation from Groningen in 1922 Oort served as a research assistant at the Yale University observatory. At Yale he was influenced by the work of the Harvard astronomer Harlow Shapley, who demonstrated that the size of the galaxy was larger than previously supposed. Oort returned to the Netherlands in 1924 and received his Ph.D. in 1926 from the University of Groningen.

In 1926 Oort became an instructor at the University of Leiden, where he remained for the rest of his career. In 1927 he investigated the theories of the astronomer Bertil Lindblad and affirmed that the Milky Way did indeed rotate. Oort employed complex mathematical equations to prove that the galaxy exhibited differential rotation, whereby stars near the centre of the Milky Way moved faster than those farther away. These calculations allowed him to deduce the mass and size of the galaxy, which he measured to be 100,000 light years across and 20,000 light years deep. Even more importantly, Oort demonstrated that our solar system was not at the centre of the galaxy, but rather lingered on the outside, some 30,000 light years from the centre. This insight forever changed scientists' conception of the Earth in space. After being appointed a professor of astronomy at Leiden, Oort postulated the existence of dark matter (or missing matter), mass that could be detected neither visually nor by contemporary calculations.
After World War II, during which Oort had been forced into hiding, he was appointed director of the Leiden Observatory. In the 1950s he collaborated with Hendrik van de Hulst in the development of radio astronomy, which offered a degree of precision in studying the galaxy, because radio waves, unlike visible light, are unaffected by dust or gas. In 1951 the pair discovered the 21-centimetre hydrogen line that cuts through the Milky Way. They were able to use this information to confirm that the galaxy completes a full rotation once every 225 million years. Oort also proposed the existence of a cloud of icy objects that surrounds the Sun, accounting for the appearance of comets. This phenomenon was later named the Oort cloud in his honour.

During his career Oort headed several astronomical groups, including the International Astronomical Union from 1958 to 1961. He co-founded the European Southern Observatory in 1962. Although he retired in 1970, Oort continued to research and publish papers on the Milky Way and other topics in astronomy. He died at the age of 92 on November 5, 1992.

%-----------------------------------------------------------------
\subsection[Jim \scshape{Peebles}]{Phillip James Edwin Peebles (1935)}\label{bio:peebles}
Jim Peebles was born in Winnipeg and graduated from the University of Manitoba in 1958. He then went to Princeton University as a graduate student in physics, and he has been there ever since, currently as Albert Einstein Professor of Science Emeritus.

With Robert Dicke and others he predicted the existence of the cosmic background radiation (previously predicted by Ralph Alpher and Robert Herman) and planned to seek it just before it was found by Arno Penzias and Robert Wilson. He has investigated characteristics of this radiation and how it may be used to constrain models of the universe. He has led statistical studies of clustering and super-clustering of galaxies. He has calculated the universal abundances of helium and other light elements, demonstrating agreement between big bang theory and observation. He has provided evidence of the existence of large quantities of dark matter in the haloes of galaxies, and he continues to work on the origin of galaxies. Peebles was one of the first to resurrect Einstein's cosmological constant, suggesting it was needed in the 1980s. His books on physical cosmology have had a significant impact in convincing physicists that the time has come to study cosmology as a respectable branch of physics.

%-----------------------------------------------------------------
\subsection[Edward \scshape{Pickering}]{Edward Charles Pickering (1846--1919)}\label{bio:pickering}
The American astronomer Edward Charles Pickering (1846-1919) was a pioneer in the fields of stellar spectroscopy and photometry.

Edward Pickering was born on 19th July, 1846, in Boston, Massachusetts, of a distinguished New England family. After studying at Boston Latin School, he attended Lawrence Scientific School, graduating \textit{summa cum laude} in 1865. He taught mathematics at that institution for a year and then moved to Massachusetts Institute of Technology, becoming Thayer professor of physics in 1868. He married Elizabeth Wardsworth Sparks in 1874.

In 1876 Pickering accepted the directorship of the Harvard Observatory, an appointment that both surprised and angered many, for he had no experience as an observational astronomer. The choice of a physicist, however, placed Harvard in the leadership of the trend, growing since the mid-century, towards a ‘new astronomy’ which used the methods of the physicist to seek a knowledge of stellar structure and its evolution. The day of the observer, who noted the positions of heavenly bodies, was virtually over, and Pickering's appointment to such an important post may well have symbolised the victory of the new astronomy over the old.

The most important achievement of Pickering's directorate was in stellar photometry, a field barely explored with large instruments at the time. When he began the work, even the magnitudes of the stars were not fixed on any generally accepted scale. Pickering established a widely accepted scale and employed instruments, at least one - the meridian photometer - of his own invention, to achieve unprecedented accuracy in determining the magnitudes of 80,000 stars.

Pickering's second work, begun in 1885, was the compilation of a ‘photographic library’, as he called it, giving a complete photographic chart of the stellar universe down to the eleventh magnitude on some 300,000 glass plates. From such plates the past record of the stars may be studied; Pickering, for example, was able to plot the path of Eros in the sky from photographs taken 4 years before this asteroid was discovered.

Pickering was also a leader in stellar spectroscopy, laying the foundation for the method of spectral classification now universally accepted and obtaining the material for the \textit{Draper Catalogue}, containing 200,000 stars. He twice received the Gold Medal of the Royal Astronomical Society, was a member of the National Academy of Sciences, and was a founder in 1898 of the American Astronomical Society, of which he was later president. By the time of his death, on 3rd February, 1919, he was generally recognised as one of the two or three outstanding astronomical researchers in America.

%-----------------------------------------------------------------
\subsection[Harlow \scshape{Shapley}]{Harlow Shapley (1885--1972)}\label{bio:shapley}
For discovering that the Earth's solar system was not at the centre of the Milky Way galaxy the astronomer Harlow Shapley was called a latter-day Nicolaus Copernicus. His contributions to astronomy and humanitarian causes were numerous. Shapley concluded that the Milky Way was much larger than had previously been assumed and devoted considerable effort to the study of nebulae (a term that at that time referred to any celestial body not identified as a star). He worked tirelessly to save Jewish scientists from Nazi Germany and after the war became committed to the cause of peace. During his tenure as the director of the Harvard University Observatory he played a significant role in making that observatory the pre-eminent centre for astronomical research in the United States.

Shapley and his twin brother, Horace, were born on November 2, 1885, in Nashville, Missouri. Their father, Willis Shapley, was a farmer and a teacher who died when the boys were young. Their mother, Sarah Stowell Shapley, raised the family single-handedly thereafter. In 1914 Harlow Shapley wed Martha Betz, a classmate who not only was a mathematician and astronomer herself, but also co-authored several scientific papers with her husband. The couple had five children.

Shapley enrolled at the University of Missouri at the age of 20 and obtained his B.A. in mathematics and physics in 1910 and his M.A. in 1911. He then moved to the Princeton University Observatory, where he received his Ph.D. in astronomy in 1913. His dissertation on eclipsing binary stars, which he later expanded and published, was considered an influential text.

Shapley's first position was at the Mount Wilson Observatory in Pasadena, California, where he worked under George Ellery Hale from 1913 to 1920. He made some of his more profound discoveries during this period. He built upon the work of the astronomer Henrietta Swan Leavitt, who had demonstrated that the rate at which a Cepheid (a type of star) pulsed was directly related to its brightness. Shapley used this insight to determine a star's absolute brightness and then compare that to its observed brightness. Employing this method, he was able to calculate the distance from Earth of any given Cepheid. With this measuring system he also determined the placement of Earth's solar system within the Milky Way. In 1915 he estimated the galaxy's centre to be some 50,000 light-years from the Sun, in the constellation Sagittarius. It was later established that this “Shapley Centre” was 33,000 light-years from the Sun. Shapley also proposed that the diameter of the Milky Way was 10 times greater than previously thought.

In 1921 Shapley left California for the Harvard Observatory, where he served as director until 1952. He continued to pursue his research, but his more significant contribution at Harvard was improving the reputation of the observatory itself. Not only did he create a graduate program in astronomy, but he also actively courted the best scientists from around the world to the observatory. In fact, he spearheaded a movement to rescue Jewish scientists from the horrors of the Nazi regime. He saw to it that dozens of these persecuted intellectuals took positions at Harvard instead of perishing in concentration camps.

After the war Shapley continued to focus on international politics. He was a key figure in the creation of the United Nations Educational, Scientific, and Cultural Organisation (UNESCO), and he represented the United States at the writing of its charter in 1945. He strongly advocated scientific cooperation with the Soviet Union and strove to undermine the House Committee on Un-American Activities, which accused him of sympathising with communists. He chaired the Committee of One Thousand, an organisation devoted to First Amendment rights, which counted Albert Einstein among its members. Shapley taught at Harvard until 1956 and then lectured on astronomy, politics, and philosophy across the country.

Shapley's findings about the size and structure of the Milky Way shifted the current conception of the Earth's solar system and its place in the galaxy. His tenure as director of Harvard's Observatory helped shape the development of astronomy in the United States. He saved a great many lives and provided a passionate example of how science and politics could work together for positive causes. Shapley died on October 20, 1972, in Boulder, Colorado.


\end{document}

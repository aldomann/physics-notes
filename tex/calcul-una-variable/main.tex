%----------------------------------------------------------------------------------------
%    PACKAGES AND OTHER DOCUMENT CONFIGURATIONS
%----------------------------------------------------------------------------------------
\documentclass[paper=a4, fontsize=12pt, twoside=semi]{scrartcl}
\usepackage[T1]{fontenc}
\usepackage[utf8]{inputenc}
\usepackage{lmodern}
\usepackage{slantsc}
\usepackage{microtype}
\usepackage[catalan]{babel}
\usepackage[fixlanguage]{babelbib}
\selectbiblanguage{catalan}

\addtokomafont{sectioning}{\normalfont\scshape}
\usepackage{tocstyle}
\usetocstyle{standard}
\renewcommand*\descriptionlabel[1]{\hspace\labelsep\normalfont\bfseries{#1}}

\usepackage{fancyhdr}
\pagestyle{fancyplain}
\fancyhead[LO]{\thepage}
\fancyhead[CO]{}
\fancyhead[RO]{\nouppercase{\mytitle}}
\fancyhead[LE]{\nouppercase{\leftmark}}
\fancyhead[CE]{}
\fancyhead[RE]{\thepage}
\fancyfoot{}
\renewcommand{\headrulewidth}{0.3pt}
\renewcommand{\footrulewidth}{0pt}
\setlength{\headheight}{13.6pt}

\usepackage{etoolbox}
\pretocmd{\section}{\cleardoubleevenemptypage}{}{}
\pretocmd{\part}{\cleardoubleevenemptypage\thispagestyle{empty}}{}{}

\renewcommand\partheadstartvskip{\clearpage\null\vfil}
\renewcommand\partheadmidvskip{\par\nobreak\vskip 20pt\thispagestyle{empty}}

\setlength{\parindent}{0pt}
\setlength{\parskip}{0.3\baselineskip plus2pt minus2pt}
\newcommand{\sk}{\medskip\noindent}

\usepackage{hyperref}
\hypersetup{colorlinks, citecolor=black, filecolor=black, linkcolor=black, urlcolor=black}

%----------------------------------------------------------------------------------------
%    MATHS AND ENVIRONMENTS
%----------------------------------------------------------------------------------------
\usepackage{amsmath,amsfonts,amsthm,amssymb}
\usepackage{xfrac}
\usepackage[a]{esvect}
\usepackage{pgfplots}
\usepackage{tikz}
\usepackage{color}
		\makeatletter
				\color{black}
				\let\default@color\current@color
		\makeatother
\usepackage{thmtools}
\usepackage{environ}

\usepackage{siunitx}

\newcommand*{\dif}{\mathrm{d}}
\newcommand*{\diff}{\mathop{}\!\mathrm{d}}
\newcommand*{\vnabla}{\vec{\nabla}}

\numberwithin{equation}{section}
\numberwithin{figure}{section}
\numberwithin{table}{section}

\usepackage{enumerate}
\usepackage{booktabs}
\usepackage{float}
		% \setlength{\intextsep}{8pt}

\makeatletter
\renewcommand*\env@matrix[1][*\c@MaxMatrixCols c]{%
		\let\@ifnextchar\new@ifnextchar
		\array{#1}}
\makeatother

\usepackage{ccicons}
\usepackage{lipsum}

\makeatletter
\newcommand*\NoIndentAfterEnv[1]{%
	\AfterEndEnvironment{#1}{\par\@afterindentfalse\@afterheading}}
\makeatother
%\NoIndentAfterEnv{thm}
\NoIndentAfterEnv{defi}
\NoIndentAfterEnv{example}
\NoIndentAfterEnv{table}

%----------------------------------------------------------------------------------------
%    THEOREMS
%----------------------------------------------------------------------------------------
\makeatletter
\providecommand{\@fourthoffour}[4]{#4}
\newcommand\fixstatement[2][\proofname\space del]{%
		\ifcsname thmt@original@#2\endcsname
				\AtEndEnvironment{#2}{%
						\xdef\pat@label{\expandafter\expandafter\expandafter
								\@fourthoffour\csname thmt@original@#2\endcsname\space\@currentlabel}%
						\xdef\pat@proofof{\@nameuse{pat@proofof@#2}}%
				}%
		\else
				\AtEndEnvironment{#2}{%
						\xdef\pat@label{\expandafter\expandafter\expandafter
								\@fourthoffour\csname #1\endcsname\space\@currentlabel}%
						\xdef\pat@proofof{\@nameuse{pat@proofof@#2}}%
				}%
		\fi
		\@namedef{pat@proofof@#2}{#1}%
}
\globtoksblk\prooftoks{1000}
\newcounter{proofcount}
\NewEnviron{proofatend}{%
		\edef\next{%
				\noexpand\begin{proof}[\pat@proofof\space\pat@label]%
				\unexpanded\expandafter{\BODY}}%
		\global\toks\numexpr\prooftoks+\value{proofcount}\relax=\expandafter{\next\end{proof}}
		\stepcounter{proofcount}}
\def\printproofs{%
			\count@=\z@
			\loop
				\the\toks\numexpr\prooftoks+\count@\relax
					 \ifnum\count@<\value{proofcount}%
					 \advance\count@\@ne
			\repeat}
\makeatother

\declaretheorem[style=plain,name=Teorema,qed=$\square$,numberwithin=section]{thm}
\declaretheorem[style=plain,name=Corol·lari,qed=$\square$,sibling=thm]{cor}
\declaretheorem[style=plain,name=Lemma,qed=$\square$,sibling=thm]{lem}
		\fixstatement{thm}
		\fixstatement[Demostració del]{lem}

\declaretheorem[style=definition,name=Definició,qed=$\blacksquare$,numberwithin=section]{defi}
\declaretheorem[style=definition,name=Exemple,qed=$\blacktriangle$,numberwithin=section]{example}

%----------------------------------------------------------------------------------------
%    ELA MOTHERFUCKING GEMINADA
%----------------------------------------------------------------------------------------
\def\xgem{%
	 \ifmmode
		 \csname normal@char\string"\endcsname l%
	 \else
		 \leftllkern=0pt\rightllkern=0pt\raiselldim=0pt
		 \setbox0\hbox{l}\setbox1\hbox{l\/}\setbox2\hbox{.}%
		 \advance\raiselldim by \the\fontdimen5\the\font
		 \advance\raiselldim by -\ht2
		 \leftllkern=-.25\wd0%
		 \advance\leftllkern by \wd1
		 \advance\leftllkern by -\wd0
		 \rightllkern=-.25\wd0%
		 \advance\rightllkern by -\wd1
		 \advance\rightllkern by \wd0
		 \allowhyphens\discretionary{-}{}%
		 {\kern\leftllkern\raise\raiselldim\hbox{.}%
			 \kern\rightllkern}\allowhyphens
	 \fi
	 }
\def\Xgem{%
	 \ifmmode
		 \csname normal@char\string"\endcsname L%
	 \else
		 \leftllkern=0pt\rightllkern=0pt\raiselldim=0pt
		 \setbox0\hbox{L}\setbox1\hbox{L\/}\setbox2\hbox{.}%
		 \advance\raiselldim by .5\ht0
		 \advance\raiselldim by -.5\ht2
		 \leftllkern=-.125\wd0%
		 \advance\leftllkern by \wd1
		 \advance\leftllkern by -\wd0
		 \rightllkern=-\wd0%
		 \divide\rightllkern by 6
		 \advance\rightllkern by -\wd1
		 \advance\rightllkern by \wd0
		 \allowhyphens\discretionary{-}{}%
		 {\kern\leftllkern\raise\raiselldim\hbox{.}%
			 \kern\rightllkern}\allowhyphens
	 \fi
	 }

\expandafter\let\expandafter\saveperiodcentered
	 \csname T1\string\textperiodcentered \endcsname

\DeclareTextCommand{\textperiodcentered}{T1}[1]{%
	 \ifnum\spacefactor=998
		 \Xgem
	 \else
		 \xgem
	 \fi#1}

%----------------------------------------------------------------------------------------
%    PDF INFO
%----------------------------------------------------------------------------------------
\newcommand*{\mytitle}{Càlcul en una variable}
\newcommand*{\myauthor}{Alfredo Hernández Cavieres}
\newcommand*{\myuni}{Universitat Autònoma de Barcelona, Departament de Física}
\newcommand*{\mydate}{\normalsize 2012-2013}

\usepackage{hyperxmp}
\hypersetup{pdfauthor={\myauthor}, pdftitle={\mytitle}}

%----------------------------------------------------------------------------------------
%    TITLE SECTION AND DOCUMENT BEGINNING
%----------------------------------------------------------------------------------------
\newcommand{\horrule}[1]{\rule{\linewidth}{#1}}
\title{
		\normalfont
		\small \scshape{\myuni} \\ [25pt]
		\horrule{0.5pt} \\[0.4cm]
		\huge \mytitle \\
		\horrule{2pt} \\[0.5cm]
}
\author{\myauthor}
\date{\mydate}

\begin{document}

\clearpage\maketitle
\thispagestyle{empty}
\addtocounter{page}{-1}

%----------------------------------------------------------------------------------------
%    LLICÈNCIA
%----------------------------------------------------------------------------------------
\section*{}\thispagestyle{empty}
\begin{centering}
		\huge \ccbyncsaeu

		\normalsize Aquesta obra està subjecta a una llicència de

		Reconeixement-NoComercial-CompartirIgual 4.0

		Internacional de Creative Commons.

\end{centering}

%----------------------------------------------------------------------------------------
%    TABLE OF CONTENTS
%----------------------------------------------------------------------------------------
\cleardoubleevenemptypage
\pdfbookmark[1]{\contentsname}{toc}
\tableofcontents

%----------------------------------------------------------------------------------------
%    SECTIONS
%----------------------------------------------------------------------------------------
% \part*{Càlcul I}
		% %----------------------------------------------------------------------------------------
%    ELS NÚMEROS REALS
%----------------------------------------------------------------------------------------
\section{Els números reals}
\subsection{Insuficiència dels números racionals}

%----------------------------------------------------------------------------------------
\subsection{El cos dels números reals}

%----------------------------------------------------------------------------------------
\subsection{L'arrel d'un número real}

%----------------------------------------------------------------------------------------
\subsection{Successions de números reals}

%----------------------------------------------------------------------------------------
\subsection{Expressió decimal d'un número real}

%----------------------------------------------------------------------------------------
\subsection{Altres propietats dels números reals}
		%----------------------------------------------------------------------------------------
%    FUNCIONS D'UNA VARIABLE REAL
%----------------------------------------------------------------------------------------
\section{Funcions d'una variable real}
Funció real $\equiv f:D \to \mathbb{R}, \quad D \subseteq \mathbb{R}$.
\begin{itemize}
    \item Domini $\equiv \{ x \in D \mid \exists f(x) \}$.
    \item Imatge $\equiv \{ f(x) \mid x \in D \}$.
    \item Gràfic $\equiv \{ (x, f(x)) \mid x \in D \} \subseteq D \times \mathbb{R}$.
\end{itemize}

\subsubsection*{Operacions}
Siguin $f$ i $g$ funcions definides a $D$ i $E$ respectivament,
\begin{itemize}
    \item Suma: $(f+g)(x) \equiv f(x) + g(x), \quad \forall x \in D \cap E$.
    \item Producte: $(fg)(x) \equiv f(x) g(x), \quad \forall x \in D \cap E$.
    \item Composició: $(g \circ f)(x) \equiv g(f(x)), \quad \forall x \in D,$ tal que $f(x) \in E$.
\end{itemize}

%----------------------------------------------------------------------------------------
\subsection{Límit d'una funció}
$\lim\limits_{x \to a} = l$, si $\forall \varepsilon > 0, \quad \exists \delta > 0$ tal que si $|x-a| < \delta \Rightarrow |f(x) - l | < \varepsilon \Leftrightarrow$ és possible expressar una inequació entre $\delta$ i $\varepsilon$. 
L'existència del límit no depèn del comportament de $f(x)$ en $a$, sinó al seu voltant.

\subsubsection*{Propietats}
\begin{itemize}
    \item El $\lim f(x)$ és únic.
    \item Si $\exists \lim\limits_{x \to a} f(x) \Rightarrow f$ és fitada en algun $\varepsilon ^{\ast} (a, \delta)$.
    \item Si $\lim\limits_{x \to a} f(x) = k$ i $\lim\limits_{x \to a} g(x) = l$,
    \begin{itemize}
        \item $\lim (f(x) + g(x)) = k + l$.
        \item $\lim (f(x) g(x)) = kl$.
        \item $\lim (\sfrac{f(x)}{g(x)}) = \sfrac{k}{l}$ (si $l \neq 0$ i $g(x) \neq 0$ en $\varepsilon ^{\ast} (a, \delta)$.
        \item $f (x) \leq g(x)$ en algun $\varepsilon ^{\ast} (a, \delta) \Rightarrow k \neq l$.
    \end{itemize}
\end{itemize} 

\subsubsection*{Límits per la dreta i per l'esquerra (procediment)}
\begin{itemize}
    \item Dreta: $ x = a + \delta , \quad \lim\limits_{x \to a^{+}} \Rightarrow \lim\limits_{\delta \to 0^{-}} $.
    \item Esquerra: $ x = a - \delta , \quad \lim\limits_{x \to a^{-}} \Rightarrow \lim\limits_{\delta \to 0^{+}} $.
\end{itemize}

%----------------------------------------------------------------------------------------
\subsection{Continuïtat d'una funció}
Si $x \in \varepsilon (a, \delta) \Rightarrow f(x) \in \varepsilon (f(a), \varepsilon ) \Rightarrow \lim\limits_{x \to a} f(x) = f(\lim\limits_{x \to a} x)$.

\subsubsection*{Discontinuïtats}
\begin{itemize}
    \item Evitables: Quan $\exists \lim\limits_{x \to a} f(x)$ finit, però $\neq f(a)$. La discontinuïtat es pot eliminar igualant $f(a)$ a $\lim\limits_{x \to a} f(x)$.
    \item Inevitables: $\lim\limits_{x \to a} f(x)$ és $\pm \infty$ o $\nexists \lim\limits_{x \to a} f(a)$.
    \begin{itemize}
        \item De salt: $\lim\limits_{x \to a^{-}} f(x) \neq \lim\limits_{x \to a^{+}} f(x)$.
        \item Oscil·lant: $\nexists$ algun o els dos límits laterals, però $f$ és fitada a $\varepsilon ^{\ast} (a, \varepsilon)$.
        \item Infinita: $f$ no és fitada a cap $\varepsilon ^{\ast} (a, \delta)$, tant si no existeixen els límits laterals com si són infinits.
    \end{itemize}
\end{itemize}

\subsubsection*{Teorema del màxim i el mínim}
Si $f$ és contínua al compacte $K \Rightarrow f(K)$ té màxim i mínim.

\subsubsection*{Teorema del valor intermedi de Bolzano}
Si $f$ és contínua a $[a,b]$, amb $f(a) \neq f(b)$ i $y_{0} \in (f(a),f(b)) \Rightarrow \exists! c \in (a,b)$ tal que $f(c) = y_{0}$. 

\subsubsection*{Teorema de la continuïtat de la funció inversa}
Si $f$ és invertible a l'interval $I \Rightarrow f$ creix o decreix estrictament a $I$ i $f^{-1}$ és contínua a $f(I)$.

\subsubsection*{Continuïtat uniforme}
$f$ és uniformement contínua si $\forall \varepsilon > 0, \quad \exists \delta > 0$ tal que $\frac{f(x) - f(x')}{x - x'} < \frac{\varepsilon}{\delta}$.

\subsubsection*{Funció de Lipschitz}
És útil en molts casos (no sempre) per establir si una $f$ és uniformement contínua en un interval.
\begin{align*}
\begin{gathered}
    \text{Lipschitz } \Rightarrow \text{ uniformement contínua.} \\
    \text{No Lipschitz } \not \Rightarrow \text{ no uniformement contínua.}
\end{gathered}
\end{align*}
Sigui $f: A \rightarrow \mathbb{R}$. Si $\exists k \geq 0$ tal que $\displaystyle \frac{| f(a_{1}) - f(a_{2})|}{|a_{1} - a_{2}|} \leq k, \quad \forall a_{1}, a_{2} \in A$, amb $a_{1} \neq a_{2} \Rightarrow f$ és uniformement contínua a $A$.

%----------------------------------------------------------------------------------------
\subsection{Infinitèsims}
$f(x)$ és un infinitèsim quan $x \to a \Leftrightarrow \lim\limits_{x \to a} f(x) = 0$.

Si $f(x)$ i $g(x)$ són dos infinitèsims quan $x \to a$, 
\begin{align}
    \lim\limits_{x \to a} = 
    \begin{cases} 
        0 \quad \rightarrow f(x) \text{ és d'ordre superior a } g(x) \\ 
        k \quad \rightarrow f(x) \text{ i } g(x) \text{ són del mateix ordre} \\ 
        \pm \infty \quad \rightarrow f(x) \text{ és d'ordre inferior a } g(x) \\ 
        \nexists \quad \rightarrow f(x) \text{ i } g(x) \text{ no són comparables}
    \end{cases}
\end{align}
Prenent $g(x) = (x-a)^{n}, \; n \in \mathbb{N}$ com a infinitèsim de referència, 
\begin{align}
    f(x) \text{ és un infinitèsim d'ordre }
        \begin{cases} 
            > n \\ 
            n \\ 
            < n 
        \end{cases} 
    \text{ si } \lim\limits_{x \to a} \frac{f(x)}{(x-a)^{n}} =  
        \begin{cases}
            0 \\
            k \\
            \pm \infty
        \end{cases}
\end{align}
Per tant, l'ordre d'un infinitèsim mesura la rapidesa amb què $f(x)$ tendeix a zero.
\begin{align}
\begin{gathered}
    f(x) \text{ d'ordre } >n \Rightarrow f(x) = o[(x-a)]^{n} \\
    f(x) \text{ d'ordre } \geq n \Rightarrow f(x) = O[(x-a)]^{n}
\end{gathered}
\end{align}

\subsubsection*{Alguns infinitèsims equivalents quan $x \to 0$}
\begin{itemize}
    \item $f(x) \approx x$.
    \item $\sin(x) \approx x$.
    \item $\tan(x) \approx x$.
    \item $1-\cos(x) \approx \frac {x^2}{2}$.
    \item $\arcsin(x) \approx x$.
    \item $\arctan (x) \approx x$.
    \item $e^{x}-1 \approx x$.
    \item $\ln(1+x) \approx x$.
\end{itemize}
		%----------------------------------------------------------------------------------------
%    LA DERIVADA
%----------------------------------------------------------------------------------------
\section{La derivada}
\subsection{El problema del pendent}
Quina és la recta que millor s'ajusta al gràfic d'una funció en un punt donat? La intuïció ens diu que és la recta tangent. No obstant, no sempre es pot parlar de recta tangent. El desenvolupament d'aquesta idea ens portarà als conceptes de derivada i funció derivable. Només per a aquestes funcions es podrà parlar de pendent i de recta tangent.

%----------------------------------------------------------------------------------------
\subsection{Derivada}
\subsubsection*{Derivada d'una funció en un punt}
Sigui $f$ una funció definida en algun entorn d'un punt $a$. Definim la derivada de $f$ en el punt $a$ com el límit (si existeix):
\begin{align}
    f'(a) \equiv \lim\limits_{x \to a} \frac{f(x)-f(a)}{x-a} = \lim\limits_{h \to 0} \frac{f(a+h) - f(a)}{h}
\end{align}    
Si aquest límit existeix, direm que la funció $f$ és derivable en el punt $a$.

\subsubsection*{Derivada i pendent de la recta tangent}
De totes les línies rectes que passen pel punt $(a, f(a))$, la recta tangent al gràfic de $f$ en aquest punt és la que millor s'ajusta al gràfic en el punt esmentat. Aquesta recta és:
\begin{align}
    y = f(a) + f'(a)(x-a)
\end{align}
Així doncs, derivabilitat $\Leftrightarrow$ existència de recta tangent.
\subsubsection*{Diferencial}
Podem reexpressar la recta tangent, $y = f(a) + f'(a)(x-a)$, en termes de nous eixos de coordenades, $(\dif x,\dif y)$, paral·leles a $(x,y)$ amb l'origen traslladat al punt $(a,f(a))$:
\begin{align}
    \dif y = f'(a) \diff x
\end{align}
Aquesta funció lineal s'anomena diferencial de la funció $f$ en el punt $a$. Els conceptes de derivabilitat i diferenciabilitat de $f$ en un punt són equivalents. Si $f$ és diferenciable en tots els punts de $D \subseteq \mathbb{R}$, escriurem:
\begin{align}
    f'(x)= \frac{\dif y}{\dif x} = \frac{\dif f}{\dif x} = \frac{\dif }{\dif x}f(x)
\end{align}
Aquesta notació, anomenada de Liebnitz, és molt útil perquè permet manipular els diferencials com a fraccions, respectant les propietats de les derivades. Pel que fa a derivades d'ordre superior $n$, escriurem:
\begin{align}
    f^{(n)}(x)= \frac{\dif ^{n}y}{\dif x^{n}} = \frac{\dif ^{n}f}{\dif x} = \frac{\dif }{\dif x^{n}}f(x)
\end{align}

\subsubsection*{Derivabilitat i continuïtat}
Si $f$ és derivable en el punt $a$, la funció $f(x)-f(a)$ ha de ser un infinitèsim (d'ordre igual o superior a 1) quan $x \to a$. Per tant:
\begin{align}
    \lim\limits_{x \to a} f(x) = f(a)
\end{align}
Així doncs, $f$ derivable en el punt $a \Leftrightarrow f$ contínua en el punt $a$.

\subsubsection*{Propietats de la derivada}
\begin{itemize}
    \item Linealitat de la derivada:
        \subitem $(f+g)'(a) = f'(a) + g'(a)$.
        \subitem $(kf)'(a) = kf'(a), \quad \forall k \in \mathbb{R}$.
    \item Derivada del producte:
        \subitem $(fg)'(a) = f'(a)g(a) + f(a)g'(a)$.
    \item Derivada del quocient:
        \subitem $(f/g)'(a)= [f'(a)g(a)-f(a)g'(a)]/[g(a)]^{2}$
    \item Derivada d'una funció composta (regla de la cadena):
        \subitem $(g \circ f)'(a) = g'[f(a)] f'(a)$
    \item Derivada de la funció inversa:
        \subitem $(f^{-1})'(a) = 1/[f'[f^{-1}(a)]]$
\end{itemize}

%----------------------------------------------------------------------------------------
\subsection{Derivades elementals}
\subsubsection*{Funcions polinòmiques}
\begin{itemize}
    \item $f(x) = k , \quad f'(x) = 0$
    \item $f(x) = x^{n} , \quad f'(x) = n x^{n-1}$
\end{itemize}

\subsubsection*{Funcions exponencials}
\begin{itemize}
    \item $f(x) = \ln x , \quad f'(x) = \frac{1}{x}$
    \item $f(x) = e^{x} , \quad f'(x) = e^{x}$
    \item $f(x) = \log_{a} x, \quad f'(x) = \frac{1}{x \ln a}$
    \item $f(x) = a^{x} , \quad f'(x) = a^{x} \ln a$
\end{itemize}

\subsubsection*{Funcions trigonomètriques}
\begin{itemize}
    \item $f(x) = \sin x , \quad f'(x) = \cos x$
    \item $f(x) = \cos x , \quad f'(x) = - \sin x $
    \item $f(x) = \tan x , \quad f'(x) = \frac{1}{\cos^{2} x} $
    \item $f(x) = \arcsin x , \quad f'(x) = \frac{1}{\sqrt{1 - x^{2}}}$
    \item $f(x) = \arccos x , \quad f'(x) = - \frac{1}{\sqrt{1 - x^{2}}}$
    \item $f(x) = \arctan x , \quad f'(x) = \frac{1}{1 + x^{2}}$
\end{itemize}

\subsubsection*{Funcions hiperbòliques}
\begin{itemize}
    \item $f(x) = \sinh x , \quad f'(x) = \cosh x$
    \item $f(x) = \cosh x , \quad f'(x) = \sinh x $
    \item $f(x) = \tanh x , \quad f'(x) = \frac{1}{\cosh^{2} x} $
    \item $f(x) = \sinh^{-1} x, \quad f'(x) = \frac{1}{\sqrt{1 + x^{2}}}$
    \item $f(x) = \cosh^{-1} x, \quad f'(x) = \frac{1}{\sqrt{-1 + x^{2}}}$
    \item $f(x) = \tanh^{-1} x, \quad f'(x) = \frac{1}{1-x^{2}}$
\end{itemize}

%----------------------------------------------------------------------------------------
\subsection{Teoremes del valor mitjà}
\subsubsection*{Teorema del valor mitjà de Rolle}
Si $f$ és contínua a l'interval $[a,b]$ i derivable en el seu interior, 
\begin{align}    
    f(a) = f(b) \Rightarrow \exists c \in (a,b) \text{ tal que } f'(c) = 0 
\end{align}
En efecte, al ser $f$ contínua a $[a,b]$, ha de tenir màxim i mínim.

\subsubsection*{Teorema del valor mitjà de Cauchy}
Si $f$ i $g$ són contínues a l'interval $[a,b]$ iderivables en el seu interior, i es compleix que $g(a) \neq g(b)$ i que $f'(x)$ i $g'(x)$ no s'anul·len simultàniament en cap punt $\in [a,b]$
\begin{align}
    \Rightarrow \exists c \in (a,b) \text{ tal que } \frac{f(b)-f(b)}{g(b)-g(a)} = \frac{f'(c)}{g'(c)}
\end{align}

\subsubsection*{Teorema del valor mitjà de Lagrange}
Si $f$ és contínua a l'interval $[a,b]$ i derivable en el seu interior
\begin{align}
    \Rightarrow \exists c \in (a,b) \text{ tal que } \frac{f(b)-f(b)}{b-a} = f'(c)
\end{align}
És un cas particular del teorema anterior; només cal prendre $g(x)=x$.

%----------------------------------------------------------------------------------------
\subsection{Creixement i concavitat}
\subsubsection*{Creixement i decreixement}
\begin{itemize}
    \item $f$ és creixent a $(a,b) \Leftrightarrow f'(x) \geq 0, \quad \forall x \in (a,b)$.
    \item $f$ és decreixent a $(a,b) \Leftrightarrow f'(x) \leq 0, \quad \forall x \in (a,b)$.
\end{itemize}
    
\subsubsection*{Extrems relatius}
\begin{itemize}
    \item Hi ha un màxim al punt $c \in (a,b)$ si $f'(c)=0$, $f$ és creixent a $(a,c)$ i decreixent a $(c,b)$.
    \item Hi ha un mínim al punt $c \in (a,b)$ si $f'(c)=0$, $f$ és decreixent a $(a,c)$ i creixent a $(c,b)$.
\end{itemize}

\subsubsection*{Concavitat i convexitat}
\begin{itemize}
    \item $f$ és còncava a $(a,b)$ si la seva derivada és monòtonament decreixent a l'interval.
    \item $f$ és convexa a $(a,b)$ si la seva derivada és monòtonament creixent a l'interval.
\end{itemize}

\subsubsection*{Punts d'inflexió}
Les condicions següents són equivalents:
\begin{itemize}
    \item Hi ha un punt d'inflexió al punt $c \in (a,b)$ si $f'(c)=0$ i $f$ és creixent (o decreixent) tant a $(a,c)$ com a $(c,b)$.
    \item Hi ha un punt d'inflexió al punt $c \in (a,b)$ si $f'(c)=0$ i $f$ és convexa a $(a,c)$ i còncava a $(a,c)$ (o a l'inrevés).
\end{itemize}

%----------------------------------------------------------------------------------------
\subsection{Regles de l'Hôpital}
Amb aquest nom s'apleguen diversos teoremes que són de gran utilitat per al càlcul de límits de funcions. Les regles de l'Hôpital permeten resoldre moltes de les indeterminacions $0/0$ i $\infty / \infty$.

\subsubsection*{Els casos $0/0$ i $\infty / \infty$}
Si el límit d'una funció és una indeterminació d'aquest tipus, es pot aplicar l'Hôpital un nombre finit de vegades, $n$, fins desfer la indeterminació (només si és possible). El valor obtingut després de $n$ reiteracions és el valor del límit de la funció inicial:
\begin{align}
    \lim\limits_{x \to a} \frac{f(x)}{g(x)} \text{ és indeterminat, però }\lim\limits_{x \to a} \frac{f^{n}(x)}{g^{n}(x)}=l \Rightarrow \lim\limits_{x \to a} \frac{f(x)}{g(x)}=l
\end{align}
Pel que fa a la resta d'indeterminacions, es poden reduir als casos $0/0$ o $\infty / \infty$.

\subsubsection*{El cas $0 \cdot \infty$}
$\displaystyle \lim\limits_{x \to a} f(x)g(x)$ és indeterminat, però
\begin{align}
    \lim\limits_{x \to a} \frac{f(x)}{1/g(x)} = \frac{\infty}{\infty} \text{ o } \frac{0}{0} 
\end{align}

\subsubsection*{El cas $\infty - \infty$}
$\displaystyle \lim\limits_{x \to a} f(x) - g(x)$ és indeterminat, però
\begin{align}
    \lim\limits_{x \to a} \frac{1/f(x) - 1/g(x)}{1/f(x)g(x)} = \frac{\infty}{\infty} \text{ o } \frac{0}{0}
\end{align}

\subsubsection*{Els casos $0^{0}$, $\infty^{0}$ i $1^{\infty}$}
$\displaystyle \lim\limits_{x \to a} f(x)^{g(x)}$ és indeterminat, però
\begin{align}
    \lim\limits_{x \to a} e^{g(x) \ln f(x)} \text{ i }\lim\limits_{x \to a} g(x) \ln f(x) = 0 \cdot \infty
\end{align}

%----------------------------------------------------------------------------------------
\subsection{Fórmula de Taylor}
\subsubsection*{Polinomi de Taylor de grau $n$}
Suposem que $f$ és derivable $n$ vegades en el punt $a$, busquem ara el polinomi $P_{n}(x)$, de grau $n$, que millor s'ajusti a $f$ en aquest punt.
\begin{align}
    f(x) \approx P_{n}^{(a)}(x) = \sum\limits_{k=0}^{n} \frac{f^{(k)}(a)}{k!} (x-a)^{k}
\end{align}
Si $a=0$, parlem del polinomi de Maclaurin.

\subsubsection*{Fórmula de Taylor}
\begin{align}
    f(x) = P_{n}^{(a)}(x) + R_{n}(x,a)
\end{align}

\subsubsection*{Resta de Lagrange}
Es pot avaluar l'error màxim que es comet quan es fa el polinomi de Taylor avaluant $R_{n}(x,a)$ en funció de $c$.
\begin{align}
    R_{n}(x,a) = \frac{f^{(n+1)}(c)}{(n+1)!}(x-a)^{n+1}
\end{align}

%----------------------------------------------------------------------------------------
\subsection{Mètode Newton-Raphson}
Permet estimar la solució d'una equació d'una variable real amb un alt grau de precisió.

Sigui $f(x)$ una funció derivable que té una solució en $x=a$ i sigui $x_{1}$ una estimació arbitrària de $a$. La tangent a la gràfica en $(x_{1}, f(x_{1}))$ té per equació $y= f(x) + f'(x_{1})(x-x_{1})$ i talla l'eix $x$ en $0=f(x_{1})+f'(x_{1})(x_{2}-x_{1}) \Rightarrow x_{2}= x_{1} - \frac{f(x_{1})}{f'(x_{1})} \Rightarrow$ 
\begin{align}
    x_{n+1}= x_{n} - \frac{f(x_{n})}{f'(x_{n})}
\end{align}
Sota hipòtesis adequades, si es fa una reiteració del procés, la successió $\{ x_{1}, x_{2}, \dots , x_{n}, \dots \}$ convergeix ràpidament a $a$.

%----------------------------------------------------------------------------------------
\subsection{Representació gràfica d'una funció}
Propietats d'una funció que cal tenir en compte  a l'hora de representar-la gràficament:
\begin{enumerate}[i)]
    \item Domini d'existència de la funció.
    \item Simetries i periodicitat.
    \item Punts de tall amb els eixos.
    \item Creixement i decreixement.
    \item Màxims, mínims i punts d'inflexió.
    \item Concavitat i convexitat.
    \item Asímptotes verticals.
    \item Asímptotes oblíqües.
    \item Asímptotes horitzontals.
    \item Comportament parabòlic.
\end{enumerate}

% \part*{Càlcul II}
		%----------------------------------------------------------------------------------------
%    INTEGRAL DE RIEMANN
%----------------------------------------------------------------------------------------
\section{Integral de Riemann}
\subsection{El problema de l'àrea}
Sigui $f$ una funció acotada en $[a , b]$. Quant val l'àrea sota el gràfic entre els punts $a$ i $b$?

\subsubsection*{Particions}
Una partició de $\Pi$ de $f$ és:
\begin{align}
\begin{gathered}
    \Pi \equiv \{ x_{0} , x_{1} , \dots , x_{n-1} , x_{n} \}, \\
    a = x_{0} < x_{1} < \dots < x_{n-1} < x_{n} = b
\end{gathered}
\end{align}

Així doncs, la partició $\Pi$ descompon l'interval $[a , b]$ en $n$ intervals $I_{i}$, on $I_{i} = [x_{i-1} , x_{i}]$. Denotarem la seva llargada com a $\Delta x_{i}$:
\begin{align}
    \Delta x_{i} = (x_{i} - x_{i-1}) \Rightarrow \sum\limits_{i=1}^{n} \Delta x_{i} = b - a
\end{align}

\subsubsection*{Sumes superiors i inferiors}
Com que $f$ és fitada en $[a , b]$, també ho és en cadascun dels seus intervals $I_{i}$. Així doncs, hi haurà un ínfim i un suprem per a cada interval:
\begin{align}
    M_{i} = \sup \{ f(x) \mid x \in I_{i} \}, \qquad m_{i} = \inf \{ f(x) \mid x \in I_{i} \}
\end{align}

Definim la suma superior i la suma inferior de $f$ associades a la partició $\Pi$ com:
\begin{align}
    S(f , \Pi) \equiv \sum\limits_{i=1}^{n} M_{i} \Delta x_{i}, \qquad s(f , \Pi) \equiv \sum\limits_{i=1}^{n} m_{i} \Delta x_{i} \\
    m_{i} \leq M_{i} \Rightarrow s(f , \Pi) \leq S(f , \Pi), \quad \forall \Pi
\end{align}
Direm que $\Pi '$ és més fina que $\Pi$ si $\Pi '$ s'obté afegint punts a $\Pi$. Llavors escriurem $\Pi ' > \Pi$. Així doncs, es compleix:
\begin{itemize}
    \item $S(f , \Pi ') \leq S(f , \Pi)$.
    \item $s(f , \Pi ') \geq s(f , \Pi)$.
    \item $\Pi_{1} , \Pi_{2} \Rightarrow s(f , \Pi_{1}) \leq S(f , \Pi_{2})$.
\end{itemize}

\subsubsection*{Integral de Riemann}
Siguin $s(f) \equiv \{ \text{sumes inferiors} \}$ i $S(f) \equiv \{ \text{sumes superiors} \}$.
\begin{align}
\begin{gathered}
    \underline{\int_{a}^{b}} f \equiv \sup (s(f)) \equiv \text{Integral inferior de Riemann} \\ 
    \overline{\int_{a}^{b}} f \equiv \inf (S(f)) \equiv \text{Integral superior de Riemann}
\end{gathered}
\end{align}
En general,
\begin{align}
    \underline{\int_{a}^{b}} f \leq \overline{\int_{a}^{b}} f
\end{align}

%----------------------------------------------------------------------------------------
\subsection{Integrabilitat d'una funció}
\subsubsection*{1r criteri d'integrabilitat}
Diem que $f$ és integrable (de Riemann) si es compleix que:
\begin{align}
    \underline{\int_{a}^{b}} f = \overline{\int_{a}^{b}} f \Leftrightarrow \forall \varepsilon > 0, \quad \exists \Pi \text{ de } [a , b] \text{ tal que } S(f , \Pi) - s(f , \Pi) < \varepsilon
\end{align}
\subsubsection*{Condicions per a la integrabilitat}
\begin{itemize}
    \item Si $f$ és contínua a $[a , b] \Rightarrow f$ és integrable a $[a , b]$.
    \item Si $f$ és monòtona a $[a , b] \Rightarrow f$ és integrable a $[a , b]$.
    \item Si $f$ és integrable a $[a , b] , f([a , b]) \subseteq [c , d]$, i $g$ és contínua a $[c , d] \Rightarrow g \circ f$ és integrable a $[a , b]$.
    \item Si $f$ és integrable a $[a , b]$, llavors també ho és en qualsevol subinterval $[c , d] \subseteq [a , b]$.
\end{itemize}

%----------------------------------------------------------------------------------------
\subsection{Integral com a límit de sumes de Riemann}
Sigui $f$ una funció definida i fitada a $[a , b]$, sigui $\Pi$ una partició de $[a , b]$ en $n$ intervals $I_{i}$ i siguin $z_{i} \in I_{i}$ una col·lecció de punts. Anomenem suma de Riemann de $f$, associada a la partició $\Pi$ i als punts $z_{i}$ a la quantitat:
\begin{align}
    \sum\limits_{i=1}^{n} f(z_{i}) \Delta x_{i}
\end{align}
Evidentment, sempre es compleix que $S(f , \Pi) \leq \sum\limits_{i=1}^{n} f(z_{i}) \Delta x_{i} \leq s(f , \Pi)$.
\subsubsection*{2n criteri d'integrabilitat}
Diem que $f$ és integrable (de Riemann) si es compleix que:
\begin{align}
    \forall \varepsilon > 0, \exists \Pi \text{ tal que } S(f , \Pi) - s(f , \Pi) < \varepsilon \Leftrightarrow \exists \lim_{\Pi} \sum\limits_{i=1}^{n} f(z_{i}) \Delta x_{i} = \int_{a}^{b} f
\end{align}

%----------------------------------------------------------------------------------------
\subsection{Propietats de la integral}
Si les funcions $f$ i $g$ són integrables en $[a , b]$, llavors:
\begin{itemize}
    \item Linealitat de la integral:
        \subitem La funció $f + g$ és integrable en $[a , b]$ i $\int_{a}^{b} (f + g) = \int_{a}^{b} f + \int_{a}^{b} g$.
        \subitem Si $k \in \mathbb{R} \Rightarrow$ la funció $k f$ és integrable en $[a , b]$ i $\int_{a}^{b} (k f) = k \int_{a}^{b} f$.
    \item Integrabilitat del producte i del quocient:
        \subitem $fg$ és integrable a $[a , b]$.
        \subitem Si $g(x) \geq k > 0 \Rightarrow \frac{f}{g}$ és integrable a $[a , b]$.
    \item Àrea sota un punt:
        \subitem Si $f$ i $g$ són integrables a $[a , b]$ i $f(x) = g(x)$ en tot l'interval $[a , b]$ excepte en un nombre finit de punts $\Rightarrow \int_{a}^{b} f = \int_{a}^{b} g$.
    \item Additivitat dels intervals d'integració:
        \subitem Si $c \in (a , b) \Rightarrow f$ és integrable en $[a , c]$ i en $[c , b]$, i es té $\int_{a}^{b} f = \int_{a}^{c} f + \int_{c}^{b} f$.
    \item Desigualtats:
        \subitem Si $f(x) \leq g(x) \, \forall x \in [a , b] \Rightarrow \int_{a}^{b} f \leq \int_{a}^{b} g$.
        \subitem La funció $|f|$ és integrable en $[a , b]$ i $|\int_{a}^{b} f| \leq \int_{a}^{b} |f|$.
    \item Valor mitjà d'una funcio integrable:
        \subitem $\left< f \right>_{[a , b]} \equiv \frac{1}{b-a} \int_{a}^{b} f$.
        \subitem Teorema del valor mitjà: $\exists c \in [a , b]$ tal que $f(c) = \left< f \right>_{[a , b]}$.
\end{itemize}
En tot el desenvolupament de la teoria de la integral de Riemann hem suposat que $a < b$. Es pot estendre el concepte d'integral al cas en què el límit inferior d'integració sigui més gran que el superior. Només cal definir:
\begin{align}
    \int_{b}^{a} f \equiv - \int_{a}^{b} f
\end{align}
Tret de les desigualtats, les propietats anteriors segueixen sent vàlides quan el límit inferior és més gran que el superior.
    
%----------------------------------------------------------------------------------------
\subsection{Integració i derivació}
Suposem que $f$ és integrable a $[a , b]$, $\forall x \in [a , b]$ definim la funció àrea:
\begin{align}
    S(x) \equiv \int_{a}^{x} f
\end{align}
\begin{itemize}
    \item $S$ és uniformement contínua a $[a , b]$.
    \item En els punts de $[a , b]$ on $f$ és contínua, la funció $S$ és derivable i la seva derivada coincideix amb $f$.
\end{itemize}

%----------------------------------------------------------------------------------------
\subsection{Teorema fonamental del càlcul}
Si $f$ és integrable a $[a , b]$ i $F$ és una primitiva de $f$:
\begin{align}
    \int_{a}^{b} f = F(b) - F(a) \equiv \left. F \right|_{a}^{b}
\end{align}

%----------------------------------------------------------------------------------------
\subsection{Notacions per a la integració i la derivació}
\subsubsection*{Notació de Lagrange}
Una de les notacions més modernes per a la derivació de funcions és la de Lagrange, en què utilitza el símbol \textit{prima} per a indicar la derivada. Sigui $f$ una funció, llavors, tenim:
\begin{align}
    f' , f'', f''' , f^{(4)} \dots , f^{(n)}
\end{align}

\subsubsection*{Notació de Liebniz}
La notació original emprada per Gottfried Leibniz es fa servir en matemàtiques. És particularment comú quan l'equació $y = f(x)$ és utilitzada per a referir-se a la relació funcional entre les variables dependents i independents $y$ i $x$. En aquest cas la derivada es pot escriure com a:
\begin{align}
    \frac{\dif y}{\dif x} \equiv \frac{\dif}{\dif x}(f(x))
\end{align}
En general, la derivada $n$-èsima de $f(x)$ s'escriu:
\begin{align}
    \frac{\dif^{n}y}{\dif x^{n}} \equiv \frac{\dif^{n}}{\dif x^{n}}(f(x))
\end{align}
Per a la integral, fa servir la següent notació:
\begin{align}
    \int_{a}^{b} f(x) \diff x \equiv \int_{a}^{b} f
\end{align}

\subsubsection*{Notació de Newton}
Sir Isaac Newton fa servir el concepte de \textit{fluent} per a la seva notació de integrals i derivades:
\begin{align*}
    x^{'''} , \; x^{''} , \; x^{'} , \; x , \; \ddot{x} , \; \ddot{x} , \; \dddot{x}
\end{align*}
Per passar d'equacions de l'esquerra a equacions de la dreta, Newton aplica \textit{fluxions} (i.e., deriva). En general, quan Newton fa fluxions, deriva respecte de $t$. Sigui $x = f(t)$, llavors, tenim:
\begin{align}
    \dot{x} \equiv \frac{\dif x} {\dif t} , \quad x^{'} \equiv \int x \diff t
\end{align}
En l'actualitat aquesta notació s'utilitza principalment en mecànica i altres àrees de la física per indicar la derivació respecte el temps. 

%----------------------------------------------------------------------------------------
\subsection{Expressió integral de la resta de Taylor}
Si $f^{(n+1)} (t)$ és integrabl entre $a$ i $x$:
\begin{align}
    f(x) = \sum\limits_{k=0}^{n} \frac{f^{k}(a)}{k!} (x-a)^{k} + R_{n} (x,a), \\
    \text{on } \quad R_{n} (x,a) = \int_{a}^{x} \frac{f^{n+1}(t)}{n!} (x-t)^{n} \diff t
\end{align}

\subsubsection*{Fitar $R_{n} (x,a)$}
\begin{example}
    $f(x) = e^{-x} \Rightarrow f^{(n)}(x) = (-1)^{n} e^{-x}$; $f^{(n)}(0) = (-1)^{n}$
    
    $\Rightarrow e^{-x} = 1 - \frac{1}{1!}x + \frac{1}{2!}x^{2} + \dots + \frac{(-1)^{n}}{n!}x^{n}+ \frac{1}{n!} \int_{0}^{x} (x-t)^{n} (-1)^{(n+1)}(t) \diff t$
    
    $\Rightarrow |R_{n}| = \frac{1}{n!} \int_{0}^{x} (x-t)^{n} e^{-x} \diff t \leq \frac{1}{n!} \int_{0}^{x} (x-t)^{n} \diff t = \left. \frac{-1}{n!} \frac{(x-t)^{n+1}}{n+1} \right|_{0}^{x}$
    
    $\Rightarrow |R_{n}| = \frac{x^{n+1}}{(n+1)!}$
\end{example}

%----------------------------------------------------------------------------------------
\subsection{Aplicacions de la integral}
Sigui $f(x)$ una funció contínua a $[a , b]$.

\subsubsection*{Longitud d'un arc de corba}
\begin{align}
    L = \int_{a}^{b} \sqrt{1 + f'(x)^{2}} \diff x
\end{align}

\subsubsection*{Superfície lateral d'un cos de revolució}
\begin{align}
    S_{L} = 2 \pi \int_{a}^{b} f(x) \sqrt{1 + f'(x)^{2}}  \diff x
\end{align}

\subsubsection*{Volum d'un cos de revolució}
\begin{align}
    V = \pi \int_{a}^{b} f(x)^{2} \diff x
\end{align}
		%----------------------------------------------------------------------------------------
%    MÈTODES D'INTEGRACIÓ
%----------------------------------------------------------------------------------------
\section{Mètodes d'integració}
\subsection{Primitives immediates}
\begin{table}[H]
    \centering
    \begin{tabular}{ll}
        \toprule
        \toprule
        Derivades & Primitives \\
        \midrule
        $\displaystyle x^{n} \to n x^{n-1} $ & $\displaystyle  \int x^{n} \diff x = \frac{x^{n+1}}{n+1} + C , \quad (n \neq -1) $ \\
        $\displaystyle \ln x \to \frac{1}{x} $ & $\displaystyle  \int \frac{1}{x} \diff x = \ln x + C $ \\
        $\displaystyle e^{x} \to e^{x} $ & $\displaystyle  \int e^{x} \diff x = e^{x} + C $ \\
        $\displaystyle \sin x \to \cos x $ & $\displaystyle  \int \cos x \diff x = \sin x + C $ \\
        $\displaystyle \cos x \to - \sin x $ & $\displaystyle  \int \sin x \diff x = - \cos x + C $ \\
        $\displaystyle \tan x \to \frac{1}{\cos^{2} x} $ & $\displaystyle  \int \frac{1}{\cos^{2} x} \diff x = \tan x + C $ \\
        $\displaystyle \arcsin x \to \frac{1}{\sqrt{1 - x^{2}}} $ & $\displaystyle  \int \frac{1}{\sqrt{1 - x^{2}}} \diff x = \arcsin x + C $ \\
        $\displaystyle \arccos x \to - \frac{1}{\sqrt{1 - x^{2}}} $ & $\displaystyle  \int - \frac{1}{\sqrt{1 - x^{2}}} \diff x = \arccos x + C $ \\
        $\displaystyle \arctan x \to \frac{1}{1 + x^{2}} $ & $\displaystyle  \int \frac{1}{1 + x^{2}} \diff x = \arctan x + C $ \\
        $\displaystyle \operatorname{arcsinh} x \to \frac{1}{\sqrt{1 + x^{2}}} $ & $\displaystyle  \int \frac{1}{\sqrt{1 + x^{2}}} \diff x = \operatorname{arcsinh} x + C $ \\
        $\displaystyle \operatorname{arccosh} x \to - \frac{1}{\sqrt{- 1 + x^{2}}} $ & $\displaystyle  \int - \frac{1}{\sqrt{- 1 + x^{2}}} \diff x = \operatorname{arccosh} x + C $ \\
        $\displaystyle \operatorname{arctanh} x \to \frac{1}{1 - x^{2}} $ & $\displaystyle  \int \frac{1}{1 - x^{2}} \diff x = \operatorname{arctanh} x + C $ \\
    \bottomrule
    \end{tabular}
\end{table}

%----------------------------------------------------------------------------------------
\subsection{Canvi de variable o substitució}
És una conseqüència de la regla de la cadena de la derivació:
\begin{align}
    \int_{a}^{b} f(g(x)) g'(x) \diff x = \int_{g(a)}^{g(b)} f(u) \diff u
\end{align}
Aquest mètode pot simplificar el càlcul d'una integral, a priori, no trivial.
\begin{example}
    $\displaystyle \int_{1}^{e} \left[ \ln (x) \right]^{2} \frac{1}{x} \diff x = \int_{0}^{1} u^{2} \diff u = \left. \frac{u^{3}}{3}\right|_{0}^{1} = \frac{1}{3}$

    Es fan els canvis $\begin{cases} u = \ln (x) \\ \dif u = \frac{1}{x} \diff x \end{cases}$
\end{example}

%----------------------------------------------------------------------------------------
\subsection{Integració per parts}
És una conseqüència de la regla del producte de funcions de la derivació:
\begin{align}
    \int_{a}^{b} u \diff v = \left. uv \right|_{a}^{b} - \int_{a}^{b} v \diff u
\end{align}

\subsubsection*{Mètode LIATE}
Aquesta regla proporciona un mètode mnemotècnic que facilita l'elecció de $u$ i $\dif v$.
\begin{itemize}
    \item L: Logarithmic functions.
    \item I: Inverse trigonometric functions.
    \item A: Algebraic functions.
    \item T: Trigonometric functions.
    \item E: Exponential functions.
\end{itemize}
Donada una funció qualsevol, $u$ serà aquella del tipus que es trobi primer a la llista.
\begin{example}
    $\displaystyle \int x e^{x} \diff x = x e^{x} - \int e^{x} \diff x = e^{x} (x - 1) + C$

    Es fan els canvis $\begin{cases} u = x & \quad \dif u = \dif x\\ \dif v = e^{x} \diff x & \quad v = e^{x} \end{cases}$
\end{example}

%----------------------------------------------------------------------------------------
\subsection{Polinomis trigonomètrics}
Per calcular integrals del tipus $\int \cos^{n} x \, \cos^{m} x \diff x$, distingim tres casos:
\begin{enumerate}[i)]
    \item $n = 1$ (o $m = 1$). Es resol amb el canvi $u = \cos x$ (o $u = \sin x$).
    \item $n$ o $m$ senar. Utilitzant la identitat $\sin^{2} x + \cos^{2} x = 1$ es redueix al cas anterior.
    \item $n$ i $m$ parells. Es redueix als casos anteriors aplicant les fórmules de l'angle meitat, $\cos^{2} x = (1 + \cos 2x)/2$ i $\sin^{2} x = (1 - \cos 2x)/2$.
\end{enumerate}

%----------------------------------------------------------------------------------------
\subsection{Funcions racionals}


\subsubsection*{Completar el quadrat}
\begin{align}
    a x^{2} + b x + c = a ((x + \alpha)^{2} + \beta )
\end{align}

%----------------------------------------------------------------------------------------
\subsection{Funcions racionals d'exponencials}
Integrals de funcions del tipus $R(e^{x})$.

Es poden reduir a integrals de funcions racionals amb el canvi de variable:
\begin{align}
    u = e^{x}, \quad \dif x = \frac{\dif u}{u}
\end{align}

%----------------------------------------------------------------------------------------
\subsection{Integrals trigonomètriques}
Inetgrals de funcions del tipus $R(\sin x , \cos x)$.
\begin{enumerate}[i)]
    \item Cas universal (canvi de Weirstrass).
        \begin{align}
            u = \tan \frac{x}{2} , \quad \dif x = \frac{2 \diff u}{1 + u^{2}} , \quad \sin x = \frac{2u}{1 + u^{2}} , \quad \cos x = \frac{1 - u^{2}}{1 + u^{2}}
        \end{align}
    \item $R$ senar en $\sin x$, $R(- \sin x , \cos x) = -R(\sin x , \cos x)$.
        \begin{align}
            u = \cos x , \quad \dif u = - \sin x \diff x, \quad \sin^{2} x = 1 - u^{2}
        \end{align}
    \item $R$ senar en $\cos x$, $R(\sin x , - \cos x) = -R(\sin x , \cos x)$.
        \begin{align}
            u = \sin x , \quad \dif u = \cos x \diff x, \quad \cos^{2} x = 1 - u^{2}
        \end{align}
    \item $R$ senar en $\sin x i \cos x$, $R(- \sin x , - \cos x) = R(\sin x , \cos x)$.
        \begin{align}
            u = \tan x , \quad \dif x = \frac{\dif u}{1 + u^{2}} , \quad \sin x = \frac{u}{\sqrt{1 + u^{2}}} , \quad \cos x = \frac{1}{\sqrt{1 + u^{2}}}
        \end{align}
\end{enumerate}

%----------------------------------------------------------------------------------------
\subsection{Funcions amb potències fraccionàries}
Integrals de potències del tipus $\displaystyle \left( \frac{ax + b}{cx + d} \right)^{r_{i}}, \; \; r_{1} , \dots , r_{n} \in \mathbb{Q}$.

Es poden reduir a integrals de funcions racionals amb el canvi de variable:
\begin{align}
    u^{m} = \frac{ax + b}{cx + d}, \quad m \equiv \text{ mínim comú denominador de $r_{i}$}.
\end{align}

%----------------------------------------------------------------------------------------
\subsection{Radicals d'expressions quadràtiques}
Integrals de funcions del tipus $R(x , \sqrt{ax^{2} + bx + c})$.

Després d'eliminar el terme lineal completant el quadrat, el radical es redueix a un dels casos:
\begin{enumerate}[i)]
    \item $\sqrt{k^{2}-x^{2}}$. Canvi: $x = k \sin t$.
    \item $\sqrt{k^{2}+x^{2}}$. Canvi: $x = k \tan t$.
    \item $\sqrt{x^{2}-k^{2}}$. Canvi: $x = k \frac{1}{\cos t}$.
\end{enumerate}

%----------------------------------------------------------------------------------------
\subsection{Identitats trigonomètriques útils}
\begin{align}
\begin{split}
    \sin ^{2} \theta + \cos ^{2} \theta = 1 \\
    \tan ^{2} \theta + 1 = \sec ^2 \theta
\end{split}
\end{align}
\begin{align}
\begin{split}
    \sin (\alpha \pm \beta) &= \sin \alpha \cos \beta \pm \cos \alpha \sin \beta \\
    \cos (\alpha \pm \beta) &= \cos \alpha \cos \beta \mp \sin \alpha \sin \beta
\end{split}
\end{align}
\begin{align}
\begin{split}
    \sin^{2} \theta &= \frac{1 - \cos 2 \theta}{2} \\
    \cos^{2} \theta &= \frac{1 + \cos 2 \theta}{2}
\end{split}
\end{align}

		%----------------------------------------------------------------------------------------
%    INTEGRALS IMPRÒPIES
%----------------------------------------------------------------------------------------
\section{Integrals impròpies}
\subsection{Restriccions de la integral de Riemann}
La teoria de la integral de Riemann només és aplicable a funcions definides i fitades en un interval tancat finit $[a, b]$. La integral impròpia estén la teoria d'integració a funcions que no compleixen aquests requisits, com ara $\int_{0}^{+\infty} xe^{-x} \diff x$ o $\int_{0}^{1} \frac{1}{\sqrt{x}} \diff x$.

%----------------------------------------------------------------------------------------
\subsection{Integral impròpia d'una funció localment integrable}
\subsubsection*{Funcions localment integrables}
Direm que $f$ és una funció localment integrable a $[a, b)$ si és integrable en qualsevol subinterval finit de $[a, b)$, és a dir, si $\forall z \in (a,b), \, \exists \int_{a}^{z}$.

Similarment, direm que $f$ és una funció localment integrable a $(a, b]$ si és integrable en qualsevol subinterval finit de $(a, b]$, és a dir, si $\forall z \in (a,b), \, \exists \int_{z}^{b}$.

\subsubsection*{Integrals impròpies}
Si $f$ és localment integrable a $[a, b)$, definim la integral impròpia de $f$ en aquest interval com:
\begin{align}
    \int_{a}^{\triangleright b} f(x) \diff x \equiv \lim_{z \to b^{-}} \int_{a}^{z} f(x) \diff x
\end{align}
Similarment, es defineix la integral impròpia quan $f$ és localment integrable en $(a, b]$:
\begin{align}
    \int_{a \triangleleft}^{b} f(x) \diff x \equiv \lim_{z \to a^{+}} \int_{z}^{b} f(x) \diff x
\end{align}

\subsubsection*{Linealitat de les integrals impròpies}
Si $f$ i $g$ són integrables a $[a, b)$ i $\lambda, \mu \in \mathbb{R}$, es compleix:
\begin{align}
    \int_{a}^{\triangleright b} [ \lambda f(x) + \mu g(x) ] \diff x = \lambda \int_{a}^{\triangleright b} f(x) \diff x + \mu \int_{a}^{\triangleright b} g(x) \diff x
\end{align}
Les integrals impròpies del tipus $\int_{a \triangleleft}^{b}$ tenen propietats de linealitat similars.

\subsubsection*{Integrals doblement impròpies}
Quan $f$ és localment integrable a $(a, b)$, tenim una integral doblement impròpia. En aquest cas es defineix:
\begin{align}
    \int_{a \triangleleft}^{\triangleright b} f(x) \diff x = \int_{a \triangleleft}^{c} f(x) \diff x + \int_{c}^{\triangleright b} f(x) \diff x , \quad \forall c \in (a, b)
\end{align}
En particular, estudiarem el cas $f(x) = 1/x^{p}$, que serà de molta utilitat a l'hora de comparar funcions:
\begin{align}
    \int_{0 \triangleleft}^{1} \frac{dx}{x^{p}} 
    \begin{cases} \text{convergeix} & \text{si } p < 1 \\ \text{divergeix} & \text{si } p \geq 1 \end{cases} \\
    \int_{1}^{\triangleright \infty} \frac{dx}{x^{p}} 
    \begin{cases} \text{convergeix} & \text{si } p > 1 \\ \text{divergeix} & \text{si } p \leq 1 \end{cases}
\end{align}

\subsubsection*{Integrabilitat absoluta}
Si $f$ és localment integrable a $[a, b)$, direm que és absolutament impròpia en aquest interval si $\int_{a}^{\triangleright b} |f (x)| \diff x$ és convergent. Llavors:
\begin{align}
    \int_{a}^{\triangleright b} |f(x)| \diff x \text{ convergent} \Rightarrow \int_{a}^{\triangleright b} f(x) \diff x \text{ convergent.}
\end{align}

%----------------------------------------------------------------------------------------
\subsection{Integrals impròpies de funcions no negatives}
Si $f$ és localment integrable i no negativa a $[a, b)$, i $S(z) = \int_{a}^{z} f(x) \diff x$, llavors:
\begin{align}
    \int_{a}^{\triangleright b}f(x) \diff x \text{ convergeix} \Leftrightarrow S(z) \text{ és fitada superiorment a } [a, b)
\end{align}
Cal notar que aquesta condició també és vàlida en el cas en què $f$ estigui definida a $(a, b]$.

\subsubsection*{Criteri de comparació}
Si $f$ i $g$ són localment integrables i no negatives a $[a ,b)$, i $\lambda \in \mathbb{R}$ tal que, en algun entorn de $b$ es compleix que $f(x) < \lambda g(x)$, llavors:
\begin{align}
    \int_{a}^{\triangleright b} g(x) \diff x \text{ convergent} \Rightarrow \int_{a}^{\triangleright b} f(x) \diff x \text{ convergent}.
\end{align}
(i consegüentment, $\int_{a}^{\triangleright b} f$ divergent $\Rightarrow$ $\int_{a}^{\triangleright b} g$ divergent).

La següent condició és una conseqüència directa del criteri de comparació:

Si $f$ i $g$ són localment integrables i no negatives a $[a ,b)$, i $A = \lim_{x \to b^{-}} \frac{f(x)}{g(x)}$, llavors:
\begin{enumerate}[i)]
    \item Si $A \neq \infty$: $\int_{a}^{\triangleright b} g$ convergent $\Rightarrow \int_{a}^{\triangleright b} f$ convergent.
    \item Si $A \neq 0$: $\int_{a}^{\triangleright b} f$ convergent $\Rightarrow \int_{a}^{\triangleright b} g$ convergent.
\end{enumerate}
En conseqüència, si $A \neq 0, \infty$, les integrals $\int_{a}^{\triangleright b} f$ i $\int_{a}^{\triangleright b} g$ són ambdues convergents o divergents.

Aquests criteris també es compleixen quan l'interval és $(a, b]$.
%----------------------------------------------------------------------------------------
\subsection{La funció $\Gamma$ d'Euler}
Per $x > 0$ es defineix
\begin{align}
    \Gamma (x) \equiv \int_{0}^{\infty} t^{x-1} e^{-t} \diff t
\end{align}
Es tracta d'una integral impròpia definida a l'interval $(0 , \infty)$ que convergeix $\forall x > 0$ i, per tant, $\Gamma (x)$ està definida a $(0 , \infty)$.

\subsubsection*{Fórmula de recurrència: extensió a $x<0$}
\begin{align}
    \Gamma (x + 1) = \int_{0}^{\infty} t^{x} e^{-t} \diff t = \left. \left[ -t e^{-t} \right] \right|_{0}^{\infty} + x \int_{0}^{\infty} t^{x-1} e^{-t} \diff t = x \Gamma (x)
\end{align}
Hem trobat, doncs, una fórmula de recurrència que relaciona els valors de $\Gamma$ en dos punts que distin $1$ entre ells:
\begin{align}
    \Gamma (x) = \frac{\Gamma (x+1)}{x}
\end{align}
D'aquesta manera tenim definida $\Gamma (x)$ a tota la recta real excepte ens els enters no positius. Notem, però, que per a $x < 0$ la funció $\Gamma (x)$ no ve donada per la integral $\int_{0}^{\infty} t^{x-1} e^{-t} \diff t$, ja que aquesta integral és divergent quan $x < 0$.

\subsubsection*{Funció factorial}
Quan $x = n \in \mathbb{N}$ la fórmula de recurrència aplicada $n$ vegades ens dóna:
\begin{align*}
    \Gamma (n+1) = n \Gamma (n) = n(n-1) \Gamma (n-1) = \dots = n! \Gamma (1)
\end{align*}
D'altra banda, $\Gamma (1) = 1$. Per tant, 
\begin{align}
    \Gamma (n+1) = n!
\end{align}
Es pot utilitzar aquesta relació com a definició del factorial de qualsevol número real que no sigui un enter negatiu, és a dir,
\begin{align}
    x! \equiv \Gamma (x+1) , \quad (x \neq -1, -2, -3, \dots)
\end{align}

\subsubsection*{Fórmula de Stirling}
Aquesta relació és molt útil per al càlcul aproximat del factorial de números grans:
\begin{align}
    n! \approx n^{n} e^{-n} \sqrt{2 \pi n}
\end{align}
		%----------------------------------------------------------------------------------------
%    SÈRIES NUMÈRIQUES
%----------------------------------------------------------------------------------------
\section{Sèries numèriques}
\subsection{Sèries de números reals}
Sigui $\{a_{n}\}$ una successió de números reals. Considerem la successió de sumes parcials $\{S_{n}\}$, on $S_{n} = \sum_{k=1}^{n} a_{k}$, llavors definim la suma de la sèrie $\sum_{k=1}^{\infty} a_{k}$ com el límit (si existeix):
\begin{align}
    \sum\limits_{k=1}^{\infty} a_{k} \equiv \lim \{S_{n}\} \equiv \lim_{n \to \infty} \left( \sum\limits_{k=1}^{n} a_{k} \right)
\end{align}
En altres paraules, el significat de «suma infinita» és, per definició,
\begin{align}
    \sum\limits_{k=1}^{\infty} \equiv \lim_{n \to \infty} \sum\limits_{k=1}^{n} 
\end{align}
Si el límit és finit direm que la sèrie és sumable o convergent. Si el límit és $\pm \infty$ direm que és divergent, i si el límit no existeix direm que la sèrie no és sumable.

\subsubsection*{Linealitat de les sèries convergents}
Si $\sum_{k=1}^{\infty} a_{k}$ i $\sum_{k=1}^{\infty} b_{k}$ són convergents i $\lambda, \mu \in \mathbb{R}$, es compleix:
\begin{align}
    \sum\limits_{k=1}^{\infty} (\lambda a_{k} + \mu b_{k}) = \lambda \sum\limits_{k=1}^{\infty} a_{k} + \mu \sum\limits_{k=1}^{\infty} b_{k}
\end{align}

\subsubsection*{Criteri general de convergència d'una sèrie}
Ja hem vist que $\sum_{k=1}^{\infty} a_{k}$ és convergent si i només si la successió de sumes parcials $\{S_{n}\}$ és convergent. Però això equival a dir que la successió $\{S_{n}\}$ és de Cauchy, és a dir,
\begin{align}
\begin{gathered}
    \forall \varepsilon > 0 , \quad \exists n_{0} \in \mathbb{N} \text{ tal que } | a_{n+1} + a_{n+2} + \dots + a_{n+p} | < \varepsilon , \\
    \quad \forall n > n_{0}, \forall p \geq 1
\end{gathered}
\end{align}
Aquesta és, doncs, una condició necessària i suficient per a la convergència de la sèrie.

Una conseqüència d'aquest criteri de convergència és que
\begin{align}
    \sum\limits_{k=1}^{\infty} a_{k} \text{ convergent } \Rightarrow \lim \{a_{n}\} = 0
\end{align}

%----------------------------------------------------------------------------------------
\subsection{Exemples de sèries numèriques}
\subsubsection*{Sèrie geomètrica}
\begin{align} 
    & \sum\limits_{k=0}^{n} r^{k} = 
    \begin{cases} 
        n+1 & \text{si } r = 1. \\ 
        (r^{n+1} - 1)/(r - 1) & \text{si } r \neq 1. 
    \end{cases}  \\
    & \sum\limits_{k=0}^{\infty} r^{k} = 
    \begin{cases} 
        1/(1 - r) & \text{si } |r| < 1. \\ 
        \text{divergent} & \text{si } r \geq 1. \\ 
        \text{no sumable} & \text{si } r \leq -1. 
    \end{cases} 
\end{align}

\subsubsection*{Sèrie aritmètica}
\begin{align} 
    & \sum\limits_{k=1}^{n} k = \frac{1}{2} n (n+1). \\ 
    & \sum\limits_{k=1}^{\infty} k \text{ divergeix}.
\end{align}

\subsubsection*{Sèrie harmònica}
\begin{align} 
    & \sum\limits_{k=1}^{n} \frac{1}{k} = H_{k} \equiv \int_{0}^{1} \frac{1-x^{k}}{1-x} \, dx. \\
    & \sum\limits_{k=1}^{\infty} \frac{1}{k} \text{ divergeix}.
\end{align}

\subsubsection*{Sèrie harmònica generalitzada}
\begin{align} 
    & \sum\limits_{k=1}^{n} \frac{1}{k^{p}} = H_{k}^{(p)}. \\
    & \sum\limits_{k=1}^{\infty} \frac{1}{k^{p}}
    \begin{cases} 
        \text{convergeix} & \text{si } p > 1. \\ 
        \text{divergeix} & \text{si } p \leq 1. \\
    \end{cases} 
\end{align}

%----------------------------------------------------------------------------------------
\subsection{Sèries de termes no negatius}
Un cas particular interessant de les sèries numèriques és el del cas de les sèries de termes no negatius, és a dir, que compleixen $a_{k} \geq 0$. En aquest cas hi ha una nova condició necessària i suficient per a la sumabilitat:

Si $\sum_{k=1}^{\infty} a_{k}$ és una sèrie de termes no negatius, llavors:
\begin{align}
    \sum\limits_{k=1}^{\infty} a_{k} \text{ convergeix } \Leftrightarrow \text{la successió } \{S_{n}\} \text{ és fitada superiorment.}
\end{align}

\subsubsection*{Criteri de comparació}
Si $\sum_{k=1}^{\infty} a_{k}$ i $\sum_{k=1}^{\infty} b_{k}$ són sèries de termes no negatius i $\lambda \in \mathbb{R}$ tal que, a partir d'un subíndex, es compleix que $a_{k} < \lambda b_{k}$, llavors:
\begin{align}
    \sum\limits_{k=1}^{\infty} b_{k} \text{ convergent} \Rightarrow \sum\limits_{k=1}^{\infty} a_{k} \text{ convergent}.
\end{align}
(i consegüentment, $\sum\limits_{k=1}^{\infty} a_{k}$ divergent $\Rightarrow$ $\sum\limits_{k=1}^{\infty} b_{k}$ divergent).

La següent condició és una conseqüència directa del criteri de comparació:

Si $\sum_{k=1}^{\infty} a_{k}$ i $\sum_{k=1}^{\infty} b_{k}$ són sèries de termes no negatius, i $A = \lim \frac{a_{n}}{b_{n}}$, llavors:
\begin{enumerate}[i)]
    \item Si $A \neq \infty$: $\sum\limits_{k=1}^{\infty} b_{k}$ convergent $\Rightarrow \sum\limits_{k=1}^{\infty} a_{k}$ convergent.
    \item Si $A \neq 0$: $\sum\limits_{k=1}^{\infty} a_{k}$ convergent $\Rightarrow \sum\limits_{k=1}^{\infty} b_{k}$ convergent.
\end{enumerate}

En conseqüència, si $A \neq 0, \infty$, les integrals $\sum\limits_{k=1}^{\infty} a_{k}$ i $\sum\limits_{k=1}^{\infty} b_{k}$ són ambdues convergents o divergents.

\subsubsection*{Criteri de la integral}
Sigui $n_{0} \in \mathbb{N}$, i $f(x)$ una funció localment integrable, no negativa i decreixent a l'interval $[n_{0}, + \infty)$, llavors:
\begin{align}
    \sum\limits_{k=n_{0}}^{\infty} f(x) \text{ convergent} \Leftrightarrow \int_{k=n_{0}}^{\infty} f(x) \, dx \text{ convergent}.
\end{align}

\subsubsection*{Criteri de l'arrel (o de Cauchy)}
Si $\sum_{k=1}^{\infty} a_{k}$ és una sèrie qualsevol i $\alpha = \lim \{\sqrt[n]{|a_{n}|}\}$, llavors:
\begin{itemize}
    \item $\alpha < 1$: la sèrie és convergent.
    \item $\alpha = 1$: no es pot concloure res.
    \item $\alpha > 1$: la sèrie és divergent.
\end{itemize}

\subsubsection*{Criteri del quocient (o d'Alambert)}
Si $\sum_{k=1}^{\infty} a_{k}$ és una sèrie qualsevol i $\alpha = \lim \{ \frac{|a_{n+1}|}{|a_{n}|} \}$, llavors:
\begin{itemize}
    \item $\alpha < 1$: la sèrie és convergent.
    \item $\alpha = 1$: no es pot concloure res.
    \item $\alpha > 1$: la sèrie és divergent.
\end{itemize}
Si els límits (de Cauchy i d'Alambert) existeixen, ambdós tenen el mateix valor.
%----------------------------------------------------------------------------------------
\subsection{Sèries alternades}
Són sèries del tipus
\begin{align}
    \sum\limits_{k=1}^{\infty} (-1)^{k+1} a_{k} = a_{1} - a_{2} + a_{3} - \dots 
\end{align}
o també
\begin{align}
    \sum\limits_{k=1}^{\infty} (-1)^{k} a_{k} = - a_{1} + a_{2} - a_{3} + \dots 
\end{align}

\subsubsection*{Convergència absoluta}
Direm que una sèrie $\sum_{k=1}^{\infty} a_{k}$ és absolutament convergent si la sèrie $\sum_{k=1}^{\infty} |a_{k}|$ és convergent. Com a conseqüència d'aquest criteri de convergència, tenim que:
\begin{align}
    \sum_{k=1}^{\infty} a_{k} \text{ absolutament convergent } \Rightarrow \sum_{k=1}^{\infty} a_{k} \text{ és convergent}.
\end{align}

\subsubsection*{Convergència condicional}
És possible que $\sum_{k=1}^{\infty} a_{k}$ sigui convergent, però que $\sum_{k=1}^{\infty} |a_{k}|$ sigui divergent. En aquest cas, una condició suficient per a la convergència de les sèries alternades ens ve donada pel següent teorema:

Si $\{a_{n}\}$ és una successió decreixent de termes positius, que convergeix cap a 0
\begin{align}
    \Rightarrow \text{ la sèrie alternada } \sum_{k=1}^{\infty} (-1)^{k+1} a_{k} \text{ és convergent}.
\end{align}

\subsubsection*{Propietat commutativa}
\begin{itemize}
    \item Si $\sum_{k=1}^{\infty} a_{k}$ és absolutament convergent, totes les reordenacions convergeixen a la mateixa suma.
    \item Si $\sum_{k=1}^{\infty} a_{k}$ és condicionalment convergent, és possible reordenar la sèrie, però aquesta pot convergir a qualsevol real o, fins i tot, divergir.
\end{itemize}
		%----------------------------------------------------------------------------------------
%    SUCCESSIONS I SÈRIES DE FUNCIONS
%----------------------------------------------------------------------------------------
\section{Successions i sèries de funcions}
\subsection{Successions de funcions}
Si denotem per $\mathcal{F}_{D}$ el conjunt de funcions definides en un domini $D$ de la recta real, una successió de funcions de $D$ és una aplicació del conjunt de números reals sobre el conjunt $\mathcal{F}_{D}$:
\begin{align}
    \begin{matrix}
        \mathbb{N} & \to & \mathcal{F}_{D} \\
        n & \mapsto & f_{n} (x)
    \end{matrix}
\end{align}
que denotarem $\{ f_{n} (x) \}_{D} = \{ f_{1}(x), f_{2}(x), f_{3}(x), \dots \}_{D}$.

\subsubsection*{Convergència puntual}
Una successió de funcions $\{ f_{n} (x) \}_{D}$ convergeix puntualment en el domini $D$, cap ala funció $f(x)$ si $\forall x \in D$ es complei que $\lim_{n \to \infty} \{ f_{n} (x) \}_{D} = f(x)$. Ho escriurem així:
\begin{align}
    f(x) = \lim_{\text{punt}(D)} f_{n}(x) \quad \text{o, també} \quad \{ f_{n} (x)\} \overset{\text{punt}(D)}{\longrightarrow} f(x)
\end{align}
Per tant, $\{f_{n}(x)\}$ convergeix puntualment a $f(x)$ si:
\begin{align}
    \begin{gathered}
        \forall \varepsilon > 0 \text{ i } \forall x \in D, \exists n_{0}(x,\varepsilon) \in \mathbb{N} \\
        \text{ tal que } |f_{n} (x) - f(x) | < \varepsilon, \forall n > n_{0}
    \end{gathered}
\end{align}
Exemples:
\begin{enumerate}[i)]
    \item La successió de funcions $f_{n}(x) = x/n$ convergeix puntualment, a tot $\mathbb{R}$, cap a la funció $f(x) = 0$.
    \item La successió de funcions $f_{n}(x) = x^{n}$ convergeix puntualment, a l'interval $(-1, 1]$, cap a la funció $f(x) = \begin{cases} 0 & \text{si } x \in (-1,1) \\ 1 & \text{si } x = 1 \end{cases}$. Notem que el límit no és una funció contínua.
    \item La successió de funcions $f_{n} (x) = \sin (n^{2}x)/n$ convergeix puntualment, a tot $\mathbb{R}$, a la funció $f(x) = 0$. En aquest exemple, és interessant notar que la successió de derivades, $f_{n}'(x) = n \cos (n^{2} x)$, no convergeix puntualment cap a $f'(x) = 0$.
\end{enumerate}
Com es pot veure en els exemples anteriors, la convergència puntal no transmet ni la continuitat, ni la integrabilitat, ni la derivabilitat de les funcions de la successió cap al seu límit. Introduirem ara un concepte més fort de convergència, la convergència uniforme, que sí que ho fa.

\subsubsection*{Convergència uniforme}
Si $n_{0}$ d'una funció contínua puntualment depèn únicament del valor $\varepsilon$ i no de $x$, parlarem d'una funció contínua uniformement. Ho escriurem així:
\begin{align}
    f(x) = \lim_{\text{unif}(D)} f_{n}(x) \quad \text{o, també} \quad \{ f_{n} (x)\} \overset{\text{unif}(D)}{\longrightarrow} f(x)
\end{align}
Per tant, $\{f_{n}(x)\}$ convergeix uniformement a $f(x)$ si:
\begin{align}
    \forall \varepsilon > 0 \text{ i } \forall x \in D, \exists n_{0}(\varepsilon) \in \mathbb{N} \text{ tal que } |f_{n} (x) - f(x) | < \varepsilon, \forall n > n_{0}
\end{align}
De la definició és desprèn que:
\begin{align}
    \{ f_{n} (x)\} \overset{\text{unif}(D)}{\longrightarrow} f(x) \Rightarrow \{ f_{n} (x)\} \overset{\text{punt}(D)}{\longrightarrow} f(x)
\end{align}
El recíproc no és cert, com comprovarem amb els exemples anteriors:
\begin{enumerate}[i)]
    \item La successió de funcions $f_{n}(x) = x/n$ no convergeix uniformement cap a la funció $f(x) = 0$, a $\mathbb{R}$. Sí que ho fa, en canvi, a qualsevol interval finit de $\mathbb{R}$.
    \item La successió $f_{n}(x) = x^{n}$ no convergeix uniformement, a l'interval $(-1, 1]$, cap a la funció $f(x)$. Sí que ho fa, en canvi, en qualsevol subinterval tancat de $(-1,1)$.
    \item La successió de funcions $f_{n} (x) = \sin (n^{2}x)/n$ convergeix uniformement, a tot $\mathbb{R}$, a la funció $f(x) = 0$. 
\end{enumerate}

\subsubsection*{Teoremes sobre la convergència uniforme de successions}
Els teoremes següents estableixen que la continuïtat i la integrabilitat es transmeten de les funcions cap a la funció límit quan la convergència és uniforme. La derivabilitat és més complicada; no es transmet necessàriament a la funció límit. Cal que la successió de les derivades sigui també uniformement convergent.
\begin{itemize}
    \item Teorema de la continuïtat:
        \subitem Si $f_{n}$ són contínues a $D$, i $\{ f_{n} (x) \} \overset{\text{unif}(D)}{\longrightarrow} f(x)$
        \begin{align}
            \Rightarrow f(x) \text{ és contínua a } D
        \end{align}
    \item Teorema de la integrabilitat terme a terme:
        \subitem Si $f_{n}$ són integrables a $[a, b] \subseteq D$, i $\{ f_{n} (x) \} \overset{\text{unif}(D)}{\longrightarrow} f(x)$
        \begin{align}
            \Rightarrow f(x) \text{ és integrable a } [a, b] \subseteq D
        \end{align}
        \subitem i es compleix
        \begin{align}
            \left\{ \int_{a}^{b} f_{n} (x) \diff x \right\} \to \int_{a}^{b} f (x) \diff x
        \end{align}
    \item Teorema de la derivabilitat terme a terme:
        \subitem Si $f_{n}$ són derivables a $D$, $\{ f_{n}' (x) \} \overset{\text{unif}(D)}{\longrightarrow} g(x)$, i $\{ f_{n} (x) \}$ convergeix en algun punt de $D$
        \begin{align}
            \Rightarrow \{ f_{n} (x) \} \overset{\text{unif}(D)}{\longrightarrow} f(x) \text{ (derivable)} \quad \text{i} \quad f'(x) = g(x)
        \end{align}
\end{itemize}

%----------------------------------------------------------------------------------------
\subsection{Sèries de funcions}
Com en el cas de les sèries numèriques, definirem la suma d'una sèrie de funcions definides en un domini $D$ com el límit, si existeix, de la successió de sumes parcials:
\begin{align}
    \sum\limits_{k=1}^{\infty} f_{k} (x) \equiv \lim \{ F_{n} (k) \} \quad \text{on} \quad F_{n} (x) = \sum\limits_{k=1}^{n} f_{k} (x)
\end{align}
Si $\{ F_{n} (x) \} \overset{\text{unif}(D)}{\longrightarrow} F(x)$ direm que la sèrie $\sum_{k=1}^{\infty} f_{k} (x)$ convergeix uniformement en el domini $D$, i que $F(x)$ és la seva suma en el sentit uniforme. Ho escriurem:
\begin{align}
    \sum\limits_{k=1}^{\infty} f_{k} (x) \overset{\text{unif}(D)}{=} F(x)
\end{align}
Per tant, $\{F_{n}(x)\}$ convergeix uniformement a $F(x)$ si:
\begin{align}
    \begin{gathered}
        \forall \varepsilon > 0 \text{ i } \forall x \in D, \exists n_{0}(\varepsilon) \in \mathbb{N} \\
        \text{ tal que } | \sum\limits_{k=1}^{\infty} f_{k} (x) - F(x) | < \varepsilon, \forall n > n_{0}
    \end{gathered}
\end{align}

\subsubsection*{Criteri de Weierstrass}
Si $|f_{n} (x)| \leq C_{n} \forall x \in D$, i $\sum_{n \geq 1} f_{n} C_{n}$ convergeix, llavors:
\begin{align}
    \sum\limits_{n \geq 1} f_{n} (x) \text{ convergeix uniformement en } D
\end{align}

\subsubsection*{Teoremes sobre la convergència uniforme de sèries}
\begin{itemize}
    \item Teorema de la continuitat:
        \subitem Si $f_{n}$ són contínues a $D$, i $\sum_{k = 1}^{\infty} f_{k} (x) \overset{\text{unif}(D)}{=} F(x)$
        \begin{align}
            \Rightarrow F(x) \text{ és contínua a } D
        \end{align}
    \item Teorema de la integrabilitat terme a terme:
        \subitem Si $f_{n}$ són integrables a $[a, b] \subseteq D$, i $\sum_{k = 1}^{\infty} f_{k} (x) \overset{\text{unif}(D)}{=} F(x)$
        \begin{align}
            \Rightarrow F(x) \text{ és integrable a } [a, b] \subseteq D
        \end{align}
        \subitem i es compleix
        \begin{align}
            \sum\limits_{k = 1}^{\infty} \int_{a}^{b} f_{n} (x) \diff x = \int_{a}^{b} F(x) \diff x
        \end{align}
    \item Teorema de la derivabilitat terme a terme:
        \subitem Si $f_{n}$ són derivables a $D$, $\sum_{k = 1}^{\infty} f_{k}' (x) \overset{\text{unif}(D)}{=} G(x)$, i $\sum_{k = 1}^{\infty} f_{k} (x)$ és sumable en algun punt de $D$
        \begin{align}
            \Rightarrow \sum\limits_{k = 1}^{\infty} f_{k} (x) \overset{\text{unif}(D)}{=} F(x) \text{ (derivable)} \quad \text{i} \quad F'(x) = G(x)
        \end{align}
\end{itemize}

%----------------------------------------------------------------------------------------
\subsection{Sèries de potències}
Es tracta de sèries del tipus $\displaystyle \sum\limits_{n=0}^{\infty} a_{n} (x-c)^{n}$, que anomenarem sèrie de potències al voltant del punt $c$.

Cal notar que qualsevol sèrie de potències al voltant d'un punt $c$, amb el canvi de variable $x' = x-c$ es pot convertir en una sèrie de potències al volant del punt $0$:
\begin{align}
    \sum\limits_{n=0}^{\infty} a_{n} (x-c)^{n} = \sum\limits_{n=0}^{\infty} a_{n} x'^{n}
\end{align}

\subsubsection*{Teorema de la convergència absoluta de les sèries de potències}
Si $\sum_{n=0}^{\infty} a_{n} x^{n}$ és convergent en el punt $x = x_{0}$, 
\begin{align}
    \Rightarrow \sum\limits_{n=0}^{\infty} a_{n} x^{n} \text{és absolutament convergent $\forall x$ tal que } |x| < |x_{0}|
\end{align}

\subsubsection*{Radi de convergència d'una sèrie de potències}
En una serie de potències $\sum\limits_{n=0}^{\infty} a_{n} (x-c)^{n}$, pot passar el següent:
\begin{enumerate}[i)]
    \item Convergeix a tot $x \in \mathbb{R}$.
    \item Només convergeix al punt $x = c$.
    \item $\exists R \geq 0$, anomenat radi de convergència, tal que la sèrie convergeix $\forall x \in (c-R , c+R)$.
\end{enumerate}
En realitat, tots casos es poden resumir en el tercer: al primer $R = \infty$, al segon $R = 0$, i al tercer $R$ és un escalar finit no nul.

\subsubsection*{Càlcul del radi de convergència}
Sigui $\sum\limits_{n=0}^{\infty} a_{n} x^{n}$ una sèrie de potències. Llavors, podem calcular el radi de convergència a partir del criteri de l'arrel i del criteri del quocient:
\begin{align}
    |x| < R = \left[ \lim \left\{ \sqrt[n]{|a_{n}|} \right\} \right]^{-1} = \left[ \lim \left\{ \frac{|a_{n+1}|}{|a_{n}|} \right\} \right]^{-1}
\end{align}
Ara bé, si tenim una sèrie que no depèn exactament de $x^{n}$, el radi serà lleugerament diferent (ja que la definció rigurosa de $R$ és sobre elements consecutius de la suma, no únicament dels índexs $a_{n}$).
\begin{example}
Sigui $\sum_{n}^{\infty} a_{n} x^{2n+1} = \sum_{n}^{\infty} a_{n} (x^{2})^{n} x$. Llavors, en fer $R = \lim \left\{ \frac{|a_{n}|}{|a_{n+1}|} \right\}$, estem calculant el radi de convergència per a $|x^{2}| < R$. Així doncs, el domini de convergència no serà $(-R, R)$ sinó $(-\sqrt{R}, \sqrt{R})$.
\end{example}
Cal notar que el càlcul del radi de convergència no ens assegura quin comportament té la funció als extrems de l'interval $[-R, R]$, de manera que a priori només podem assegurar que la sèrie $\sum_{n=0}^{\infty} a_{n} x^{n}$ convergeix $\forall x \in (-R , R)$.

\subsubsection*{Radi de convergència de la sèrie de derivades}
La sèrie de potències $\sum_{n=1}^{\infty} na_{n} x^{n-1}$ té el mateix radi de convergència que la sèrie $\sum_{n=0}^{\infty} a_{n} x^{n}$.

\subsubsection*{Radi de convergència de la sèrie de primitives}
La sèrie de potències $\sum_{n=0}^{\infty} \frac{a_{n}}{n+1} x^{n+1}$ té el mateix radi de convergència que la sèrie $\sum_{n=0}^{\infty} a_{n} x^{n}$.

\subsubsection*{Teorema de la convergència uniforme de les sèries de potències}
Si $\sum_{n=0}^{\infty} a_{n} x^{n}$ té radi de convergència $R$ i $[-A, A] \subseteq (-R, R)$
\begin{align}
    \Rightarrow \sum\limits_{n=0}^{\infty} a_{n} x^{n} \text{ convergeix uniformement a } [-A, A]
\end{align}

\subsubsection*{Teorema d'Abel}
Si $\sum_{n=0}^{\infty} a_{n} x^{n}$ convergeix en $l > 0$
\begin{align}
    \Rightarrow \sum\limits_{n=0}^{\infty} a_{n} x^{n} \text{ convergeix uniformement en } [0, l]
\end{align}
El mateix teorema és aplicable si $l' < 0$, amb un interval de convergència $[l', 0]$.

\subsubsection*{Teorema de la integrabilitat terme a terme de les sèries de potències}
Si $\sum_{n=0}^{\infty} a_{n} t^{n} = f(t), \quad \forall t \in (-R, R)$
\begin{align}
    \Rightarrow \sum\limits_{n=0}^{\infty} \frac{a_{n}}{n+1} x^{n+1} = \int_{0}^{x} f(t) \diff t , \quad \forall x \in (-R, R)
\end{align}

\subsubsection*{Teorema de la derivabilitat terme a terme de les sèries de potències}
Si $\sum_{n=0}^{\infty} a_{n} x^{n} = f(x), \quad \forall x \in (-R, R)$
\begin{align}
    \Rightarrow f(x) \text{ és derivable} \quad \text{i } \sum\limits_{n=0}^{\infty} na_{n} x^{n-1} = f'(x), \quad \forall x \in (-R, R)
\end{align}

\subsubsection*{Unicitat de les sèries de potències}
Si $f(x) = \sum_{n=0}^{\infty} a_{n} x^{n}$ a l'interval $(-R, R)$, derivant-la reiteradament obtenim
\begin{align}
\begin{split}
    f(x) & = \sum\limits_{n=0}^{\infty} a_{n} x^{n} \\
    f(x)' & = \sum\limits_{n=0}^{\infty} na_{n} x^{n-1} \\
    f(x)'' & = \sum\limits_{n=0}^{\infty} n(n-1)a_{n} x^{n-2} \\
    f(x)''' & = \sum\limits_{n=0}^{\infty} n(n-1)(n-2)a_{n} x^{n-3} \\
    & \, \, \, \vdots \\
    f(x)^{(k)} & = \sum\limits_{n=0}^{\infty} n(n-1) \dots (n-k+1) a_{n} x^{n-k} = \frac{n!}{(n-k)!} a_{n} x^{n-k}
\end{split}
\end{align}
Substituint $x=0$ a les igualtats anteriors obtenim
\begin{align}
\begin{split}
    f(0) & = a_{0}\\
    f'(0) & = a_{1}\\
    f''(0) & = 2 a_{2}\\
    f'''(0) & = 3 \cdot 2 a_{3}\\
    &\, \, \, \vdots \\
    f^{(n)}(0) & = n! a_{n}
\end{split}
\end{align}
Tenim, doncs, la següent relació entre els coeficients d'una sèrie de potències i les derivades de la funció cap a la qual convergeix (dins l'interval de convergència).
\begin{align}
    f(x) = \sum\limits_{n=0}^{\infty} a_{n} x^{n} \text{ a l'interval } (-R, R) \Rightarrow a_{n} = \frac{f^{(n)}(0)}{n!}
\end{align}
D'aquí podem concloure la unicitat de les sèries de potències:
\begin{align}
    f(x) = \sum\limits_{n=0}^{\infty} a_{n} x^{n} = \sum\limits_{n=0}^{\infty} b_{n} x^{n} \Rightarrow a_{n} = b_{n}
\end{align}

\subsubsection*{Sèrie de Taylor}
Donada una funció $f(x)$ $\infty$-derivable en el punt $c$, anomenem sèrie de Taylor de $f(x)$ al voltant del punt $c$ a la sèrie de potències
\begin{align}
    \sum\limits_{n=0}^{\infty} \frac{f^{(n)}(c)}{n!} (x-c)^{n} \quad \text{si } |x-c| < R
\end{align}
No obstant això, hi ha funcions $\infty$-derivables que no són la suma de la seva sèrie de Talyor, com ara $f(x) = \exp (-1/x^{2})$, que és és $\infty$-derivable $\forall x \in \mathbb{R}$, però totes les seves derivades a l'origen són 0 i, per tant, la sèrie de Taylor al voltant de l'origen és 0.

%----------------------------------------------------------------------------------------
\subsection{Funcions analítiques}
Quan serà una funció $\infty$-derivable expressable com la suma d'una sèrie de Taylor? Per obtenir la resposta escrivim la Fórmula de Taylor de $f(x)$ al voltant del punt $c$:
\begin{align}
    f(x) \sum\limits_{k=0}^{n} \frac{f^{(k)}(c)}{k!} (x-c)^{k} + R_{n}(x,c)
\end{align}
Observem que els polinomis de Taylor (i.e., $\sum_{k=0}^{n} \frac{f^{(k)}(c)}{k!} (x-c)^{k}$, de grau $n$) són les sumes parcials de la sèrie de Taylor. Per tant, la sèrie de Taylor serà convergent $n \to \infty \Rightarrow R_{n} (x,c) \to 0$.

Així doncs, les funcions analítiques són aquelles funcions $f(x)$ tals que
\begin{align}
    \sum\limits_{n=0}^{\infty} \frac{f^{(n)}(c)}{n!} (x-c)^{n} = f(x)
\end{align}

\subsubsection*{Exemples}
\begin{itemize}
    \item $\displaystyle e^{x} = \sum\limits_{n=0}^{\infty} \frac{x^{n}}{n!}, \quad \forall x \in \mathbb{R}$
    \item $\displaystyle \sin x = \sum\limits_{n=0}^{\infty} (-1)^{n} \frac{x^{2n+1}}{(2n+1)!}, \quad \forall x \in \mathbb{R}$
    \item $\displaystyle \cos x = \sum\limits_{n=0}^{\infty} (-1)^{n} \frac{x^{2n}}{(2n)!}, \quad \forall x \in \mathbb{R}$
\end{itemize}
De l'expressió d'aquestes funcions com a sèries de potències podem deduir moltes de les seves propietats, com ara $\sin ' x = \cos x$, l'expressió de $\sin (x+y)$, etc.

Tanmateix, podem definir noves funcions a partir de les seva expressió com a sèrie de potències, i partir de l'expressió d'altres funcions com a sèries trobar una expressió general de les noves funcions:
\begin{itemize}
    \item $\displaystyle \sum\limits_{n=0}^{\infty}  \frac{x^{2n+1}}{(2n+1)!} \equiv \sinh x = \frac{e^{x} - e^{-x}}{2} $
    \item $\displaystyle \sum\limits_{n=0}^{\infty} \frac{x^{2n}}{(2n)!} \equiv \cosh x = \frac{e^{x} + e^{-x}}{2} $
\end{itemize}
\bigskip
\begin{example}
    Sigui $\displaystyle f(x) = \frac{1}{1 + x^{2}} = \sum\limits_{n}^{\infty} a_{n} x^{n}$. Intentar trobar la sèrie de Taylor derivant indefinidament pot ser molt complicat, però tenim altres mètodes per fer-ho.
    
    Sabem que $\displaystyle\frac{1}{1 - y} = \sum\limits_{n}^{\infty} y^{n}, \quad |y| < 1 $, llavors fem $y = -x^{2}, \quad |x| < 1$ $\Rightarrow$ $\displaystyle\frac{1}{1 + x^{2}} = \sum\limits_{n}^{\infty} (-x^{2})^{n} = \sum\limits_{n}^{\infty} (-1)^{n} x^{2n} $. 
    
    A partir d'aquesta nova expressió podem definir una nova funció. Sabem que $\displaystyle \frac{\dif}{\dif x} (\arctan x) = \frac{1}{1+x^{2}}$, llavors
    \begin{align}
        \int_{0}^{x} f(t) \diff t \equiv \arctan x = \sum\limits_{0}^{\infty} (-1)^{n} \frac{x^{2n+1}}{2n+1}
    \end{align}
\end{example}
Com hem vist, l'expressió de funcions analítiques com a sèries de potències són de gran utilitat per estudiar les seves propietats i ens poden ajudar a definir noves funcions a partir de funcions de les quals sabem la seva expressió com a sèrie de potències.
%----------------------------------------------------------------------------------------
\subsection{Sèries de Fourier}
Les sèries de Fourier són molt importants a la Física. Les escriurem de la següent forma:
\begin{align}
    S(x) = \frac{a_{0}}{2} + \sum\limits_{k=1}^{\infty} \left( a_{k} \cos \frac{k \pi x}{L} + b_{k} \sin \frac{k \pi x}{L} \right)
\end{align}
Aquesta sèrie funcional pot ser o no convergent, i en cas de convergir, pot fer-ho puntualment o uniforme. Com que les funcions $\cos (k \pi x / L)$ i $\sin (k \pi x / L)$ són periòdiques amb període $2L$, si la sèrie convergeix cap a la funció $S(x)$, aquesta també serà periòdica, és a dir,
\begin{align}
    S(x) = S(x + 2L)
\end{align}

\subsubsection*{Càlcul dels coeficients}
A partir de la funció $f(x)$ donada, calculem les integrals següents:
\begin{align}
    a_{k} = \frac{1}{L} \int_{c}^{c+2L} f(x) \cos \frac{k \pi x}{L} \dif x , \quad (k = 0, 1, 2 \dots)
\end{align}
\begin{align}
    b_{k} = \frac{1}{L} \int_{c}^{c+2L} f(x) \sin \frac{k \pi x}{L} \dif x , \quad (k = 1, 2, 3 \dots)
\end{align}
on $c$ és un punt qualsevol. De fet, podem integrar sobre qualsevol interval de longitud $2L$. Pel que fa a l'existència de les integrals, n'hi ha prou amb que $f$ sigui integrable al llarg d'un període.

En particular, 
\begin{align}
    a_{0} = \frac{1}{L} \int_{c}^{c+2L} f(x) \dif x
\end{align}

\subsubsection*{Teorema de convergència}
Si $f(x)$ i $f'(x)$ són contínues a l'interval $[-L, L]$ (és a dir, només tenen un nombre finit de discontinuïtats de salt o evitables), llavors:
\begin{align}
    \frac{a_{0}}{2} + \sum\limits_{k=1}^{\infty} \left( a_{k} \cos \frac{k \pi x}{L} + b_{k} \sin \frac{k \pi x}{L} \right) \overset{\text{punt}}{\longrightarrow} \frac{1}{2} \left[ f(x^{+}) + f(x^{-}) \right]
\end{align}
on $f(x^{\pm})$ són respectivament els límits per la dreta i per l'esquerra de la funció $f$ en el punt $x$. Per tant, els punts on $f$ sigui contínua, la sèrie convergirà cap a $f(x)$.

Una condició suficient per a la convergència uniforme és que $f(x)$ sigui contínua i que les sèries numèriques $\sum a_{k}$ i $\sum b_{k}$ siguin absolutament sumables.

\subsubsection*{Funcions parelles i senars}
En el cas que ens trobem amb funcions parelles o senars, l'expressió de les sèries de Fourier se simplifica força.
\begin{itemize}
    \item Funcions parelles: $f(-x) = f(x) \sim \frac{a_{0}}{2} + \sum a_{k} \dots$
        \subitem $\displaystyle b_{k} = 0 $
        \subitem $\displaystyle a_{k} = \frac{1}{L} \int_{c}^{c+2L} f(x) \cos \frac{k \pi x}{L} \dif x $
    \item Funcions senars: $f(-x) = -f(x) \sim \sum b_{k} \dots$
        \subitem $\displaystyle a_{k} = 0$
        \subitem $\displaystyle b_{k} = \frac{1}{L} \int_{c}^{c+2L} f(x) \sin \frac{k \pi x}{L} \dif x $
\end{itemize}

%----------------------------------------------------------------------------------------
\end{document}

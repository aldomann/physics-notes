%-----------------------------------------------------------------
%   BASIC DOCUMENT LAYOUT
%-----------------------------------------------------------------
\documentclass[paper=a4, fontsize=12pt, twoside=semi]{scrartcl}
\usepackage[T1]{fontenc}
\usepackage[utf8]{inputenc}
\usepackage{lmodern}
\usepackage{slantsc}
\usepackage{microtype}
\usepackage[catalan]{babel}
\usepackage[fixlanguage]{babelbib}
\selectbiblanguage{catalan}

% Sectioning layout\\
\addtokomafont{sectioning}{\normalfont\scshape}
\usepackage{tocstyle}
\usetocstyle{standard}
\renewcommand*\descriptionlabel[1]{\hspace\labelsep\normalfont\bfseries{#1}}

% Empty pages
\usepackage{etoolbox}
\pretocmd{\section}{\cleardoubleevenemptypage}{}{}
\pretocmd{\part}{\cleardoubleevenemptypage\thispagestyle{empty}}{}{}
\renewcommand\partheadstartvskip{\clearpage\null\vfil}
\renewcommand\partheadmidvskip{\par\nobreak\vskip 20pt\thispagestyle{empty}}

% Paragraph indentation behaviour
\setlength{\parindent}{0pt}
\setlength{\parskip}{0.3\baselineskip plus2pt minus2pt}
\newcommand{\sk}{\medskip\noindent}

% Fancy header and footer
\usepackage{fancyhdr}
\pagestyle{fancyplain}
\fancyhead[LO]{\thepage}
\fancyhead[CO]{}
\fancyhead[RO]{\nouppercase{\mytitle}}
\fancyhead[LE]{\nouppercase{\leftmark}}
\fancyhead[CE]{}
\fancyhead[RE]{\thepage}
\fancyfoot{}
\renewcommand{\headrulewidth}{0.3pt}
\renewcommand{\footrulewidth}{0pt}
\setlength{\headheight}{13.6pt}

%-----------------------------------------------------------------
%   MATHS AND SCIENCE
%-----------------------------------------------------------------
\usepackage{amsmath,amsfonts,amsthm,amssymb}
\usepackage{xfrac}
\usepackage[a]{esvect}
\usepackage{chemformula}
\usepackage{graphicx}

\usepackage[arrowdel]{physics}
    \renewcommand{\vnabla}{\vec{\nabla}}
    % \renewcommand{\vectorbold}[1]{\boldsymbol{#1}}
    % \renewcommand{\vectorarrow}[1]{\vec{\boldsymbol{#1}}}
    % \renewcommand{\vectorunit}[1]{\hat{\boldsymbol{#1}}}
    \renewcommand{\vectorarrow}[1]{\vec{#1}}
    \renewcommand{\vectorunit}[1]{\hat{#1}}
    \renewcommand*{\grad}[1]{\vnabla #1}
    \renewcommand*{\div}[1]{\vnabla \vdot \va{#1}}
    \renewcommand*{\curl}[1]{\vnabla \cp \va{#1}}
    \let\rot\curl

% SI units
\usepackage[separate-uncertainty=true, alsoload=astro, alsoload=hep]{siunitx}
\sisetup{range-phrase = \text{--}, range-units = brackets}
\DeclareSIPrePower\quartic{4}
    \DeclareSIUnit\atm{atm}

% Smaller trig functions
\newcommand{\Sin}{\trigbraces{\operatorname{s}}}
\newcommand{\Cos}{\trigbraces{\operatorname{c}}}
\newcommand{\Tan}{\trigbraces{\operatorname{t}}}

% Operator-style notation for matrices
\newcommand*{\mat}[1]{\hat{#1}}

% Matrices in (A|B) form via [c|c] option
\makeatletter
\renewcommand*\env@matrix[1][*\c@MaxMatrixCols c]{%
  \hskip -\arraycolsep
  \let\@ifnextchar\new@ifnextchar
  \array{#1}}
\makeatother

% Shorter \mathcal and \mathbb
\newcommand*{\mc}[1]{\mathcal{#1}}
\newcommand*{\mbb}[1]{\mathbb{#1}}

% Shorter ^\ast and ^\dagger
\newcommand*{\sast}{^{\ast}{}}
\newcommand*{\sdag}{^{\dagger}{}}

%-----------------------------------------------------------------
%   OTHER PACKAGES
%-----------------------------------------------------------------
\usepackage{environ}

%Left numbered equations
\makeatletter
    \NewEnviron{Lalign}{\tagsleft@true\begin{align}\BODY\end{align}}
\makeatother

% Plots and graphics
\usepackage{pgfplots}
\usepackage{tikz}
\usepackage{color}
    \makeatletter
        \color{black}
        \let\default@color\current@color
    \makeatother

% Richer enumerate, figure, and table support
\usepackage{enumerate}
\usepackage[shortlabels]{enumitem}
\usepackage{float}
\usepackage{tabularx}
\usepackage{booktabs}
    % \setlength{\intextsep}{8pt}
\numberwithin{equation}{section}
\numberwithin{figure}{section}
\numberwithin{table}{section}

% No indentation after certain environments
\makeatletter
\newcommand*\NoIndentAfterEnv[1]{%
    \AfterEndEnvironment{#1}{\par\@afterindentfalse\@afterheading}}
\makeatother
% \NoIndentAfterEnv{thm}
\NoIndentAfterEnv{defi}
\NoIndentAfterEnv{example}
\NoIndentAfterEnv{table}

% Misc packages
\usepackage{ccicons}
\usepackage{lipsum}

%-----------------------------------------------------------------
%   THEOREMS
%-----------------------------------------------------------------
\usepackage{thmtools}

% Theroems layout
\declaretheoremstyle[
    spaceabove=6pt, spacebelow=6pt,
    headfont=\normalfont,
    notefont=\mdseries, notebraces={(}{)},
    bodyfont=\small,
    postheadspace=1em,
]{small}

\declaretheorem[style=plain,name=Teorema,qed=$\square$,numberwithin=section]{thm}
\declaretheorem[style=plain,name=Corol·lari,qed=$\square$,sibling=thm]{cor}
\declaretheorem[style=plain,name=Lema,qed=$\square$,sibling=thm]{lem}
\declaretheorem[style=definition,name=Definició,qed=$\blacksquare$,numberwithin=section]{defi}
\declaretheorem[style=definition,name=Exemple,qed=$\blacktriangle$,numberwithin=section]{example}
\declaretheorem[style=small,name=Demostració,numbered=no,qed=$\square$]{sproof}

%-----------------------------------------------------------------
%   ELA MOTHERFUCKING GEMINADA
%-----------------------------------------------------------------
\def\xgem{%
    \ifmmode
        \csname normal@char\string"\endcsname l%
    \else
        \leftllkern=0pt\rightllkern=0pt\raiselldim=0pt
        \setbox0\hbox{l}\setbox1\hbox{l\/}\setbox2\hbox{.}%
        \advance\raiselldim by \the\fontdimen5\the\font
        \advance\raiselldim by -\ht2
        \leftllkern=-.25\wd0%
        \advance\leftllkern by \wd1
        \advance\leftllkern by -\wd0
        \rightllkern=-.25\wd0%
        \advance\rightllkern by -\wd1
        \advance\rightllkern by \wd0
        \allowhyphens\discretionary{-}{}%
        {\kern\leftllkern\raise\raiselldim\hbox{.}%
            \kern\rightllkern}\allowhyphens
    \fi
}
\def\Xgem{%
    \ifmmode
        \csname normal@char\string"\endcsname L%
    \else
        \leftllkern=0pt\rightllkern=0pt\raiselldim=0pt
        \setbox0\hbox{L}\setbox1\hbox{L\/}\setbox2\hbox{.}%
        \advance\raiselldim by .5\ht0
        \advance\raiselldim by -.5\ht2
        \leftllkern=-.125\wd0%
        \advance\leftllkern by \wd1
        \advance\leftllkern by -\wd0
        \rightllkern=-\wd0%
        \divide\rightllkern by 6
        \advance\rightllkern by -\wd1
        \advance\rightllkern by \wd0
        \allowhyphens\discretionary{-}{}%
        {\kern\leftllkern\raise\raiselldim\hbox{.}%
            \kern\rightllkern}\allowhyphens
    \fi
}

\expandafter\let\expandafter\saveperiodcentered
    \csname T1\string\textperiodcentered \endcsname

\DeclareTextCommand{\textperiodcentered}{T1}[1]{%
    \ifnum\spacefactor=998
        \Xgem
    \else
        \xgem
    \fi#1}

%-----------------------------------------------------------------
%   PDF INFO AND HYPERREF
%-----------------------------------------------------------------
\usepackage{hyperref}
\usepackage{cleveref}
\hypersetup{colorlinks, citecolor=black, filecolor=black, linkcolor=black, urlcolor=black}

\newcommand*{\mytitle}{Estructura de la matèria i termodinàmica}
\newcommand*{\myauthor}{Alfredo Hernández Cavieres}
\newcommand*{\myuni}{Universitat Autònoma de Barcelona, Departament de Física}
\newcommand*{\mydate}{\normalsize 2012-2013}

\usepackage{hyperxmp}
\hypersetup{pdfauthor={\myauthor}, pdftitle={\mytitle}}

%----------------------------------------------------------------------------------------
%    TITLE SECTION AND DOCUMENT BEGINNING
%----------------------------------------------------------------------------------------
\newcommand{\horrule}[1]{\rule{\linewidth}{#1}}
\title{
    \normalfont
    \small \scshape{\myuni} \\ [25pt]
    \horrule{0.5pt} \\[0.4cm]
    \huge \mytitle \\
    \horrule{2pt} \\[0.5cm]
}
\author{\myauthor}
\date{\mydate}

\begin{document}

\clearpage\maketitle
\thispagestyle{empty}
\addtocounter{page}{-1}

%----------------------------------------------------------------------------------------
%    LLICÈNCIA
%----------------------------------------------------------------------------------------
\section*{}\thispagestyle{empty}
\begin{centering}
    \huge \ccbyncsaeu

    \normalsize Aquesta obra està subjecta a una llicència de

    Reconeixement-NoComercial-CompartirIgual 4.0

    Internacional de Creative Commons.

\end{centering}

%----------------------------------------------------------------------------------------
%    TABLE OF CONTENTS
%----------------------------------------------------------------------------------------
\cleardoubleevenemptypage
\pdfbookmark[1]{\contentsname}{toc}
\tableofcontents

%----------------------------------------------------------------------------------------
%    SECTIONS
%----------------------------------------------------------------------------------------
%-----------------------------------------------------------------
%	CONSTANTS I FACTORS DE CONVERSIÓ
%	!TEX root = ./../main.tex
%-----------------------------------------------------------------
\section{Constants i factors de conversió}
\subsection*{Constants universals\footnote{Per raons històriques, a l'astrofísica és estàndard fer servir el sistema CGS (\textit{centimetre--gram--second}) d'unitats.}}
\begin{itemize}[leftmargin=*]
	\item Velocitat e la llum al buit: $c = \SI{3 e 10}{\cm \per \s}$.
	\item Permitivitat al buit: $\varepsilon_{0} = \SI{8.854 e -12}{\square\C \per \newton \per \square\metre}$.
	\item Permeabilitat al buit: $\mu_{0} = \SI{4\pi e -7}{\m\kg \per \square\C}$.
	\item Constant de gravitació: $G = \SI{6.67 e-8}{\dyn \square\cm \per \square\g}$.
	\item Constant de Planck: $h = \SI{6.626 e -27}{\erg \s} = \SI{4.134 e-15}{\eV \s}$.
	\item Constant de Planck reduïda: $\hbar \equiv h/(2\pi) = \SI{1.055 e -27}{\erg \s} = \SI{6.579 e-16}{\eV \s}$.
	\item Constant d'estructura fina: $\alpha = \si{\elementarycharge\squared \per \planckbar \per \clight} \approx \num{1/137.036}$.
\end{itemize}

%-----------------------------------------------------------------
\subsection*{Constants atòmiques}
\begin{itemize}[leftmargin=*]
	\item Càrrega de l'electró: $e = \SI{1.602 e-19}{\C} = \SI{4.803 e-10}{\esu}$.
	\item Constant de Boltzmann: $k_{B} = \SI{1.381 e-16}{\erg \per \K} = \SI{8.617 e-5}{\eV \per \K}$.
	\item Constant dels gasos: $R \equiv k_{B}/N_{A} = \SI{8.314}{\J \per \K \per \mol} = \SI{0.082}{\atm \litre \per \K \per \mol}$.
	\item Nombre d'Avogadro: $N_{A} = \SI{6.022 e23}{\per\mol}$.
	\item Constant d'Stefan--Boltzmann: $\sigma = \SI{5.670 e-5}{\erg \per \square\cm \per \s \per \quartic\K}$.
	\item Constant de radiació: $a \equiv \pi^{2}k_{B}^{4}/(15c^3\hbar^{3}) = \SI{7.566 e-15}{\erg \per \cubic\cm \per \quartic\K}$.
	\item Secció eficaç de Thomson: $\sigma_{T} \equiv 8\pi e^4/(3m_{e}^2c^4) = \SI{6.652 e-25}{\cm^{2}}$.
	\item Radi de Bohr: $a_{0} \equiv \hbar^{2}/(m_{e}e^{2}) = \SI{5.292 e-9}{\cm}$.
	\item Longitud d'ona de Compton de l'electró: $\lambda_{c} \equiv \hbar/(m_{e}c) = \SI{2.426 e-14}{\cm}$.
\end{itemize}

%-----------------------------------------------------------------
\subsection*{Masses atòmiques}
\begin{itemize}[leftmargin=*]
	\item Massa de l'electró: $m_{e} = \SI{9.109 e-28}{\g} = \SI{0.511}{\mega \eVperc \squared}$.
	\item Massa del protó: $m_{p} = \SI{1.673 e-24}{\g} = \SI{938.272}{\mega \eVperc \squared}$.
	\item Massa del neutró: $m_{n} = \SI{1.675 e-24}{\g} = \SI{939.56}{\mega \eVperc \squared}$.
	\item Massa del deuteró: $m_{d} = \SI{1875.612}{\eVperc \squared}$.
	\item Massa de la partícula alfa: $m(\ch{^{4}_{2}H}) = \SI{1.225}{\giga \eVperc \squared}$.
	\item Massa de l'àtom d'hidrogen: $m_{H} = \SI{938.386}{\mega \eVperc \squared}$.
	\item Massa atòmica: $\si{\amu} \equiv m(\ch{^{12} C}) =  \SI{931.494}{\mega \eVperc \squared}$.
	% \item Massa del deuteró: $m_{d} = \SI{3.344 e-24}{\g} = \SI{1875.612}{\eVperc \squared}$.
	% \item Massa de la partícula alfa: $m(\ch{^{4}_{2}H}) = \SI{6.646 e-24}{\g} = \SI{1.225}{\giga \eVperc \squared}$.
	% \item Massa de l'àtom d'hidrogen: $m_{H} = \SI{1.673 e-24}{\g} = \SI{938.386}{\mega \eVperc \squared}$.
	% \item Massa atòmica: $\si{\amu} \equiv m(\ch{^{12} C}) = \SI{1.66054 e-24}{\g} = \SI{931.494}{\mega \eVperc \squared}$.
\end{itemize}

%-----------------------------------------------------------------
\subsection*{Constants astronòmiques}
\begin{itemize}[leftmargin=*]
	\item Massa del Sol: $M_{\odot} = \SI{1.988 e33}{\g}$.
	\item Radi del Sol: $R_{\odot} = \SI{6.955 e10}{\cm}$.
	\item Lluminositat del Sol: $L_{\odot} = \SI{3.846 e33}{\erg\per\s}$.
	\item Constant solar: $F_{\odot} = \SI{1.360 e6}{\erg \per\s \per\square\cm}$.
	\item Temperatura de la superfície solar: $T_{\odot} = \SI{5780}{\K}$.
	\item Magnitud absoluta del Sol: $M_{\odot} = \num{4.76}$.
	\item Massa de la Terra: $M_{\oplus} = \SI{5.972 e 27}{\g}$.
	\item Radi de la Terra: $R_{\oplus} = \SI{6.367 e8}{\cm}$.
	\item Massa de la Lluna: $M_{L} = \SI{7.346 e 25}{\g}$.
	\item Radi de la Lluna: $R_{L} = \SI{1.738 e8}{\cm}$.
	\item Constant de Hubble: $H_{0} = \SI{100}{\planck \km \per \s \per \mega\parsec}$, on $h \in [0.5,0.9]$.
	\item Temps de Hubble: $H_{0}^{-1} = \SI{3.086 e17}{\per \planck \s} = \SI{9.778}{\per \planck \giga \year}$.
	\item Radi de Hubble: $c H_{0}^{-1} = \SI{2997.9}{\per \planck \mega \parsec}$.
	% Note that h in here is not planck's constant, but the expansion rate. I use it here for pure convenience.
\end{itemize}

%-----------------------------------------------------------------
\subsection*{Unitats de Planck}
\begin{itemize}[leftmargin=*]
	\item Massa de Planck: $M_{Pl} = \sqrt{\hbar c / G} \approx \SI{2.2 e-5}{\g}$.
	\item Longitud de Planck: $l_{Pl} = \sqrt{\hbar G / c^{3}} \approx \SI{1.6 e-33}{\cm}$.
	\item Temps de Planck: $t_{Pl} = \sqrt{\hbar G / c^{5}} \approx \SI{5.4 e-44}{\s}$.
	\item Temperatura de Planck: $T_{Pl} = \sqrt{\hbar c^{5}/ G} / k_{B} \approx \SI{1.4 e32}{\K}$.
\end{itemize}

%-----------------------------------------------------------------
\subsection*{Constants numèriques}
\begin{itemize}[leftmargin=*]
	\item Pi: $\pi = \num{3.141593}$.
	\item Nombre d'Euler: $e = \num{2.718282}$.
	\item Constant d'Euler--Mascheroni: $\gamma = \num{0.577216}$.
	\item Zeta de Riemann: $\zeta(3) = \num{1.202057}$.
	\item Logaritmes: $\ln 10 = \num{2.302585}$, $\log e = \num{0.434294}$, $\log 2 = \num{0.301030}$, $\log 3 = \num{0.477121}$.
\end{itemize}

%-----------------------------------------------------------------
\subsection*{Factors de conversió}
\begin{itemize}[leftmargin=*] %\item $\SI{1}{} = \SI{}{}$.
	\item $\SI{1}{\eV} = \SI{1.602 e-12}{\erg}$.
	\item $\SI{1}{\erg} = \SI{1}{\dyn \cm} = \SI{6.242 e11}{\eV} = \SI{e-7}{\J}$.
	\item $\SI{1}{\dyn} = \SI{e-5}{\N}$.
	\item $\SI{1}{\cal} = \SI{4.184}{\J}$.
	\item $\SI{1}{\J} = \SI{0.239}{\cal}$.
	\item $\SI{1}{\esu} = \SI{3.336 e-10}{\coulomb}$.
	\item $\SI{1}{\atm} = \SI{1.013 e5}{\Pa} = \SI{760}{\mmHg}$.
	\item $\SI{1}{\bar} = \SI{e5}{\Pa}$.
	\item $\SI{1}{\torr} = \SI{1}{\mmHg} = \SI{133.3}{\Pa}$.
	\item $\SI{1}{\kg} = \SI{5.610 e29}{\mega \eVperc \squared}$.
	\item $\SI{1}{\mega \eVperc \squared} = \SI{1.783 e-27}{\g}$.
	\item $\SI{1}{\amu} = \SI{1.66054 e-27}{\kg} = \SI{931.494}{\mega \eVperc \squared}$.
	\item $\SI{1}{\parsec} = \SI{3.262}{\lightyear} = \SI{2.063e5}{\au} = \SI{3.086 e18}{\cm}$.
	\item $\SI{1}{\lightyear} = \SI{9.461 e17}{\cm}$.
	\item $\SI{1}{\au} = \SI{1.496 e13}{\cm}$.
	\item $\SI{1}{\year} = \SI{3.1558149984 e7}{\s}$.
	\item $\SI{1}{\arcsecond} = \SI{4.848 e-6}{\radian}$.
\end{itemize}

\part*{Estructura de la matèria}
\addcontentsline{toc}{part}{Estructura de la matèria}
    %----------------------------------------------------------------------------------------
%    ELS INICIS DE LA FÍSICA QUÀNTICA
%----------------------------------------------------------------------------------------
\section{Els inicis de la física quàntica}
\subsection{Estudis previs}
\subsubsection*{Gustav Kirchhoff, 1859}

\begin{figure}[H]
\centering
    WIP: GRAFIC BONIC 
\caption{Gràfic il·lustratiu}
\end{figure}

\subsubsection*{Joseph von Fraunhofer, 1830}
Els seus estudis suposen un gran canvi filosòfic, ja que permeten saber de què estan fetes les estrelles.
\begin{figure}[H]
\centering
    WIP: GRAFIC BONIC 
\caption{Gràfic il·lustratiu}
\end{figure}

\subsubsection*{James Clerk Maxell, 1865: Teoria electromagnètica de la llum}
\begin{align}
    p = \frac{1}{3} \frac{U}{V} \equiv \text{ pressió}
\end{align}

\subsubsection*{Stefan-Boltzmann, 1879}
\begin{align}
    q = \frac{\dot{Q}}{A} = \varepsilon \sigma T^{4}
\end{align}

\begin{itemize}
    \item Stefan arriba a la fórmula experimentalment.
    \item Boltzmann relaciona l'exponent 4 amb $p$. 
    $\begin{cases} \text{si } p = \frac{1}{3} \frac{U}{V} \Rightarrow q \approx T^{\alpha + 1} \\ q = \frac{U}{V} \frac{c}{4}\end{cases}$
\end{itemize}

\subsubsection*{Wilhelm Wien, 1896}
\begin{figure}[H]
\centering
    WIP: GRAFIC BONIC 
\caption{Gràfic il·lustratiu}
\end{figure}

%----------------------------------------------------------------------------------------
\subsection{Max Planck, inici de la física quàntica}

%----------------------------------------------------------------------------------------
\subsection{Albert Einstein}

%----------------------------------------------------------------------------------------
\subsection{Walther Nernst}

%----------------------------------------------------------------------------------------
\subsection{Efecte Compton}

%----------------------------------------------------------------------------------------
\subsection{Louis de Broglie}
\subsubsection*{Longitut d'ona de de Broglie, 1923}
Hem vist que la llum ona es comporta també com a corpuscle. Podria ser que els corpuscles es comportin també com a ona? De Broglie suposa que sí i proposa:
\begin{align}
    \lambda = \frac{h}{p} = \frac {h}{mv}
\end{align}

\subsubsection*{Comprovació experimental amb electrons: Davisson i Germer, 1926}
\begin{figure}[H]
\centering
    WIP: GRAFIC BONIC 
\caption{Gràfic il·lustratiu}
\end{figure}

%----------------------------------------------------------------------------------------
\subsection{Dualitat ona-corpuscle}
\paragraph{Principi de complementaritat (Bohr, 1926)}
La llum no és ni ona ni partícula, sinó que es manifesta com a ona o com a partícula.

\subsubsection*{Conseqüències de la dualitat}
    %----------------------------------------------------------------------------------------
%    FÍSICA ATÒMICA
%----------------------------------------------------------------------------------------
\section{Física atòmica}
\subsection{Apartat 1}

%----------------------------------------------------------------------------------------
\subsection{Apartat 2}

%----------------------------------------------------------------------------------------
\subsection{Apartat 3}

%----------------------------------------------------------------------------------------
\subsection{Apartat 4}

%----------------------------------------------------------------------------------------
\subsection{Apartat 5}
    %----------------------------------------------------------------------------------------
%    LA TAULA PERIÒDICA
%----------------------------------------------------------------------------------------
\section{La taula periòdica}
Ordenem els elements químics segons el núm. atòmic $\Rightarrow$ posa en manifest una repetició periòdica de les propietats físico-químiques dels elements.

%----------------------------------------------------------------------------------------
\subsection{Principi d'exclusió de Pauli}
En un àtom no pot haver dos electrons amb el mateix conjunt de nombres quàntics $(n,l,m_{z},s)$.

\begin{figure}[H]
\centering
    WIP: GRAFIC BONIC 
\caption{Gràfic il·lustratiu}
\end{figure}

Les propietats químiques estan relacionades amb la capa exterior de l'àtom.

%----------------------------------------------------------------------------------------
\subsection{Equació de Schrödinger (1926)}
\subsubsection*{Laplaciana}
\begin{align}
    E \Psi(\vec{r}) = \frac{-\hbar^2}{2m}\nabla^2 \Psi(\vec{r}) + V({r}) \Psi(\vec{r})
\end{align}
\subsubsection*{Coordenades esfèriques}
\begin{align}
\begin{aligned}
    E \Psi = & -\frac{\hbar^{2}}{2m} \left[ \frac{1}{r^{2}} \frac{\partial}{\partial r} \left( r \frac{\partial \Psi}{\partial r} \right) + \frac{1}{r^{2} \sin^{2} \theta} \left( \frac{\partial}{\partial \theta} sin \theta \frac{\partial \Psi}{\partial \theta} \right) \right. \\ 
    & + \left. \frac{1}{r^{2} \sin^{2} \theta} \frac{\partial^{2} \Psi}{\partial \varphi^{2}} \right] - \frac{kZe^{2}}{r} \Psi
\end{aligned}
\end{align}

\begin{align}
     \Psi({r,\theta,\varphi}) = ({r}) Y_{l,m_{z}}({\theta,\varphi})\Rightarrow E_{n,l,m_{z}}
\end{align}
    %----------------------------------------------------------------------------------------
%    FÍSICA NUCLEAR
%----------------------------------------------------------------------------------------
\section{Física nuclear}
\subsection{Comparació entre física atòmica i física nuclear}
\subsubsection*{Diferències}
\begin{itemize}
    \item El radi del nucli es unes $10^{4}$ vegades més petit que el radi de l'àtom $\Rightarrow$ $E$nuclears $\propto$ $10^{8}$ $E$atòmiques.
    \item Els protagonistes de la física atòmica són els electrons; els de la física nuclear són els protons i neutrons (1932: descobriment del neutró).
    \item La força rellevant en la física atòmica és la força electromagnètica; les de la física nuclear són la força hadrònica (nuclear forta) i la força nuclear feble.
\end{itemize}

\subsubsection*{Analogies}
\begin{itemize}
    \item Són sistemes típicament quàntics: nivells d'energia i salts entre nivells.
    \item Protons i neutrons tenen espin $\sfrac{1}{2}$ (com els electrons) i satisfan el principi d'exclusió de Pauli.
\end{itemize}

%----------------------------------------------------------------------------------------
\subsection{Descripció dels nuclis}
\begin{align}
\begin{split} 
    Z & \equiv \text{nre. de protons.} \\
    N & \equiv \text{nre. de neutrons.} \\
    A & \equiv \text{nre. màssic = nre. de nucleons.}
\end{split}
\end{align}
\subsubsection*{Representació d'un element químic $X$}
\begin{align}
    _{Z}^{A}X
\end{align}
\paragraph*{Isòtops} Elements amb mateix $Z$ i diferent $N$. La majoria d'isòtops d'un element són inestables.
(e.g., $_{2}^{3} He$ ($2p1n$); $_{2}^{4} He$ ($2p2n$)).
\begin{figure}[H]
\centering
    WIP: GRAFIC BONIC 
\caption{Gràfic il·lustratiu}
\end{figure}

%----------------------------------------------------------------------------------------
\subsection{Força hadrònica}
Recordem que per a dos protons, l'energia potencial electrostàtica (repulsió) val: $U_{d} = \frac{ke^{2}}{r}$; $r \downarrow \Rightarrow U_{d} \uparrow$.
\begin{figure}[H]
\centering
    WIP: GRAFIC BONIC 
\caption{Gràfic il·lustratiu}
\end{figure}

\subsubsection*{Energia d'enllaç dels nuclis}
\begin{itemize}
    \item $E$ necessària per descompondre un nucli en els seus components ($p$ i $n$) per separat.
    \item La força hadrònica és la mateixa: $pp$, $nn$ i $pn$.
    \item Relacionada amb el defecte de massa dels nuclis. $\Delta m = \sum\limits_{i} m_{i} - M_{total} = Zm{p} + Nm_{n} - M_{total} $.
\end{itemize}
\begin{align}
    \begin{gathered}
        E_{enllaç} = \Delta m c^{2} \text{ i, en concret, } \SI{1}{\amu} c^{2} = \SI{931.5}{\MeV} \\
        \frac{E_{enllaç}}{A} = E_{enllaç} \text{ per nucleó.}
    \end{gathered}
\end{align}

\begin{figure}[H]
\centering
    WIP: GRAFIC BONIC 
\caption{Gràfic il·lustratiu}
\end{figure}

\subsubsection*{Nivells d'energia}
\begin{figure}[H]
\centering
    WIP: GRAFIC BONIC 
\caption{Nivells atòmics d'energia}
\end{figure}

\part*{Termodinàmica}
\addcontentsline{toc}{part}{Termodinàmica}
    %-----------------------------------------------------------------
%	TEMA
%	!TEX root = ./../main.tex
%-----------------------------------------------------------------
\section{Tema 1}
\subsection{Apartat 1}

%-----------------------------------------------------------------
\subsection{Apartat 2}

%-----------------------------------------------------------------
\subsection{Apartat 3}

    %----------------------------------------------------------------------------------------
%    TEMA 2
%----------------------------------------------------------------------------------------
\section{Tema 2}
\subsection{Apartat 1}

%----------------------------------------------------------------------------------------
\subsection{Apartat 2}

%----------------------------------------------------------------------------------------
\subsection{Apartat 3}

%----------------------------------------------------------------------------------------
\subsection{Apartat 4}

%----------------------------------------------------------------------------------------
\subsection{Apartat 5}

%----------------------------------------------------------------------------------------
\end{document}

%----------------------------------------------------------------------------------------
%    LA TAULA PERIÒDICA
%----------------------------------------------------------------------------------------
\section{La taula periòdica}
Ordenem els elements químics segons el núm. atòmic $\Rightarrow$ posa en manifest una repetició periòdica de les propietats físico-químiques dels elements.

%----------------------------------------------------------------------------------------
\subsection{Principi d'exclusió de Pauli}
En un àtom no pot haver dos electrons amb el mateix conjunt de nombres quàntics $(n,l,m_{z},s)$.

\begin{figure}[H]
\centering
    WIP: GRAFIC BONIC 
\caption{Gràfic il·lustratiu}
\end{figure}

Les propietats químiques estan relacionades amb la capa exterior de l'àtom.

%----------------------------------------------------------------------------------------
\subsection{Equació de Schrödinger (1926)}
\subsubsection*{Laplaciana}
\begin{align}
    E \Psi(\vec{r}) = \frac{-\hbar^2}{2m}\nabla^2 \Psi(\vec{r}) + V({r}) \Psi(\vec{r})
\end{align}
\subsubsection*{Coordenades esfèriques}
\begin{align}
\begin{aligned}
    E \Psi = & -\frac{\hbar^{2}}{2m} \left[ \frac{1}{r^{2}} \frac{\partial}{\partial r} \left( r \frac{\partial \Psi}{\partial r} \right) + \frac{1}{r^{2} \sin^{2} \theta} \left( \frac{\partial}{\partial \theta} sin \theta \frac{\partial \Psi}{\partial \theta} \right) \right. \\ 
    & + \left. \frac{1}{r^{2} \sin^{2} \theta} \frac{\partial^{2} \Psi}{\partial \varphi^{2}} \right] - \frac{kZe^{2}}{r} \Psi
\end{aligned}
\end{align}

\begin{align}
     \Psi({r,\theta,\varphi}) = ({r}) Y_{l,m_{z}}({\theta,\varphi})\Rightarrow E_{n,l,m_{z}}
\end{align}
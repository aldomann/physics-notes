%----------------------------------------------------------------------------------------
%    ELS INICIS DE LA FÍSICA QUÀNTICA
%----------------------------------------------------------------------------------------
\section{Els inicis de la física quàntica}
\subsection{Estudis previs}
\subsubsection*{Gustav Kirchhoff, 1859}

\begin{figure}[H]
\centering
    WIP: GRAFIC BONIC 
\caption{Gràfic il·lustratiu}
\end{figure}

\subsubsection*{Joseph von Fraunhofer, 1830}
Els seus estudis suposen un gran canvi filosòfic, ja que permeten saber de què estan fetes les estrelles.
\begin{figure}[H]
\centering
    WIP: GRAFIC BONIC 
\caption{Gràfic il·lustratiu}
\end{figure}

\subsubsection*{James Clerk Maxell, 1865: Teoria electromagnètica de la llum}
\begin{align}
    p = \frac{1}{3} \frac{U}{V} \equiv \text{ pressió}
\end{align}

\subsubsection*{Stefan-Boltzmann, 1879}
\begin{align}
    q = \frac{\dot{Q}}{A} = \varepsilon \sigma T^{4}
\end{align}

\begin{itemize}
    \item Stefan arriba a la fórmula experimentalment.
    \item Boltzmann relaciona l'exponent 4 amb $p$. 
    $\begin{cases} \text{si } p = \frac{1}{3} \frac{U}{V} \Rightarrow q \approx T^{\alpha + 1} \\ q = \frac{U}{V} \frac{c}{4}\end{cases}$
\end{itemize}

\subsubsection*{Wilhelm Wien, 1896}
\begin{figure}[H]
\centering
    WIP: GRAFIC BONIC 
\caption{Gràfic il·lustratiu}
\end{figure}

%----------------------------------------------------------------------------------------
\subsection{Max Planck, inici de la física quàntica}

%----------------------------------------------------------------------------------------
\subsection{Albert Einstein}

%----------------------------------------------------------------------------------------
\subsection{Walther Nernst}

%----------------------------------------------------------------------------------------
\subsection{Efecte Compton}

%----------------------------------------------------------------------------------------
\subsection{Louis de Broglie}
\subsubsection*{Longitut d'ona de de Broglie, 1923}
Hem vist que la llum ona es comporta també com a corpuscle. Podria ser que els corpuscles es comportin també com a ona? De Broglie suposa que sí i proposa:
\begin{align}
    \lambda = \frac{h}{p} = \frac {h}{mv}
\end{align}

\subsubsection*{Comprovació experimental amb electrons: Davisson i Germer, 1926}
\begin{figure}[H]
\centering
    WIP: GRAFIC BONIC 
\caption{Gràfic il·lustratiu}
\end{figure}

%----------------------------------------------------------------------------------------
\subsection{Dualitat ona-corpuscle}
\paragraph{Principi de complementaritat (Bohr, 1926)}
La llum no és ni ona ni partícula, sinó que es manifesta com a ona o com a partícula.

\subsubsection*{Conseqüències de la dualitat}
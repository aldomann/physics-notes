%----------------------------------------------------------------------------------------
%    FÍSICA NUCLEAR
%----------------------------------------------------------------------------------------
\section{Física nuclear}
\subsection{Comparació entre física atòmica i física nuclear}
\subsubsection*{Diferències}
\begin{itemize}
    \item El radi del nucli es unes $10^{4}$ vegades més petit que el radi de l'àtom $\Rightarrow$ $E$nuclears $\propto$ $10^{8}$ $E$atòmiques.
    \item Els protagonistes de la física atòmica són els electrons; els de la física nuclear són els protons i neutrons (1932: descobriment del neutró).
    \item La força rellevant en la física atòmica és la força electromagnètica; les de la física nuclear són la força hadrònica (nuclear forta) i la força nuclear feble.
\end{itemize}

\subsubsection*{Analogies}
\begin{itemize}
    \item Són sistemes típicament quàntics: nivells d'energia i salts entre nivells.
    \item Protons i neutrons tenen espin $\sfrac{1}{2}$ (com els electrons) i satisfan el principi d'exclusió de Pauli.
\end{itemize}

%----------------------------------------------------------------------------------------
\subsection{Descripció dels nuclis}
\begin{align}
\begin{split} 
    Z & \equiv \text{nre. de protons.} \\
    N & \equiv \text{nre. de neutrons.} \\
    A & \equiv \text{nre. màssic = nre. de nucleons.}
\end{split}
\end{align}
\subsubsection*{Representació d'un element químic $X$}
\begin{align}
    _{Z}^{A}X
\end{align}
\paragraph*{Isòtops} Elements amb mateix $Z$ i diferent $N$. La majoria d'isòtops d'un element són inestables.
(e.g., $_{2}^{3} He$ ($2p1n$); $_{2}^{4} He$ ($2p2n$)).
\begin{figure}[H]
\centering
    WIP: GRAFIC BONIC 
\caption{Gràfic il·lustratiu}
\end{figure}

%----------------------------------------------------------------------------------------
\subsection{Força hadrònica}
Recordem que per a dos protons, l'energia potencial electrostàtica (repulsió) val: $U_{d} = \frac{ke^{2}}{r}$; $r \downarrow \Rightarrow U_{d} \uparrow$.
\begin{figure}[H]
\centering
    WIP: GRAFIC BONIC 
\caption{Gràfic il·lustratiu}
\end{figure}

\subsubsection*{Energia d'enllaç dels nuclis}
\begin{itemize}
    \item $E$ necessària per descompondre un nucli en els seus components ($p$ i $n$) per separat.
    \item La força hadrònica és la mateixa: $pp$, $nn$ i $pn$.
    \item Relacionada amb el defecte de massa dels nuclis. $\Delta m = \sum\limits_{i} m_{i} - M_{total} = Zm{p} + Nm_{n} - M_{total} $.
\end{itemize}
\begin{align}
    \begin{gathered}
        E_{enllaç} = \Delta m c^{2} \text{ i, en concret, } \SI{1}{\amu} c^{2} = \SI{931.5}{\MeV} \\
        \frac{E_{enllaç}}{A} = E_{enllaç} \text{ per nucleó.}
    \end{gathered}
\end{align}

\begin{figure}[H]
\centering
    WIP: GRAFIC BONIC 
\caption{Gràfic il·lustratiu}
\end{figure}

\subsubsection*{Nivells d'energia}
\begin{figure}[H]
\centering
    WIP: GRAFIC BONIC 
\caption{Nivells atòmics d'energia}
\end{figure}
%-----------------------------------------------------------------
%   BASIC DOCUMENT LAYOUT
%-----------------------------------------------------------------
\documentclass[paper=a4, fontsize=12pt, twoside=semi]{scrartcl}
\usepackage[T1]{fontenc}
\usepackage[utf8]{inputenc}
\usepackage{lmodern}
\usepackage{slantsc}
\usepackage{microtype}
\usepackage[catalan]{babel}
\usepackage[fixlanguage]{babelbib}
\selectbiblanguage{catalan}

% Sectioning layout\\
\addtokomafont{sectioning}{\normalfont\scshape}
\usepackage{tocstyle}
\usetocstyle{standard}
\renewcommand*\descriptionlabel[1]{\hspace\labelsep\normalfont\bfseries{#1}}

% Empty pages
\usepackage{etoolbox}
\pretocmd{\section}{\cleardoubleevenemptypage}{}{}
\pretocmd{\part}{\cleardoubleevenemptypage\thispagestyle{empty}}{}{}
\renewcommand\partheadstartvskip{\clearpage\null\vfil}
\renewcommand\partheadmidvskip{\par\nobreak\vskip 20pt\thispagestyle{empty}}

% Paragraph indentation behaviour
\setlength{\parindent}{0pt}
\setlength{\parskip}{0.3\baselineskip plus2pt minus2pt}
\newcommand{\sk}{\medskip\noindent}

% Fancy header and footer
\usepackage{fancyhdr}
\pagestyle{fancyplain}
\fancyhead[LO]{\thepage}
\fancyhead[CO]{}
\fancyhead[RO]{\nouppercase{\mytitle}}
\fancyhead[LE]{\nouppercase{\leftmark}}
\fancyhead[CE]{}
\fancyhead[RE]{\thepage}
\fancyfoot{}
\renewcommand{\headrulewidth}{0.3pt}
\renewcommand{\footrulewidth}{0pt}
\setlength{\headheight}{13.6pt}

%-----------------------------------------------------------------
%   MATHS AND SCIENCE
%-----------------------------------------------------------------
\usepackage{amsmath,amsfonts,amsthm,amssymb}
\usepackage{xfrac}
\usepackage[a]{esvect}
\usepackage{chemformula}
\usepackage{graphicx}

\usepackage[arrowdel]{physics}
    \renewcommand{\vnabla}{\vec{\nabla}}
    % \renewcommand{\vectorbold}[1]{\boldsymbol{#1}}
    % \renewcommand{\vectorarrow}[1]{\vec{\boldsymbol{#1}}}
    % \renewcommand{\vectorunit}[1]{\hat{\boldsymbol{#1}}}
    \renewcommand{\vectorarrow}[1]{\vec{#1}}
    \renewcommand{\vectorunit}[1]{\hat{#1}}
    \renewcommand*{\grad}[1]{\vnabla #1}
    \renewcommand*{\div}[1]{\vnabla \vdot \va{#1}}
    \renewcommand*{\curl}[1]{\vnabla \cp \va{#1}}
    \let\rot\curl

% SI units
\usepackage[separate-uncertainty=true, alsoload=astro, alsoload=hep]{siunitx}
\sisetup{range-phrase = \text{--}, range-units = brackets}
\DeclareSIPrePower\quartic{4}
    \DeclareSIUnit\atm{atm}
    \DeclareSIUnit\atm{cal}

% Smaller trig functions
\newcommand{\Sin}{\trigbraces{\operatorname{s}}}
\newcommand{\Cos}{\trigbraces{\operatorname{c}}}
\newcommand{\Tan}{\trigbraces{\operatorname{t}}}

% Operator-style notation for matrices
\newcommand*{\mat}[1]{\hat{#1}}

% Matrices in (A|B) form via [c|c] option
\makeatletter
\renewcommand*\env@matrix[1][*\c@MaxMatrixCols c]{%
  \hskip -\arraycolsep
  \let\@ifnextchar\new@ifnextchar
  \array{#1}}
\makeatother

% Shorter \mathcal and \mathbb
\newcommand*{\mc}[1]{\mathcal{#1}}
\newcommand*{\mbb}[1]{\mathbb{#1}}

% Shorter ^\ast and ^\dagger
\newcommand*{\sast}{^{\ast}{}}
\newcommand*{\sdag}{^{\dagger}{}}

%-----------------------------------------------------------------
%   OTHER PACKAGES
%-----------------------------------------------------------------
\usepackage{environ}

%Left numbered equations
\makeatletter
    \NewEnviron{Lalign}{\tagsleft@true\begin{align}\BODY\end{align}}
\makeatother

% Plots and graphics
\usepackage{pgfplots}
\usepackage{tikz}
\usepackage{color}
    \makeatletter
        \color{black}
        \let\default@color\current@color
    \makeatother

% Richer enumerate, figure, and table support
\usepackage{enumerate}
\usepackage[shortlabels]{enumitem}
\usepackage{float}
\usepackage{tabularx}
\usepackage{booktabs}
    % \setlength{\intextsep}{8pt}
\numberwithin{equation}{section}
\numberwithin{figure}{section}
\numberwithin{table}{section}

% No indentation after certain environments
\makeatletter
\newcommand*\NoIndentAfterEnv[1]{%
    \AfterEndEnvironment{#1}{\par\@afterindentfalse\@afterheading}}
\makeatother
% \NoIndentAfterEnv{thm}
\NoIndentAfterEnv{defi}
\NoIndentAfterEnv{example}
\NoIndentAfterEnv{table}

% Misc packages
\usepackage{ccicons}
\usepackage{lipsum}

%-----------------------------------------------------------------
%   THEOREMS
%-----------------------------------------------------------------
\usepackage{thmtools}

% Theroems layout
\declaretheoremstyle[
    spaceabove=6pt, spacebelow=6pt,
    headfont=\normalfont,
    notefont=\mdseries, notebraces={(}{)},
    bodyfont=\small,
    postheadspace=1em,
]{small}

\declaretheorem[style=plain,name=Teorema,qed=$\square$,numberwithin=section]{thm}
\declaretheorem[style=plain,name=Corol·lari,qed=$\square$,sibling=thm]{cor}
\declaretheorem[style=plain,name=Lema,qed=$\square$,sibling=thm]{lem}
\declaretheorem[style=definition,name=Definició,qed=$\blacksquare$,numberwithin=section]{defi}
\declaretheorem[style=definition,name=Exemple,qed=$\blacktriangle$,numberwithin=section]{example}
\declaretheorem[style=small,name=Demostració,numbered=no,qed=$\square$]{sproof}

%-----------------------------------------------------------------
%   ELA MOTHERFUCKING GEMINADA
%-----------------------------------------------------------------
\def\xgem{%
    \ifmmode
        \csname normal@char\string"\endcsname l%
    \else
        \leftllkern=0pt\rightllkern=0pt\raiselldim=0pt
        \setbox0\hbox{l}\setbox1\hbox{l\/}\setbox2\hbox{.}%
        \advance\raiselldim by \the\fontdimen5\the\font
        \advance\raiselldim by -\ht2
        \leftllkern=-.25\wd0%
        \advance\leftllkern by \wd1
        \advance\leftllkern by -\wd0
        \rightllkern=-.25\wd0%
        \advance\rightllkern by -\wd1
        \advance\rightllkern by \wd0
        \allowhyphens\discretionary{-}{}%
        {\kern\leftllkern\raise\raiselldim\hbox{.}%
            \kern\rightllkern}\allowhyphens
    \fi
}
\def\Xgem{%
    \ifmmode
        \csname normal@char\string"\endcsname L%
    \else
        \leftllkern=0pt\rightllkern=0pt\raiselldim=0pt
        \setbox0\hbox{L}\setbox1\hbox{L\/}\setbox2\hbox{.}%
        \advance\raiselldim by .5\ht0
        \advance\raiselldim by -.5\ht2
        \leftllkern=-.125\wd0%
        \advance\leftllkern by \wd1
        \advance\leftllkern by -\wd0
        \rightllkern=-\wd0%
        \divide\rightllkern by 6
        \advance\rightllkern by -\wd1
        \advance\rightllkern by \wd0
        \allowhyphens\discretionary{-}{}%
        {\kern\leftllkern\raise\raiselldim\hbox{.}%
            \kern\rightllkern}\allowhyphens
    \fi
}

\expandafter\let\expandafter\saveperiodcentered
    \csname T1\string\textperiodcentered \endcsname

\DeclareTextCommand{\textperiodcentered}{T1}[1]{%
    \ifnum\spacefactor=998
        \Xgem
    \else
        \xgem
    \fi#1}

%-----------------------------------------------------------------
%   PDF INFO AND HYPERREF
%-----------------------------------------------------------------
\usepackage{hyperref}
\usepackage{cleveref}
\hypersetup{colorlinks, citecolor=black, filecolor=black, linkcolor=black, urlcolor=black}

\newcommand*{\mytitle}{Inciació a la física experimental}
\newcommand*{\myauthor}{Alfredo Hernández Cavieres}
\newcommand*{\myuni}{Universitat Autònoma de Barcelona, Departament de Física}
\newcommand*{\mydate}{\normalsize 2012-2013}

\usepackage{hyperxmp}
\hypersetup{pdfauthor={\myauthor}, pdftitle={\mytitle}}

%----------------------------------------------------------------------------------------
%    TITLE SECTION AND DOCUMENT BEGINNING
%----------------------------------------------------------------------------------------
\newcommand{\horrule}[1]{\rule{\linewidth}{#1}}
\title{
    \normalfont
    \small \scshape{\myuni} \\ [25pt]
    \horrule{0.5pt} \\[0.4cm]
    \huge \mytitle \\
    \horrule{2pt} \\[0.5cm]
}
\author{\myauthor}
\date{\mydate}

\begin{document}

\clearpage\maketitle
\thispagestyle{empty}
\addtocounter{page}{-1}

%----------------------------------------------------------------------------------------
%    LLICÈNCIA
%----------------------------------------------------------------------------------------
\section*{}\thispagestyle{empty}
\begin{centering}
    \huge \ccbyncsaeu

    \normalsize Aquesta obra està subjecta a una llicència de

    Reconeixement-NoComercial-CompartirIgual 4.0

    Internacional de Creative Commons.

\end{centering}

%----------------------------------------------------------------------------------------
%    TABLE OF CONTENTS
%----------------------------------------------------------------------------------------
\cleardoubleevenemptypage
\pdfbookmark[1]{\contentsname}{toc}
\tableofcontents

%----------------------------------------------------------------------------------------
%    SECTIONS
%----------------------------------------------------------------------------------------
% \part*{Primera}
% \addcontentsline{toc}{part}{Primera}
    %----------------------------------------------------------------------------------------
%    METROLOGIA
%----------------------------------------------------------------------------------------
\section{Metrologia}
En formen part:
\begin{itemize}
    \item Magnituds físiques: atribut d’un fenomen, cos o substància que pot ser distingit qualitativament i determinat quantitativament.
    \item Els seus valors numèrics: grandària d’una magnitud particular que generalment s'expressa com un número multiplicat per una unitat de mesura.
    \item Unitats.
    \item Patrons: la comparació amb una unitat patró arbitrària permet expressar una mesura amb un cert valor numèric.
    \item Mètodes de mesura.
    \item Avaluació d'incerteses.
\end{itemize}

\subsubsection*{Expressió d'unitats}
\begin{enumerate}[a)]
  \item $(x \pm y)$ unitats \emph{(+ direcció i sentit)} $\equiv$ valor numèric $\pm$ incertesa (valor absolut).
  \item $x$ unitats $\pm y$ \emph{(+ direcció i sentit)} $\equiv$ valor numèric $\pm$ incertesa (valor relatiu).
\end{enumerate}

%----------------------------------------------------------------------------------------
\subsection{Definicions de la GUM}
\paragraph{GUM}
Guide to Uncertainty Measurement.

\paragraph{Valor veritable}
Valor real d'una magnitud al qual només podem aproximar-nos-hi. Es dóna com a {\it resultat de mesura\/} la millor estimació/aproximació al valor veritable.

\paragraph{Errors}
\begin{itemize}
    \item Sistemàtics: errors de procediment (e.g., una cinta mètrica mal calibrada). Són errors evitables.
    \item Aleatoris: errors sense font coneguda (e.g., dilatació d'una cinta mètrica). Són errors inevitables.
\end{itemize}

\paragraph{Exactitud i precisió}
\begin{itemize}
    \item Exactitud: error petit d'una mesura respecte el valor veritable.
    \item Precisió: quan les mesures no difereixen gaire entre si, donant lloc a una incertesa petita. 
\end{itemize}
A una mesura precisa se li pot fer una correcció, fent-la, alhora, exacta.$\Rightarrow$ precisió $\ll$ exactitud.

\paragraph{Xifres significatives i nivell de significació}
\begin{itemize}
    \item Les xifres significatives d’una mesura són aquelles (diferents dels zeros a l’esquerra) de les que estem totalment segurs que no variaran en repetir la mesura.
    \item El nivell de significació ve donat per la posició de la darrera xifra significativa.
\end{itemize}

\paragraph{Incertesa estàndard}
Incertesa del resultat d’una mesura expressat com una desviació estàndard. Dóna una indicació de la probabilitat que el resultat d'un mesura contingui el conjunt dels valors de les mostres experimentals.
\begin{itemize}
    \item Mai podrà ser del 100\%, a causa dels errors aleatoris. 
    \item Si no se n'especifica cap, és del 68,27\%.
\end{itemize}

\paragraph{Avaluació d'incerteses}
\begin{itemize}
    \item Tipus A: mètode d'avaluació de la incertesa a partir de l'anàlisi estadística d'una sèrie d'observacions.
    \item Tipus B: qualsevol altre mètode. (e.g., basar l'incertesa d'una mesura en l'incertesa instrumental especificada pel fabricant).
\end{itemize}

\paragraph{Incertesa estàndard combinada}
Combinació d'incerteses d'una magnitud mesurada indirectament (e.g., propagació d'incerteses per a $L$ i $T$ quan es mesura una velocitat lineal).

\paragraph{Incertesa expandida}
Incertesa diferent ($>$) de l'estàndard, definida per un factor de cobriment $k$ que normalment oscil·la entre 2 i 3.

%----------------------------------------------------------------------------------------
\subsection{Magnituds}
\paragraph{Magnituds fonamentals}
Magnituds independents entre elles que permeten constriur totes les magnituds físiques. Al Sistema Internacional en són 7:
\begin{itemize}
    \item Longitud ($L$), s'expressa en metres ($m$).
    \item Massa ($M$), s'expressa en quilograms ($kg$).
    \item Temps ($T$), s'expressa en segons ($s$).
    \item Intensitat de corrent elèctric ($I$), s'expressa en ampères ($A$).
    \item Temperatura termodinàmica ($\Theta$), s'expressa en kelvins ($K$).
    \item Intensitat lumínica ($J$), s'expressa en candeles ($cd$).
    \item Quantitat de substància ($N$), s'expressa en mols ($mol$).
\end{itemize}
\paragraph{Magnituds derivades}
Són les magnituds construïdes a partir d'una combinació de magnituds fonamentals.

%----------------------------------------------------------------------------------------
\subsection{Anàlisi dimensional}
El resultat d'una mesura expressa el valor d'una magnitud física $G$ com el producte d'un valor numèric $\{ G \}$ per una unitat $[ G ]$. El valor d'una magnitud física és sempre el mateix: $\{ G \} [ G ] = \{ G' \} [ G' ]$ (e.g., 1500 $mm$ = 58,59$"$).

Qualsevol magnitud física pot expressar-se en funció de les dimensions de les unitats fonamentals. $\boxed{[G] = L^{\alpha} M^{\beta} T^{\gamma}  I^{\delta} \Theta^{\varepsilon} J^{\zeta} N^{\eta}}$. 

L'anàlisi dimensional permet, doncs:
\begin{itemize}
    \item Trobar l'equació dimensional d'una magnitud.
    \item Verificar la coherència d'una llei física. 
    \item Trobar l'equació de la relació de proporcionalitat ($\propto$) d'una llei física desconeguda.
\end{itemize}
    %----------------------------------------------------------------------------------------
%    CÀLCUL D'INCERTESES
%----------------------------------------------------------------------------------------
\section{Càlcul d'incerteses}
\subsection{Incertesa d'un conjunt de mostres}

\subsubsection*{Millor estimació del valor real d'una magnitud física}
\begin{align}
    \boxed{\bar{x} = \frac{\sum\limits_{i=1}^{n} x_{i}}{n}}
\end{align}
On $x_{i} \equiv$valors de les mesures i $n \equiv$nre. total de mesures. Associem $\bar{x}$ amb el valor de la magnitud $\approx$ valor real.

\subsubsection*{Variança $\sigma ^{2}$ i desviació estàndard $\sigma$ d'una mostra}
\begin{align}
    \boxed{\sigma ^{2} = \frac{\sum\limits_{i=1}^{n} \left( x_{i} - \bar{x} \right) ^{2}}{n-1}}; \qquad \boxed{\sigma = \sqrt{\sigma ^{2}}}
\end{align}

\subsubsection*{Variança $s^{2}$ i desviació estàndard $s$ de la mitjana}
\begin{align}
    \boxed{s^{2} = \frac{\sigma ^{2}}{n}}; \qquad \boxed{s = \sqrt{s^{2}} = \frac{\sigma}{\sqrt{n}}}
\end{align}

\subsubsection*{Incertesa total $u_{T}$}
\begin{align}
    \boxed{u_{T} = \sqrt{u_{T}^{2}}}; \qquad \boxed{u_{T}^{2} = u_{e}^{2} + u_{i}^{2}}
\end{align}
On $u_{e} \equiv$ incertesa d'origen estadístic ($=s$) i $u_{i} \equiv$ incertesa instrumental (valor més petit de les divisions de l'aparell de mesura.

La ISO recomana utilitzar dues xifres significatives per a les incerteses$\Rightarrow$ el valor es talla tal que tingui el mateix nivell de significació que la incertesa (e.g., $\bar{x} = 9,28823$ $mm$ i $u_{T} = 0,01825$ $mm$ $\Rightarrow$ el valor serà $\left( 9,288 \pm 0,018 \right)$ $mm$.

%----------------------------------------------------------------------------------------
\subsection{Incertesa combinada}
Sigui $y = f(x_{1} , \cdots , x_{n})$. L'incertesa combinada $u_{c}$ de $y$ es calcula de la forma següent:
\begin{align}
    \boxed{u_{c}^{2} \left( y \right) = \sum\limits_{i=1}^{n} \left( \frac{\partial f}{\partial x_{i}} \right) ^{2} u_{x_{i}}^{2}}
\end{align}

%----------------------------------------------------------------------------------------
\subsection{Incertesa expandida}
\begin{align}
    \boxed{U = k \cdot u_{y}}
\end{align}
on, $k$ és el factor de cobriment.
    %----------------------------------------------------------------------------------------
%    REGRESSIÓ LINEAL
%----------------------------------------------------------------------------------------
\section{Regressió lineal}
\subsection{Introducció}
\subsubsection*{Linealització d'una funció}
Sigui $f: X \to Y$, es pot esbrinar $f$? En general, tota relació funcional es pot linealitzar:
\begin{itemize}
    \item Coulomb: $F = f(d) \Rightarrow F = f(\frac{1}{d^{2}})$ (parabòlica $\to$ lineal).
    \item Radioactivitat: $A = A_{0} \exp [- \lambda t] \Rightarrow \ln A = - \lambda t \ln A_{0}$ (exponencial $\to$ lineal).
\end{itemize}

\subsubsection*{Mètode de linealització}
Principi de màxima probabilitat $\Rightarrow$ mètode dels mínims quadrants $\Rightarrow$ regressió lineal.

%----------------------------------------------------------------------------------------
\subsection{Principi de màxima probabilitat}

%----------------------------------------------------------------------------------------
\subsection{Mètode dels mínims quadrants}

%----------------------------------------------------------------------------------------
\subsection{Coeficient de regressió}
    %----------------------------------------------------------------------------------------
%    DISTRIBUCIONS
%----------------------------------------------------------------------------------------
\section{Distribució}
\subsection{Conceptes previs}
\begin{itemize}
    \item Sèrie de mesures: $x_{1}, x_{2}, \dots , x_{i}, \dots , x_{n-1}, x_{n}$.
    \item Residus: $d_{i} = x_{i} - \bar{x}$.
    \item Desviacions: $\varepsilon_{i} = x_{i} - x_{real}$.
\end{itemize}
\begin{align}
    \bar{x} = \frac{\sum\limits_{i=1}^{n} x_{i}}{n}; \quad f_{i}= \frac{\# x_{i}}{n} \Rightarrow \bar{x} = \sum\limits_{i=1}^{n} f_{i} x_{i}
\end{align}

%----------------------------------------------------------------------------------------
\subsection{Distribució binomial}
Experiment: $\begin{cases} A \quad \text{èxit} \to p \equiv P(A) \\ B \quad \text{fracàs} \to q \equiv P(B) \end{cases}$. L'experiment es pot repetir $\Rightarrow$ intents.

$\#$èxits si intentem $n$ vegades l'experiment = probabilitat de tenir $k$ èxits si fem $n$ intents $\equiv P(k)$.
\begin{align}
    P(k) = \begin{pmatrix} n \\ k \end{pmatrix} p^{k} q^{n-k} = \begin{pmatrix} n \\ k \end{pmatrix} p^{k} (1-p)^{n-k} \Rightarrow \sum\limits_{k=0}^{n} P(k) = 1
\end{align}
\begin{align}
    \bar{k} & = \sum\limits_{k=0}^{n} P(k)k = pn \\
    \sigma^{2} & = pn (1-p)
\end{align}
Exemple: tirar un da 10 vegades i treure $x\geq 3$.
\begin{align*}
\begin{split}
    k \to & P(k) \\
    0 \to & 0,0002 \\
    1 \to & 0,0034 \\
    2 \to & 0,0031 \\
    3 \to & 0,0163 \\
    4 \to & 0,0569 \\
    5 \to & 0,1366 \\
    6 \to & 0,2276 \\
    7 \to & 0,2601 \\
    8 \to & 0,1951 \\
    9 \to & 0,0867 \\
    10 \to & 0,0173
\end{split}
\end{align*}
La màxima $P(k)$ és quan $k=7$; $\quad \bar{k} = \frac{2}{3} \cdot 10 = 6,\bar{6}$.

%----------------------------------------------------------------------------------------
\subsection{Distribució de Poisson}
Es realitza el mateix experiment, però $\begin{cases} n \to \infty \\ p \to 0 \end{cases}$ i $pn = \text{finit}$.
\begin{align}
    P(k)= \frac{(pn)^{k} \exp [-pn]}{k!} = \frac{\bar{k}^{k} \exp [-\bar{k}]}{k!}
\end{align}
\begin{align}
    \bar{k} & = pn \\
    \sigma^{2} & = \bar{k} (1-p) = \bar{k}
\end{align}
Exemple: detectar 1 àtom radioactiu entre 10$^{\text{molt}}$ àtoms.

Si fem 10 mesures durant 10 minuts i obtenim el registre de 250 desintegracions $\equiv \bar{k} \Rightarrow \sigma = \sqrt{\bar{k}} = 5$.

%----------------------------------------------------------------------------------------
\subsection{Distribució de Gauss}
S'utilitza quan en comptes de fer servir variables discretes ($\mathbb{N}$) es fan servir variables contínues ($\mathbb{R}$) $\Rightarrow k \to x$. Tanmateix $p \to 0$.
\begin{align}
    P(x)= \frac{1}{\sqrt{2 \pi \bar{x}}} \exp \left[ - \frac{(x - \bar{x})^{2}}{2 \bar{x}} \right] = \frac{1}{\sigma \sqrt{2 \pi}} \exp \left[ - \frac{(x - \bar{x})^{2}}{2 \sigma^{2}} \right]
\end{align}
\begin{align}
    \bar{x} & = pn \\
    \sigma^{2} & = \bar{x}
\end{align}
\begin{figure}[H]
\centering
    % WIP: GRAFIC BONIC
\caption{Gràfic d'una distribució de Gauss}
\end{figure}


%----------------------------------------------------------------------------------------
\end{document}

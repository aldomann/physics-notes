%----------------------------------------------------------------------------------------
%    REGRESSIÓ LINEAL
%----------------------------------------------------------------------------------------
\section{Regressió lineal}
\subsection{Introducció}
\subsubsection*{Linealització d'una funció}
Sigui $f: X \to Y$, es pot esbrinar $f$? En general, tota relació funcional es pot linealitzar:
\begin{itemize}
    \item Coulomb: $F = f(d) \Rightarrow F = f(\frac{1}{d^{2}})$ (parabòlica $\to$ lineal).
    \item Radioactivitat: $A = A_{0} \exp [- \lambda t] \Rightarrow \ln A = - \lambda t \ln A_{0}$ (exponencial $\to$ lineal).
\end{itemize}

\subsubsection*{Mètode de linealització}
Principi de màxima probabilitat $\Rightarrow$ mètode dels mínims quadrants $\Rightarrow$ regressió lineal.

%----------------------------------------------------------------------------------------
\subsection{Principi de màxima probabilitat}

%----------------------------------------------------------------------------------------
\subsection{Mètode dels mínims quadrants}

%----------------------------------------------------------------------------------------
\subsection{Coeficient de regressió}
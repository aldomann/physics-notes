%----------------------------------------------------------------------------------------
%    METROLOGIA
%----------------------------------------------------------------------------------------
\section{Metrologia}
En formen part:
\begin{itemize}
    \item Magnituds físiques: atribut d’un fenomen, cos o substància que pot ser distingit qualitativament i determinat quantitativament.
    \item Els seus valors numèrics: grandària d’una magnitud particular que generalment s'expressa com un número multiplicat per una unitat de mesura.
    \item Unitats.
    \item Patrons: la comparació amb una unitat patró arbitrària permet expressar una mesura amb un cert valor numèric.
    \item Mètodes de mesura.
    \item Avaluació d'incerteses.
\end{itemize}

\subsubsection*{Expressió d'unitats}
\begin{enumerate}[a)]
  \item $(x \pm y)$ unitats \emph{(+ direcció i sentit)} $\equiv$ valor numèric $\pm$ incertesa (valor absolut).
  \item $x$ unitats $\pm y$ \emph{(+ direcció i sentit)} $\equiv$ valor numèric $\pm$ incertesa (valor relatiu).
\end{enumerate}

%----------------------------------------------------------------------------------------
\subsection{Definicions de la GUM}
\paragraph{GUM}
Guide to Uncertainty Measurement.

\paragraph{Valor veritable}
Valor real d'una magnitud al qual només podem aproximar-nos-hi. Es dóna com a {\it resultat de mesura\/} la millor estimació/aproximació al valor veritable.

\paragraph{Errors}
\begin{itemize}
    \item Sistemàtics: errors de procediment (e.g., una cinta mètrica mal calibrada). Són errors evitables.
    \item Aleatoris: errors sense font coneguda (e.g., dilatació d'una cinta mètrica). Són errors inevitables.
\end{itemize}

\paragraph{Exactitud i precisió}
\begin{itemize}
    \item Exactitud: error petit d'una mesura respecte el valor veritable.
    \item Precisió: quan les mesures no difereixen gaire entre si, donant lloc a una incertesa petita. 
\end{itemize}
A una mesura precisa se li pot fer una correcció, fent-la, alhora, exacta.$\Rightarrow$ precisió $\ll$ exactitud.

\paragraph{Xifres significatives i nivell de significació}
\begin{itemize}
    \item Les xifres significatives d’una mesura són aquelles (diferents dels zeros a l’esquerra) de les que estem totalment segurs que no variaran en repetir la mesura.
    \item El nivell de significació ve donat per la posició de la darrera xifra significativa.
\end{itemize}

\paragraph{Incertesa estàndard}
Incertesa del resultat d’una mesura expressat com una desviació estàndard. Dóna una indicació de la probabilitat que el resultat d'un mesura contingui el conjunt dels valors de les mostres experimentals.
\begin{itemize}
    \item Mai podrà ser del 100\%, a causa dels errors aleatoris. 
    \item Si no se n'especifica cap, és del 68,27\%.
\end{itemize}

\paragraph{Avaluació d'incerteses}
\begin{itemize}
    \item Tipus A: mètode d'avaluació de la incertesa a partir de l'anàlisi estadística d'una sèrie d'observacions.
    \item Tipus B: qualsevol altre mètode. (e.g., basar l'incertesa d'una mesura en l'incertesa instrumental especificada pel fabricant).
\end{itemize}

\paragraph{Incertesa estàndard combinada}
Combinació d'incerteses d'una magnitud mesurada indirectament (e.g., propagació d'incerteses per a $L$ i $T$ quan es mesura una velocitat lineal).

\paragraph{Incertesa expandida}
Incertesa diferent ($>$) de l'estàndard, definida per un factor de cobriment $k$ que normalment oscil·la entre 2 i 3.

%----------------------------------------------------------------------------------------
\subsection{Magnituds}
\paragraph{Magnituds fonamentals}
Magnituds independents entre elles que permeten constriur totes les magnituds físiques. Al Sistema Internacional en són 7:
\begin{itemize}
    \item Longitud ($L$), s'expressa en metres ($m$).
    \item Massa ($M$), s'expressa en quilograms ($kg$).
    \item Temps ($T$), s'expressa en segons ($s$).
    \item Intensitat de corrent elèctric ($I$), s'expressa en ampères ($A$).
    \item Temperatura termodinàmica ($\Theta$), s'expressa en kelvins ($K$).
    \item Intensitat lumínica ($J$), s'expressa en candeles ($cd$).
    \item Quantitat de substància ($N$), s'expressa en mols ($mol$).
\end{itemize}
\paragraph{Magnituds derivades}
Són les magnituds construïdes a partir d'una combinació de magnituds fonamentals.

%----------------------------------------------------------------------------------------
\subsection{Anàlisi dimensional}
El resultat d'una mesura expressa el valor d'una magnitud física $G$ com el producte d'un valor numèric $\{ G \}$ per una unitat $[ G ]$. El valor d'una magnitud física és sempre el mateix: $\{ G \} [ G ] = \{ G' \} [ G' ]$ (e.g., 1500 $mm$ = 58,59$"$).

Qualsevol magnitud física pot expressar-se en funció de les dimensions de les unitats fonamentals. $\boxed{[G] = L^{\alpha} M^{\beta} T^{\gamma}  I^{\delta} \Theta^{\varepsilon} J^{\zeta} N^{\eta}}$. 

L'anàlisi dimensional permet, doncs:
\begin{itemize}
    \item Trobar l'equació dimensional d'una magnitud.
    \item Verificar la coherència d'una llei física. 
    \item Trobar l'equació de la relació de proporcionalitat ($\propto$) d'una llei física desconeguda.
\end{itemize}
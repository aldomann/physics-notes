%-----------------------------------------------------------------
%	BASIC DOCUMENT LAYOUT
%-----------------------------------------------------------------
\documentclass[paper=a4, fontsize=12pt, twoside=semi]{scrartcl}
\usepackage[T1]{fontenc}
\usepackage[utf8]{inputenc}
\usepackage{lmodern}
\usepackage{microtype}
\usepackage[catalan]{babel}
\usepackage[fixlanguage]{babelbib}
\selectbiblanguage{catalan}

% Sectioning layout
\addtokomafont{sectioning}{\normalfont\scshape}
\usepackage{tocstyle}
\usetocstyle{standard}
\renewcommand*\descriptionlabel[1]{\hspace\labelsep\normalfont\bfseries{#1}}

% Empty pages
\usepackage{etoolbox}
\pretocmd{\section}{\cleardoubleevenemptypage}{}{}
\pretocmd{\part}{\cleardoubleevenemptypage\thispagestyle{empty}}{}{}
\renewcommand\partheadstartvskip{\clearpage\null\vfil}
\renewcommand\partheadmidvskip{\par\nobreak\vskip 20pt\thispagestyle{empty}}

% Paragraph indentation behaviour
\setlength{\parindent}{0pt}
\setlength{\parskip}{0.3\baselineskip plus2pt minus2pt}
\newcommand{\sk}{\medskip\noindent}

% Fancy header and footer
\usepackage{fancyhdr}
\pagestyle{fancyplain}
\fancyhead[LO]{\thepage}
\fancyhead[CO]{}
\fancyhead[RO]{\nouppercase{\mytitle}}
\fancyhead[LE]{\nouppercase{\leftmark}}
\fancyhead[CE]{}
\fancyhead[RE]{\thepage}
\fancyfoot{}
\renewcommand{\headrulewidth}{0.3pt}
\renewcommand{\footrulewidth}{0pt}
\setlength{\headheight}{13.6pt}

%-----------------------------------------------------------------
%	MATHS AND SCIENCE
%-----------------------------------------------------------------
\usepackage{amsmath,amsfonts,amsthm,amssymb}
\usepackage{xfrac}
\usepackage[a]{esvect}
\usepackage{chemformula}
\usepackage{graphicx}

% SI units
\usepackage[separate-uncertainty=true]{siunitx}
	%\DeclareSIUnit\micron{\micro\metre}

% Custom commands and operators
\newcommand*{\dif}{\mathrm{d}}
\newcommand*{\diff}{\mathop{}\!\mathrm{d}}
\newcommand*{\der}[3][]{\frac{\dif^{#1}#2}{\dif #3^{#1}}}
\newcommand*{\pder}[3][]{\frac{\partial^{#1}#2}{\partial #3^{#1}}}

\newcommand*{\abs}[1]{\left| #1 \right|}
\newcommand*{\avg}[1]{\left< #1 \right>}
\newcommand*{\norm}[1]{\| #1 \|}
\newcommand*{\eval}[1]{\left. #1 \right|}

\newcommand*{\vnabla}{\vec{\nabla}}
\newcommand*{\grad}[1]{\vnabla #1}
\let\divsymb=\div
\renewcommand*{\div}[1]{\vnabla \cdot #1}
\newcommand*{\rot}[1]{\vnabla \times #1}

% Dirac quantum notation
%\newcommand{\ket}[1]{\left| #1 \right>} % for Dirac kets
%\newcommand{\bra}[1]{\left< #1 \right|} % for Dirac bras
%\newcommand{\braket}[2]{\left< #1 \vphantom{#2} \right| \left. #2 \vphantom{#1} \right>} % for Dirac brackets
%\newcommand{\matrixel}[3]{\left< #1 \vphantom{#2#3} \right| #2 \left| #3 \vphantom{#1#2} \right>} % for Dirac matrix elements

% FLA-style notation for matrices
\usepackage{stackengine}
\newcommand*{\vecsign}{\mathchar"017E}
\newcommand*{\dvecsign}{\smash{\stackon[-2.60pt]{\vecsign}{\rotatebox{180}{$\vecsign$}}}}
\newcommand*{\mat}[1]{\def\useanchorwidth{T}\stackon[-5.4pt]{\mathcal{#1}}{\,\dvecsign}}
\stackMath

% Matrices in (A|B) form via [c|c] option
\makeatletter
\renewcommand*\env@matrix[1][*\c@MaxMatrixCols c]{%
  \hskip -\arraycolsep
  \let\@ifnextchar\new@ifnextchar
  \array{#1}}
\makeatother

%-----------------------------------------------------------------
%	OTHER PACKAGES
%-----------------------------------------------------------------
\usepackage{environ}

% Plots and graphics
\usepackage{pgfplots}
\usepackage{tikz}
\usepackage{color}
	\makeatletter
		\color{black}
		\let\default@color\current@color
	\makeatother

% Richer enumerate, figure, and table support
\usepackage{enumerate}
\usepackage{float}
\usepackage{booktabs}
	%\setlength{\intextsep}{8pt}
\numberwithin{equation}{section}
\numberwithin{figure}{section}
\numberwithin{table}{section}

% No indentation after certain environments
\makeatletter
\newcommand*\NoIndentAfterEnv[1]{%
	\AfterEndEnvironment{#1}{\par\@afterindentfalse\@afterheading}}
\makeatother
%\NoIndentAfterEnv{thm}
\NoIndentAfterEnv{defi}
\NoIndentAfterEnv{example}
\NoIndentAfterEnv{table}

% Misc packages
\usepackage{ccicons}
\usepackage{lipsum}

%-----------------------------------------------------------------
%	THEOREMS
%-----------------------------------------------------------------
\usepackage{thmtools}

% Proofatend environment
\makeatletter
\providecommand{\@fourthoffour}[4]{#4}
\newcommand\fixstatement[2][\proofname\space del]{%
	\ifcsname thmt@original@#2\endcsname
		\AtEndEnvironment{#2}{%
			\xdef\pat@label{\expandafter\expandafter\expandafter
				\@fourthoffour\csname thmt@original@#2\endcsname\space\@currentlabel}%
			\xdef\pat@proofof{\@nameuse{pat@proofof@#2}}%
		}%
	\else
		\AtEndEnvironment{#2}{%
			\xdef\pat@label{\expandafter\expandafter\expandafter
				\@fourthoffour\csname #1\endcsname\space\@currentlabel}%
			\xdef\pat@proofof{\@nameuse{pat@proofof@#2}}%
		}%
	\fi
	\@namedef{pat@proofof@#2}{#1}%
}
\globtoksblk\prooftoks{1000}
\newcounter{proofcount}
\NewEnviron{proofatend}{%
	\edef\next{%
		\noexpand\begin{proof}[\pat@proofof\space\pat@label]%
		\unexpanded\expandafter{\BODY}}%
	\global\toks\numexpr\prooftoks+\value{proofcount}\relax=\expandafter{\next\end{proof}}
	\stepcounter{proofcount}}
\def\printproofs{%
	\count@=\z@
	\loop
		\the\toks\numexpr\prooftoks+\count@\relax
			\ifnum\count@<\value{proofcount}%
			\advance\count@\@ne
	\repeat}
\makeatother

% Theroems layout
\declaretheoremstyle[
    spaceabove=6pt, spacebelow=6pt,
    headfont=\normalfont,
    notefont=\mdseries, notebraces={(}{)},
    bodyfont=\small,
    postheadspace=1em,
]{small}

\declaretheorem[style=plain,name=Teorema,qed=$\square$,numberwithin=section]{thm}
\declaretheorem[style=plain,name=Corol·lari,qed=$\square$,sibling=thm]{cor}
\declaretheorem[style=plain,name=Lemma,qed=$\square$,sibling=thm]{lem}
\declaretheorem[style=definition,name=Definició,qed=$\blacksquare$,numberwithin=section]{defi}
\declaretheorem[style=definition,name=Exemple,qed=$\blacktriangle$,numberwithin=section]{example}
\declaretheorem[style=small,name=Demostració,numbered=no,qed=$\square$]{sproof}
\fixstatement{thm}
\fixstatement[Demostració del]{lem}

%-----------------------------------------------------------------
%	ELA MOTHERFUCKING GEMINADA
%-----------------------------------------------------------------
\def\xgem{%
	\ifmmode
		\csname normal@char\string"\endcsname l%
	\else
		\leftllkern=0pt\rightllkern=0pt\raiselldim=0pt
		\setbox0\hbox{l}\setbox1\hbox{l\/}\setbox2\hbox{.}%
		\advance\raiselldim by \the\fontdimen5\the\font
		\advance\raiselldim by -\ht2
		\leftllkern=-.25\wd0%
		\advance\leftllkern by \wd1
		\advance\leftllkern by -\wd0
		\rightllkern=-.25\wd0%
		\advance\rightllkern by -\wd1
		\advance\rightllkern by \wd0
		\allowhyphens\discretionary{-}{}%
		{\kern\leftllkern\raise\raiselldim\hbox{.}%
			\kern\rightllkern}\allowhyphens
	\fi
}
\def\Xgem{%
	\ifmmode
		\csname normal@char\string"\endcsname L%
	\else
		\leftllkern=0pt\rightllkern=0pt\raiselldim=0pt
		\setbox0\hbox{L}\setbox1\hbox{L\/}\setbox2\hbox{.}%
		\advance\raiselldim by .5\ht0
		\advance\raiselldim by -.5\ht2
		\leftllkern=-.125\wd0%
		\advance\leftllkern by \wd1
		\advance\leftllkern by -\wd0
		\rightllkern=-\wd0%
		\divide\rightllkern by 6
		\advance\rightllkern by -\wd1
		\advance\rightllkern by \wd0
		\allowhyphens\discretionary{-}{}%
		{\kern\leftllkern\raise\raiselldim\hbox{.}%
			\kern\rightllkern}\allowhyphens
	\fi
}

\expandafter\let\expandafter\saveperiodcentered
	\csname T1\string\textperiodcentered \endcsname

\DeclareTextCommand{\textperiodcentered}{T1}[1]{%
	\ifnum\spacefactor=998
		\Xgem
	\else
		\xgem
	\fi#1}

%-----------------------------------------------------------------
%	PDF INFO AND HYPERREF
%-----------------------------------------------------------------
\usepackage{hyperref}
\hypersetup{colorlinks, citecolor=black, filecolor=black, linkcolor=black, urlcolor=black}

\newcommand*{\mytitle}{Equacions diferencials}
\newcommand*{\mysubtitle}{}
\newcommand*{\myauthor}{Alfredo Hernández Cavieres}
\newcommand*{\myuni}{Universitat Autònoma de Barcelona, Departament de Física}
\newcommand*{\mydate}{\normalsize 2013-2014}

\pdfstringdefDisableCommands{\def\and{i }}

\usepackage{hyperxmp}
\hypersetup{pdfauthor={\myauthor}, pdftitle={\mytitle}}

%-----------------------------------------------------------------
%	TITLE SECTION AND DOCUMENT BEGINNING
%-----------------------------------------------------------------
\newcommand{\horrule}[1]{\rule{\linewidth}{#1}}
\title{
	\normalfont
	\small \scshape{\myuni} \\ [25pt]
	\horrule{0.5pt} \\[0.4cm]
	\huge \mytitle \\
	%\Large \scshape{\mysubtitle} \\
	\horrule{2pt} \\[0.5cm]
}
\author{\myauthor}
\date{\mydate}

\begin{document}

\clearpage\maketitle
\thispagestyle{empty}
\addtocounter{page}{-1}

%-----------------------------------------------------------------
%	LICENCE
%-----------------------------------------------------------------
\section*{}\thispagestyle{empty}
\begin{centering}
	\href{http://creativecommons.org/licenses/by-nc-sa/4.0/deed.ca}{\huge \ccbyncsaeu}

	\normalsize Aquesta obra està subjecta a una llicència de

	Reconeixement-NoComercial-CompartirIgual 4.0

	Internacional de Creative Commons.

\end{centering}

%----------------------------------------------------------------------------------------
%    TABLE OF CONTENTS
%----------------------------------------------------------------------------------------
\cleardoubleevenemptypage
\pdfbookmark[1]{\contentsname}{toc}
\tableofcontents

%----------------------------------------------------------------------------------------
%    SECTIONS
%----------------------------------------------------------------------------------------
% \part*{Primera}
% \addcontentsline{toc}{part}{Primera}
	%-----------------------------------------------------------------
%	INTRODUCCIÓ
%	!TEX root = ./../main.tex
%-----------------------------------------------------------------
\section{Introducció}
\subsection{Història}
% WIP: 7 wanderers
La setmana té set dies perquè, abans de la invenció del telescopi, es veien set objectes movent-se per l'esfera celeste: els 7 astres errants o planetes, com els hi anomenaven els antics. A cadascun li correspon un dia de la setmana: la Lluna, Mart, Mercuri, Júpiter, Venus, Saturn, i el Sol.

Abans de l'ús del telescopi en l'astronomia, aquesta es va concentrar en estudiar la posició i moviment dels cossos celestes, formats suposadament per la quinta essència.

No obstant, la llei de la gravitació (Newton), l'estudi de les ratlles espectrals, i la composició dels meteorits ens confirmen que la naturalesa dels cossos celestes extraterrestres no difereix essencialment dels terrestres, eliminant la necessitat de la quinta essència.

\subsubsection*{Moviment dels astres: epicicles}
L'epicicle és un dels elements geomètrics bàsics del sistema geocèntric de Claudi Ptolemeu, basat en un Terra que ocupa el centre de l'Univers.
\begin{figure}[h]
	\centering
	\includegraphics[width=0.7\textwidth]{./images/1-epicycles-a}
	% \textcolor{gray}{\rule{8cm}{4cm}}
	\caption{Problema dels epicicles}
	\label{fig:epicycles-a}
\end{figure}

La teoria dels epicicles va ser dissenyada per Apol·loni de Pèrgam a finals del segle III aC i perfeccionada posteriorment per Ptolemeu. En aquest model, per explicar les variacions de velocitat i direcció del moviment aparent dels planetes, tots els cossos celestes es mouen al voltant de la Terra en cercles petits, anomenats \textit{epicicles}, que al seu torn, es mouen sobre cercles majors anomenats \textit{deferents}. Són els cercles deferents els que estan centrats a la Terra. Ptolemeu, a més, per explicar més precisament els moviments planetaris, va introduir el punt \textit{equant} i va desplaçar el centre del cercle deferent del centre de la Terra.
\begin{figure}[H]
	\centering
	\includegraphics[width=0.3\textwidth]{./images/1-epicycles-b}
	\caption{Elements bàsics del sistema planetari de Claudi Ptolemeu}
	\label{fig:epicycles-b}
\end{figure}

% FIXME: copernicus@sec:bio
L'antiga concepció dels epicicles va ser eliminada amb el desenvolupament de la teoria heliocèntrica de Nicolau Copèrnic (1473--1543) i l'explicació del moviment planetari en òrbites el·líptiques va ser donada per Johannes Kepler.

Tant el telescopi òptic, el descobriment de les ratlles espectrals, el radiotelescopi, així com l'aplicació de la física i les matemàtiques a l'estudi dels cossos celestes van suposar avenços especialment notables, com ara:
\begin{itemize}
	\item Naturalesa i evolució de les estrelles i els seus estats finals (nana blanca, estrella de neutrons, forats negres).
	\item Galàxies més enllà de la Via Làctia.
	\item Expansió de l'Univers.
	\item Radiació de fons de microones.
	\item Expansió accelerada de l'Univers.
\end{itemize}

\subsubsection*{Aberració estel·lar}
% FIXME: bradley@sec:bio
La descripció de l'aberració estel·lar per James Bradley (1728) va confirmar per primer cop la teoria heliocèntrica de Copèrnic.

S'entén per aberració estel·lar la petita diferència angular entre la posició aparent d'una estrella al firmament i la seva vertadera posició.

Aquest fenomen es deu a l'efecte conjunt del moviment de l'observador terrestre i el fet que la velocitat de la llum és finita. El desplaçament (en la posició) i la seva direcció depenen de la velocitat de l'observador i de la seva direcció de moviment.
\begin{figure}[h]
	\centering
	\includegraphics[width=0.3\textwidth]{./images/1-aberration}
	\caption{Aberració estel·lar deguda al moviment de la Terra, on (a) és la font de llum real, i (b) és la font de llum aparent}
	\label{fig:aberration}
\end{figure}
\begin{figure}[h]
	\centering
	\includegraphics[width=0.45\textwidth]{./images/1-aberration-analogy}
	\caption{Analogia de l'aberració estel·lar amb un cotxe en moviment: (a) cotxe en repòs, (b) cotxe en moviment}
	\label{fig:aberration-analogy}
\end{figure}

L'aberració anual es deu al moviment orbital de la Terra en torn al Sol. Existeix, a més a més, l'\textit{aberració diürna} (molt més petita) deguda a la rotació de la Terra en torn al seu eix.

L'aberració és més fàcilment advertida per a estrelles llunyanes a l'eclíptica. Bradley va trobar aberracions de quasi $\SI{20.5}{\arcsecond}$ al pol celeste. És immediat veure que
\begin{align}
	\tan \theta = \frac{v}{c}
\end{align}
cosa que va permetre a Bradley a determinar la velocitat de la llum amb un error de només el $\SI{2}{\percent}$.

Coneguda la velocitat de la llum, la relació anterior permet esbrinar la velocitat mitjana orbital de la Terra:
\begin{align*}
	v = c \tan(\Delta \theta_{\max}) = \SI{3 e5}{\km \per \s} \tan(\SI{20.5}{\arcsecond}) \approx \SI{29.8}{\km\per\s}
\end{align*}
Bradley, doncs, va confirmar per primera vegada les idees de Copèrnic i Galileu que la Terra orbita en torn al Sol (i no del revés).
\begin{example}
	Si es coneix el producte $G \Msun$, la distància (mitjana) Terra--Sol es pot determinar mitjançant
	\begin{align*}
		d = \frac{G \Msun}{v^{2}} = \SI{1.49 e13}{\cm} \equiv \SI{1}{\au}
	\end{align*}
\end{example}

\subsubsection*{Mètode}
El mètode de l'astrofísica consisteix en observació (telescopi, radiotelescopi, interferòmetre, espectrògraf, etc.) juntament amb teoria.

Pot ser útil considerar la següent analogia: la nostra observació de l'Univers (a escala temporal) seria equivalent a una nau extraterrestre observant la Terra en la seva òrbita durant uns 2 minuts.

%-----------------------------------------------------------------
\subsection{Contingut de l'Univers}
A primera vista predominen les estrelles de massa $\SI{0.08}{\Msun} \lesssim M \lesssim \SI{100}{\Msun}$ aïllades (freqüentment) o en grups. L'estrella més propera al Sol és Proxima Centauri [\href{http://apod.nasa.gov/apod/ap051204.html}{APOD~051204}], situada a $\approx \SI{4.24}{\lightyear}$.

Hi ha una gran varietat en massa, mida, i lluminositat. Sírius, una estrella doble de la constel·lació del Can, és la més brillant amb una magnitud aparent de $m = -1.46$ [\href{http://apod.nasa.gov/apod/ap000611.html}{APOD~000611}].

\subsubsection*{Medi interestel·lar}
El medi interestel·lar està format per
\begin{itemize}
	\item Núvols de gas i pols de molt baixa densitat.
	\item Raigs còsmics girant en camps magnètics.
	\item Regions H II (e.g., nebulosa del Cranc [\href{http://apod.nasa.gov/apod/ap130905.html}{APOD\footnote{\textit{Astronomy Picture of the Day} (APOD) és un lloc web proporcionat per la NASA i la Universitat Tecnològica de Michigan (MTU). Segons el lloc web, «\textit{Cada dia una imatge diferent o fotografia del nostre Univers apareix, juntament amb una breu explicació escrita per un astrònom professional}». La sintàxi de les entrades del web és \url{http://apod.nasa.gov/apod/apYYMMDD.html}.} 130905}]).
	\item Nebuloses planetàries (e.g., nebulosa de l'Ull de Gat [\href{http://apod.nasa.gov/apod/ap141109.html}{APOD~141109}]).
\end{itemize}

\subsubsection*{Cúmuls estel·lars}
Els grups d'estrelles s'anomenen núvols o cúmuls (clústers) estel·lars. En podem trobar dels següents tipus:
\begin{itemize}
	\item Open clusters $\sim 10^{5}$ estrelles (e.g., Plèiades [\href{http://apod.nasa.gov/apod/ap120903.html}{APOD~120903}]).
	\item Globular clusters $\sim 10^{6}$ estrelles (e.g, estrelles velles).
\end{itemize}
Els cúmuls formen, en conjunt, estructures estel·lars que anomenem supercúmuls (superclusters) (e.g., Coma Supercluster [\href{http://apod.nasa.gov/apod/ap080616.html}{APOD~080616}]).

\subsubsection*{Galàxies}
Són sistemes de $\sim \numrange{e10}{e12}$ estrelles. La Via Làctia pertany al Grup Local (format per més de 54 galàxies, essent la majoria d'elles galàxies nanes).

La Galàxia d'Andròmeda [\href{http://apod.nasa.gov/apod/ap140730.html}{APOD~140730}], a uns $\SI{2}{\mega \lightyear}$ i amb una extensió de $\SI{e5}{\lightyear}$, és la més prominent del Grup Local.

\subsubsection*{Regions H I}
Reben aquest nom núvols molt tènues d'àtoms neutres d'hidrogen en zones interestel·lars. Aquests núvols orbiten en torn al centre de la Galàxia amb velocitat dependent de la distància al centre de la mateixa.

Van ser detectades gràcies al fotó en la transició paral·lel$\to$antiparal·lel dels espins del protó i l'electró de l'àtom. La diferència d'energia entre ambdues configuracions és minúscula ($\Delta E = \SI{6 e-6}{\eV}$). Per això, la longitud d'ona del protó emès sigui aproximadament d'un pam:
\begin{align}
	\lambda = \frac{hc}{\Delta E} = \SI{21.2}{\cm}
\end{align}
Aquesta transició paral·lel--antiparal·lel és poc freqüent, ja que la seva probabilitat és molt baixa ($\approx \SI{2.7 e-15}{\per\s}$).

A l'any 1945, van de Hulst va predir que ones de ràdio de la corresponent freqüència ($\SI{1.40 e9}{\Hz}$) podrien ser detectades, les quals serien indicis de l'existència de les regions H I. Els àtoms en elles estarien al seu nivell fonamental d'energia i formarien un gas fred (entre 10 i 100 kelvin).

La mida dels núvols varia entre 1 i 10 parsecs; el nombre d'àtoms per centímetre cúbic, entre 1 i 100.

Gràcies a l'existència d'aquestes regions, amb l'ajuda de radiotelescopis s'ha pogut conèixer (en bona part) l'estructura de la Galàxia.

\subsubsection*{Altres}
\begin{itemize}
	\item Matèria fosca.
	\item Bany de radiació còsmica a $\SI{2.7}{\K}$ (CBM): COBE, WMAP, Planck (vegeu la figura \ref{fig:wmap-planck}).
	\item Bany de neutrins.
	\item Ones gravitacionals: \textit{Laser Interferometer Space Antenna} (LISA, \url{http://lisa.nasa.gov/}).
\end{itemize}
\begin{figure}[H]
	\centering
	\includegraphics[width=0.7\textwidth]{./images/1-wmap-planck}
	\caption{Comparació de la capacitat de resolució del WMAP (2001) i del \textit{Planck Satellite} (2009) de la radiació còsmica de fons}
	\label{fig:wmap-planck}
\end{figure}

%-----------------------------------------------------------------
\subsection{Instrumentació}
\subsubsection*{Telescopi òptic}
Un telescopi és un sistema òptic que permet veure objectes llunyans, tot ampliant-ne la seva mida angular i la seva lluminositat aparents. Probablement els telescopis són l'eina més important en astronomia i astrofísica. Tot i que amb la paraula \textit{telescopi} hom s'acostuma a referir als telescopis òptics, hi ha telescopis per a gairebé totes les freqüències de l'espectre electromagnètic.

Tot telescopi òptic està format per un objectiu i un ocular. L'objectiu forma una imatge (normalment real) de l'objecte llunyà sobre el seu pla focal; aquesta imatge és llavors ampliada per l'ocular o bé impressionada sobre una pel·lícula fotogràfica o detectada per una càmera CCD (\textit{charge-coupled device}). Si l'objectiu és una lent es parla de telescopi refractor, si l'objectiu és un mirall còncau es parla de telescopi reflector; si utilitza una combinació de lents i miralls s'anomena telescopi catadiòptric.
\begin{figure}[h]
	\centering
	\includegraphics[width=0.6\textwidth]{./images/1-yerkes}
	\caption{Albert Einstein a l'Observatori Yerkes el 6 de maig de 1921}
	\label{fig:yerkes}
\end{figure}

% FIXME: hale@sec:bio
L'Observatori Yerkes (figura \ref{fig:yerkes}) és un observatori astronòmic operat per la Universitat de Chicago. L'observatori, que s'autodenomina \textit{el lloc de naixement de l'astrofísica moderna} va ser fundat al 1897 per George Hale i finançat per Charles Yerkes. Va representar un canvi en la manera de pensar sobre els observatoris, des de ser un mer habitatge per als telescopis i els observadors, a la concepció moderna de l'equip d'observació integrat amb espai de laboratori per a la física i la química.

% FIXME: hubble@sec:bio
% FIXME: chandra@sec:bio
L'observatori té el major telescopi refractor del món utilitzat amb èxit per l'astronomia i una col·lecció de més de 170,000 plaques fotogràfiques. Astrònoms notables han dut a terme investigacions al Yerkes, com ara Edwin Hubble (que va fer el seu treball de postgrau al Yerkes i per a qui el \textit{Hubble Space Telescope} va ser nomenat), Subrahmanyan Chandrasekhar (per a qui el \textit{Chandra Space Telescope} va ser nomenat), el prolífic astrònom rus-americà Otto Struve, i el famós divulgador de l'astronomia Carl Sagan.

\subsubsection*{Radiotelescopi}
Karl Jansky (1931) va descobrir senyals de ràdio de més enllà del Sistema Solar, fonamentalment de la constel·lació de Sagitari. Aquest descobriment va marcar el naixement de la radioastronomia.

Els senyals de ràdio s'originen en la interacció de partícules carregades amb camps electromagnètics.
La unitat que s'utilitza per mesurar la densitat de flux espectral és el Jansky. En particular, al CGS, $\SI{1}{\jansky} = \SI{e-3}{\erg \per \square\m \per \Hz}$.

El disc d'un radiotelescopi reflexa a una antena l'energia de les ones rebudes de la font extraterrestre. A continuació el senyal és amplificat i processat donant lloc a un ràdio-mapa del cel a la longitud d'ona en qüestió.

\subsubsection*{Criteri de Rayleigh}
El criteri de Rayleigh especifica la separació mínima entre dues fonts de llum que es poden resoldre en dos objectes diferents.
\begin{align}
	D \approx 1.22\frac{\lambda}{\theta}
\end{align}
\begin{example}
	Si $\lambda = \SI{21}{\cm}$ (corresponent a una regió H I, per \textit{spin-flip}), per obtenir una resolució de $\SI{1}{\arcsecond} = \SI{4.85 e-6}{\radian}$, caldria un radiotelescopi de diàmetre
	\begin{align*}
	D \approx 1.22 \frac{0.21}{4.85\times 10^{6}} \approx \SI{52.82}{\km}
	\end{align*}
	No obstant això, observem que radiotelescopis moderns com ara del Arecibo Observatory [\href{http://apod.nasa.gov/apod/ap981129.html}{APOD~981129}], té un diàmetre de $D = \SI{300}{\m}$.
\end{example}

\subsubsection*{Interferometria}
\begin{figure}[h]
	\centering
	\includegraphics[width=0.3\textwidth]{./images/1-interferometry}
	\caption{Diagrama simplificat de la tècnica de radiointerferometria}
	\label{fig:interferometry}
\end{figure}
Per obtenir una resolució suficient es recorre a l'interferometria, fent ús de dos o més radiotelescopis. Les ones estan en fase si
\begin{align}
	\begin{cases} L = n \lambda & \text{(màxim)}\\ L = \qty(n - \dfrac{1}{2}) \lambda & \text{(mínim)} \end{cases}
\end{align}
ja que $\sin \theta = \dfrac{L}{d}$. Llavors, es pot determinar amb una molt bona aproximació la posició de la font. Actualment, s'aconsegueixen resolucions de la mil·lèsima de segons d'arc.


El VLA (\textit{Very Large Array}), localitzat a Nou Mèxic, és un conjunt de 27 radiotelescopis disposats en $Y$ sobre rails. Tenen un diàmetre de $D = \SI{25}{\m}$. El senyal de cada radiotelescopi és combinat amb la resta i analitzat mitjançant un ordinador per, finalment, produir un ràdio-mapa d'alta resolució del cel.
\begin{figure}[h]
	\centering
	\includegraphics[width=0.6\textwidth]{./images/1-vla}
	\caption{Fotografia del \textit{Very Large Array}}
	\label{fig:vla}
\end{figure}

%-----------------------------------------------------------------
\subsection{Finestres observacionals}
Per a longituds d'ona fora de l'espectre visible i de ràdio, l'atmosfera es mostra opaca (en diversa mesura) a la radiació electromagnètica (figura \ref{fig:observational-windows}).
\begin{figure}[H]
	\centering
	\includegraphics[width=0.65\textwidth]{./images/1-observational-windows}
	\caption{Informació obtinguda mitjançant l'observació en diferents regions de l'espectre electromagnètic}
	\label{fig:observational-windows}
\end{figure}

\subsubsection*{Zona infraroja}
\begin{itemize}
	\item El vapor d'aigua absorbeix la major part de la radiació infraroja.
	\item Mauna Kea (situat a \SI{4205}{\m} sobre el nivell del mar) és un volcà que, per la seva localització, és un lloc ideal per fer observacions d'infrarojos. De fet el \textit{Canada--France--Hawaii Telescope} (CFHT) és el millor telescopi terrestre d'IR en termes resolució d'imatge.
	\item Globus i satèl·lits.
	\item El telescopi i els instruments annexos han de ser refredats suficientment per evitar que la seva emissió IR contamini el senyal extraterrestre (a $\SI{300}{\K} \Rightarrow \lambda \approx \SI{10}{\micro\m})$ .
	\item \textit{Infrared Astronomy Satellite} (IRAS) (1983): $\SI{10}{\micro\m} \lesssim \lambda \lesssim \SI{12}{\micro\m}$, ha detectat pols orbitant estrelles joves, el que suggereix possible formació de planetes.
\end{itemize}

\subsubsection*{Zona de microones}
\begin{itemize}
	\item \textit{Cosmic Background Explorer} (COBE) (1989): \url{http://lambda.gsfc.nasa.gov/product/cobe/}.
	\item \textit{Wilkinson Microwave Anisotropy Probe} (WMAP) (2001): \url{http://map.gsfc.nasa.gov/}.
	\item \textit{Planck Satellite} (2009): \url{http://www.esa.int/Our_Activities/Space_Science/Planck}.
\end{itemize}

\subsubsection*{Zona ultraviolada}
\begin{itemize}
	\item \textit{International Ultraviolet Explorer} (IUE) (1978): $\SI{1200}{\angstrom} < \lambda$.
	\item \textit{Extreme Ultraviolet Explorer} (EUVE) (1992): $\SI{60}{\angstrom} \leq \lambda \leq \SI{740}{\angstrom} \Rightarrow$ informació sobre nanes blanques i púlsars.
\end{itemize}

\subsubsection*{Zona de raigs $X$}
\begin{itemize}
	\item \textit{Röntgensatellit} (ROSAT) (1990): $\SI{5.1}{\angstrom} \leq \lambda \SI{124}{\angstrom} \Rightarrow$ informació sobre quàsars i estrelles de neutrons.
	\item \textit{Chandra X-Ray Observatory} (CXO) (1999) \url{http://www.nasa.gov/mission_pages/chandra/main/}.
\end{itemize}

\subsubsection*{Zona de raigs $\gamma$}
\begin{itemize}
	\item \textit{Compton Gamma Ray Observatory} (CGRO) (1991) té quatre diferents detectors amb energies als intervals
	\begin{align*}
	\begin{aligned}
		\SI{20}{\keV} &\leq E_{\gamma} \leq \SI{600}{\keV} \\
		\SI{100}{\keV} &\leq E_{\gamma} \leq \SI{10}{\MeV} \\
		\SI{1}{\MeV} &\leq E_{\gamma} \leq \SI{30}{\MeV} \\
		\SI{20}{\MeV} &\leq E_{\gamma} \leq \SI{30}{\GeV} \quad \text{(EGRET).} \\
	\end{aligned}
	\end{align*}
\end{itemize}

\begin{table}[H]
	\centering
		\begin{tabularx}{0.9\textwidth}{cX}
		\toprule
		Longitud d'ona & Objecte característic \\
		\midrule
		Raig Gamma & Objectes compactes en col·lisió (estrelles de neutrons, ...)? \\
		Raig $X$ & Estrelles de neutrons \\
		Ultraviolat & Estrelles calents, quàsars \\
		Visible & Estrelles \\
		Infraroig & Gegants vermelles, nuclis galàctics \\
		IR llunyà & Protoestrelles, pols estel·lar, planetes \\
		Mil·límetre & Pols fred, núvols mol·leculars \\
		Ràdio de \si{\cm} & Línia H I \SI{21}{\cm}, púlsars \\
		\bottomrule
		\end{tabularx}
	\caption{Comparació dels fenòmens característics que són estudiats a diferents longituds d'ona}
	\label{tab:wavelenghts}
\end{table}

    %----------------------------------------------------------------------------------------
%    EQUACIONS DIFERENCIALS DE PRIMER ORDRE
%----------------------------------------------------------------------------------------
\section{Equacions diferencials de primer ordre}
\subsection{Equacions homogènies}
\subsubsection*{Funció homogènia}
\begin{defi}
$F(x,y)$ és homogènia de grau $n \Leftrightarrow F(tx, ty) = t^{n} F(x,y)$, $\quad \forall t$.
\end{defi}
Propietats:
\begin{enumerate}[i)]
    \item Si $F(x,y)$ és homogènia de grau $n$ i $G(x,y)$ és homogènia de grau $m$, llavors $FG$ i $\displaystyle \frac{F}{G}$ són homogènies de grau $n+m$ i $n-m$ respectivament.
    \item Si $F(x,y)$ és de grau zero, llavors $F(x,y)$ és funció únicament de $\displaystyle \frac{y}{x}$.
\end{enumerate}

\subsubsection*{Equació diferencial homogènia}
\begin{defi}
\begin{align}
    M(x,y) \diff x + N(x,y) \diff y = 0
\end{align}
amb $M$, $N$ homogènies del mateix grau.
\end{defi}

\subsubsection*{Resolució}
Es resolen per separació de variables.
\begin{example}
    $\displaystyle \frac{\dif y}{\dif x} = - \frac{M(x,y)}{N(x,y)} = f \left( \frac{y}{x} \right)$ i fent $y = vx$, $\displaystyle v + \frac{\dif v}{\dif x} x = f(v)$

    $\Rightarrow \boxed{\frac{\dif x}{x} = \frac{\dif v}{f(v) - v}}$.
\end{example}

\subsubsection*{Representació gràfica}
\begin{defi}[Homotècia]
$(x,y) \mapsto (kx, ky)$.
\end{defi}
Les corbes integrals es transformen unes en altres mitjançant homotècia.
\begin{example}
 $\displaystyle \frac{\dif y}{\dif x} = - \frac{x^{2} + y^{2}}{2x^{2}} = - \frac{1}{2} - \frac{y^{2}}{2x^{2}}$, funció només de $\displaystyle \frac{y}{x}$. Llavors, fem $\displaystyle v = \frac{y}{x}$ i separem $x, v$: $\displaystyle \frac{\dif v}{- \frac{1}{2} - \frac{v^{2}}{2} - v} = \frac{\dif x}{x} \Rightarrow - \frac{2 \diff v}{(1+v)^{2}} = \frac{\dif x}{x} \Rightarrow \ln x = \frac{2}{1+v} + C$

 $\Rightarrow \boxed{y = \left( \frac{2x}{\ln x + C} - x \right)}$.
\end{example}
\begin{example}
    $\displaystyle ky = \frac{2kx}{\ln kx + C} - kx \Rightarrow \boxed{y = \frac{2x}{\ln k + \ln x - C} - x} \Rightarrow$ fent $C' = C - \ln k$, obtenim una altra corba de la família.
\end{example}
%----------------------------------------------------------------------------------------
\subsection{Equacions lineals}
\begin{defi}
\begin{align}\label{ed-lineal}
   \frac{\dif y}{\dif x} + P(x) y = Q(x)
\end{align}
\end{defi}

\subsubsection*{Resolució}
Observem que $\displaystyle \frac{\dif}{\dif x} \left( y e^{\int P(x) \diff x} \right) = e^{\int P(x) \diff x} \left( \frac{\dif y}{\dif x} + y P(x) \right)$.

A partir de \eqref{ed-lineal} tenim: $\displaystyle e^{\int P(x) \diff x} \left( \frac{\dif y}{\dif x} + P(x) y \right) = Q(x) e^{\int P(x) \diff x}$. Integrant: $\displaystyle y e^{\int P(x) \diff x} = \int Q(x) e^{P(x) \diff x} \diff x + C \Rightarrow$
\begin{align}
    \boxed{y = C e^{-\int P(x) \diff x} + e^{-\int P(x) \diff x} \int Q(x) e^{\int P(x) \diff x} \diff x}
\end{align}

\begin{example}
	$\displaystyle x y' + (1 - x) y = e^{2x}$.

	$\displaystyle \Rightarrow P(x) \equiv \frac{1 - x}{x}, \, Q(x) = \frac{e^{2x}}{x} \Rightarrow \int P(x) \diff x = \ln x - x \Rightarrow y = C \frac{e^{x}}{x} + \frac{e^{x}}{x} \int \frac{x e^{-x}}{x} e^{2x} \diff x$

	$\Rightarrow \boxed{y = C \frac{e^{x}}{x} + \frac{e^{2x}}{x}}$.
\end{example}

\subsubsection*{Propietats de les solucions de l'equació diferencial de 1r ordre}
\begin{enumerate}[i)]
	\item Si $y_{1}$ és una solució particular de la reduïda, la solució general de la reduïda és $y = C y_{1}$.
	\item Si $y_{1}$ és una solució particular de la reduïda i $y_{2}$ una solució particular de la completa, llavors la solució general de la completa és $y = C y_{1} + y_{2}$.
	\item Si $y_{1}$, $y_{2}$ són dues solucions particulars diferents de la completa, la solució general de la completa és $y = y_{2} + C(y_{2} - y_{1})$.
\end{enumerate}
\begin{example}
	$y' + y = 10$, amb una solució particular de la reduïda $y = e^{-x}$; de la completa $y = 10$. Llavors, $\boxed{y = C e^{-x} + 10}$.
\end{example}
\begin{example}
	$\displaystyle y' + \frac{y}{x^{2}} = 2x + 1$, amb una solució particular de la completa $y = x^{2}$. Llavors, $\boxed{y = C e^{\sfrac{1}{x}} + x^{2}}$.
\end{example}

%----------------------------------------------------------------------------------------
\subsection{Equació de Bernoulli}
\begin{defi}
	\begin{align}
		\frac{\dif y}{\dif x} + P(x) y = Q(x) y^{n}
	\end{align}
\end{defi}

\subsubsection*{Resolució}
Aquesta ED ja és lineal per a $n = 0$ i $n = 1$. A la resta de casos, es pot linealitzar amb el canvi de variable $\boxed{z = y^{1-n}}$.

Tenim $\displaystyle \frac{\dif z}{\dif x} = (1-n) y^{-n} \frac{\dif y}{\dif x}$, i l'ED queda: $\displaystyle y^{-n} \frac{\dif y}{\dif x} + P(x) y^{1-n} = Q(x) \Rightarrow$
\begin{align}
    \boxed{\frac{\dif z}{\dif x} + (1-n) P(x) z = (1-n) Q(x)}
\end{align}
\begin{example}
	$\displaystyle \frac{\dif y}{\dif x} + \frac{y}{x} = y^{2} \frac{x \cos x - \sin x}{x}$.

	És una equació de Bernoulli amb $\displaystyle n = 2 \Rightarrow z = \frac{1}{y}$. Llavors, queda $\displaystyle \frac{\dif z}{\dif x} - \frac{z}{x} = \frac{\sin x - x \cos x}{x}.$

	Resolem l'ED lineal: $\displaystyle P(x) = - \frac{1}{x} \Rightarrow e^{- \int P(x) \diff x} = x \Rightarrow z = Cx + x \int \frac{\sin x - x \cos x}{x^{s}} \diff x = Cx - \sin x \Rightarrow \boxed{y = \frac{1}{Cx - \sin x}}$
\end{example}

%----------------------------------------------------------------------------------------
\subsection{Equació de Ricatti}
\begin{defi}
	\begin{align}
		\frac{\dif y}{\dif x} = P(x) y^{2} + Q(x) y + R(x)
	\end{align}
\end{defi}
\subsubsection*{Resolució}
És una ED no resoluble en general, però sí a partir d'una solució particular $y_{p}$. Sigui $\displaystyle y_{p}' = P(x) y^{2}_{p} + Q(x) y_{p} + R(x)$.

Cerquem $\boxed{z = y - y_{p}} \Rightarrow z' = Pz (y + y_{p}) + Qz \Leftrightarrow$
\begin{align}
    \boxed{z' - (2y_{p} P + Q)z = Pz^{2}}
\end{align}
que és una equació de Bernoulli amb $n = 2$, que es resol amb el canvi $\displaystyle w = \frac{1}{z}$.
\begin{example}
	$\displaystyle y' = x^{3} (y-x)^{2} + \frac{y}{x}$, amb $y_{p} = x$.

	Tenim $P(x) = x^{3}$, $\displaystyle Q(x) = -2x^{4} + \frac{1}{x}$, $R(x) = x^{5}$.
	\begin{enumerate}[i)]
		\item Ricatti $\mapsto$ Bernoulli:
			\subitem $\displaystyle z = y - x \Rightarrow z' - \left(2x^{4} - 2x^{4} + \frac{1}{x} \right) z  = x^{3} z^{2} \Rightarrow z' - \frac{z}{x} = x^{3} z^{2} \Rightarrow P = - \frac{1}{x}, \, Q = x^{3}, \, n=2$.
		\item Bernoulli $\mapsto$ lineal:
			\subitem $\displaystyle w = \frac{1}{z} \Rightarrow w' + \frac{w}{x} = -x^{3}$.
			\subitem Resolem: $\displaystyle \Rightarrow w = \frac{C}{x} - \frac{x^{4}}{5}$.
		\item Retorn $w \mapsto z \mapsto y$.
			\subitem $\boxed{y = \frac{5x}{5C - x^{5}}}$.
	\end{enumerate}
\end{example}

%----------------------------------------------------------------------------------------
\subsection{Equacions exactes}
\begin{defi}
	\begin{align}
    \begin{gathered}
		M(x,y) \diff x + N(x,y) \diff y = 0 \text{ és exacta} \Leftrightarrow \\
        \exists f(x,y) \mid \dif f \equiv M \diff x + N \diff y
    \end{gathered}
	\end{align}
\end{defi}
\begin{thm}[d'Euler]
	Una ED de la forma $M(x,y) \diff x + N(x,y) \diff y = 0$ és exacta si i només si $\displaystyle \frac{\partial M}{\partial y} = \frac{\partial N}{\partial x}$.
\end{thm}

\subsubsection*{Resolució}
Si l'ED és exacta, llavors $\dif f = 0$. Així doncs,
\begin{align}
    \boxed{f(x,y) \equiv C = \int_{a}^{x} M(x,y) \diff x + \int N(a,y) \diff y}
\end{align}
Alternativament, podem considerar el següent: $\displaystyle f = \int M(x,y) \diff x + g(y)$ i $\displaystyle g'(y) \equiv N(x,y) - \frac{\partial}{\partial y} \left( \int M(x,y) \diff x \right)$. Llavors,
\begin{align}
	\boxed{C = \int M \diff x + \int \left( N - \frac{\partial}{\partial y} \int M \diff x \right) \diff y}
\end{align}

\begin{example}
	$\displaystyle \left( \frac{y}{x} + y^{3} \right) \diff x + \left( \ln x + 3x y^{2} + 4y \right) \diff y = 0$.

	Comprovem que sigui exacta: $\displaystyle \frac{\partial M}{\partial y} = \frac{1}{x} + 3y^{2} = \frac{\partial N}{\partial x} = \frac{1}{x} + 3y^{2}$.

	Tenim  $\displaystyle f(x,y) = \int_{a}^{x} M(x,y) \diff x + \int N(a,y) \diff y \Rightarrow f = y \ln x + x y^{3} + 2y^{2}$. La solució és $\boxed{y \ln x + x y^{3} + 2y^{2} = C}$.

    Altrament podem fer $f = y \ln x + xy^{3} + g(y) \Rightarrow g'(y) = \ln x + 3xy^{2} + 4y - \ln x  3xy^{2} = 4y \Rightarrow g(y) = 2y^{2}$. La solució és $\boxed{y \ln x + x y^{3} + 2y^{2} = C}$.
\end{example}

%----------------------------------------------------------------------------------------
\subsection{Factors integrants}
\begin{defi}
	Una funció $\mu(x,y)$ és factor integrant de l'equació $M \diff x + N \diff y = 0 \Leftrightarrow$
	\begin{align}
		\mu M(x,y) \diff x + \mu N(x,y) \diff y = 0 \text{ és exacta}
	\end{align}
\end{defi}
\begin{example}
	$2y \diff x + x \diff y = 0$ no és exacta, però agafant $\mu = x \Rightarrow 2xy \diff x + x^{2} \diff y = 0$ és exacta.
\end{example}
Propietat: $\exists$ sempre una infinitat de factors integrants. Ja que $\exists$ una solució $f(x,y) = C$, es té $\displaystyle \frac{\partial f}{\partial x} \diff x + \frac{\partial f}{\partial y} \diff y = 0$. Llavors, només cal fer:
\begin{align}
	\boxed{\mu = \frac{\partial f / \partial x}{M} = \frac{\partial f / \partial y}{N}}
\end{align}
\begin{example}
	Per a $2y \diff x + x \diff y = 0$, qualsevol factor de la forma $x F(x^{2}y)$ és factor integrant.
\end{example}

\subsubsection*{Mètode per trobar $\mu$}
Segons convingui s'ha de trobar una funció $G$ o una funció $H$ tal que $e^{\int G}$ o $e^{\int H}$ sigui factor integrant.
\begin{align}
	\boxed{G \equiv \frac{1}{N} \left( \frac{\partial M}{\partial y} - \frac{\partial N}{\partial x} \right) \Rightarrow \mu = \mu = e^{\int G}}
\end{align}
\begin{align}
	\boxed{H \equiv \frac{1}{M} \left( \frac{\partial N}{\partial x} - \frac{\partial M}{\partial y} \right) \Rightarrow \mu = \mu = e^{\int H}}
\end{align}
\begin{example}
	$(4x^{2} + y) \diff x - x \diff y = 0$ no exacta.

	$\displaystyle G = - \frac{1}{x} (1 - (-1)) = - \frac{2}{x} \Rightarrow \mu = e^{-2 \ln x} = \frac{1}{x^{2}}$

	Resolent la nova equació exacta, obtenim: $\boxed{4x - \frac{y}{x} = C}$.
\end{example}
%----------------------------------------------------------------------------------------
\subsection{Equacions de 2n ordre resoltes per mètodes de 1r ordre}
Les ED de la forma $\displaystyle f \left( x, y, \frac{\dif y}{\dif x}, \frac{\dif^{2} y}{\dif x^{2}} \right) = 0$ es resol per mètodes de 1r ordre ens dos casos:
\begin{enumerate}[i)]
	\item $\displaystyle f \left( x, \frac{\dif y}{\dif x}, \frac{\dif^{2} y}{\dif x^{2}} \right) = 0$, només cal fer el canvi $\displaystyle v = \frac{\dif y}{\dif x}$.
	\item $\displaystyle f \left( y, \frac{\dif y}{\dif x}, \frac{\dif^{2} y}{\dif x^{2}} \right) = 0$, només cal fer el canvi $\displaystyle v = \frac{\dif y}{\dif x}$ i $\displaystyle v \frac{\dif v}{\dif y} = \frac{\dif^{2} y}{\dif x^{2}}$.
\end{enumerate}
\begin{example}
	$\displaystyle y \frac{\dif^{2} y}{\dif x^{2}} = 2 \left( \frac{\dif y}{\dif x} \right)^{2}$.

	Fent els canvis de variable, tenim $\displaystyle y v \frac{\dif v}{\dif y} = 2v^{2}$. Separant les variables i integrant, obtenim $2 \ln y = \ln v + A$. Fent $\ln B = A$ i desfent els logaritmes tenim $y^{2} = Bv$.

	Desfem el canvi de variable i obtenim $\displaystyle y^{2} = B \frac{\dif y}{\dif x} \Rightarrow \dif x = B \frac{\dif y}{y^{2}} \Rightarrow x = - \frac{B}{y} + C \Rightarrow y = - \frac{B}{x-C} \Rightarrow \boxed{y = \frac{1}{C_{1} x + C_{2}}}$.
\end{example}

%----------------------------------------------------------------------------------------
\subsection{Equació de Clairaut}
\begin{defi}
	Sigui la família de rectes no paral·leles a un paràmtre.
	\begin{align}
		y = Cx + f(C)
	\end{align}
	L'ED més general que compleix aquesta solució és la que s'obté fent $\displaystyle \frac{\dif y}{\dif x} = C$:	\begin{align}
		y = \frac{\dif y}{\dif x} x + f\left( \frac{\dif y}{\dif x} \right)
	\end{align}
\end{defi}

\begin{example}\label{sol-sing}
	$\displaystyle y = x \frac{\dif y}{\dif x} - \frac{1}{4} \left( \frac{\dif y}{\dif x} \right)^{2}$. La solució és $\boxed{y = Cx - \frac{1}{4} C^{2}}$.
\end{example}
\begin{example}
	$\displaystyle y = x \frac{\dif y}{\dif x} - \frac{\dif x}{\dif y}$. La solució és $\boxed{y = Cx - C^{-1}}$.
\end{example}
\begin{example}
	$\displaystyle \left(y - x \frac{\dif y}{\dif x} \right) + \frac{\dif y}{\dif x} = \left( \frac{\dif y}{\dif x} \right)^{2}$.

    La solució és $\boxed{y = Cx \pm \sqrt{C^{2} - C}}$.
\end{example}

\begin{defi}[Solució singular]
	És una solució adicional de l'ED, no inclosa en la solució general.
\end{defi}
\begin{example}
	A l'equació de l'exemple~\ref{sol-sing}, $\boxed{y = x^{2}}$ és una solució singular.
\end{example}

\subsubsection*{Corba envolvent}
\begin{defi}
	És una corba tangent en cadascun dels seus punts a alguna de les rectes. Si $\exists$ l'envolvent, aquesta és solució de l'ED.
\end{defi}
Una condició necessària de l'envolvent és que $\displaystyle x = - \frac{\dif f(C)}{\dif C}, \quad \forall x$. Com que l'envolvent també verifica $y = Cx + f(C)$, l'envolvent s'obté aïllant $C$ a les dues equacions.
\begin{example}
	$y = Cx - C^{-1}$.

	$\displaystyle x = \frac{\dif}{\dif C} (C^{-1}) = - C^{-2} \Rightarrow C^{2} = -x^{-1}$.

	Pel binomi de Newton $\Rightarrow y^{2} = C^{2} x^{2} - 2x + C^{-2} \Rightarrow \boxed{y^{2} = -4x}$ és l'envolvent.
\end{example}
\begin{example}[amb dues envolvents]
	$\displaystyle y = \frac{\dif y}{\dif x} x - \frac{1}{3} \left( \frac{\dif y}{\dif x} \right)^{3} \Rightarrow$

    $\boxed{y = Cx - \frac{1}{3} C^{3}}$ és la solució general, però $\boxed{y = \pm x^{\sfrac{3}{2}}}$ són dues solucions singulars.
\end{example}
Observació: les solucions que trobem per aquest mètode són candidats a envolvents; s'ha de comprovar directament a l'ED per substitució.

%----------------------------------------------------------------------------------------
\subsection{Família de corbes a $n$ paràmetres}
\begin{defi}
	Una relació de la forma $F(x,y,C_{1}, \dots, C_{n}) = 0$ representa una família de corbes a $n$ paràmetres. Satisfan una ED d'ordre $n$ (si els $n$ paràmetres són essencials), la qual s'obté derivant $n$ vegades, és a dir, fins a $\displaystyle F(x,y,C_{1}, \dots, C_{n}, \frac{\dif y}{\dif x}, \dots , \frac{\dif^{n} y}{\dif x^{n}})$ i eliminant $C_{1}, \dots , C_{n}$ entre aquestes $n+1$ relacions.
\end{defi}
\begin{example}
	ED de tots els cercles de radi unitat: $(x-a)^{2} + (y-b)^{2} = 1$.
	\begin{enumerate}[i)]
	    \item $\displaystyle 2(x-a) \diff x + 2 (y-b) \diff y = 0 \Rightarrow x-a = (b-y) \frac{\dif y}{\dif x}$.
	    \item $\displaystyle 1 = -1 \left( \frac{\dif y}{\dif x} \right)^{2} + (b-y) \frac{\dif^{2} y}{\dif x^{2}} \Rightarrow y-b = \frac{1+y'^{2}}{-y''}$.
	    \item Substituint l'ED del pas i) a l'equació del cercle, obtenim $\displaystyle (y-b)^{2} \left[ 1 + \left( \frac{\dif y}{\dif x} \right) \right]$.
	    \item Finalment, considerant els resultats dels passos ii) i iii), obtenim $\boxed{(1 + y'^{2})^{3} = y''^{2}}$.
	\end{enumerate}
	Observem, llavors, que el radi de curvatura d'una corba és $\boxed{\frac{(1 + y'^{2})^{\sfrac{3}{2}}}{|y''|}}$.
\end{example}

    %----------------------------------------------------------------------------------------
%    EQUACIONS DIFERENCIALS LINEALS
%----------------------------------------------------------------------------------------
\section{Equacions diferencials lineals}
\subsection{Equacions reduïdes i completes}
\begin{defi}[Equació diferencial d'ordre $n$]
    \begin{align}\label{ed-lineal-n}
        \frac{\dif^{n} y}{\dif x^{n}} + P_{1}(x) \frac{\dif^{n-1} y}{\dif x^{n-1}} + \cdots + P_{n-1}(x) \frac{\dif y}{\dif x} + P_{n}(x) y = R(x)
	\end{align}
	\begin{itemize}
		\item Si $R(x) = 0$, es tracta d'una ED reduïda.
		\item Si $R(x) \neq 0$, es tracta d'una ED completa.
	\end{itemize}
\end{defi}

\subsubsection*{Propietats de les equacions reduïdes i completes}
\begin{enumerate}[i)]
    \item Si una solució de la reduïda s'anul·la, al igual que les seves $n-1$ derivades en algun punt $x_{0}$, aquesta solució és $y(x) \equiv 0$.
	\item Si $u_{1}, \dots , u_{k}$ són solucions de la reduïda, també ho són $Cu_{1}, \dots , Cu_{k}$.
	\item Si $y_{1}, y_{2}$ són solucions de la completa, $y_{2} - y_{1}$ és solució de la reduïda.
\end{enumerate}
\begin{cor}
    Si $y_{1}$ és una solució particular de la completa, i $u_{1}$ la solució general de la reduïda, llavors $y_{1} + u_{1}$ és la solució general de la completa.
\end{cor}
Sabem per resultats de capítols anteriors que:
\begin{itemize}
	\item Solució general d'una ED d'ordre $n$: funció $y(x)$ amb $n$ constants arbitràries.
	\item $\exists! y(x)$ amb valors predeterminats per a $y(x_{0}), y'(x_{0}), \dots , y^{(n-1)}(x_{0})$.
\end{itemize}

\begin{thm}
	La solució general d'una ED lineal completa s'obté sumant la solució general de la reduïda i una solució particular de la completa.
\end{thm}
\begin{defi}[Wronskià]
	\begin{align}
		W(u_{1}(x), \dots , u_{n}(x)) \equiv \begin{vmatrix} u_{1} & \dots & u_{n} \\ u'_{1} & \dots & u'_{n} \\ \vdots & & \vdots \\ u^{(n-1)}_{1} & \dots & u^{(n-1)}_{n} \end{vmatrix} \equiv f(x)
	\end{align}
\end{defi}
\begin{thm}
	Si $n$ funcions són linealment dependents (LD) i $\exists$ les seves derivades fins a $n-1$, el seu wronskià és $\equiv 0$.
\end{thm}
\begin{thm}
	Tota solució de l'ED reduïda es pot expressar com a combinació lineal de $n$ solucions linealment independents (LI) de la reduïda.
\end{thm}
\begin{cor}
	La forma de trobar la solució general de l'ED reduïda és trobar $n$ solucions particulars LI (mitjançant el wronskià és fàcil comprovar si ho són).
\end{cor}

%----------------------------------------------------------------------------------------
\subsection{Equacions reduïdes amb coeficients constants}
\begin{defi}
	\begin{align}\label{ed-red-ct-n}
		\frac{\dif^{n} y}{\dif x^{n}} + p_{1} \frac{\dif^{n-1} y}{\dif x^{n-1}} + \dots + p_{n-1} \frac{\dif y}{\dif x} + p_{n} y = 0
	\end{align}
\end{defi}

\subsubsection*{Resolució}
\begin{defi}[Equació auxiliar]
	$\exists$ solucions de l'ED~\eqref{ed-red-ct-n} de la forma $y = e^{mx}$ on $m$ és arrel de l'equació auxiliar $m^{n} + p_{1} m^{n-1} + \dots + p_{n-1} m + p_{n} = 0$.
\end{defi}
Segons com siguin les arrels de l'equació auxiliar, tindrem diferents solucions:
\begin{enumerate}[i)]
	\item Si són $n$ arrels reals diferents: $m_{1}, \dots , m_{n}$.
		\subitem $\Rightarrow \boxed{y = C_{1}e^{m_{1}x} + \dots + C_{n}e^{m_{n}x}}$ és solució.
	\item Si són $n$ arrels reals, però alguna múltiple: $m_{k}$ amb multiplicitat $r$.
		\subitem Se substitueixen els $r$ sumands amb $e^{m_{k}x}$ de la solució anterior per
		\subitem $\boxed{(B_{1} + B_{2} x + \dots + B_{r} x^{r-1}) e^{m_{k}x}}$.
	\item Si hi ha alguna arrel complexa: $m = \alpha \pm \beta \imath$.
		\subitem Se substitueixen els termes $C e^{(\alpha + \beta \imath) x} + D e^{(\alpha + \beta \imath) x}$ per $\boxed{e^{\alpha x} (A \cos \beta + B \sin \beta)}$.
	\item Si hi ha alguna arrel complexa múltiple: $m_{k} = \alpha \pm \beta \imath$ arrel de multiplicitat $r$
		\subitem Se substitueixen $2r$ termes de la primera solució per
		\subitem $\boxed{e^{\alpha x} \left[ (A_{1} + A_{2} x + \dots + A_{r} x^{r-1})\cos \beta x \right]}$
		\subitem $+ \boxed{e^{\alpha x} \left[ (B_{1} + B_{2} x + \dots + B_{r} x^{r-1})\sin \beta x \right]}$.
\end{enumerate}

\begin{example}
	$y'' + 4y' + 4y = x + 1 \Rightarrow m^2 + 4m + 4 = 0 \Rightarrow m = -2$ (doble).

	$\Rightarrow \boxed{y = e^{-2x} (A + Bx)}$ és solució de la reduïda.
\end{example}
\begin{example}
	$y^{(4)} + 5y'' + 4y = 0 \Rightarrow m^{4} + 5m^{2} + 4 = 0 \Rightarrow m_{1} = \pm \imath, m_{2} = \pm 2 \imath$.

	$\Rightarrow \boxed{y = A \cos x + B \sin x + C \cos 2x + D \sin 2x}$ és solució.
\end{example}

%----------------------------------------------------------------------------------------
\subsection{Equacions completes amb coeficients constants}
\begin{defi}
	\begin{align}\label{ed-comp-ct-n}
		\frac{\dif^{n} y}{\dif x^{n}} + p_{1} \frac{\dif^{n-1} y}{\dif x^{n-1}} + \dots + p_{n-1} \frac{\dif y}{\dif x} + p_{n} y = R(x)
	\end{align}
\end{defi}
\subsubsection*{Mètode dels anihiladors}
\begin{defi}
    Emprant la notació d'Euler ($y^{(n)} = D^{n}y$), l'equació~\eqref{ed-comp-ct-n} s'escriurà:
    \begin{align}
        L(y) = (D^{n} + p_{1} D^{n-1} + \dots + p_{n-1} D + p_{n}) y = R(x)
    \end{align}
    On $L$ és l'operador lineal diferencial $D^{n} + p_{1} D^{n-1} + \dots + p_{n-1} D + p_{n}$. És fàcil veure que $L = (D-m_{1}) (D-m_{2}) \dots (D-m_{n})$, on $m_{i}$ śon les arrels de l'equació auxiliar.
\end{defi}
\begin{defi}[Operador anihilador]
Un operador anihilador és un operador lineal $L_{i}$ que anihila $R(x)$, és a dir $L_{1} L_{2} \dots L_{k} R(x) = 0$.
\begin{itemize}
    \item $\boxed{(D-\alpha)^{n+1}}$ anihila les funcions que tenen forma de $\boxed{x^{n} e^{\alpha x}}$.
    \item $\boxed{(D^{2}-2\alpha D + \alpha^{2} + \beta^{2})^{n+1}}$ anihila les funcions que tenen forma de

    $\boxed{x^{n} e^{\alpha x} (C_{1} \cos \beta x + C_{2} \sin \beta x)}$.
\end{itemize}
\end{defi}
\subsubsection*{Resolució}
\begin{enumerate}[i)]
    \item Solucionar l'ED reduïda $L(y) = 0$.
    \item Multiplicar pels operadors anihiladors a ambdues bandes de l'ED:

    $L'(y) \equiv (L_{1}L_{2} \dots L)(y) = L_{1}L_{2} \dots R(x) \equiv 0$.
    \item Trobar les arrels de $L'$ i expressar solució general corresponent a l'equació $L'(y) = 0$.
    \item Els sumands de la solució de $L'(y) = 0$ que no siguin ja a la solució de la reduïda són la solució particular de la completa que busquem. És a dir, tenim $y_{p} = C_{1} P_{1}(x) + C_{2} P_{2}(x) + \dots$.
    \item Substituir $y$ de l'ED inicial per $y_{p}$, aplicar l'operador $L$ i trobar els valors de les constants $C_{i}$ per tal que es compleixi $L(y_{p}) = R(x)$.
\end{enumerate}
\begin{example}
    $y''' - y' = x e^{x} \Rightarrow (D^{3} - D) y = xe^{x}$

    Reduïda: $(D^{3} - D) = D (D-1) (D+1) = 0 \Rightarrow \boxed{y = A + Be^{x} + Ce^{-x}}$.

    L'operador que anihila $x e^{x}$ és $(D-1)^{2} \Rightarrow D (D-1)^{3} (D+1) = 0$. Llavors tenim $m_{1} = 0$, $m_{2} = 1$ (triple), $m_{3} = -1$. La solució general és $\boxed{y = (B + Ex + Fx^{2})e^{x} + A + Ce^{-x}}$.

    $\Rightarrow \boxed{y_{p} = Exe^{x} + Fx^{2} e^{x}} \Rightarrow (D^{3}-D) (Exe^{x} + Fx^{2} e^{x}) = xe^{x} = (2E + 6F)e^{x} + 4Fxe^{x}$.

    $\displaystyle \Rightarrow E = -\frac{3}{4}, F = \frac{1}{4} \Rightarrow \boxed{y_{p} = -\frac{3}{4} xe^{x} + \frac{1}{4} x^{2} e^{x}}$.
\end{example}

%----------------------------------------------------------------------------------------
\subsection{Mètode de la variació de paràmetres (2n ordre)}
En general és un mètode vàlid $\forall$ funció $R(x)$. A més no es limita al cas dels coeficients constants, sinó que és vàlid sempre que s'hagi resolt l'ED reduïda.
\begin{defi}
	\begin{align}\label{ed-var-par-2}
		y'' + P(x) y' + Q(x) y = R(x)
	\end{align}
\end{defi}

\subsubsection*{Resolució}
Siguin $u_{1}, u_{2}$ solucions LI de la reduïda ($u''_{i} + Pu'_{i} + Q_{i} = 0$). Cerquem una solució de la concreta de la forma
\begin{align*}
	y = t_{1}(x) u_{1} + t_{2}(x) u_{2}
\end{align*}
i, a més, compleix la següent propietat: $t'_{1}(x) u_{1} + t'_{2}(x) u_{2} = 0$.
\\
Derivant, substituint a~\eqref{ed-var-par-2}, tenim: $t'_{1}(x) u'_{1} + t'_{2}(x) u'_{2} = R \Rightarrow \exists t_{1}, t_{2} \mid y = t_{1}(x) u_{1} + t_{2}(x) u_{2}$. Així doncs, tenim el següent sistema d'equacions algebràiques:
\begin{align*}
	\begin{cases} t'_{1}(x) u_{1} + t'_{2}(x) u_{2} = 0 \\ t'_{1}(x) u'_{1} + t'_{2}(x) u'_{2} = R \end{cases}
\end{align*}
La solució del sistema és $\displaystyle t'_{1} = - \frac{Ru_{1}}{W}, t'_{2} = \frac{Ru_{2}}{W}$, on el wronskià val $W = u_{1} u'_{2} - u'_{1} u_{2}$. Integrant les $t'_{i}$ respecte $x$, tenim:
\begin{align}
	\boxed{y_{p} = - u_{1}(x) \int_{a}^{x} \frac{R(x) u_{2}(x)}{W(x)} \diff x + u_{2}(x) \int_{a}^{x} \frac{R(x) u_{1}(x)}{W(x)} \diff x}
\end{align}

\begin{example}
    $\displaystyle y'' + y = \frac{1}{\sin x}$. $m = \pm \imath$.
    La solució de la reduïda és $\boxed{y = A \cos x + B \sin x}$. Llavors definim $u_{1} = \cos x$ i $u_{2} = \sin x$.

    $W(u_{1}(x), u_{2}(x)) = \cos^{2} x + \sin^{2} x = 1 \neq 0$.

    $\displaystyle y_{p} = \cos x \int_{a}^{x} \frac{\sin x}{\sin x} \diff x + \sin x \int_{a}^{x} \frac{\cos x}{\sin x} \diff x$

    $y_{p} = \cos x (a-x) + \sin x \ln (\sin x) - \sin x (\ln (\sin a))$,

    però $a \cos x$ i  $\sin x (\ln (\sin a))$ s'absorveixen en la solució de la reduïda.

    Llavors, $\boxed{y_{p} = - x \cos x + \sin x \ln (\sin x)}$ és una solució particular de la completa.

    $\Rightarrow \boxed{y = A \cos x + B \sin x - x \cos x + \sin x \ln (\sin x)}$.
\end{example}

%----------------------------------------------------------------------------------------
%\subsection{Mètode de la variació de paràmetres (ordre $n$)} %WIP

%----------------------------------------------------------------------------------------
\subsection{Reducció de l'ordre d'una equació}
\subsubsection*{Equació lineal de 2n ordre}
Sigui $y'' + P(x) y' + Q(x) y = R(x)$ i sigui $u(x)$ no $\equiv 0$ una solució de la reduïda.

Fem el canvi de variable $\boxed{y = tu}$. Llavors $y' = tu' + t'u$, $y'' = tu'' + 2t'u' + t''u$.

Substituïm en l'ED completa: $t(u'' + Pu' + Qu) + t'(2u' + Pu) + t'' u = R$. Com que $u$ és solució de la reduïda, $u'' + Pu' + Qu = 0 \Rightarrow t(u'' + Pu' + Qu) + t'' u = R$ i fem el canvi de variable $\boxed{v = t'} \Rightarrow$
\begin{align}
    \boxed{v(2u' +P(x)u) + vu' = R(x)}
\end{align}
Llavors, determinem $v$, desfem el canvi per trobar $t$ i tornem a desfer el canvi per trobar la solució de $y(x)$.
\begin{example}
$x y'' - 2 (x + 1) y' + (x+2) y = x^{3} e^{2x}$, amb $u = e^{x}$.

$\displaystyle \Rightarrow v \left( 2 - 2 \frac{x + 1}{x} \right) + v' = x^{2} e^{x} \Rightarrow - \frac{2v}{x} + v' = x^{2} e^{x} \Rightarrow \frac{v' x^{2} - 2vx}{x^{4}} = e^{x} \Rightarrow e^{x} = \frac{\dif}{\dif x} \left( \frac{v}{x^{2}} \right) \Rightarrow \frac{v}{x^{2}} = e^{x} + C \Rightarrow \boxed{v = C x^{2} + x^{2} e^{x}}$

$\displaystyle t = (x^{2} - 2x + 2) e^{x} + \frac{C}{3} + D \Rightarrow \boxed{ y = \left[ (x^{2} - 2x + 2)e^{x} + \frac{C}{3} + D \right] e^{x}}$.
\end{example}
Altrament, si tenim $y'' + P(x) y' + Q(x) y = R(x)$ i $y_{1}(x)$ no $\equiv 0$ és una solució. Llavors, $y_{2} = y_{1} t$ és una altra solució linealment independent respecte $y_{1}$.

En particular, $\displaystyle y_{2} = y_{1} \int \frac{e^{-\int P(x) \diff x}}{y_{1}^{2}} \diff x$. I per tant, la solució general és $\boxed{y = C_{1}y_{1} + C_{2} y_{2}}$.
\begin{example}
    $\displaystyle y'' + \frac{1}{x} y' + \left( 1 - \frac{1}{4x^{2}} \right) y = 0$, amb solució $\displaystyle y_{1} = \frac{\sin x}{\sqrt{x}}$.

    $\Rightarrow \displaystyle y_{2} = y_{1} \int \frac{e^{-\int \frac{1}{x} \diff x}}{y_{1}^{2}} \diff x = - \frac{\cos x}{\sqrt{x}}$

    $\Rightarrow \boxed{y = C_{1} \frac{\sin x}{\sqrt{x}} + C_{2} \frac{\cos x}{\sqrt{x}}}$ és la solució general de l'ED (notem que el signe negatiu de $y_{2}$ ja el conté la constant).
\end{example}

%\subsubsection*{Equació d'ordre $n$} %WIP
%Sigui una ED reduïda d'ordre $n$:
%\begin{align}\label{ed-reduc-n}
%    	y^{(n)} + y^{(n-1)} P_{1}(x) + \dots + P_{n}(x) y = 0
%\end{align}
%\begin{thm}
%    Si es coneixen $u_{1}(x), \dots u_{p}(x)$ LI, l'equació~\eqref{ed-reduc-n} es pot reduir a una equació d'ordre $n-p$ en la variable $v$.
%    \begin{align*}
%        v = \begin{vmatrix} u_{1} & \dots & u_{p} & y \\  \vdots & &  & \vdots \\ u_{1}^{(p)} & \dots & u_{p}^{(p)} & y^{(p)} \end{vmatrix} \phi(x)
%    \end{align*}
%    on $\phi(x)$ és una funció arbitrària.
%\end{thm}

%\subsubsection*{Casos}
%1) Es coneix $u(x)$ no $\equiv 0$.

%$v = \begin{vmatrix} u & y \\ u' & y' \end{vmatrix} \phi(x)$, convé fer $\displaystyle \phi(x) = \frac{1}{u(x)^{2}}$.

%$v = \left( \frac{y}{u} \right)' \Rightarrow y = tu$ amb $t' = v$.

%----------------------------------------------------------------------------------------
\subsection{Equació de Cauchy-Euler}
\begin{defi}
\begin{align}
    x^{2} \frac{\partial^{2} y}{\partial x^{2}} + px \frac{\partial y}{\partial x} + qy = R(x)
\end{align}
\end{defi}
Es redueix a una de coeficients constants amb el canvi de variable $x= e^{t}$ o $t = \ln x$. Així doncs, per la regla de la cadena, l'equació queda
\begin{align}
    \frac{\partial^{2} y}{\partial t^{2}} - \frac{\partial y}{\partial t} + p \frac{\partial y}{\partial t} + qy = R(t)
\end{align}

\begin{example}
    $x^{2}y'' -4y' +6y = 2x + 5$ queda, fent el canvi $x = e^{t}$, $y'' -5 y' + 6y = 2e^{t} + 5$.

    $\displaystyle \Rightarrow \boxed{y = Ae^{2t} + Be^{3t} + e^{t} + \frac{5}{6}}$.
\end{example}

%----------------------------------------------------------------------------------------
\subsection{Equacions exactes de 2n ordre}
\begin{defi}
    \begin{align}
        P_{2}(x) \frac{\dif^{2} y}{\dif x^{2}} + P_{1}(x) \frac{\dif y}{\dif x} + P_{0} y = R(x)
    \end{align}
\end{defi}
\begin{thm}
    Una ED de segon ordre és exacta $\Leftrightarrow P_{2}''(x) - P_{1}'(x) + P_{0}(x) \equiv 0$.
\end{thm}
\subsubsection*{Resolució}
Per a les equacions exactes de segon ordre hi haurà prou amb resoldre una equació lineal de primer ordre. En particular, l'ED que s'ha de resoldre és
\begin{align}
    P_{2}(x)y' + [P_{1}(x) - P_{2}'(x)] y = C + \int R(x) \diff x
\end{align}
\begin{example}
    $(x+3) y'' + (2x+8) y' + 2y = 2$ és exacta ja que $P_{2}'' - P_{1}' + P_{0} = 0 -2 +2 \equiv 0$.

    Llavors hem de resoldre $(x+3) y' + [2x-8-x]y = C + 2 \int \diff x = C + 2x$. Estandaritzant la seva forma, tenim $\displaystyle y' + \frac{x-8}{x+3} y = C + \frac{2x}{x+3}$, i la seva solució és $\boxed{y = \frac{C e^{-2x} + x + B}{x+3}}$.
\end{example}

%----------------------------------------------------------------------------------------
%\subsection{Equacions fraccionàries lineals} %WIP
%\subsubsection*{Estudi geomètric de l'ED fraccional lineal}

    %----------------------------------------------------------------------------------------
%    TRANSFORMADES DE LAPLACE
%----------------------------------------------------------------------------------------
\section{Transformades de Laplace}
\subsection{Transformada d'una funció}
La transformada de Laplace és un operador integral. Pot ser útil quan es resolen equacions diferencials, ja que transforma una equació diferencial és una equació algebraica ordinària.
\begin{defi}
    La transformada de Laplace d'una funció $f(t)$ ve donada per
    \begin{align}
        \mathcal{L}[f(t)] \equiv \int_{0}^{\infty} e^{-st} f(t) \diff t
    \end{align}
    Denotem la funció resultant com $\mathcal{L}[f(t)] = F(s)$.
\end{defi}
La següent llista resumeix la transformada de Laplace d'algunes funcions utilitzades freqüentment.
\begin{itemize}
    \item $\mathcal{L}[1] = \displaystyle \frac{1}{s}$.
    \item $\mathcal{L}[t^{n}] = \displaystyle \frac{n!}{s^{n+1}}$.
    \item $\mathcal{L}[\sin at] = \displaystyle \frac{a}{s^2 + a^2}$.
    \item $\mathcal{L}[\cos at] = \displaystyle \frac{s}{s^2 + a^2}$.
    \item $\mathcal{L}[\sinh at] = \displaystyle \frac{a}{s^2 - a^2}$.
    \item $\mathcal{L}[\cosh at] = \displaystyle \frac{s}{s^2 - a^2}$.
\end{itemize}

\subsubsection*{Propietats}
\begin{enumerate}[i)]
    \item $\mathcal{L}\left[a f(t) + b g(t) \right] = a \mathcal{L}[f(t)] + b \mathcal{L}[g(t)]$.
    \item $\mathcal{L}\left[ e^{ct} f(t) \right] = F(s - c)$.
    \item $\mathcal{L}\left[ f^{(n)}(t) \right] = s^{n} \mathcal{L}\left[f(t)\right] - s^{n-1} f(0) - s^{n-2} f'(0) - \dots - f^{(n-1)}(0)$.
\end{enumerate}
\begin{defi}[Convolució]
    Definim la convolució de $f(t)$ amb $g(t)$ com
    \begin{align}
        f(t) \ast g(t) = \int_{0}^{t} f(t - \tau) g(\tau) \diff \tau = \int_{0}^{t} f(t) g(t - \tau) \diff \tau
    \end{align}
\end{defi}
La significança d'aquesta operació per a les transformades de Laplace és perquè es compleix
\begin{align}
    \mathcal{L}\left[ f(t) \ast g(t) \right] = \mathcal{L}\left[ f(t) \right] \cdot \mathcal{L}\left[ g(t) \right]
\end{align}

%----------------------------------------------------------------------------------------
\subsection{Transformada inversa d'una funció}
\begin{defi}
    \begin{align}
        \mathcal{L}^{-1} \left[ F(s) \right] = f(t) \quad \text{si} \quad \mathcal{L} \left[ f(t) \right] = F(s)
    \end{align}
\end{defi}
La següent llista resumeix algunes transformades inverses interessants:
\begin{enumerate}[i)]
    \item $\displaystyle \mathcal{L}^{-1}\left[ \frac{\dif}{ \dif s} F(s) \right] = -t \, f(t)$.
    \item $\mathcal{L}^{-1}\left[ e^{-cs} F(S) \right] = f(t-c) \,H(t-a)$.
\end{enumerate}
\begin{defi}[Funció esglaó de Heaviside]
    \begin{align}
        H(t-c) = \begin{cases} 1, & \quad t \geq c. \\ 0, & \quad t < c. \end{cases}
    \end{align}
    La seva transformada de Laplace és
    \begin{align*}
        \mathcal{L}[H(t-c)] = \int_{c}^{\infty} e^{-st} \diff t = \frac{e^{-cs}}{s}
    \end{align*}
\end{defi}

%----------------------------------------------------------------------------------------
\subsection{Transformada d'una equació diferencial}
Gràcies a la linealitat de la transformada de Laplace es poden resoldre equacions diferencials de manera molt senzilla.

\begin{example}
    Sigui $y'' -2y' + 2y = e^{-t}$, amb $y(0) = 0$, $y'(0) = 1$.
    
    Denotant $\mathcal{L}[y] = Y$, tenim $\displaystyle s^{2} Y - 1 - 2sY + 2Y = \frac{1}{s+1}$. Solucionant l'equació per $Y$, obtenim
    \begin{align*}
        Y = \frac{s+2}{(s+1)(s^2-2s+2)}
    \end{align*}
    La seva descomposició en fraccions parcials és
    \begin{align*}
    \begin{aligned}
        Y & = \frac{1}{5(s+1)} + \frac{1}{5}\frac{8-s}{(s-1)^{2} + 1} \\
        & = \frac{1}{5(s+1)} - \frac{1}{5}\frac{s-1}{(s-1)^{2} + 1} + \frac{7}{5}\frac{1}{(s-1)^{2} + 1}
    \end{aligned}
    \end{align*}
    Llavors, fent la transformada inversa, obtenim trivialment
    \begin{align*}
        \boxed{y(t) = \frac{1}{5} (e^{-t} - e^{t} \cos t + 7e^{t} \sin t)}.
    \end{align*}
\end{example}

    %----------------------------------------------------------------------------------------
%    EQUACIONS AMB SOLUCIONS EN SÈRIES
%----------------------------------------------------------------------------------------
\section{Equacions amb solucions en sèries}
\subsection{Desenvolupament en sèrie en torn a un punt ordinari}

%----------------------------------------------------------------------------------------
\subsection{Equacions hipergeomètriques}

%----------------------------------------------------------------------------------------
\subsection{Equacions de Legendre}

%----------------------------------------------------------------------------------------
\subsection{Equacions de Bessel}

%----------------------------------------------------------------------------------------
\subsection{Equacions de Laguerre}

%----------------------------------------------------------------------------------------
\subsection{Equacions d'Hermite}
    %----------------------------------------------------------------------------------------
%    TEMA
%----------------------------------------------------------------------------------------
\section{Teoria d'Sturm--Liouville}
\subsection{Sèries de Fourier}
\begin{defi}
    Les sèries de Fourier són molt importants a la Física. Les escriurem de la següent forma:
    \begin{align}
        S(x) = \frac{a_{0}}{2} + \sum\limits_{n=1}^{\infty} \left( a_{n} \cos \frac{n \pi x}{L} + b_{n} \sin \frac{n \pi x}{L} \right)
    \end{align}
\end{defi}
Aquesta sèrie funcional pot ser o no convergent, i en cas de convergir, pot fer-ho puntualment o uniforme. Com que les funcions $\cos (n \pi x / L)$ i $\sin (n \pi x / L)$ són periòdiques amb període $2L$, si la sèrie convergeix cap a la funció $S(x)$, aquesta també serà periòdica, és a dir,
\begin{align}
    S(x) = S(x + 2L)
\end{align}

\subsubsection*{Càlcul dels coeficients}
A partir de la funció $f(x)$ donada, calculem les integrals següents:
\begin{align}
    a_{n} = \frac{1}{L} \int_{c}^{c+2L} f(x) \cos \frac{n \pi x}{L} \dif x , \quad (n = 0, 1, 2 \dots)
\end{align}
\begin{align}
    b_{n} = \frac{1}{L} \int_{c}^{c+2L} f(x) \sin \frac{n \pi x}{L} \dif x , \quad (n = 1, 2, 3 \dots)
\end{align}
on $c$ és un punt qualsevol. De fet, podem integrar sobre qualsevol interval de longitud $2L$. Pel que fa a l'existència de les integrals, n'hi ha prou amb que $f$ sigui integrable al llarg d'un període.

\subsubsection*{Funcions parelles i senars}
En el cas que ens trobem amb funcions parelles o senars, l'expressió de les sèries de Fourier se simplifica força.
\begin{itemize}
    \item Funcions parelles: $f(-x) = f(x) \sim \frac{a_{0}}{2} + \sum a_{n} \dots$
        \subitem $\displaystyle b_{n} = 0 $
        \subitem $\displaystyle a_{n} = \frac{1}{L} \int_{c}^{c+2L} f(x) \cos \frac{n \pi x}{L} \dif x $
    \item Funcions senars: $f(-x) = -f(x) \sim \sum b_{n} \dots$
        \subitem $\displaystyle a_{n} = 0$
        \subitem $\displaystyle b_{n} = \frac{1}{L} \int_{c}^{c+2L} f(x) \sin \frac{n \pi x}{L} \dif x $
\end{itemize}

\subsubsection*{Identitat de Parseval}
En anàlisi matemàtica, la identitat de Parseval és un resultat fonamental sobre la suma de certes sèries obtingudes a partir de la sèrie de Fourier d'una funció.
\begin{defi}
    \begin{align}
        \frac{a_{0}^{2}}{2} + \sum_{n=1}^{\infty} (a_{n}^{2} + b_{n}^{2} ) = \frac{1}{L} \int_{c}^{c+2L} \left[ f(x) \right]^{2} \diff x
    \end{align}
\end{defi}

%----------------------------------------------------------------------------------------
\subsection{Problemes d'Sturm--Liouville}
Un problema general d'Sturm--Liouville pot ser escrit com
\begin{align}\label{eq:S-L}
    (p(x)y')' + (q(x) + \lambda w(x)) y = 0, \quad \text{amb} \begin{cases} C_{1} y(a) + C_{2}y'(a) = 0 \\ C_{3}y(b) + C_{4}y'(b) = 0  \end{cases}
\end{align}
On $p(x)$, $q(x)$ i $w(x)$ són funcions donades, $C_{1}y(a) \dots = 0$, $C_{2}y(b) \dots = 0$ són el que anomenem condicions de contorn i $\lambda$ és una constant que només pot prendre certs valors: els valors propis corresponents al problema. La funció $w(x)$ s'anomena funció pes i té un paper important per estudiar les propietats del problema.

\begin{defi}[Valors i funcions pròpies]
    Una manera intuïtiva d'entendre els problemes d'Sturm--Liouville és mirar el problema des d'una altra perspectiva. L'equació \eqref{eq:S-L} es pot expressar com
    \begin{align}
        Ly = \lambda y
    \end{align}
    En aquesta equació $L$ és un operador lineal que aplicat a una funció resulta la funció multiplicada per un escalar. Així doncs, $\lambda$ és un valor propi i $y$ la seva funció pròpia associada.
\end{defi}

\begin{example}
    $y'' + \lambda y = 0$, amb $\begin{cases} y(0) = 0 \\ y(L) = 0  \end{cases}$. Resolent l'equació auxiliar $m^2 + \lambda = 0$, veiem que les arrels són $m = \pm\sqrt{-\lambda}$. Ens interessa estudiar la solució segons si $\lambda$ és positiva, negativa o zero.
    \begin{itemize}
        \item Els casos $\lambda = 0$ i $\lambda<0$, aplicant les condicions de contorn, només porten a la solució $y\equiv0$ (solució trivial) i altres casos no interessants.
        \item Estudiem $\lambda>0$. Com que $m = \pm\sqrt{-\lambda}$, tenim que $m = \pm \imath \sqrt{\lambda}$; d'aquí veiem que la solució serà de la forma $y =  C_{1} \sin (\sqrt{\lambda} x) + C_{2} \cos (\sqrt{\lambda} x)$.

        Considerem les condicions de contorn:
        \subitem $y(0) = C_{2} \cos (0) = C_{2} = 0 \Rightarrow C_{2} = 0$.
        \subitem $y(L) = C_{1} \sin (\sqrt{\lambda} L) = 0 \Rightarrow$ les possibilitats són $\begin{cases} C_{1} = 0. \\ \sqrt{\lambda} L = n \pi. \end{cases}$
    \end{itemize}
    Llavors, els valors propis del problema són $\boxed{\lambda_{n} = \left( \frac{n\pi}{L} \right)^{2}}$ i les funcions pròpies

    $\boxed{y_{n} = C_{n} \sin \left( \frac{n\pi x}{L} \right)}$.
\end{example}

\subsubsection*{Trobar la forma auto-adjunta}
No sempre ens trobarem les ED de la forma auto-adjunta, com a \eqref{eq:S-L}; en general ens trobarem amb funcions de la forma
\begin{align}
    y'' + P(x)y' + [Q(x) + R(x) \lambda]y = 0
\end{align}
Per treballar amb aquestes equacions multipliquem tota l'equació pel factor $e^{\int P(x) \diff x} = p(x)$, de manera que la podem re-agrupar:
\begin{align}
    (p(x) y')' + [p(x)Q(x) + p(x)R(x)\lambda]y = 0
\end{align}
Observem que trobar la forma auto-adjunta d'una ED ens permet identificar la funció pes $w(x) = p(x)R(x)$.

\begin{example}
    Sigui $\displaystyle xy'' + y' + \frac{\lambda}{x} y = 0 \Rightarrow y'' + \frac{1}{x}y' + \frac{\lambda}{x^{2}} y = 0$. Trobem el factor $p(x) = e^{\int x^{-1} \diff x} = x$. Així doncs, multiplicant tota l'equació pel factor $p(x) = x$, trobem la forma auto-adjunta: $\boxed{(x y')' + \frac{\lambda}{x}y = 0}$.
\end{example}

\subsubsection*{Ortogonalitat de les funcions pròpies}
\begin{thm}
    Diem que dues funcions pròpies són ortogonals si el seu producte interior s'anul·la:
    \begin{align}
        \left< y_{n} \mid y_{m} \right> \equiv \int_{a}^{b} w(x) y_{n}(x) y_{m}(x) \diff x = 0
    \end{align}
    Aquest teorema és vàlid $\forall y_{n}, y_{m}$ amb $n \neq m$, ja que totes les funcions pròpies de valors propis diferents generen una base ortogonal.
\end{thm}

\begin{thm}
    Diem que una funció pròpia és normalitzada si la seva norma és la unitat:
    \begin{align}
        \| y_{n} \|^{2} \equiv \int_{a}^{b} w(x) y_{n}^{2}(x) \diff x = 1
    \end{align}
    Aquelles funcions que siguin normalitzades les denotarem com $u_{n}(x)$.
\end{thm}

La següent llista resumeix les integrals típiques que surten en aquests càlculs:
\begin{itemize}
    \item $\displaystyle \int_{0}^{L} \sin \left( \frac{n \pi x}{L} \right) \sin \left( \frac{m \pi x}{L} \right) = \int_{0}^{L} \cos \left( \frac{n \pi x}{L} \right) \cos \left( \frac{m \pi x}{L} \right) = \frac{L}{2} \delta_{nm}$.
    \item $\displaystyle \int_{0}^{L} \sin \left( \frac{n \pi x}{L} \right) \cos \left( \frac{n \pi x}{L} \right) = \begin{cases} 0, & \text{si } n+m \text{ parell.} \\ \displaystyle \frac{2mL}{(m^{2}-n^{2})\pi}, &  \text{si } n+m \text{ senar.} \end{cases}$
    \item $\displaystyle \int_{0}^{2L} \sin \left( \frac{n \pi x}{L} \right) \sin \left( \frac{m \pi x}{L} \right) = \int_{0}^{2L} \cos \left( \frac{n \pi x}{L} \right) \cos \left( \frac{m \pi x}{L} \right) = L \delta_{nm}$.
    \item $\displaystyle \int_{0}^{2L} \sin \left( \frac{n \pi x}{L} \right) \cos \left( \frac{n \pi x}{L} \right) = 0$.
\end{itemize}

%----------------------------------------------------------------------------------------
\subsection{Sèrie de Fourier generalitzada}
Sigui $f(x)$ una funció que volem expressar com a sèrie de Fourier i $u(x)$ la funció pròpia d'un problema d'Sturm--Liouville. Llavors, tenim la sèrie de Fourier generalitzada:
\begin{align}
    S(x) = \sum_{n=1}^{\infty} a_{n} y_{n}(x); \quad \text{on } a_{n} = \frac{\left< f \mid y_{n} \right>}{\| y_{n} \|^{2}}
\end{align}
O alternativament:
\begin{align}
    S(x) = \sum_{n=1}^{\infty} a_{n} u_{n}(x); \quad \text{on } a_{n} = \left< f \mid u_{n} \right>
\end{align}

\begin{example}
    Sigui $\displaystyle y'' + \frac{\lambda}{4x^{2}} y = 0$, amb condicions de contorn $y(1) = 0$, $y(e^{\pi}) = 0$. Volem calcular la sèrie de Fourier de $f(x) = \sqrt{x}$ associada a aquest problema d'Sturm--Liouville.

    Resolent el problema dels valors i funcions pròpies, obtenim $\lambda_{n} = 4n^{2} + 1$ i $y_{n}(x) = C_{n} \sqrt{x} \sin (n \ln x)$.

    A continuació veurem com els dos mètodes per trobar la sèrie de Fourier de $f(x)$ són equivalents.
    \begin{enumerate}[i)]
        \item $\displaystyle a_{n} = \frac{\left< f \mid y_{n} \right>}{\| y_{n} \|^{2}} = \frac{\int_{1}^{e^{\pi}}\frac{1}{4x^{2}} \sqrt{x} \sqrt{x} \sin (n \ln x) \diff x}{\int_{1}^{e^{\pi}}\frac{1}{4x^{2}} \sqrt{x}^{2} \sin^{2} (n \ln x) \diff x} = \frac{\int_{1}^{e^{\pi}} \frac{1}{x} \sin (n \ln x) \diff x}{\int_{1}^{e^{\pi}} \frac{1}{x} \sin^{2} (n \ln x) \diff x}$. Fent el canvi $t = \ln x$ arribem fàcilment a $\boxed{a_{n} = \frac{1-(-1)^{n}}{n \pi/2}}$.
        \item Imposant que $\| y_{n} \|^{2} = 1$, podem determinar que $ C_{n} = 2\sqrt{\frac{2}{\pi}}$. Llavors $u_{n}(x) = 2\sqrt{\frac{2}{\pi}} \sqrt{x} \sin (n \ln x)$. Així doncs, $ a_{n} = \left< f \mid u_{n} \right> = \int_{1}^{e^{\pi}} \frac{1}{4 x^{2}} 2 \sqrt{x} \sqrt{\frac{2}{\pi}} \sqrt{x} \sin (n \ln x) \diff x = \frac{1}{\sqrt{2\pi}} \int_{1}^{e^{\pi}} \frac{1}{x} \sin (n \ln x) \diff x$. Llavors arribem a $\boxed{a_{n} = \frac{1-(-1)^{n}}{n \sqrt{2 \pi}}}$.
    \end{enumerate}
    Tot i que els dos resultats semblen diferents, com que construïm la sèrie amb $y_{n}(x)$ i $u_{n}(x)$, respectivament, a partir d'ambdós mètodes arribem a la mateixa expressió general:
    \begin{align*}
        \boxed{\sqrt{x} = \sum_{n=1}^{\infty} [1 - (-1)^{n}] \frac{2}{n \pi} \sqrt{x} \sin (n \ln x)}
    \end{align*}
    A l'enllaç següent es pot veure la representació gràfica de $\sqrt{x}$ i la seva sèrie $S(x)$: \url{https://www.desmos.com/calculator/m5cb68noq1}. Notem que la sèrie només convergeix per $1 < x < e^{\pi}$, que són precisament els límits de les condicions de contorn del problema d'Sturm--Liouville.
\end{example}


%\section*{Demostracions}
%\printproofs

%----------------------------------------------------------------------------------------
\end{document}

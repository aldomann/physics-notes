%----------------------------------------------------------------------------------------
%    TEMA
%----------------------------------------------------------------------------------------
\section{\mytitle}
\subsection{Termodinàmica}
\begin{align}
    U = Q - pV, \quad \dif U = T\diff S - p\diff V
\end{align}
\begin{align}
    H = U + pV
\end{align}
\begin{align}
    \begin{aligned}
        A = U - TS \\
        G = H - TS
    \end{aligned}
\end{align}
\begin{align}
    \dif G = V\diff p - S \diff T + \sum \mu_{i} \diff n_{i}
\end{align}

\begin{align}
    \dif S_s = \frac{\dif q_{\text{rev}}}{T_s}
\end{align}
\begin{align}
    \dif S_e = -\frac{\dif q_{\text{rev}}}{T_e} = -\frac{\dif H}{T_e}
\end{align}
%----------------------------------------------------------------------------------------
\subsection{Canvis de fase}
\begin{align}
    \text{Clausius--Clapeiron: } \frac{\dif p}{\dif T} = \frac{L_{m}}{T \Delta v_{m}}
\end{align}
\begin{align}
    \ln \left( \frac{p_A}{p_B} \right) = \frac{L}{R} \left( \frac{1}{T_B} - \frac{1}{T_A} \right)
\end{align}
%----------------------------------------------------------------------------------------
\subsection{Dissolucions}
\begin{align}
    \begin{aligned}
        \text{Dalton: } & p_{i} = p \chi_{i} \quad \text{(gasos ideals)} \\
        \text{Raoult: } & p_{i} = p_{i}^{\ast} \chi_{i} \quad \text{líquids ideals)} \\
        \text{Henry: } & p_{i} = k_{i} \chi_{i} \quad \text{(líquid diluït)}
    \end{aligned}
\end{align}
\begin{align}
    \begin{aligned}
        \text{Gasos id.: } & \mu_{i} = \mu_{o}^{g} + RT \ln \chi_{i} \\
        \text{Líquids id.: } & \mu_{i} = \mu^{\ast} + RT \ln \chi_{i} \\
        \text{Líquids dil.: } & \mu_{i} = \mu_{o}^{l} + RT \ln \chi_{i} \\
        \text{Líquids reals: } & \chi_{i} \mapsto a_{i} = \gamma_{i} \chi_{i}
    \end{aligned}
\end{align}

\begin{align}
    \text{Solubilitats: } \frac{p_A}{p_B} = \frac{S_A}{S_B}
\end{align}
%----------------------------------------------------------------------------------------
\subsection{Propietats col·ligatives}
\begin{align}
    \Delta T_e = k_b m
\end{align}
\begin{align}
    \Delta T_c = k_f m
\end{align}
\begin{align}
    \Delta p_1 = p_1^{\ast} \chi_2
\end{align}
\begin{align}
    \pi = mRT
\end{align}
%----------------------------------------------------------------------------------------
\subsection{Estructura de la matèria}
\begin{itemize}
    \item $n$: nre. quàntic principal ($0, \dots, \infty$).
    \item $l$: nre. quàntic azimutal ($0, \dots, n-1$).
    \item $m_l$: nre. quàntic magnètic ($-l, \dots, l$).
    \item $m_s$: nre. quàntic d'spin ($-1/2, +1/2$).
\end{itemize}

\subsubsection*{Orbitals}
\begin{itemize}
    \item $l=0$: orbital $s$ (2 electrons màxim).
    \item $l=1$: orbital $p$ (6 electrons màxim). 
    \item $l=2$: orbital $d$ (10 electrons màxim).
    \item $l=3$: orbital $f$ (14 electrons màxim).
\end{itemize}
%----------------------------------------------------------------------------------------
\newpage

\subsection{Canvi químic}
\subsubsection*{Equilibri químic}
Sigui \ch{a A + b B <=> c C + d D} una reacció química.
\begin{align}
    \Delta_{r} G = \Delta_{r} G^{0} + RT \ln Q_{p}
\end{align}
\begin{align*}
    \Delta_{r} G   \begin{cases} < 0 & \text{reacciona cap a la dreta} \\ = 0 & \text{està en l'equilibri} \\ > 0 & \text{reacciona cap a l'esquerra} \end{cases}
\end{align*}
on $Q$ és el quocient de reacció. Quan reacciona cap a la dreta es diu que la reacció és espontània.
\begin{align}
    \Delta_{r} G = \Delta_{r} H - T \Delta_{r} S
\end{align}

Per a gasos ideals tenim:    
\begin{align}
    Q_{p} = \frac{\prod (p_{prod}/ p^{0})^{\nu_{prod}}}{\prod (p_{reac}/ p^{0})^{\nu_{reac}}}
\end{align}
on $p^{0} = \SI{1}{atm}$. 

Per a dissolucions ideals tenim:
\begin{align}
    Q_{c} = \frac{\prod (C_{prod}/C^{0})^{\nu_{prod}}}{\prod (C_{reac}/ C^{0})^{\nu_{reac}}}
\end{align}
on $C^{0} = \SI{1}{M}$.

\begin{align*}
    Q_{x} \begin{cases} < K_{x} & \text{reacciona cap a la dreta} \\ = K_{x} & \text{està en l'equilibri} \\ > K_{x} & \text{reacciona cap a l'esquerra} \end{cases}
\end{align*}

Es pot demostrar que existeix la següent relació entre les constants $k_p$ en funció de la temperatura:
\begin{align}
    \ln \frac{k_{p}^{0}({T_2})}{k_{p}^{0}(T_1)} = - \frac{\Delta H^{0}}{R} \left( \frac{1}{T_{2}} - \frac{1}{T_{1}} \right)
\end{align}
%---------------------------------
\subsubsection*{Reaccions àcid--base}
Hi ha deferents definicions d'àcid:
\begin{itemize}
    \item Brønsted: que dóna protons ($H^{+}$).
    \item Lewis: que agafa un parell d'electrons.
\end{itemize}

\begin{align}
\begin{aligned}
    K_{w} &= [H_{3}O^{+}] [OH^{-}] \\ &= K_{a} K_{b} = 10^{-14} \\
    14 &= pH + pOH
\end{aligned}
\end{align}
\begin{align}
\begin{aligned}
    pH &= -\log [H_{3}O^{+}] \\
    pOH &= - \log [OH^{-}]
\end{aligned}
\end{align}
%---------------------------------
\subsubsection*{Reaccions de precipitació}
Sigui \ch{cAB -> a A+ + b B-} una reacció química.
\begin{align}
    Q_{ps} = [A^+]^{a} [B^-]^{b}
\end{align}
\begin{align*}
    Q_{ps}  \begin{cases} \leq K_{ps} & \text{no precipita} \\ > K_{ps} & \text{precipita} \end{cases}
\end{align*}
on $Q_{ps}$ s'anomena el producte de solubilitat.
%---------------------------------
\subsubsection*{Cel·les electroquímiques (reaccions d'oxidació--reducció)}
Si una cel·la opera reversiblement, es compleix
\begin{align}
\begin{aligned}
    W_{\text{elec}} &= - \nu F E_{\text{cel·la}} \\ &= \Delta_{r} G = - RT \ln K
\end{aligned}
\end{align}
on $F$ és la constant de Faraday (\SI{9.6485 e4}{\coulomb \per \mol}), $E_{\text{cel·la}}$ és la for\c{c}a electromotriu de la cel·la, i $\nu$ és el nre. de mols d'electrons transferits en la reacció de la cel·la.
\begin{align}
    \text{Eq. Nernst: } E_{\text{c}} = E^{0}_{\text{c}} - \frac{RT}{\nu F} \ln Q
\end{align}
on $Q$ és el quocient de reacció.
\begin{align}
    E^{0}_{\text{cel·la}} = E^{0}_{\text{càtode}} - E^{0}_{\text{ànode}}
\end{align}
\begin{align*}
    E^{0}_{\text{cel·la}}   \begin{cases} < 0 & \text{no espontània} \\ > 0 & \text{espontània} \end{cases}
\end{align*}
on el càtode és a on es produeix la reducció (guany d'electrons) i l'ànode és a on es produeix l'oxidació (pèrdua d'electrons).
%----------------------------------------------------------------------------------------
\subsection{Cinètica química}
\subsubsection*{Velocitat d'una reacció}
Sigui \ch{a A + b B -> c C + d D} una reacció química.
\begin{align}
\begin{aligned}
    v &= -\frac{1}{a} \frac{\dif [A]}{\dif t} = -\frac{1}{b} \frac{\dif [B]}{\dif t} \\ &= \frac{1}{c} \frac{\dif [C]}{\dif t} = \frac{1}{d} \frac{\dif [D]}{\dif t}
\end{aligned}
\end{align}
on $v$ és la velocitat de la reacció.
%---------------------------------
\subsubsection*{Ordre d'una reacció}
\begin{align}
    \text{Llei de velocitat: } v = k [A]^{x} [B]^{y}
\end{align}
on $k$ s'anomena constant de velocitat, que, com el seu nom indica, és constant $\forall t$; $x$ i $y$ són els ordres de cada reactiu i són determinats experimentalment; poden o no coincidir amb el coeficient estequiomètric. S'anomena ordre de reacció a la suma $x + y$.
%---------------------------------
\subsubsection*{Reaccions d'ordre zero}
Sigui \ch{aA -> bB} una reacció química. Mitjan\c{c}ant el càlcul podem arribar a la següent expressió:
\begin{align}
    [A]_{t} = [A]_{0} - akt
\end{align}
%---------------------------------
\subsubsection*{Reaccions de primer ordre}
Sigui \ch{aA -> bB} una reacció química. Mitjan\c{c}ant el càlcul podem arribar a la següent expressió:
\begin{align}
\begin{gathered}
    \ln [A]_{t} = \ln [A]_{0} -akt \\
    \text{ o alternativament } \\
    \ln ([B]_{\infty} - [B]_{t}) = \ln [B]_{\infty} -akt 
\end{gathered}
\end{align}
%---------------------------------
\subsubsection*{Reaccions de segon ordre}
Sigui \ch{aA -> bB} una reacció química. Mitjan\c{c}ant el càlcul podem arribar a la següent expressió:
\begin{align}
    \frac{1}{[A]_{t}} = \frac{1}{[A]_{0}} + akt 
\end{align}
%---------------------------------
\subsubsection*{Reaccions elementals}
Sigui \ch{aA -> bB} una reacció química. Llavors, l'ordre de la reacció és el coeficient estequiomètric \ch{a}.
%---------------------------------
\subsubsection*{Energia d'activació}
\begin{align}
    \text{Eq. Arrhenius: } k = A e^{-E_{a}/RT}
\end{align}
on $A$ és la freqüència de col·lisió i $E_{a}$ és l'energia d'activació de la reacció.

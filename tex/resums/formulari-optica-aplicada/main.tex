%-----------------------------------------------------------------
%	BASIC DOCUMENT LAYOUT
%-----------------------------------------------------------------
\documentclass[paper=a4, fontsize=9pt, twocolumn]{scrartcl}
\usepackage[T1]{fontenc}
\usepackage[utf8]{inputenc}
\usepackage{lmodern}
\usepackage{microtype}
\usepackage[catalan]{babel}
\usepackage[fixlanguage]{babelbib}
\selectbiblanguage{catalan}

% Sectioning layout
\addtokomafont{sectioning}{\normalfont\scshape}
\usepackage{tocstyle}
\usetocstyle{standard}
\renewcommand*\descriptionlabel[1]{\hspace\labelsep\normalfont\bfseries{#1}}

% Empty pages
\usepackage{etoolbox}

% Paragraph indentation behaviour
\setlength{\parindent}{0pt}
\setlength{\parskip}{0.3\baselineskip plus2pt minus2pt}
\newcommand{\sk}{\medskip\noindent}

% Fancy header and footer
\usepackage{fancyhdr}
\pagestyle{fancyplain}
\fancyhead[L]{\mytitle}
\fancyhead[R]{\myauthor}
\fancyfoot[C]{\thepage}
\fancypagestyle{firststyle}
{
    \fancyhead[L]{\mytitle}
    \fancyhead[R]{\myauthor}
    \fancyfoot[C]{\thepage}
}
\renewcommand{\headrulewidth}{0.3pt}
\renewcommand{\footrulewidth}{0pt}
\setlength{\headheight}{13.6pt}

%-----------------------------------------------------------------
%	MATHS AND SCIENCE
%-----------------------------------------------------------------
\usepackage{amsmath,amsfonts,amsthm,amssymb}
\usepackage{xfrac}
\usepackage[a]{esvect}
\usepackage{chemformula}
\usepackage{graphicx}

\usepackage[arrowdel]{physics}
	\renewcommand{\vnabla}{\vec{\nabla}}
	% \renewcommand{\vectorbold}[1]{\boldsymbol{#1}}
	% \renewcommand{\vectorarrow}[1]{\vec{\boldsymbol{#1}}}
	% \renewcommand{\vectorunit}[1]{\hat{\boldsymbol{#1}}}
	\renewcommand{\vectorarrow}[1]{\vec{#1}}
	\renewcommand{\vectorunit}[1]{\hat{#1}}
	\renewcommand*{\grad}[1]{\vnabla #1}
	\renewcommand*{\div}[1]{\vnabla \vdot \va{#1}}
	\renewcommand*{\curl}[1]{\vnabla \cp \va{#1}}
	\let\rot\curl

% SI units
\usepackage[separate-uncertainty=true, alsoload=astro, alsoload=hep]{siunitx}
\sisetup{range-phrase = \text{--}, range-units = brackets}
\DeclareSIPrePower\quartic{4}
	%\DeclareSIUnit\micron{\micro\metre}

% Smaller trig functions
\newcommand{\Sin}{\trigbraces{\operatorname{s}}}
\newcommand{\Cos}{\trigbraces{\operatorname{c}}}
\newcommand{\Tan}{\trigbraces{\operatorname{t}}}

% Operator-style notation for matrices
\newcommand*{\mat}[1]{\hat{#1}}

% Matrices in (A|B) form via [c|c] option
\makeatletter
\renewcommand*\env@matrix[1][*\c@MaxMatrixCols c]{%
  \hskip -\arraycolsep
  \let\@ifnextchar\new@ifnextchar
  \array{#1}}
\makeatother

% Shorter \mathcal and \mathbb
\newcommand*{\mc}[1]{\mathcal{#1}}
\newcommand*{\mbb}[1]{\mathbb{#1}}
\newcommand*{\TF}[1]{\mathcal{F}\qty{#1}}
\newcommand*{\TFI}[1]{\mathcal{F}^{-1}\qty{#1}}

% Shorter ^\ast and ^\dagger
\newcommand*{\sast}{^{\star}{}}
\newcommand*{\sdag}{^{\dagger}{}}

% Blackboard bold identity
\usepackage{bbm}
\newcommand*{\bbid}{\mathbbm{1}}

\newcommand*{\Scale}[2][4]{\scalebox{#1}{$#2$}}%

%-----------------------------------------------------------------
%	OTHER PACKAGES
%-----------------------------------------------------------------
\usepackage{environ}

%Left numbered equations
\makeatletter
	\NewEnviron{Lalign}{\tagsleft@true\begin{align}\BODY\end{align}}
\makeatother

% Plots and graphics
\usepackage{pgfplots}
\usepackage{tikz}
\usepackage{color}
	\makeatletter
		\color{black}
		\let\default@color\current@color
	\makeatother

% Richer enumerate, figure, and table support
\usepackage{enumerate}
\usepackage[shortlabels]{enumitem}
\usepackage{float}
\usepackage{tabularx}
\usepackage{booktabs}
	%\setlength{\intextsep}{8pt}
\numberwithin{equation}{section}
\numberwithin{figure}{section}
\numberwithin{table}{section}

% No indentation after certain environments
\makeatletter
\newcommand*\NoIndentAfterEnv[1]{%
	\AfterEndEnvironment{#1}{\par\@afterindentfalse\@afterheading}}
\makeatother
%\NoIndentAfterEnv{thm}
\NoIndentAfterEnv{defi}
\NoIndentAfterEnv{example}
\NoIndentAfterEnv{table}

% Misc packages
\usepackage{lipsum}

%-----------------------------------------------------------------
%	THEOREMS
%-----------------------------------------------------------------
\usepackage{thmtools}

% Theroems layout
\declaretheoremstyle[
	spaceabove=6pt, spacebelow=6pt,
	headfont=\normalfont,
	notefont=\mdseries, notebraces={(}{)},
	bodyfont=\small,
	postheadspace=1em,
]{small}

\declaretheorem[style=plain,name=Teorema,qed=$\square$,numberwithin=section]{thm}
\declaretheorem[style=plain,name=Corol·lari,qed=$\square$,sibling=thm]{cor}
\declaretheorem[style=plain,name=Lema,qed=$\square$,sibling=thm]{lem}
\declaretheorem[style=definition,name=Definició,qed=$\blacksquare$,numberwithin=section]{defi}
\declaretheorem[style=definition,name=Exemple,qed=$\blacktriangle$,numberwithin=section]{example}
\declaretheorem[style=small,name=Demostració,numbered=no,qed=$\square$]{sproof}

%-----------------------------------------------------------------
%	ELA MOTHERFUCKING GEMINADA
%-----------------------------------------------------------------
\def\xgem{%
	\ifmmode
		\csname normal@char\string"\endcsname l%
	\else
		\leftllkern=0pt\rightllkern=0pt\raiselldim=0pt
		\setbox0\hbox{l}\setbox1\hbox{l\/}\setbox2\hbox{.}%
		\advance\raiselldim by \the\fontdimen5\the\font
		\advance\raiselldim by -\ht2
		\leftllkern=-.25\wd0%
		\advance\leftllkern by \wd1
		\advance\leftllkern by -\wd0
		\rightllkern=-.25\wd0%
		\advance\rightllkern by -\wd1
		\advance\rightllkern by \wd0
		\allowhyphens\discretionary{-}{}%
		{\kern\leftllkern\raise\raiselldim\hbox{.}%
			\kern\rightllkern}\allowhyphens
	\fi
}
\def\Xgem{%
	\ifmmode
		\csname normal@char\string"\endcsname L%
	\else
		\leftllkern=0pt\rightllkern=0pt\raiselldim=0pt
		\setbox0\hbox{L}\setbox1\hbox{L\/}\setbox2\hbox{.}%
		\advance\raiselldim by .5\ht0
		\advance\raiselldim by -.5\ht2
		\leftllkern=-.125\wd0%
		\advance\leftllkern by \wd1
		\advance\leftllkern by -\wd0
		\rightllkern=-\wd0%
		\divide\rightllkern by 6
		\advance\rightllkern by -\wd1
		\advance\rightllkern by \wd0
		\allowhyphens\discretionary{-}{}%
		{\kern\leftllkern\raise\raiselldim\hbox{.}%
			\kern\rightllkern}\allowhyphens
	\fi
}

\expandafter\let\expandafter\saveperiodcentered
	\csname T1\string\textperiodcentered \endcsname

\DeclareTextCommand{\textperiodcentered}{T1}[1]{%
	\ifnum\spacefactor=998
		\Xgem
	\else
		\xgem
	\fi#1}

%-----------------------------------------------------------------
%	PDF INFO AND HYPERREF
%-----------------------------------------------------------------
\usepackage{hyperref}
\hypersetup{colorlinks, citecolor=black, filecolor=black, linkcolor=black, urlcolor=black}

\newcommand*{\mytitle}{Formulari òptica aplicada}
\newcommand*{\myauthor}{Alfredo Hernández Cavieres}
\newcommand*{\mydate}{\normalsize \today}

\pdfstringdefDisableCommands{\def\and{i }}

\usepackage{hyperxmp}
\hypersetup{pdfauthor={\myauthor}, pdftitle={\mytitle}}

%-----------------------------------------------------------------
%	DOCUMENT BODY
%-----------------------------------------------------------------
\begin{document}

%\tableofcontents
\thispagestyle{firststyle}
%----------------------------------------------------------------------------------------
%    TEMA
%----------------------------------------------------------------------------------------
\section{\mytitle}
\subsection{Tensor d'inèrcia}
\begin{align}
    \mat{I} = \hat{\int} \dif m \begin{pmatrix} x^{2} & -xy & -xz \\ -yx & y^{2} & -yz \\ -zx & -zy & z^{2} \end{pmatrix}
\end{align}
on $x^{2} \equiv y^{2} + z^{2}$, $y^{2} \equiv x^{2} + z^{2}$, i $z^{2} \equiv x^{2} + y^{2}$.

\begin{align}
    T = T_{tras} + T_{rot}
\end{align}
\begin{align}
    T_{rot} = \frac{1}{2} \lrbra{\vec{\omega}^{t} \cdot \mat{I} \cdot \vec{\omega}}
\end{align}

\begin{align}
    \vec{L} = \vec{L}_{cm} + \vec{L}_{rot}
\end{align}
\begin{align}
    \vec{L}_{rot} = \mat{I} \times \vec{\omega} \mid \vec{L} = \vec{r} \times \vec{p}
\end{align}
%----------------------------------------------------------------------------------------
\subsection{Angles d'Euler (rotacions)}
Rotació: $\mat{O} \cdot \vec{x}$.
\begin{align}
    \text{Eix $z_{f}$: } \mat{O}_{\varphi} = \begin{pmatrix} \Cos \varphi & \Sin \varphi & 0 \\ -\Sin \varphi & \Cos \varphi & 0 \\ 0 & 0 & 1 \end{pmatrix}
\end{align}
\begin{align}
    \text{Eix $x_{girat}$: } \mat{O}_{\theta} = \begin{pmatrix} 1 & 0 & 0 \\ 0 & \Cos \theta & \Sin \theta \\ 0 & -\Sin \theta & \Cos \theta \end{pmatrix}
\end{align}
\begin{align}
    \text{Eix $z_{m}$: } \mat{O}_{\psi} = \begin{pmatrix} \Cos \psi & \Sin \psi & 0 \\ -\Sin \psi & \Cos \psi & 0 \\ 0 & 0 & 1 \end{pmatrix}
\end{align}

\begin{align}
    \begin{pmatrix} N_{1} \\ N_{2} \\ N_{3} \end{pmatrix} = \begin{pmatrix} I_{1} \dot{\omega}_{1} + (I_{3} - I_{2}) \omega_{2} \omega_{3} \\ I_{2} \dot{\omega}_{2} + (I_{1} - I_{3}) \omega_{3} \omega_{1} \\ I_{3} \dot{\omega}_{3} + (I_{2} - I_{1}) \omega_{1} \omega_{2} \end{pmatrix}
\end{align}

\begin{align}
    \text{Precessió: } \vec{N} = \vec{\Omega} \times \vec{L}
\end{align}

%----------------------------------------------------------------------------------------
\subsection{Sistemes de referència mòbils}
\begin{align}
\begin{aligned}
    m \ddot{\vec{r}}' &=  m \vec{a}_{f} \overbrace{- m \ddot{\vec{R}}_{f} - m \dot{\vec{\omega}} \times \vec{r}'}^{\text{arrossegament}} \\
    & \underbrace{- 2m \vec{\omega} \times \dot{\vec{r}}'}_{\text{coriol·lis}} \underbrace{- m \vec{\omega} \times \lrpar{\vec{\omega} \times \vec{r}'}}_{\text{centrífuga}} \\
\end{aligned}
\end{align}
$m \vec{a}_{f}$ és la for\c{c}a al SR inercial, mentre que $m \ddot{\vec{r}}'$ és la for\c{c} aparent al SR mòbil.

\begin{align}
\begin{aligned}
    \vec{\omega}_{T} &\mapsto \text{SRm (per coriolis): } \\
    \vec{\omega} &= \omega \lrpar{-\Cos \lambda, 0, \Sin \lambda} \text{(nord)} \\
    \vec{\omega} &= \omega \lrpar{-\Cos \lambda, 0, -\Sin \lambda} \text{(sud)}
\end{aligned}
\end{align}

%----------------------------------------------------------------------------------------
\subsection{Foucault}
\begin{align}
    \begin{pmatrix} x \\ y \end{pmatrix} = \begin{pmatrix} \Cos \omega_{z} t & \Sin \omega_{z} t \\ -\Sin \omega_{z} t & \Cos \omega_{z} t \end{pmatrix} \begin{pmatrix} x_{0} \\ y_{0} \end{pmatrix}
\end{align}
\begin{align}
    \Delta t = \frac{2 \pi}{ \omega_{z}} = \frac{2 \pi}{ \omega \Sin \lambda}
\end{align}
on $\omega_{z}$ és la component $z$ de $\omega$ al SRm i $\lambda$ és la latitud.

%----------------------------------------------------------------------------------------
\subsection{Punt de percussió}
\begin{align}
    p_{p} = \frac{I}{m R}
\end{align}
on $R$ és la distància del centre de masses a l'extrem.

% %-----------------------------------------------------------------
%	FORMULARI
%	!TEX root = main.tex
%-----------------------------------------------------------------
\setcounter{section}{1}
\setcounter{page}{2}

\section{\mytitle}
\subsection{Equació d'Schrödinger}
\begin{align*}
	\qty[\frac{p^{2}}{2m} + V(x)] \psi_{n}(x) = E_{n}\psi_{n}(x) \qc p = -ih \pdv{x}
\end{align*}

%-----------------------------------------------------------------
\subsection{Pou infinit}
\begin{align*}
	\qty[0,L] \Rightarrow
	\begin{cases}
		\phi_{n}(x) = \sqrt{\dfrac{2}{L}} \sin \qty(\dfrac{n\pi x}{L}) & x \in [0,L] \\
		\phi_{n}(x) = 0 & x \not\in [0,L]
	\end{cases}
	\\
	\qty[-\dfrac{L}{2}, \dfrac{L}{2}] \Rightarrow
	\begin{cases}
		\phi_{n}(x) = \sqrt{\dfrac{2}{L}} \cos \qty(\dfrac{n\pi x}{L}) & n \text{ senar} \\
		\phi_{n}(x) = \sqrt{\dfrac{2}{L}} \sin \qty(\dfrac{n\pi x}{L}) & n \text{ parell}
	\end{cases}
\end{align*}
\begin{align*}
	E_{n} = \dfrac{n^{2} \pi^{2} \hbar^{2}}{2m L^{2}}
\end{align*}

%-----------------------------------------------------------------
\subsection{Harmònics 1D}
\begin{align*}
\begin{gathered}
	\text{Gaussiana: } \psi(x) = \frac{1}{(2\pi \sigma_{x}^{2})^{1/4}} \exp[- \frac{x^{2}}{4 \sigma_{x}^{2}}] \\% \qc \int g(x) \dd{x} \equiv 1
	\boxed{\phi_{n}(x) = C_{n} H_{n}(\tilde{x}) \exp[-\dfrac{\tilde{x}^{2}}{2}]} % \qc \int \phi_{n}\sast(x) \phi_{n}(x) \dd{x} \equiv 1
\end{gathered}
\end{align*}

\begin{align*}
	\tilde{x} = \dfrac{x}{a_{0}} \qc a_{0} = \sqrt{\dfrac{\hbar}{m\omega}} \qc C_{n} = \qty(\frac{1}{\pi a_{0}^{2}})^{1/4} \qty(\frac{1}{2^{n} n!})^{1/2}
\end{align*}

\begin{align*}
	\begin{aligned}
		H_{0}(\tilde{x}) &= 1 \\
		H_{1}(\tilde{x}) &= 2\tilde{x} \\
	\end{aligned}
	\quad
	\begin{aligned}
		H_{2}(\tilde{x}) &= 4\tilde{x}^{2} - 2 \\
		H_{3}(\tilde{x}) &= 8\tilde{x}^{3} -12 \tilde{x}
	\end{aligned}
\end{align*}

% Sigui $\psi (x) = \alpha_{0} \phi_{0} + \alpha_{1} \phi_{1} + \alpha_{2} \phi_{2} + \dots$. Llavors,
\begin{align*}
\begin{gathered}
	E_{n} = \hbar \omega \qty(n + \frac{1}{2}) \qc p(E = E_{n})_{\psi} = \abs{\alpha_{n}}^{2}\\
	% (\Delta E)_{\psi} = \sqrt{\ev{E^{2}}_{\psi} - \ev{E}_{\psi}^{2}}
\end{gathered}
\end{align*}
% \begin{align*}
% 	\ev{\hat{H}}_{\psi} = \sum \abs{\alpha_{n}}^{2} E_{n} \qc \ev{\hat{H}^{2}}_{\psi} = \sum \abs{\alpha_{n}}^{2} E_{n}^{2}
% \end{align*}

% \begin{align*}
% \begin{gathered}
% 	\ev{x}_{\psi} = \int_{-\infty}^{\infty} \psi\sast x \psi \dd{x} \qc \ev{p}_{\psi} = \int_{-\infty}^{\infty} - \psi\sast i \hbar \pdv{x} \psi \dd{x} \\
% 	\Delta x \Delta p \geq \dfrac{\hbar}{2}
% \end{gathered}
% \end{align*}

% %-----------------------------------------------------------------
% \subsubsection*{Evolució temporal}
% \begin{align*}
% 	\psi(x,t) = \sum_{n} a_{n} \phi_{n}(x,0) \exp[-\frac{i E_{n}}{\hbar} t]
% \end{align*}

%-----------------------------------------------------------------
\subsection{Potencials centrals}
\begin{align*}
	\boxed{\phi_{nlm}(r, \theta, \varphi) = R_{n}^{l}(r) Y_{l}^{m}(\theta, \varphi)}
	\qc
	\begin{cases}
		l = 0, 1, \cdots, n-1 \\
		m = -l, -l-1, \cdots, l
	\end{cases}
\end{align*}

\begin{align*}
	\hat{H} \ket{\phi_{nlm}} &= -R_{y}/n^{2} \, \ket{\phi_{nlm}} \qc R_{y} = \SI{13.6}{\eV} \\
	\hat{L^{2}} \ket{\phi_{nlm}} &= \hbar^{2} l(l+1) \, \ket{\phi_{nlm}} \\
	\hat{L_{z}} \ket{\phi_{nlm}} &= \hbar m \, \ket{\phi_{nlm}} \\
	E_{n} &= - \frac{m}{2\hbar^{2}} \qty(\frac{z e^{2}}{4\pi \varepsilon_{0} n})^{2} \quad \text{(en general)}
\end{align*}

\subsubsection*{Harmònics 3D}
\begin{align*}
\begin{aligned}
	Y_{0}^{0} &= \sqrt{\frac{1}{4\pi}} \\
	Y_{1}^{0} &= \sqrt{\frac{3}{4\pi}} \cos\theta \\
	Y_{1}^{\pm 1} &= \mp\sqrt{\frac{3}{8\pi}} \sin\theta e^{\pm i\varphi} \\
\end{aligned}
\quad
\begin{aligned}
	Y_{2}^{0} &= \sqrt{\frac{5}{16\pi}} (3 \cos^{2}\theta-1) \\
	Y_{2}^{\pm 1} &= \mp \sqrt{\frac{15}{8\pi}} \sin\theta \cos\theta e^{\pm i\varphi} \\
	Y_{2}^{\pm 2} &= \sqrt{\frac{15}{32\pi}} \sin^{2}\theta e^{\pm 2i\varphi} \\
\end{aligned}
\end{align*}

\subsubsection*{Funcions d'ona radials}
\begin{align*}
	R_{1}^{0} &= 2 \qty(\frac{1}{a_{0}})^{3/2} \exp[-\frac{r}{a_{0}}] \\
	R_{2}^{1} &= \frac{1}{\sqrt{3}} \qty(\frac{1}{2 a_{0}})^{3/2} \qty(\frac{r}{a_{0}}) \exp[-\frac{r}{2a_{0}}] \\
	R_{2}^{0} &= 2 \qty(\frac{1}{2 a_{0}})^{3/2} \qty(1 - \frac{r}{2a_{0}}) \exp[-\frac{r}{2a_{0}}] \\
	R_{3}^{2} &= \frac{2\sqrt{2}}{27\sqrt{5}} \qty(\frac{1}{3 a_{0}})^{3/2} \qty(\frac{r}{a_{0}})^{2} \exp[-\frac{r}{3a_{0}}] \\
	R_{3}^{1} &= \frac{4\sqrt{2}}{3} \qty(\frac{1}{3 a_{0}})^{3/2} \qty(\frac{r}{a_{0}}) \qty(1 - \frac{r}{6 a_{0}}) \exp[-\frac{r}{3a_{0}}] \\
	R_{3}^{0} &= 2 \qty(\frac{1}{3 a_{0}})^{3/2} \qty(1 - \frac{2r}{3a_{0}} + \frac{2 r^{2}}{27 a_{0}^{2}}) \exp[-\frac{r}{3a_{0}}]
\end{align*}

\subsubsection*{Àtom d'hidrogen}
\begin{align*}
	V(r) = - \frac{e^{2}}{4\pi \varepsilon_{0}r} \quad \text{(energia potencial)}
\end{align*}

\subsubsection*{Probabilitats marginals}
\begin{align*}
	\rho(r,\theta,\varphi) \dd{V} =
	\begin{cases}
		\rho(r) \dd{r} \\
		\rho(\theta) \dd{\theta} \\
		\rho(\varphi) \dd{\varphi}
	\end{cases}
	\Rightarrow
	P[\zeta \leq a] = \int_{0}^{a} \rho(\zeta) \dd{\zeta}
\end{align*}
on $\zeta \text{ pot ser } r, \theta, \varphi$. Recordem $(x,y,z) = (r\Sin \theta \Cos \varphi, r \Sin \theta \Sin \varphi, r\Cos \theta)$ i $\dd{V} = r^{2} \Sin \theta \dd{r} \dd{\theta} \dd{\varphi}$.
\begin{align*}
\begin{gathered}
	\zeta = \zeta_{\max} \Leftrightarrow \dv{\rho(\zeta)}{\zeta} \equiv 0
\end{gathered}
\end{align*}

%-----------------------------------------------------------------
\subsection{Teoremes}
\begin{align*}
	\dv{\ev{x}}{t} &= \frac{\ev{p}}{m} \qc \dv{\ev{p}}{t} = \ev{\vec{F}(x)} = - \ev{\grad{V(x)}} \tag{Ehrenfest} \\
	\dv{\ev{\va{r} \vdot \va{p}}}{t} &= \frac{\ev{p^{2}}}{m} - \ev{\va{r} \cdot \grad{V(x)}} \tag{Virial} \\
	2 \ev{K} &= \frac{\ev{p^{2}}}{m} = \ev{\va{r} \cdot \grad{V(x)}} \tag{Virial; estat estacionari}
\end{align*}

%-----------------------------------------------------------------
\subsection{Integrals i coses maques}
\begin{align*}
\begin{gathered}
	\int_{-\infty}^{\infty} e^{-ax^2} x^{2} \dd{x} = \dfrac{1}{2a} \sqrt{\dfrac{\pi}{a}} \qc \int_{-\infty}^{\infty} e^{-ax^2} x^{4} \dd{x} = \dfrac{3}{4a^{2}} \sqrt{\dfrac{\pi}{a}} \\
	\int_{0}^{\infty} e^{-ar} r^{n} \dd{r} = \dfrac{n!}{(a)^{n+1}} \qc 0 \leq n \\
\end{gathered}
\end{align*}
\begin{align*}
	\comm{A}{f(B)} = f'(B) \comm{A}{B}
\end{align*}

%-----------------------------------------------------------------
\subsection{Trigonometria}
\begin{align*}
	\begin{aligned}
		\sin^{2} \theta &= \frac{1}{2}(1 - \cos 2\theta) \\
		\cos^{2} \theta &= \frac{1}{2}(1 + \cos 2\theta) = \frac{3 \cos^{2} \theta - 1 + \sin^{2} \theta}{2} \\
		\sin(\theta \pm \phi) &= \sin \theta \cos \phi \pm \cos \theta \sin \phi \\
		\cos(\theta \pm \phi) &= \cos \theta \cos \phi \mp \sin \theta \sin \phi
	\end{aligned}
\end{align*}

%-----------------------------------------------------------------
\newpage
\subsection{Operadors sobre estats de Fock}
\begin{align*}
	x = \sqrt{\frac{\hbar}{2m\omega}} \qty(a\sdag + a) \qc p = i \sqrt{\frac{\hbar m \omega}{2}} \qty(a\sdag - a)
\end{align*}
\subsubsection*{Posició}
\begin{align*}
\begin{aligned}
	x \ket{0} &\to \ket{1}, \\
	x \ket{1} &\to \ket{0} + \sqrt{2} \ket{3}, \\
	x \ket{2} &\to \sqrt{2} \ket{1} + \sqrt{3} \ket{3}, \\
	x \ket{3} &\to \sqrt{3} \ket{2} + \sqrt{4} \ket{4}, \\
	x \ket{4} &\to \sqrt{4} \ket{3} + \sqrt{5} \ket{5}, \\
	x \ket{5} &\to \sqrt{5} \ket{4} + \sqrt{6} \ket{6},
\end{aligned} \quad
\begin{aligned}
	x^{2} \ket{0} &\to \ket{0} + \sqrt{2} \ket{2} \\
	x^{2} \ket{1} &\to 3 \ket{1} + \sqrt{6} \ket{3} \\
	x^{2} \ket{2} &\to \sqrt{2} \ket{0} + 5 \ket{2} + 2 \sqrt{3} \ket{4} \\
	x^{2} \ket{3} &\to \sqrt{6} \ket{1} + 7 \ket{3} + 2 \sqrt{5} \ket{5} \\
	x^{2} \ket{4} &\to 2 \sqrt{3} \ket{2} + 9 \ket{4} + \sqrt{30} \ket{6} \\
	x^{2} \ket{5} &\to 2 \sqrt{5} \ket{3} + 11 \ket{5} + \sqrt{42} \ket{7}
\end{aligned}
\end{align*}

\subsubsection*{Moment}
\begin{align*}
\begin{aligned}
	p \ket{0} &\to \ket{1}, \\
	p \ket{1} &\to -\ket{0} + \sqrt{2} \ket{3}, \\
	p \ket{2} &\to -\sqrt{2} \ket{1} + \sqrt{3} \ket{3}, \\
	p \ket{3} &\to -\sqrt{3} \ket{2} + \sqrt{4} \ket{4}, \\
	p \ket{4} &\to -\sqrt{4} \ket{3} + \sqrt{5} \ket{5}, \\
	p \ket{5} &\to -\sqrt{5} \ket{4} + \sqrt{6} \ket{6},
\end{aligned} \quad
\begin{aligned}
	p^{2} \ket{0} &\to -\ket{0} + \sqrt{2} \ket{2} \\
	p^{2} \ket{1} &\to -3 \ket{1} + \sqrt{6} \ket{3} \\
	p^{2} \ket{2} &\to \sqrt{2} \ket{0} - 5 \ket{2} + 2 \sqrt{3} \ket{4} \\
	p^{2} \ket{3} &\to \sqrt{6} \ket{1} - 7 \ket{3} + 2 \sqrt{5} \ket{5} \\
	p^{2} \ket{4} &\to 2 \sqrt{3} \ket{2} - 9 \ket{4} + \sqrt{30} \ket{6} \\
	p^{2} \ket{5} &\to 2 \sqrt{5} \ket{3} - 11 \ket{5} + \sqrt{42} \ket{7}
\end{aligned}
\end{align*}


\end{document}

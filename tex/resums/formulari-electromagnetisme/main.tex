%-----------------------------------------------------------------
%	BASIC DOCUMENT LAYOUT
%-----------------------------------------------------------------
\documentclass[paper=a4, fontsize=8pt, twocolumn]{scrartcl}
\usepackage[T1]{fontenc}
\usepackage[utf8]{inputenc}
\usepackage{lmodern}
\usepackage{microtype}
\usepackage[catalan]{babel}
\usepackage[fixlanguage]{babelbib}
\selectbiblanguage{catalan}

% Sectioning layout
\addtokomafont{sectioning}{\normalfont\scshape}
\usepackage{tocstyle}
\usetocstyle{standard}
\renewcommand*\descriptionlabel[1]{\hspace\labelsep\normalfont\bfseries{#1}}

% Empty pages
\usepackage{etoolbox}

% Paragraph indentation behaviour
\setlength{\parindent}{0pt}
\setlength{\parskip}{0.3\baselineskip plus2pt minus2pt}
\newcommand{\sk}{\medskip\noindent}

% Fancy header and footer
\usepackage{fancyhdr}
\pagestyle{fancyplain}
\fancyhead[L]{\mytitle}
\fancyhead[R]{\myauthor}
\fancyfoot[C]{\thepage}
\fancypagestyle{firststyle}
{
    \fancyhead[L]{\mytitle}
    \fancyhead[R]{\myauthor}
    \fancyfoot[C]{\thepage}
}
\renewcommand{\headrulewidth}{0.3pt}
\renewcommand{\footrulewidth}{0pt}
\setlength{\headheight}{13.6pt}

%-----------------------------------------------------------------
%	MATHS AND SCIENCE
%-----------------------------------------------------------------
\usepackage{amsmath,amsfonts,amsthm,amssymb}
\usepackage{xfrac}
\usepackage[a]{esvect}
\usepackage{chemformula}
\usepackage{graphicx}

\usepackage[arrowdel]{physics}
    \renewcommand{\vnabla}{\vec{\nabla}}
    % \renewcommand{\vectorbold}[1]{\boldsymbol{#1}}
    % \renewcommand{\vectorarrow}[1]{\vec{\boldsymbol{#1}}}
    % \renewcommand{\vectorunit}[1]{\hat{\boldsymbol{#1}}}
    \renewcommand{\vectorarrow}[1]{\vec{#1}}
    \renewcommand{\vectorunit}[1]{\hat{#1}}
    \renewcommand*{\grad}[1]{\vnabla #1}
    \renewcommand*{\div}[1]{\vnabla \vdot \va{#1}}
    \renewcommand*{\curl}[1]{\vnabla \cp \va{#1}}
    \let\rot\curl

% SI units
\usepackage[separate-uncertainty=true]{siunitx}
\sisetup{range-phrase = \text{--}}
\DeclareSIPrePower\quartic{4}
	%\DeclareSIUnit\micron{\micro\metre}

% Smaller trig functions
\newcommand{\Sin}{\trigbraces{\operatorname{s}}}
\newcommand{\Cos}{\trigbraces{\operatorname{c}}}
\newcommand{\Tan}{\trigbraces{\operatorname{t}}}

% Operator-style notation for matrices
\newcommand*{\mat}[1]{\hat{#1}}

% Matrices in (A|B) form via [c|c] option
\makeatletter
\renewcommand*\env@matrix[1][*\c@MaxMatrixCols c]{%
  \hskip -\arraycolsep
  \let\@ifnextchar\new@ifnextchar
  \array{#1}}
\makeatother

% Shorter \mathcal and \mathbb
\newcommand*{\mc}[1]{\mathcal{#1}}
\newcommand*{\mbb}[1]{\mathbb{#1}}

% Shorter ^\ast and ^\dagger
\newcommand*{\sast}{^{\ast}{}}
\newcommand*{\sdag}{^{\dagger}{}}

%-----------------------------------------------------------------
%	OTHER PACKAGES
%-----------------------------------------------------------------
\usepackage{environ}

%Left numbered equations
\makeatletter
    \NewEnviron{Lalign}{\tagsleft@true\begin{align}\BODY\end{align}}
\makeatother

% Plots and graphics
\usepackage{pgfplots}
\usepackage{tikz}
\usepackage{color}
	\makeatletter
		\color{black}
		\let\default@color\current@color
	\makeatother

% Richer enumerate, figure, and table support
\usepackage{enumerate}
\usepackage[shortlabels]{enumitem}
\usepackage{float}
\usepackage{tabularx}
\usepackage{booktabs}
	%\setlength{\intextsep}{8pt}
\numberwithin{equation}{section}
\numberwithin{figure}{section}
\numberwithin{table}{section}

% No indentation after certain environments
\makeatletter
\newcommand*\NoIndentAfterEnv[1]{%
	\AfterEndEnvironment{#1}{\par\@afterindentfalse\@afterheading}}
\makeatother
%\NoIndentAfterEnv{thm}
\NoIndentAfterEnv{defi}
\NoIndentAfterEnv{example}
\NoIndentAfterEnv{table}

% Misc packages
\usepackage{ccicons}
\usepackage{lipsum}

%-----------------------------------------------------------------
%	THEOREMS
%-----------------------------------------------------------------
\usepackage{thmtools}

% Theroems layout
\declaretheoremstyle[
    spaceabove=6pt, spacebelow=6pt,
    headfont=\normalfont,
    notefont=\mdseries, notebraces={(}{)},
    bodyfont=\small,
    postheadspace=1em,
]{small}

\declaretheorem[style=plain,name=Teorema,qed=$\square$,numberwithin=section]{thm}
\declaretheorem[style=plain,name=Corol·lari,qed=$\square$,sibling=thm]{cor}
\declaretheorem[style=plain,name=Lema,qed=$\square$,sibling=thm]{lem}
\declaretheorem[style=definition,name=Definició,qed=$\blacksquare$,numberwithin=section]{defi}
\declaretheorem[style=definition,name=Exemple,qed=$\blacktriangle$,numberwithin=section]{example}
\declaretheorem[style=small,name=Demostració,numbered=no,qed=$\square$]{sproof}

%-----------------------------------------------------------------
%	ELA MOTHERFUCKING GEMINADA
%-----------------------------------------------------------------
\def\xgem{%
	\ifmmode
		\csname normal@char\string"\endcsname l%
	\else
		\leftllkern=0pt\rightllkern=0pt\raiselldim=0pt
		\setbox0\hbox{l}\setbox1\hbox{l\/}\setbox2\hbox{.}%
		\advance\raiselldim by \the\fontdimen5\the\font
		\advance\raiselldim by -\ht2
		\leftllkern=-.25\wd0%
		\advance\leftllkern by \wd1
		\advance\leftllkern by -\wd0
		\rightllkern=-.25\wd0%
		\advance\rightllkern by -\wd1
		\advance\rightllkern by \wd0
		\allowhyphens\discretionary{-}{}%
		{\kern\leftllkern\raise\raiselldim\hbox{.}%
			\kern\rightllkern}\allowhyphens
	\fi
}
\def\Xgem{%
	\ifmmode
		\csname normal@char\string"\endcsname L%
	\else
		\leftllkern=0pt\rightllkern=0pt\raiselldim=0pt
		\setbox0\hbox{L}\setbox1\hbox{L\/}\setbox2\hbox{.}%
		\advance\raiselldim by .5\ht0
		\advance\raiselldim by -.5\ht2
		\leftllkern=-.125\wd0%
		\advance\leftllkern by \wd1
		\advance\leftllkern by -\wd0
		\rightllkern=-\wd0%
		\divide\rightllkern by 6
		\advance\rightllkern by -\wd1
		\advance\rightllkern by \wd0
		\allowhyphens\discretionary{-}{}%
		{\kern\leftllkern\raise\raiselldim\hbox{.}%
			\kern\rightllkern}\allowhyphens
	\fi
}

\expandafter\let\expandafter\saveperiodcentered
	\csname T1\string\textperiodcentered \endcsname

\DeclareTextCommand{\textperiodcentered}{T1}[1]{%
	\ifnum\spacefactor=998
		\Xgem
	\else
		\xgem
	\fi#1}

%-----------------------------------------------------------------
%	PDF INFO AND HYPERREF
%-----------------------------------------------------------------
\usepackage{hyperref}
\hypersetup{colorlinks, citecolor=black, filecolor=black, linkcolor=black, urlcolor=black}

\newcommand*{\mytitle}{Formulari electromagnetisme}
\newcommand*{\myauthor}{Alfredo Hernández Cavieres}
\newcommand*{\mydate}{\normalsize \today}

\pdfstringdefDisableCommands{\def\and{i }}

\usepackage{hyperxmp}
\hypersetup{pdfauthor={\myauthor}, pdftitle={\mytitle}}

%-----------------------------------------------------------------
%	DOCUMENT BODY
%-----------------------------------------------------------------
\begin{document}

%\tableofcontents
\thispagestyle{firststyle}
%----------------------------------------------------------------------------------------
%    TEMA
%----------------------------------------------------------------------------------------
\section{\mytitle}
\subsection{Tensor d'inèrcia}
\begin{align}
    \mat{I} = \hat{\int} \dif m \begin{pmatrix} x^{2} & -xy & -xz \\ -yx & y^{2} & -yz \\ -zx & -zy & z^{2} \end{pmatrix}
\end{align}
on $x^{2} \equiv y^{2} + z^{2}$, $y^{2} \equiv x^{2} + z^{2}$, i $z^{2} \equiv x^{2} + y^{2}$.

\begin{align}
    T = T_{tras} + T_{rot}
\end{align}
\begin{align}
    T_{rot} = \frac{1}{2} \lrbra{\vec{\omega}^{t} \cdot \mat{I} \cdot \vec{\omega}}
\end{align}

\begin{align}
    \vec{L} = \vec{L}_{cm} + \vec{L}_{rot}
\end{align}
\begin{align}
    \vec{L}_{rot} = \mat{I} \times \vec{\omega} \mid \vec{L} = \vec{r} \times \vec{p}
\end{align}
%----------------------------------------------------------------------------------------
\subsection{Angles d'Euler (rotacions)}
Rotació: $\mat{O} \cdot \vec{x}$.
\begin{align}
    \text{Eix $z_{f}$: } \mat{O}_{\varphi} = \begin{pmatrix} \Cos \varphi & \Sin \varphi & 0 \\ -\Sin \varphi & \Cos \varphi & 0 \\ 0 & 0 & 1 \end{pmatrix}
\end{align}
\begin{align}
    \text{Eix $x_{girat}$: } \mat{O}_{\theta} = \begin{pmatrix} 1 & 0 & 0 \\ 0 & \Cos \theta & \Sin \theta \\ 0 & -\Sin \theta & \Cos \theta \end{pmatrix}
\end{align}
\begin{align}
    \text{Eix $z_{m}$: } \mat{O}_{\psi} = \begin{pmatrix} \Cos \psi & \Sin \psi & 0 \\ -\Sin \psi & \Cos \psi & 0 \\ 0 & 0 & 1 \end{pmatrix}
\end{align}

\begin{align}
    \begin{pmatrix} N_{1} \\ N_{2} \\ N_{3} \end{pmatrix} = \begin{pmatrix} I_{1} \dot{\omega}_{1} + (I_{3} - I_{2}) \omega_{2} \omega_{3} \\ I_{2} \dot{\omega}_{2} + (I_{1} - I_{3}) \omega_{3} \omega_{1} \\ I_{3} \dot{\omega}_{3} + (I_{2} - I_{1}) \omega_{1} \omega_{2} \end{pmatrix}
\end{align}

\begin{align}
    \text{Precessió: } \vec{N} = \vec{\Omega} \times \vec{L}
\end{align}

%----------------------------------------------------------------------------------------
\subsection{Sistemes de referència mòbils}
\begin{align}
\begin{aligned}
    m \ddot{\vec{r}}' &=  m \vec{a}_{f} \overbrace{- m \ddot{\vec{R}}_{f} - m \dot{\vec{\omega}} \times \vec{r}'}^{\text{arrossegament}} \\
    & \underbrace{- 2m \vec{\omega} \times \dot{\vec{r}}'}_{\text{coriol·lis}} \underbrace{- m \vec{\omega} \times \lrpar{\vec{\omega} \times \vec{r}'}}_{\text{centrífuga}} \\
\end{aligned}
\end{align}
$m \vec{a}_{f}$ és la for\c{c}a al SR inercial, mentre que $m \ddot{\vec{r}}'$ és la for\c{c} aparent al SR mòbil.

\begin{align}
\begin{aligned}
    \vec{\omega}_{T} &\mapsto \text{SRm (per coriolis): } \\
    \vec{\omega} &= \omega \lrpar{-\Cos \lambda, 0, \Sin \lambda} \text{(nord)} \\
    \vec{\omega} &= \omega \lrpar{-\Cos \lambda, 0, -\Sin \lambda} \text{(sud)}
\end{aligned}
\end{align}

%----------------------------------------------------------------------------------------
\subsection{Foucault}
\begin{align}
    \begin{pmatrix} x \\ y \end{pmatrix} = \begin{pmatrix} \Cos \omega_{z} t & \Sin \omega_{z} t \\ -\Sin \omega_{z} t & \Cos \omega_{z} t \end{pmatrix} \begin{pmatrix} x_{0} \\ y_{0} \end{pmatrix}
\end{align}
\begin{align}
    \Delta t = \frac{2 \pi}{ \omega_{z}} = \frac{2 \pi}{ \omega \Sin \lambda}
\end{align}
on $\omega_{z}$ és la component $z$ de $\omega$ al SRm i $\lambda$ és la latitud.

%----------------------------------------------------------------------------------------
\subsection{Punt de percussió}
\begin{align}
    p_{p} = \frac{I}{m R}
\end{align}
on $R$ és la distància del centre de masses a l'extrem.


\end{document}

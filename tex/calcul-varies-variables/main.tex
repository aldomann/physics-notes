%-----------------------------------------------------------------
%	BASIC DOCUMENT LAYOUT
%-----------------------------------------------------------------
\documentclass[paper=a4, fontsize=12pt, twoside=semi]{scrartcl}
\usepackage[T1]{fontenc}
\usepackage[utf8]{inputenc}
\usepackage{lmodern}
\usepackage{slantsc}
\usepackage{microtype}
\usepackage[catalan]{babel}
\usepackage[fixlanguage]{babelbib}
\selectbiblanguage{catalan}

% Sectioning layout
\addtokomafont{sectioning}{\normalfont\scshape}
\usepackage{tocstyle}
\usetocstyle{standard}
\renewcommand*\descriptionlabel[1]{\hspace\labelsep\normalfont\bfseries{#1}}

% Empty pages
\usepackage{etoolbox}
\pretocmd{\section}{\cleardoubleevenemptypage}{}{}
\pretocmd{\part}{\cleardoubleevenemptypage\thispagestyle{empty}}{}{}
\renewcommand\partheadstartvskip{\clearpage\null\vfil}
\renewcommand\partheadmidvskip{\par\nobreak\vskip 20pt\thispagestyle{empty}}

% Paragraph indentation behaviour
\setlength{\parindent}{0pt}
\setlength{\parskip}{0.3\baselineskip plus2pt minus2pt}
\newcommand{\sk}{\medskip\noindent}

% Fancy header and footer
\usepackage{fancyhdr}
\pagestyle{fancyplain}
\fancyhead[LO]{\thepage}
\fancyhead[CO]{}
\fancyhead[RO]{\nouppercase{\mytitle}}
\fancyhead[LE]{\nouppercase{\leftmark}}
\fancyhead[CE]{}
\fancyhead[RE]{\thepage}
\fancyfoot{}
\renewcommand{\headrulewidth}{0.3pt}
\renewcommand{\footrulewidth}{0pt}
\setlength{\headheight}{13.6pt}

%-----------------------------------------------------------------
%	MATHS AND SCIENCE
%-----------------------------------------------------------------
\usepackage{amsmath,amsfonts,amsthm,amssymb}
\usepackage{xfrac}
\usepackage[a]{esvect}
\usepackage{chemformula}
\usepackage{graphicx}

% SI units
\usepackage[separate-uncertainty=true]{siunitx}
	%\DeclareSIUnit\micron{\micro\metre}

% Custom commands and operators
\newcommand*{\dif}{\mathrm{d}}
\newcommand*{\diff}{\mathop{}\!\mathrm{d}}
\newcommand*{\der}[3][]{\frac{\dif^{#1}#2}{\dif #3^{#1}}}
\newcommand*{\pder}[3][]{\frac{\partial^{#1}#2}{\partial #3^{#1}}}

\newcommand*{\abs}[1]{\left| #1 \right|}
\newcommand*{\avg}[1]{\left< #1 \right>}
\newcommand*{\norm}[1]{\| #1 \|}
\newcommand*{\eval}[1]{\left. #1 \right|}

\newcommand*{\vnabla}{\vec{\nabla}}
\newcommand*{\grad}[1]{\vnabla #1}
\let\divsymb=\div
\renewcommand*{\div}[1]{\vnabla \cdot #1}
\newcommand*{\rot}[1]{\vnabla \times #1}

% Dirac quantum notation
%\newcommand{\ket}[1]{\left| #1 \right>} % for Dirac kets
%\newcommand{\bra}[1]{\left< #1 \right|} % for Dirac bras
%\newcommand{\braket}[2]{\left< #1 \vphantom{#2} \right| \left. #2 \vphantom{#1} \right>} % for Dirac brackets
%\newcommand{\matrixel}[3]{\left< #1 \vphantom{#2#3} \right| #2 \left| #3 \vphantom{#1#2} \right>} % for Dirac matrix elements

% FLA-style notation for matrices
\usepackage{stackengine}
\newcommand*{\vecsign}{\mathchar"017E}
\newcommand*{\dvecsign}{\smash{\stackon[-2.60pt]{\vecsign}{\rotatebox{180}{$\vecsign$}}}}
\newcommand*{\mat}[1]{\def\useanchorwidth{T}\stackon[-5.4pt]{\mathcal{#1}}{\,\dvecsign}}
\stackMath

% Matrices in (A|B) form via [c|c] option
\makeatletter
\renewcommand*\env@matrix[1][*\c@MaxMatrixCols c]{%
  \hskip -\arraycolsep
  \let\@ifnextchar\new@ifnextchar
  \array{#1}}
\makeatother

%-----------------------------------------------------------------
%	OTHER PACKAGES
%-----------------------------------------------------------------
\usepackage{environ}

% Plots and graphics
\usepackage{pgfplots}
\usepackage{tikz}
\usepackage{color}
	\makeatletter
		\color{black}
		\let\default@color\current@color
	\makeatother

% Richer enumerate, figure, and table support
\usepackage{enumerate}
\usepackage{float}
\usepackage{booktabs}
	%\setlength{\intextsep}{8pt}
\numberwithin{equation}{section}
\numberwithin{figure}{section}
\numberwithin{table}{section}

% No indentation after certain environments
\makeatletter
\newcommand*\NoIndentAfterEnv[1]{%
	\AfterEndEnvironment{#1}{\par\@afterindentfalse\@afterheading}}
\makeatother
%\NoIndentAfterEnv{thm}
\NoIndentAfterEnv{defi}
\NoIndentAfterEnv{example}
\NoIndentAfterEnv{table}

% Misc packages
\usepackage{ccicons}
\usepackage{lipsum}

%-----------------------------------------------------------------
%	THEOREMS
%-----------------------------------------------------------------
\usepackage{thmtools}

% Proofatend environment
\makeatletter
\providecommand{\@fourthoffour}[4]{#4}
\newcommand\fixstatement[2][\proofname\space del]{%
	\ifcsname thmt@original@#2\endcsname
		\AtEndEnvironment{#2}{%
			\xdef\pat@label{\expandafter\expandafter\expandafter
				\@fourthoffour\csname thmt@original@#2\endcsname\space\@currentlabel}%
			\xdef\pat@proofof{\@nameuse{pat@proofof@#2}}%
		}%
	\else
		\AtEndEnvironment{#2}{%
			\xdef\pat@label{\expandafter\expandafter\expandafter
				\@fourthoffour\csname #1\endcsname\space\@currentlabel}%
			\xdef\pat@proofof{\@nameuse{pat@proofof@#2}}%
		}%
	\fi
	\@namedef{pat@proofof@#2}{#1}%
}
\globtoksblk\prooftoks{1000}
\newcounter{proofcount}
\NewEnviron{proofatend}{%
	\edef\next{%
		\noexpand\begin{proof}[\pat@proofof\space\pat@label]%
		\unexpanded\expandafter{\BODY}}%
	\global\toks\numexpr\prooftoks+\value{proofcount}\relax=\expandafter{\next\end{proof}}
	\stepcounter{proofcount}}
\def\printproofs{%
	\count@=\z@
	\loop
		\the\toks\numexpr\prooftoks+\count@\relax
			\ifnum\count@<\value{proofcount}%
			\advance\count@\@ne
	\repeat}
\makeatother

% Theroems layout
\declaretheoremstyle[
	spaceabove=6pt, spacebelow=6pt,
	headfont=\normalfont,
	notefont=\mdseries, notebraces={(}{)},
	bodyfont=\small,
	postheadspace=1em,
]{small}

\declaretheorem[style=plain,name=Teorema,qed=$\square$,numberwithin=section]{thm}
\declaretheorem[style=plain,name=Corol·lari,qed=$\square$,sibling=thm]{cor}
\declaretheorem[style=plain,name=Lemma,qed=$\square$,sibling=thm]{lem}
\declaretheorem[style=definition,name=Definició,qed=$\blacksquare$,numberwithin=section]{defi}
\declaretheorem[style=definition,name=Exemple,qed=$\blacktriangle$,numberwithin=section]{example}
\declaretheorem[style=small,name=Demostració,numbered=no,qed=$\square$]{sproof}
\fixstatement{thm}
\fixstatement[Demostració del]{lem}

%-----------------------------------------------------------------
%	ELA MOTHERFUCKING GEMINADA
%-----------------------------------------------------------------
\def\xgem{%
	\ifmmode
		\csname normal@char\string"\endcsname l%
	\else
		\leftllkern=0pt\rightllkern=0pt\raiselldim=0pt
		\setbox0\hbox{l}\setbox1\hbox{l\/}\setbox2\hbox{.}%
		\advance\raiselldim by \the\fontdimen5\the\font
		\advance\raiselldim by -\ht2
		\leftllkern=-.25\wd0%
		\advance\leftllkern by \wd1
		\advance\leftllkern by -\wd0
		\rightllkern=-.25\wd0%
		\advance\rightllkern by -\wd1
		\advance\rightllkern by \wd0
		\allowhyphens\discretionary{-}{}%
		{\kern\leftllkern\raise\raiselldim\hbox{.}%
			\kern\rightllkern}\allowhyphens
	\fi
}
\def\Xgem{%
	\ifmmode
		\csname normal@char\string"\endcsname L%
	\else
		\leftllkern=0pt\rightllkern=0pt\raiselldim=0pt
		\setbox0\hbox{L}\setbox1\hbox{L\/}\setbox2\hbox{.}%
		\advance\raiselldim by .5\ht0
		\advance\raiselldim by -.5\ht2
		\leftllkern=-.125\wd0%
		\advance\leftllkern by \wd1
		\advance\leftllkern by -\wd0
		\rightllkern=-\wd0%
		\divide\rightllkern by 6
		\advance\rightllkern by -\wd1
		\advance\rightllkern by \wd0
		\allowhyphens\discretionary{-}{}%
		{\kern\leftllkern\raise\raiselldim\hbox{.}%
			\kern\rightllkern}\allowhyphens
	\fi
}

\expandafter\let\expandafter\saveperiodcentered
	\csname T1\string\textperiodcentered \endcsname

\DeclareTextCommand{\textperiodcentered}{T1}[1]{%
	\ifnum\spacefactor=998
		\Xgem
	\else
		\xgem
	\fi#1}

%-----------------------------------------------------------------
%	PDF INFO AND HYPERREF
%-----------------------------------------------------------------
\usepackage{hyperref}
\hypersetup{colorlinks, citecolor=black, filecolor=black, linkcolor=black, urlcolor=black}

\newcommand*{\mytitle}{Càlcul en vàries variables}
\newcommand*{\mysubtitle}{}
\newcommand*{\myauthor}{Alfredo Hernández Cavieres}
\newcommand*{\myuni}{Universitat Autònoma de Barcelona, Departament de Física}
\newcommand*{\mydate}{\normalsize 2013-2014}

\pdfstringdefDisableCommands{\def\and{i }}

\usepackage{hyperxmp}
\hypersetup{pdfauthor={\myauthor}, pdftitle={\mytitle}}

%-----------------------------------------------------------------
%	TITLE SECTION AND DOCUMENT BEGINNING
%-----------------------------------------------------------------
\newcommand{\horrule}[1]{\rule{\linewidth}{#1}}
\title{
	\normalfont
	\small \scshape{\myuni} \\ [25pt]
	\horrule{0.5pt} \\[0.4cm]
	\huge \mytitle \\
	%\Large \scshape{\mysubtitle} \\
	\horrule{2pt} \\[0.5cm]
}
\author{\myauthor}
\date{\mydate}

\begin{document}

\clearpage\maketitle
\thispagestyle{empty}
\addtocounter{page}{-1}

%-----------------------------------------------------------------
%	LICENCE
%-----------------------------------------------------------------
\section*{}\thispagestyle{empty}
\begin{centering}
	\href{http://creativecommons.org/licenses/by-nc-sa/4.0/deed.ca}{\huge \ccbyncsaeu}

	\normalsize Aquesta obra està subjecta a una llicència de

	Reconeixement-NoComercial-CompartirIgual 4.0

	Internacional de Creative Commons.

\end{centering}

%-----------------------------------------------------------------
%	DOCUMENT BODY
%-----------------------------------------------------------------
\cleardoubleevenemptypage
\pdfbookmark[1]{\contentsname}{toc}
\tableofcontents

%\part*{Primera}
%\addcontentsline{toc}{part}{Primera}
	%----------------------------------------------------------------------------------------
%    L'ESPAI R^N
%----------------------------------------------------------------------------------------
\section{L'espai $\mathbb{R}^{n}$}
\subsection{El cos dels números reals}
$\mathbb{R}$ és un cos ordenat arquimedià complet.
\begin{itemize}
    \item Cos ordenat: cos $(\mathbb{R}, +, \cdot)$ amb ordenació total ($\leq$) que compleix:
        \subitem $x \leq y \Rightarrow x + z \leq y + z$.
        \subitem $x,y \geq 0 \Rightarrow xy \geq 0$.
    \item Arquimedià: no és fitat superiorment: $\forall b>0, \quad \exists a\in \mathbb{R} \mid na>b,  \quad  n  \in  \mathbb{N}$.
    \item Complet: tota successió de Cauchy és convergent en aquest cos. Això és equivalent a dir que $\mathbb{R}$ té la propietat de l'extrem.
\end{itemize}

%---------------------------------------------------------------------------------------
\subsection{Espai $\mathbb{R}^{n}$}
\begin{defi}[Producte escalar]
    És una operació $\mathbb{R}^{n} \times \mathbb{R}^{n} \to \mathbb{R}$.
    \begin{align}
        \vec{x} \cdot \vec{y} = (x_{1} y_{1}, x_{2} y_{2}, \dots , x_{n} y_{n})
    \end{align}
\end{defi}
El producte escalar compleix les següents propietats:
\begin{enumerate}[i)]
     \item És bilineal: 
        \subitem $\vec{x} \cdot (\vec{y} + \vec{z}) = \vec{x} \cdot \vec{y} + \vec{x} \cdot \vec{z}$.
        \subitem $(\vec{x} + \vec{y}) \cdot \vec{z} = \vec{x} \cdot \vec{z} + \vec{y} \cdot \vec{z}$.
        \subitem $\vec{x} \cdot (\lambda \vec{y}) = \lambda (\vec{x} \cdot \vec{y})$.
        \subitem $(\lambda \vec{x}) \cdot \vec{y} = \lambda (\vec{x} \cdot \vec{y})$.
    \item És simètric: $\vec{x} \cdot \vec{y} = -\vec{y} \cdot \vec{x}$.
    \item $\vec{x} \cdot \vec{x} > 0, \quad \forall \vec{x} \neq 0$.
\end{enumerate}

\begin{defi}[Mòdul d'un vector]
    \begin{align}
        \| \vec{x} \| \equiv \sqrt{\vec{x} \cdot \vec{x}} = \sqrt{x_{1}^{2} + x_{2}^{2} + \dots + x_{n}^{2}}
    \end{align}
    A més es compleix que $\vec{x} \cdot \vec{y} = \| \vec{x} \| \| \vec{y} \| \cos \alpha$.
\end{defi}
Propietats del mòdul:
\begin{enumerate}[i)]
    \item $\| \vec{x} \| > 0, \quad \forall \vec{x} \neq \vec{0}$.
    \item $ \| \vec{0} \| = 0$.
    \item $\| \lambda \vec{x} \| = \lambda \| \vec{x} \|$.
    \item $\|\vec{x} + \vec{y}\| \leq \| \vec{x} \| + \| \vec{y} \|$  (desigualtat triangular).
\end{enumerate}

\begin{defi}[Desigualtat de Schwarz]
    \begin{align}
        | \vec{x} \cdot \vec{y} | \leq \| \vec{x} \| \| \vec{y} \| 
    \end{align}
\end{defi}

\begin{defi}[Distància entre dos vectors]
    \begin{align}
        d(\vec{x}, \vec{y}) \| \vec{x} - \vec{y} \| = \sqrt{(x_{1} - y_{1})^{2} + (x_{2} - y_{2})^{2} + \dots + (x_{n} - y_{n})^{2}}
    \end{align}
\end{defi}
Propietats de la distància:
\begin{enumerate}[i)]
    \item $d(\vec{x}, \vec{y}) > 0, \quad \forall \vec{x} \neq \vec{y}$.
    \item $d(\vec{x}, \vec{x}) = 0$.
    \item $d(\vec{x}, \vec{y}) = d(\vec{y}, \vec{x})$.
    \item $d(\vec{x}, \vec{z}) \leq d(\vec{x}, \vec{y}) + d(\vec{y}, \vec{z})$  (desigualtat triangular).
\end{enumerate}
%----------------------------------------------------------------------------------------
\subsection{Successions a $\mathbb{R}^{n}$}
Una successió de $\mathbb{R}^{n}$ és una aplicació de $\mathbb{N}$ sobre $\mathbb{R}^{n}$: $m \mapsto \vec{x}^{(m)}$, que denotem $\{ \vec{x}^{(m)} \} = \{ \vec{x}^{(1)}, \vec{x}^{(2)}, \vec{x}^{(3)}, \dots \}$.

\begin{defi}[Successió convergent]
    $\lim \{ \vec{x}^{(m)} \} = \vec{l}$ si $\forall \varepsilon > 0, \quad \exists n_{o} \mid d\left(\vec{x}^{(m)}, \vec{l}\right) < \varepsilon, \quad \forall m > n_{0}$.
\end{defi}

\begin{defi}[Successió de Cauchy]
    $\{ \vec{x}^{(m)} \}$ és de Cauchy si $\forall \varepsilon > 0, \quad \exists n_{o} \mid d(\vec{x}^{(l)}, \vec{x}^{(m)}) < \varepsilon, \quad \forall l, m > n_{0}$.
    
    Com que totes les successions de Cauchy a $\mathbb{R}^{n}$ són convergents, $\mathbb{R}^{n}$ és complet.
\end{defi}

%----------------------------------------------------------------------------------------
\subsection{Topologia de $\mathbb{R}^{n}$}
\subsubsection*{Entorns}
\begin{itemize}
    \item Entorn de centre $\vec{a}$ i radi $r$: $\varepsilon (\vec{a}, r) \equiv \{ \vec{x} \mid d(\vec{x}, \vec{a} < r \}$.
    \item Entorn perforat: $\varepsilon^{\ast} (\vec{a}, r) = \varepsilon (\vec{a}, r) \backslash \{ \vec{a} \}$.
\end{itemize}
\subsubsection*{Tipus de punts}
\begin{itemize}
    \item Punt $\vec{a}$ interior a $A$: si $\exists r \mid \varepsilon (\vec{a}, r) \subset A$.
    \item Punt $\vec{a}$ exterior a $A$: si $\vec{a}$ és interior a $\bar{A}$.
    \item Punt $\vec{a}$ frontera de $A$: si no és interior ni exterior ($\Leftrightarrow$ tot entorn de $\vec{a}$ conté algun element de $A$ i de $\bar{A}$.
    \item Punt $\vec{a}$ d'acumulació de $A$: si tot entorn de $\vec{a}$ conté algun punt de $A$ diferent de $\vec{a}$.
\end{itemize}
\subsubsection*{Tipus de conjunts}
\begin{itemize}
    \item Conjunt obert: si tots els seus punts són interiors.
    \item Conjunt tancat: si conté tots els seus punts d'acumulació ($\Leftrightarrow$ el complementari és obert).
    \item Conjunt fitat: si està contingut en algun entorn de $\vec{0}$.
    \item Conjunt compacte: si tota successió té alguna successió parcial convergent ($\Leftrightarrow$ tancat i fitat).
\end{itemize}

%----------------------------------------------------------------------------------------
\subsection{Producte vectorial (a $\mathbb{R}^{3}$)}
\begin{defi}
    És una operació $\mathbb{R}^{3} \times \mathbb{R}^{3} \to \mathbb{R}^{3}$. Siguin $\vec{x}$ , $\vec{y} \in \mathbb{R}^{3}$, llavors
    \begin{align}
        \vec{x} \times \vec{y} = \begin{vmatrix} \hat{e}_{1} & \hat{e}_{2} & \hat{e}_{3} \\ x_{1} & x_{2} & x_{3} \\ y_{1} & y_{2} & y_{3} \end{vmatrix}
    \end{align}
\end{defi}
Propietats del producte vectorial:
\begin{enumerate}[i)]
    \item És bilineal: 
        \subitem $\vec{x} \times (\vec{y} + \vec{z}) = \vec{x} \times \vec{y} + \vec{x} \times \vec{z}$.
        \subitem $(\vec{x} + \vec{y}) \times \vec{z} = \vec{x} \times \vec{z} + \vec{y} \times \vec{z}$.
        \subitem $\vec{x} \times (\lambda \vec{y}) = \lambda (\vec{x} \times \vec{y})$.
        \subitem $(\lambda \vec{x}) \times \vec{y} = \lambda (\vec{x} \times \vec{y})$.
    \item És antisimètric: $\vec{x} \times \vec{y} = -\vec{y} \times \vec{x}$.
    \item $\| \vec{x} \times \vec{y} \| = \| \vec{x} \| \| \vec{y} \| \sin \alpha$, on $\alpha$ és l'angle que formen.
\end{enumerate}

\subsubsection*{Producte mixt}
\begin{align}
    \vec{x} (\vec{y} \times \vec{z}) = \begin{vmatrix} x_{1} & x_{2} & x_{3} \\ y_{1} & y_{2} & y_{3} \\ z_{1} & z_{2} & z_{3} \end{vmatrix}
\end{align}

	%----------------------------------------------------------------------------------------
%    FUNCIONS A R^N
%----------------------------------------------------------------------------------------
\section{Funcions a $\mathbb{R}^{n}$, límits i continuïtat}
\subsection{Funcions a $\mathbb{R}^{n}$}
Sigui $D \subseteq \mathbb{R}^{n}$. Una funció és una aplicació de $D$ sobre $\mathbb{R}^{n}$: $\vec{x} \mapsto \vec{f}(\vec{x})$.
    \begin{defi}
    Són funcions amb valors a $\mathbb{R}$.
    \begin{itemize}
        \item $n=m=1$: funcions escalars d'una variable real, $y = f(x)$.
        \item $n>1, m=1$: camps escalars, $y = f(\vec{x})$.
    \end{itemize}
\end{defi}
\begin{defi}
    Són funcions amb valors a $\mathbb{R}^{m}$.
    \begin{itemize}
        \item $n=1, m>1$: funcions vectorials d'una variable real, $\vec{y} = \vec{f}(x)$.
        \item $n>1, m>1$: camps vectorials, $\vec{y} = \vec{f}(\vec{x})$.
    \end{itemize}
\end{defi}

%----------------------------------------------------------------------------------------
\subsection{Límit d'una funció}
Sigui $\vec{f}(\vec{x})$ una funció definida a $D \subseteq \mathbb{R}^{n}$, amb valors a $\mathbb{R}^{m}$ i sigui $\vec{a}$ un punt d'acumulació de $D$. Llavors, direm que
\begin{align}
    \lim_{\vec{x} \to \vec{a}} \vec{f}(\vec{x}) = \vec{l} \text{ si } \forall \varepsilon > 0, \exists \delta > 0 \mid \text{si }d(\vec{x}, \vec{a}) < \delta \Rightarrow d(\vec{f}(\vec{x}), \vec{l}) < \varepsilon
\end{align}
Propietats dels límits de funcions: si $\lim\limits_{\vec{x} \to \vec{a}} \vec{f}(\vec{x}) = \vec{A}$ i $\lim\limits_{\vec{x} \to \vec{a}} \vec{g}(\vec{x}) = \vec{B}$
\begin{enumerate}[i)]
    \item $\lim\limits_{\vec{x} \to \vec{a}} (\vec{f}(\vec{x}) + \vec{g}(\vec{x})) = \vec{A} + \vec{B}$.
    \item $\lim\limits_{\vec{x} \to \vec{a}} ( \lambda \vec{f}(\vec{x})) = \lambda \vec{A}$.
    \item $\lim\limits_{\vec{x} \to \vec{a}} ( \vec{f}(\vec{x}) \cdot \vec{g}(\vec{x}))= \vec{A} \cdot \vec{B}$.
    \item $\lim\limits_{\vec{x} \to \vec{a}} \| \vec{f}(\vec{x}) \| = \| \vec{A} \|$.
\end{enumerate}

%----------------------------------------------------------------------------------------
\subsection{Límits direccionals}
\begin{defi}
    Els límits d'una funció de $n$ variables en un punt $\vec{a}$ són els límits d'aquesta funció quan $\vec{x} \to \vec{a}$ seguint una trajectòria rectilínia.
\end{defi}

\subsubsection*{Límits direccionals de camps escalars}
Si $f(\vec{x})$ és un camp escalar i $\vec{u}$ és un vector de $\mathbb{R}^{n}$, definim el límit de $f(\vec{x})$ quan $\vec{x} \to \vec{a}$ en la direcció $\vec{u}$ com
\begin{align}
    \lim_{\vec{x} \to \vec{a}} f(\vec{x}) \equiv \lim_{\lambda \to 0^{+}} f(\vec{a} + \lambda \vec{u})
\end{align}
Si $\exists \lim\limits_{\vec{x} \to \vec{a}} f(\vec{x}) \Rightarrow \exists$ els límits direccionals de $f(\vec{x})$ i coincideixen en el punt $\vec{a}$. El recíproc no és cert.

\subsubsection*{Límits direccionals de camps vectorials}
Tal com passa als camps escalars, si $\exists \lim\limits_{\vec{x} \to \vec{a}} \vec{f}(\vec{x}) \Rightarrow \exists$ els límits direccionals de $\vec{f}(\vec{x})$ i coincideixen en el punt $\vec{a}$. El recíproc no és cert.

%----------------------------------------------------------------------------------------
\subsection{Continuïtat}
L'existència o no del $\lim\limits_{\vec{x} \to \vec{a}} ( \vec{f}(\vec{x})$ , així com el seu propi valor, depèn dels valors de $\vec{f}(\vec{x})$ al voltant del punt $\vec{a}$ i no del seu valor en el propi punt. De la comparació del límit amb el valor de la funció en surt el concepte de continuïtat:
\begin{align}
    \vec{f}(\vec{x}) \text{ és contínua en el punt } \vec{a} \text{ si } \lim\limits_{\vec{x} \to \vec{a}} \vec{f}(\vec{x}) = \vec{f}(\vec{a})
\end{align}
Propietats de les funcions contínues:
\begin{enumerate}[i)]
    \item Si $\vec{f}(\vec{x})$ i $\vec{g}(\vec{x})$ són contínues $\Rightarrow \vec{f}(\vec{x}) + \vec{g}(\vec{x})$, $\vec{f}(\vec{x}) \cdot \vec{g}(\vec{x})$ i $\vec{f}(\vec{x}) / \vec{g}(\vec{x})$ (si $\vec{g}(\vec{x}) \neq 0)$ són contínues.
    \item Si $\vec{f}(\vec{x})$ és contínua en $\vec{x} = \vec{a}$ i $\vec{g}(\vec{y})$ és contínua en $\vec{y} = \vec{f}(\vec{x}) \Rightarrow \vec{g}(\vec{f}(\vec{x}))$ és contínua en el punt $\vec{a}$.
    \item Una funció pot ser no contínua en el punt $\vec{a}$ i, en canvi, ser-ho a cadascuna de les variables separadament.
\end{enumerate}

\subsubsection*{Continuïtat uniforme en un domini}
\begin{defi}
    $\vec{f}(\vec{x})$ és uniformement contínua a $D$ si
    \begin{align}
        \forall \varepsilon > 0, \exists \delta > 0 \mid \text{si } \vec{x}, \vec{x}' \in D \text{ i } d(\vec{x}, \vec{x}') < \delta \Rightarrow d(\vec{f}(\vec{x}), \vec{f}(\vec{x})') < \varepsilon
    \end{align}
\end{defi}
\begin{thm}
    Si $\vec{f}(\vec{x})$ és contínua en un compacte $D \Rightarrow \vec{f}(\vec{x})$ és uniformement contínua en $D$.
\end{thm}
	%----------------------------------------------------------------------------------------
%    FUNCIONS VECTORIALS D'UNA VARIABLE
%----------------------------------------------------------------------------------------
\section{Funcions vectorials d'una variable}
\subsection{Funcions $\mathbb{R} \to \mathbb{R}^{m}$}
Sigui $\vec{f}(u)$ una funció definida a $D \subseteq \mathbb{R}$, amb valors a $\mathbb{R}^{m}$: $\vec{f}(u) \equiv (f_{1}(u), f_{2}(u), \dots , f_{m}(u))$.
\begin{defi}[Continuïtat]
    $\vec{f}(u)$ és contínua en el punt $u=a$ si $\lim_{u \to a} \vec{f}(u) = \vec{f}(a)$.
\end{defi}
\begin{defi}[Derivada de $\vec{f}(u)$ en el punt $a$]
    \begin{align*}
        f'(a) \equiv \lim_{h \to 0} \frac{\vec{f}(a+h) - \vec{f}(a)}{h} = (f'(a)_{1}, f'(a)_{2} \dots , f'(a)_{m})
    \end{align*}
\end{defi}
\begin{defi}[Integral de $\vec{f}(u)$ en el punt $a$]
    \begin{align*}
        \displaystyle \int_{a}^{b} \vec{f}(u) \diff u \equiv \left( \int_{a}^{b} f_{1}(u) \diff u, \int_{a}^{b} f_{2}(u) \diff u, \dots , \int_{a}^{b} f_{m}(u) \diff u \right)
    \end{align*}
\end{defi}
\begin{thm}[Teorema fonamental del càlcul]
    Si $\vec{f}(u)$ és integrable en $[a,b]$ i $\vec{F}(u)$ és primitiva de $\displaystyle \vec{f}(u) \Rightarrow \int_{a}^{b} \vec{f}(u) \diff u = \vec{F}(b) - \vec{F}(a)$.
\end{thm}
\begin{defi}[Funció mòdul]
    \begin{align*}
        \| \vec{f} \| (u) \equiv \| \vec{f}(u) \| = \sqrt{\sum\limits_{i=1}^{m} f_{i}(u)^{2}}
    \end{align*}
\end{defi}
\begin{thm}
    Si $\vec{f}(u)$ és integrable a $[a,b] \Rightarrow$ 
    \begin{align*}
        \left\| \int_{a}^{b} \vec{f}(u) \diff u \right\| \leq \int_{a}^{b} \| \vec{f}(u) \| \diff u
    \end{align*}
\end{thm}

%----------------------------------------------------------------------------------------
\subsection{Corbes}
Si $\vec{f}(u)$ és contínua en $[a,b]$, defineix un arc de corba a $\mathbb{R}^{m}$.
\begin{defi}[Classe d'una corba]
    Una corba $\vec{f}(u)$ és de classe $C_{[a,b]}^{n}$ sí la seva $n$-èsima derivada és contínua a $[a,b]$.
\end{defi}
\begin{defi}[Corba rectificable]
    Sigui $\Pi = \{u_{0}, u_{1}, u_{2}, \dots, u_{n}\}$ una partició de $[a,b]$, construïm la quantitat $\displaystyle l(\vec{f}, \Pi) = \sum\limits_{i=1}^{n} \| \vec{f}(u_{i}), \vec{f}(u_{i-1}) \|$. Òbviament $l(\vec{f}, \Pi') \geq l(\vec{f}, \Pi)$ si $\Pi'$ és és fina que $\Pi$.
    
    Si $\left\{ l(\vec{f}, \Pi);\forall \Pi \right\}$ és fitat superiorment direm que la corba $\vec{f}(u)$ és rectificable i al suprem $l(\vec{f}) = \sup \left\{ l(\vec{f}, \Pi);\forall \Pi \right\}$ l'anomenarem longitud de la corba.
\end{defi}
\begin{thm}
    Si $\vec{f}(u)$ és de classe $C_{[a,b]}^{1} \Rightarrow \vec{f}$ és rectificable i
    \begin{align}
        l(\vec{f}) = \int_{a}^{b} \sqrt{\sum\limits_{i=1}^{m} f'_{i}(u)^{2}} \diff u = \int_{a}^{b} \| \vec{f}(u) \| \diff u
    \end{align}
\end{thm}
\begin{defi}[Paràmetre arc]
    Si $\vec{f}(u)$ és de classe $C_{[a,b]}^{1}$, definim el paràmetre arc com la funció
    \begin{align}
        s(u) \equiv \int_{a}^{u} \| \vec{f}'(u') \| \diff u'
    \end{align}
    que mesura la longitud de la corba en funció del paràmetre $u$. Notem que $\displaystyle \frac{\dif s}{\dif u} = \| \vec{f}' (u) \| \geq 0, \quad \forall u \in [a,b]$.
\end{defi}
\begin{defi}
    Una corba $\vec{f}(u)$ és no singular si és de $C_{[a,b]}^{1}$ i $\vec{f}'(u) \neq 0, \quad \forall u \in [a,b]$.
    
    Si $\vec{f}(u)$ és no singular, llavors $s(u)$ és estrictament creixent, ja que $s'(u) = \| \vec{f}'(u) \| > 0$; també ho és la seva inversa $u(s)$, ja que $u'(s) = 1/s'(u) > 0$.
\end{defi}

%----------------------------------------------------------------------------------------
\subsection{Geometria d'una corba de $\mathbb{R}^{2}$ i $\mathbb{R}^{3}$}
Sigui $\vec{f}(u)$ una corba de classe $C_{[a,b]}^{2}$. Si entenem $u$ com a un temps, geomètricament entenem $\vec{f}(u)$ com a «posició», $\vec{f}'(u)$ com a  «velocitat» i $\vec{f}''(u)$: «acceleració».
\begin{defi}[Vector tangent unitari]
    És un vector tangent a la corba $\vec{f}(u)$.
    \begin{align}
        \hat{T} = \frac{\vec{f}'(u)}{\| \vec{f}'(u) \|}
    \end{align}
\end{defi}
\begin{defi}[Curvatura]
    $\kappa$ és la curvatura de la corba, que és $\geq 0$, per definició.
        \begin{align}
        \frac{\dif \hat{T}}{\dif s} = \kappa \hat{N} \quad \text{i} \quad \kappa = \frac{\| \vec{f}'(u) \times \vec{f}''(u) \|}{\| \vec{f}'(u) \|}^{3} 
    \end{align}
    On $\hat{N}$ és el vector normal unitari, perpendicular a $\hat{T}$.
\end{defi}
\begin{defi}[Radi de curvatura]
    L'invers de la curvatura és el radi de curvatura.
    \begin{align}
        \rho \equiv \frac{1}{\kappa}
    \end{align}
\end{defi}
\begin{defi}[Vector binormal unitari]
    És un vector perpendicular a $\hat{T}$ i $\hat{N}$ alhora, que defineix el pla a on es mou la corba (pla osculador).
    \begin{align}
        \frac{\dif \hat{N}}{\dif s} = - \kappa \hat{T} + \tau \hat{B}
    \end{align}
    \begin{align}
        \hat{T} \times \hat{N} = \hat{B} \quad \text{i} \quad \hat{B} = \frac{\vec{f}'(u) \times \vec{f}''(u)}{\| \vec{f}'(u) \times \vec{f}''(u) \|}
    \end{align}
\end{defi}
\begin{defi}[Torsió]
    $\tau$ indica la variació de $\hat{B}$, és a dir, indica la variació d'orientació del pla osculador. La torsió pot ser positiva, negativa o zero.
    \begin{align}
        \frac{\dif \hat{B}}{\dif s} = - \tau \hat{N}
    \end{align}
\end{defi}
Utilitzant aquests vectors unitaris, podem parametritzar la «velocitat» i l'«acceleració»:
\begin{align*}
    \vec{f}'(u) = \| \vec{f}'(u) \| \hat{T}
\end{align*}
\begin{align*}
    \vec{f}''(u) = \| \vec{f}'(u) \|' \hat{T} + \kappa \| \vec{f}'(u) \|^{2} \hat{N}
\end{align*}

\subsubsection*{Geometria 2D}
Com que el pla osculador no canvia en funció del temps $u$, $\hat{B}$ és constant i, en particular $\tau \equiv 0$.

En una corba a $\mathbb{R}^{2}$, el radi de curvatura $\rho$ coincideix amb el radi del cercle osculador; el cercle que millor s'ajusta a la corba en el punt considerat.

\subsubsection*{Fórmules de Frénet}
\begin{defi}[Tríedre de Frénet]
    Geomètricament es compleix que $\hat{T} \times \hat{N} = \hat{B}$, $\hat{N} \times \hat{B} = \hat{T}$ i $\hat{B} \times \hat{T} = \hat{N}$. Aquestes relacions són el que anomenem tríedre de Frénet.
\end{defi}
Les derivades respecte el paràmetre arc de $\hat{T}$, $\hat{N}$ i $\hat{B}$ es poden reescriure de forma matricial:
\begin{align}
    \frac{\dif}{\dif s} \begin{pmatrix} \hat{T} \\ \hat{N} \\ \hat{B} \end{pmatrix} = \begin{pmatrix} 0 & \kappa & 0 \\ - \kappa & 0 & \tau \\ 0 & - \tau & 0 \end{pmatrix} \begin{pmatrix} \hat{T} \\ \hat{N} \\ \hat{B} \end{pmatrix}
\end{align}

\begin{example}
    Considerem la corba $\vec{f}(t) = (\cos t, \sin t, 1)$.
    \\
    $\Rightarrow \vec{f}'(t) = (-\sin t, \cos t, 1) \Rightarrow \vec{f}''(t) = (-\cos t, -\sin t, 0)$, 
    
    $\displaystyle s'(t) = \| \vec{f}'(t) \| = \sqrt{2} \Rightarrow t'(s) = \frac{1}{\sqrt{2}}$, 
    
    $\vec{f}'(t) \times \vec{f}''(t) = (\sin t, -\cos t, 1) \Rightarrow \| \vec{f}'(t) \times \vec{f}''(t) \| = \sqrt{2}$
    \begin{itemize}
        \item Vectors $\hat{T}$ i $\hat{B}$:
            \subitem $\displaystyle\hat{T} = \left( \frac{-\sin t}{\sqrt{2}}, \frac{\cos t}{\sqrt{2}}, \frac{1}{\sqrt{2}} \right)$ i $\displaystyle \hat{B} = \left( \frac{\sin t}{\sqrt{2}}, \frac{-\cos t}{\sqrt{2}}, \frac{1}{\sqrt{2}} \right)$.
        \item Centre de curvatura: 
            \subitem $\displaystyle \kappa = \frac{\| \vec{f}'(t) \times \vec{f}''(t) \|}{\| \vec{f}'(t) \|} = \frac{1}{2} \Rightarrow \rho = 2$.
        \item Torsió: 
            \subitem $\displaystyle \frac{\dif \hat{B}}{\dif s} = - \tau \hat{N} = (\frac{\cos t}{2}, \frac{\sin t}{2}, 0)$
            \subitem $\displaystyle \hat{N} = \hat{B} \times \hat{T} = (-\cos t, -\sin t, 0) \Rightarrow \tau = \frac{1}{2}$.
        \item Centre de curvatura: 
            \subitem $\vec{c} = \vec{f} + \rho \hat{N} \Rightarrow \vec{c} = (-\cos t, -\sin t, t)$.
        \end{itemize}
\end{example}

	%----------------------------------------------------------------------------------------
%    DERIVACIÓ DE CAMPS ESCALARS
%----------------------------------------------------------------------------------------
\section{Derivació de camps escalars}
L'extensió del concepte de derivada a funcions de més d'una variable no és automàtica i requereix algunes modificacions. Ho farem des de dues perspectives diferents: la derivada direccional i la diferencial.

\subsection{Derivades direccionals i derivades parcials}
\begin{defi}[Derivada direccional]
    Si $f(\vec{x})$ és un camp escalar i $\hat{u}$ un vector unitari de $\mathbb{R}^{n}$, definim la derivada en la direcció $\hat{u}$ de $f(\vec{x})$ en el punt $\vec{a}$:
    \begin{align}
        (D_{\hat{u}} f) (\vec{a}) \equiv (D_{\hat{u}} f)_{\vec{a}} \equiv \lim_{\lambda \to 0} \frac{f(\vec{a} + \lambda \hat{u}) - f(\vec{a})}{\lambda}
    \end{align}
\end{defi}
Es compleixen, per tant, els teoremes de les funcions derivables d'una variable:
\begin{thm}[del valor mitjà]\label{thm-tmv}
    Sigui $\hat{u} = (\vec{b} - \vec{a})/\| \vec{b} - \vec{a} \|$. Si $D_{\hat{u}} f \exists$ en tots els punts del segment rectilini que uneix $\vec{a}$ i $\vec{b} \Rightarrow \exists \vec{c}$ dins aquest segment que compleix
    \begin{align*}
        f(\vec{b}) - f(\vec{a}) = (D_{\hat{u}} f)_{\vec{c}} \| \vec{b} - \vec{a} \|
    \end{align*}
\end{thm}
\begin{thm}[Continuïtat de $f(\vec{a} + \lambda \hat{u})$]
    Si $\exists (D_{\hat{u}} f)_{\vec{a}} \Rightarrow f(\vec{a} + \lambda \hat{u})$ és contínua en el punt $\lambda = 0$, és a dir, $\lim\limits_{\lambda \to 0} f(\vec{a} + \lambda \hat{u}) = f(\vec{a})$.
\end{thm}

\begin{defi}[Derivades parcials]
    Les derivades parcials són les derivades direccionals en les direccions $\hat{e}_{i}$.
    \begin{align}
        \left( \frac{\partial f}{\partial x_{i}} \right)_{\vec{a}} \equiv (D_{\hat{e}_{i}} f)_{\vec{a}} \equiv (\partial_{i} f)_{\vec{a}}
    \end{align}
    La derivada parcial respecte $x_{i}$ en el punt $\vec{a}$ és la derivada de la funció $f(\vec{x})$ respecte la variable $x_{i}$ fixant les altres variables al punt $\vec{a}$:
    \begin{align}
        \left( \frac{\partial f}{\partial x_{i}} \right)_{\vec{a}} = \left[ \frac{\dif}{\dif x_{i}} f(a_{1}, \dots , x_{i}, \dots , a_{n}) \right]_{x_{i}=a_{i}}
    \end{align}
\end{defi}

\subsubsection*{Derivades direccionals i continuïtat}
El fet que de l'existència de $D_{\hat{u}}f$ (i, menys encara, la simple existència de les derivades parcials) no és suficient per garantir la continuïtat. Aquest fet evidencia que les derivades direccionals no són una extensió una extensió satisfactòria del concepte de derivada per a les funcions de $n$ variables. Un concepte més adient és el de diferenciabilitat.

%----------------------------------------------------------------------------------------
\subsection{Camps escalars diferenciables}
\subsubsection*{Infinitèsims}
\begin{defi}
    $f(\vec{x})$ és un «infinitèsim quan $\vec{x} \to \vec{a}$» si $\lim\limits_{\vec{x} \to \vec{a}} f(\vec{x}) = 0$.
\end{defi}
\begin{defi}
    $f(\vec{x})$ és un «infinitèsim d'ordre superior a $n$ quan $\vec{x} \to \vec{a}$» si $\displaystyle \lim_{\vec{x} \to \vec{a}} \frac{f(\vec{x})}{\| \vec{x} - \vec{a} \|^{n}} = 0$. Diem, llavors, que $f(\vec{x}) \to 0$ «més ràpidament» que $\| \vec{x} - \vec{a} \|^{n}$ quan $\vec{x} \to \vec{a}$, i ho expressem: 
    \begin{align*}
        f(\vec{x}) = o[\| \vec{x} - \vec{a} \|^{n}]
    \end{align*}
\end{defi}
\begin{defi}
    Quan només es pot afirmar que $\displaystyle \lim_{\vec{x} \to \vec{a}} \frac{f(\vec{x})}{\| \vec{x} - \vec{a} \|^{n}} \neq \infty$, diem que $f(\vec{x})$ és un «infinitèsim d'ordre igual o superior a $n$ (i que $f(\vec{x}) \to 0$ «tan o més ràpidament» que $\| \vec{x} - \vec{a} \|^{n}$) quan $\vec{x} \to \vec{a}$», i ho expressem:
    \begin{align*}
        f(\vec{x}) = O[\| \vec{x} - \vec{a} \|^{n}]
    \end{align*}
\end{defi}

\subsubsection*{Camps escalars diferenciables}
\begin{defi}
    Sigui $f(\vec{x})$ un camp escalar definit a $D \subseteq \mathbb{R}^{n}$ i sigui $\vec{a} \in D$. Direm que $f(\vec{x})$ és diferenciable en el punt $\vec{a}$ si $\exists \vec{K} \in \mathbb{R}^{n}$ tal que
    \begin{align}
        f(\vec{x}) = f(\vec{a}) + \vec{K} (\vec{x} - \vec{a}) + o[\| \vec{x} - \vec{a} \|]
    \end{align}
\end{defi}

\begin{thm}[Diferenciabilitat i continuïtat]
    Si $f(\vec{x})$ és diferenciable en el punt $\vec{a} \Rightarrow f(\vec{x})$ és contínua en el punt $\vec{a}$.
\end{thm}
\begin{thm}[Diferenciabilitat i l'existència de les derivades parcials]
    Si $f(\vec{x})$ és diferenciable en el punt $\vec{a} \Rightarrow \exists (D_{\hat{u}} f)_{\vec{a}}, \forall \hat{u}$ i es compleix $(D_{\hat{u}} f)_{\vec{a}} = \vec{K} \cdot \hat{u}$.
\end{thm}
\begin{cor}
    Si $f$ és diferenciable en el punt $\vec{a}$, es compleix
        \begin{align*}
            \left( \frac{\partial f}{\partial x_{i}} \right)_{\vec{a}} = (D_{\hat{e}_{i}} f)_{\vec{a}} = \vec{K}\cdot \hat{e}_{i} \Rightarrow \vec{K} \text{ és únic.}
    \end{align*}
\end{cor}

\subsubsection*{Gradient d'un camp escalar diferenciable}
Introduïm l'operador nabla: $\displaystyle \vnabla = \left( \frac{\partial}{\partial x_{1}}, \frac{\partial}{\partial x_{2}}, \dots, \frac{\partial}{\partial x_{n}} \right)$.
\begin{defi}[Gradient]
    Si $f$ és diferenciable en el punt $\vec{a}$, 
    \begin{align}
        (\operatorname{grad} f)_{\vec{a}} \equiv (\vnabla f)_{\vec{a}} = \left( \left( \frac{\partial}{\partial x_{1}} \right)_{\vec{a}}, \left( \frac{\partial}{\partial x_{2}} \right)_{\vec{a}}, \dots, \left( \frac{\partial}{\partial x_{n}} \right)_{\vec{a}} \right) 
    \end{align}
\end{defi}
El vector gradient, és doncs, en cada punt, el vector $\vec{K}$ del camp escalar diferenciable $f(\vec{x})$:
\begin{align}\label{eq:taylor1}
    f(\vec{x}) = f(\vec{a}) + (\vnabla f)_{\vec{a}} \cdot (\vec{x} - \vec{a}) + o[\| \vec{x} - \vec{a} \|]
\end{align}
O dit d'una altra manera:
\begin{align}
    \lim_{\vec{x} - \vec{a} \to \vec{0}} \frac{f(\vec{x}) - f(\vec{a}) - (\vnabla f)_{\vec{a}} \cdot (\vec{x} - \vec{a})}{\| \vec{x} - \vec{a} \|} = 0
\end{align}
La fórmula~\eqref{eq:taylor1} s'anomena també fórmula de Taylor de primer ordre del camp diferenciable $f(\vec{x})$ en el punt $\vec{a}$.

Geomètricament, el gradient es pot interpretar com el vector que té mòdul de la derivada direccional màxima i que té la direcció en què aquesta derivada direccional és màxima.

\subsubsection*{Diferencial d'un camp escalar diferenciable}
\begin{defi}[Diferencial total]
    Reexpressant el «pla» tangent a $y = f(\vec{a}) + (\vnabla f)_{\vec{a}} \cdot (\vec{x} - \vec{a})$ en un sistema de coordenades $\dif x_{1}, \dif x_{2}, \dots , \dif x_{n}$ local, tenim la diferencial o diferencial total de $f(\vec{x})$ en el punt $\vec{a}$:
    \begin{align}
        \dif y = (\vnabla f)_{\vec{a}} = \left( \frac{\partial}{\partial x_{1}} \right)_{\vec{a}} \diff x_{1} + \dots + \left( \frac{\partial}{\partial x_{n}} \right)_{\vec{a}} \diff x_{n}
    \end{align}
\end{defi}

\begin{thm}[Condició suficient per a la diferenciabilitat]
    Si $f(\vec{x})$ té derivades parcials en algun entorn del punt $\vec{a}$ i són contínues en el punt $\vec{a} \Rightarrow f(\vec{x})$ és diferenciable en el punt $\vec{a}$. 
\end{thm}
\begin{cor}
    Si $f(\vec{x})$ és de classe $C_{D}^{1} \Rightarrow f(\vec{x})$ és diferenciable a $D$.
\end{cor}

%----------------------------------------------------------------------------------------
\subsection{Regla de la cadena $1-n-1$}
Sigui $\vec{x} = \vec{g}(u)$ una funció $\mathbb{R} \to \mathbb{R}^{n}$ diferenciable en el punt $u = a$; sigui $y=f(\vec{x})$ una funció $\mathbb{R}^{n} \to \mathbb{R}$ diferenciable en el punt $\vec{x} = \vec{g}(u)$; sigui $F(u) \equiv (f \circ \vec{g})(u)$. Llavors $F(u)$ és diferenciable en el punt $u = a$ i es compleix
\begin{align}
    F'(a) = (\vnabla f)_{\vec{g}(a)} \cdot \vec{g}'(a)
\end{align}
o, també
\begin{align}
    \left( \frac{\partial F}{\partial u} \right)_{a} = \sum_{i=1}^{n}  \left( \frac{\partial f}{\partial x_{i}} \right)_{\vec{g}(a)}  \left( \frac{\partial x_{i}}{\partial u} \right)_{a}
\end{align}

\subsubsection*{Corbes de nivell}
Sigui un $y = f(x_{1},x_{2})$ un camp escalar diferenciable definit a $\mathbb{R}^{2}$. El seu gràfic és una superfície de $\mathbb{R}^{3}$. La intersecció d'aquesta superfície amb el pla $y \equiv c$ determina una corba sobre el gràfic que, projectada sobre el pla de les variables $x_{1}, x_{2}$, és la corba de nivell $f(x_{1}, x_{2}) \equiv c$. El vector gradient és, en cada punt de $\mathbb{R}^{2}$, ortogonal a les corbes de nivell.

Generalitzant aquesta propietat a un camp escalar definit a $\mathbb{R}^{n}$, $\vnabla f$ és, en cada punt, ortogonal a l'hiper-pla tangent a la hiper-superfície de nivell.
%----------------------------------------------------------------------------------------
\subsection{Derivades parcials d'ordre superior}
Si el camp escalar $f(\vec{x})$ té derivades parcials en el domini $D \subseteq \mathbb{R}^{n}$, considerem els camps escalars $\displaystyle \left( \frac{\partial f}{\partial x_{i}} \right) (\vec{x})$ que poden tenir o no derivades parcials les quals, en cas d'existir, anomenem derivades parcials de segon ordre
\begin{align}
    \left ( \frac{\partial}{\partial x_{j}} \left[ \left( \frac{\partial f}{\partial x_{i}} (\vec{x}) \right) \right] \right)_{\vec{a}} \equiv \left( \frac{\partial^{2} f}{\partial x_{j} \partial x_{i}} \right)_{\vec{a}}
\end{align}
Reiterant el procés podem parlar de derivades de tercer, quart, cinquè, etc. ordre. Cal remarcar que l'ordre en què es fan les successives derivades és, en principi, no permutable.
\begin{thm}[de Schwarz]
Si $f(\vec{x}) = f(x_{1}, \dots, x_{m})$ té derivades parcials d'ordre $m$ en algun entorn de $\vec{x}_{0}$ i són contínues en aquest punt, es compleix
\begin{align}
    \left( \frac{\partial^{m} f}{\partial x_{i_{1}} \partial x_{i_{2}} \dots \partial x_{i_{m}}} \right)_{\vec{x}_{0}} = \left( \frac{\partial^{m} f}{\partial x_{j_{1}} \partial x_{j_{2}} \dots \partial x_{j_{m}}} \right)_{\vec{x}_{0}}
\end{align}
On $x_{i_{1}}, x_{i_{2}}, \dots, x_{i_{m}}$ i $x_{j_{1}}, x_{j_{2}}, \dots, x_{j_{m}}$ són dues permutacions d'un conjunt de $m$ variables de les $n$ variables $x_{1}, x_{2}, \dots, x_{n}$.
\end{thm}
\begin{cor}
    Si $f(\vec{x})$ és de classe $C_{D}^{m}$, llavors l'ordre derivació de les derivades parcials mixtes d'ordre $m$ és indiferent.
\end{cor}
%----------------------------------------------------------------------------------------
\subsection{Fórmula de Taylor per a un camp escalar}
Sigui $f(\vec{x}$ un camp escalar de $n$ variables de classe $C^{m}$ en algun entorn del punt $\vec{a}$. Com en el cas de les funcions d'una variable, volem trobar la millor aproximació polinòmica de $f$ en aquest punt. Es tracta, doncs, de trobar un polinomi $P_{m}^{(\vec{a})} (\vec{x})$ de $n$ variables, de gray $m$, que tingui contacte d'ordre superior a $m$ amb $f(\vec{x})$ en el punt $\vec{a}$, és a dir, que compleixi
\begin{align}
    f(\vec{x}) - P_{m}^{(\vec{a})}(\vec{x}) = o [\| \vec{x} - \vec{a} \|^{m}]
\end{align}

\subsubsection*{Polinomi de Taylor de grau $m$ de $f(\vec{x})$ en el punt $\vec{a}$}
L'expressió del polinomi amb millor ajust polinòmic (de grau $m$) de la funció $f(\vec{x})$ en el punt $\vec{a}$ és
\begin{align}
    \begin{aligned}
        P_{m}^{(\vec{a})}(\vec{x}) = & f(\vec{a}) + \sum_{i_{1}} \left( \frac{\partial f}{\partial x_{i_{1}}} \right)_{\vec{a}} (x_{i_{1}} - a_{i_{1}}) \\
        & + \frac{1}{2!} \sum_{i_{1}, i_{2}} \left( \frac{\partial^{2} f}{\partial x_{i_{1}} \partial x_{i_{2}}} \right)_{\vec{a}} (x_{i_{1}} - a_{i_{1}}) (x_{i_{2}} - a_{i_{2}}) + \dots \\
        & + \frac{1}{m!} \sum_{i_{1}, \dots, i_{m}} \left( \frac{\partial^{m} f}{\partial x_{i_{1}} \dots \partial x_{i_{m}}} \right)_{\vec{a}} (x_{i_{1}} - a_{i_{1}}) \dots (x_{i_{m}} - a_{i_{m}})
    \end{aligned}
\end{align}

\subsubsection*{Fórmula de Taylor}
Si $f(\vec{x})$ és de classe $C^{m+1}$ en algun entorn del punt $\vec{a}$, el terme $o[\| \vec{x} - \vec{a} \|^{m}]$ es pot precisar més. 
\begin{thm}[de la Fórmula de Taylor]
    Si $f(\vec{x})$ és de classe $C^{m+1}$ en algun entorn del punt $\vec{a} \Rightarrow f(\vec{x}) = P_{m}^{(\vec{a})}(\vec{x}) + R_{m}^{(\vec{a})}(\vec{x})$. On $R_{m}^{(\vec{a})}(\vec{x})$ és la resta de Lagrange, que s'expressa: $\exists \tilde{t} \in (0,1)$ tal que 
    \begin{align}
        \begin{aligned}
            R_{m}^{(\vec{a})}(\vec{x}) = & \frac{1}{(m+1)!} \sum_{i_{1}, \dots, i_{m+1}} \left( \frac{\partial^{m+1} f}{\partial x_{i_{1}} \dots \partial x_{i_{m+1}}} \right)_{\vec{a} + \tilde{t} (\vec{x} - \vec{a}) } \\
            & \cdot (x_{i_{1}} - a_{i_{1}}) \dots (x_{i_{m+1}} - a_{i_{m+1}})
        \end{aligned}
    \end{align}
\end{thm}

%----------------------------------------------------------------------------------------
\subsection{Hessià}
Sigui $f(\vec{x})$ un camp escalar de, al menys, classe $C^{2}$ en algun entorn del punt $\vec{a}$. La fórmula de Taylor de segon ordre s'escriu
\begin{align}
    \begin{aligned}
        f(\vec{x}) = & f(\vec{a}) + \sum_{i} \left( \frac{\partial f}{\partial x_{i}} \right)_{\vec{a}} (x_{i} - a_{i}) \\
        & + \frac{1}{2!} \sum_{i, j} \left( \frac{\partial^{2} f}{\partial x_{i} \partial x_{j}} \right)_{\vec{a}} (x_{i} - a_{i}) (x_{j} - a_{j}) \\
        & + o[\| \vec{x} - \vec{a} \|^{2}]
    \end{aligned}
\end{align}
que podem escriure de forma matricial
\begin{align}
    f(\vec{x}) = f(\vec{a}) + (\vnabla f)_{\vec{a}} \cdot (\vec{x} - \vec{a}) + \frac{1}{2!} (\vec{x} - \vec{a}) \cdot ((H^{f})_{\vec{a}}(\vec{x} - \vec{a})) + \dots
\end{align}
On $(H^{f})_{\vec{a}}(\vec{x} - \vec{a})$ és l'hessià, o matriu hessiana, multiplicat per la matriu del vector $\vec{x} - \vec{a}$
\begin{align}
    (H^{f})_{\vec{a}}(\vec{x} - \vec{a}) = \begin{pmatrix} \displaystyle \frac{\partial^{2} f}{\partial x_{1}^{2}} & \displaystyle \frac{\partial^{2} f}{\partial x_{1} \partial x_{2}} & \dots & \displaystyle \frac{\partial^{2} f}{\partial x_{1} \partial x_{n}} \\ \displaystyle \frac{\partial^{2} f}{\partial x_{2} \partial x_{1}} & \displaystyle \frac{\partial^{2} f}{\partial x_{2}^{2}} & \dots & \displaystyle \frac{\partial^{2} f}{\partial x_{2} \partial x_{n}} \\ \vdots & \vdots & \ddots & \vdots \\ \displaystyle \frac{\partial^{2} f}{\partial x_{n} \partial x_{1}} & \displaystyle \frac{\partial^{2} f}{\partial x_{n} \partial x_{2}} & \dots & \displaystyle \frac{\partial^{2} f}{\partial x_{n}^{2}} \end{pmatrix}_{\vec{a}} \begin{pmatrix} x_{1} - a_{1} \\ \vdots \\ x_{n} - a_{n} \end{pmatrix}
\end{align}
L'hessià té un paper important en la determinació dels màxims i mínims relatius dels camps escalars.

%----------------------------------------------------------------------------------------
\subsection{Punts estacionaris}
\begin{thm}
    Si $f(\vec{x})$ té un extrem relatiu en el punt $\vec{a}$ i és diferenciable en aquest punt, llavors $(\vnabla f)_{\vec{a}} = \vec{0}$.
\end{thm}
El recíproc no és cert. Direm, però, que $\vec{a}$ és un punt estacionari o crític.

Si $f(\vec{x})$ és diferenciable en el punt $\vec{a}$ i aquest punt és estacionari hi ha dues possibilitats:
\begin{itemize}
    \item Extrem (màxim o mínim) relatiu: quan en algun entorn de $\vec{a}$ hi ha punts $\vec{x}$ on $f(\vec{x}) \leq f(\vec{a})$ o $f(\vec{x}) \geq f(\vec{a})$.
    \item Punt de sella: quan en tot entorn de $\vec{a}$ hi ha punts $\vec{x}$ on $f(\vec{x}) < f(\vec{a})$ i punts on $f(\vec{x}) > f(\vec{a})$.
\end{itemize}

\subsubsection*{Caracterització dels punts estacionaris a partir de l'hessià}
Si $f(\vec{x})$ és un camp escalar de classe $C^{2}$ en algun entorn d'un punt estacionari $\vec{a}$, $(\vnabla f)_{\vec{a}} \cdot (\vec{x} - \vec{a}) = 0$). Llavors la fórmula de Taylor de segon ordre al voltant de $\vec{a}$ la podem expressar com
\begin{align}
    f(\vec{x}) - f(\vec{a}) = \frac{1}{2!} (\vec{x} - \vec{a}) \cdot ((H^{f})_{\vec{a}}(\vec{x} - \vec{a})) + \dots
\end{align}
O dit d'una altra manera (de forma matricial)
\begin{align}
    f(\vec{x}) - f(\vec{a}) = \frac{1}{2!} h^{t} H h + \dots
\end{align}
Per tant,
\begin{itemize}
    \item Si $h^{t} H h > 0, \, \forall h$, es tracta d'un mínim relatiu.
    \item Si $h^{t} H h < 0, \, \forall h$, es tracta d'un màxim relatiu.
    \item Si per alguns $h$ tenim $h^{t} H h > 0$ i per altres $h^{t} H h < 0$, es tracta d'un punt de sella.
    \item Si per algun $h$ tenim $h^{t} H h = 0$, no és conclusiu.
\end{itemize}

\subsubsection*{Caracterització dels punts estacionaris a partir dels valors propis de l'hessià}
Un punt estacionari $\vec{a}$ de $f(\vec{x})$ (de classe $C^{2}$) és:
\begin{itemize}
    \item Mínim relatiu: si tots els valors propis de $(H^{f})_{\vec{a}}$ són $>0$.
    \item Màxim relatiu: si tots els valors propis de $(H^{f})_{\vec{a}}$ són $<0$.
    \item Punt de sella: si hi ha valors propis $>0$ i també valors propis $<0$.
    \item Si hi ha valors propis $=0$ i $>0$ (o $=0$ i $<0$) no es pot concloure res.
\end{itemize}

\subsubsection*{Criteri de Sylvester}
És un criteri pràctic que dóna una condició necessària i suficient per què els valors propis siguin estrictament positius o negatius (és fàcil de demostrar a partir del teorema de Sylvester).

Si $H^{(1)}, H^{(2)}, H^{(3)}, \dots, H^{(n)}$ són els menors principals del vèrtex superior esquerre de $(H^{f})_{\vec{a}}$.
\begin{itemize}
    \item Els valors propis són $>0 \Leftrightarrow$
        \subitem $\det H^{(1)} > 0, \det H^{(2)} > 0,  \det H^{(3)} > 0, \dots, \det H^{(n)} > 0$ (mínim relatiu).
    \item Els valors propis són $<0 \Leftrightarrow$
        \subitem $\det H^{(1)} < 0, \det H^{(2)} > 0,  \det H^{(3)} < 0, \dots,$
        \subitem $\det H^{(n)}\begin{cases} >0 & (n \text{ parell}) \\ <0 & (n \text{ senar}) \end{cases}$ (màxim relatiu).
    \item En altres situacions:
        \subitem Si $\det (H^{f})_{\vec{a}} \neq 0$, els valors propis són tots $\neq 0$, però tenen signes diferents (punt de sella).
        \subitem Si $\det (H^{f})_{\vec{a}} = 0$ algun valor propi és $0$ (no és conclusiu).
\end{itemize}

	%----------------------------------------------------------------------------------------
%    DERIVACIÓ DE CAMPS VECTORIALS
%----------------------------------------------------------------------------------------
\section{Derivació de camps vectorials}
\subsection{Matriu jacobiana}
Sigui $\vec{f}(\vec{x}) = (f_{1}(\vec{x}), f_{2}(\vec{x}), \dots, f_{m}(\vec{x}))$ un camp vectorial definit a $D \subseteq \mathbb{R}^{n}$, amb valors a $\mathbb{R}^{m}$.
\begin{defi}[Derivada direccional i derivades parcials]
    Si $\hat{u}$ és un vector unitari de $\mathbb{R}^{n}$,
    \begin{align}
        (D_{\hat{u}} \vec{f})_{\vec{a}} = \lim_{\lambda \to 0} \frac{\vec{f}(\vec{a} - \lambda \hat{u})}{\lambda} = \begin{pmatrix} D_{\hat{u}} f_{1} \\ D_{\hat{u}} f_{2} \\ \vdots \\ D_{\hat{u}} f_{m} \end{pmatrix}_{\vec{a}}; \quad \left( \frac{\partial \vec{f}}{\partial x_{i}} \right)_{\vec{a}} = \begin{pmatrix} \displaystyle \frac{\partial f_{1}}{\partial x_{i}} \\ \displaystyle \frac{\partial f_{2}}{\partial x_{i}} \\ \vdots \\ \displaystyle \frac{\partial f_{m}}{\partial x_{i}} \end{pmatrix}_{\vec{a}}
    \end{align}
\end{defi}
\begin{defi}[Matriu jacobiana]
    \begin{align}
        (\vec{J}^{\vec{f}})_{\vec{a}} \equiv \begin{pmatrix} \vnabla f_{1} \\ \vdots \\ \vnabla f_{m} \end{pmatrix}_{\vec{a}} = \begin{pmatrix} \displaystyle \frac{\partial f_{1}}{\partial x_{1}} & \dots & \displaystyle \frac{\partial f_{m}}{\partial x_{n}} \\ \vdots & \ddots & \vdots \\ \displaystyle \frac{\partial f_{1}}{\partial x_{1}} & \dots & \displaystyle \frac{\partial f_{m}}{\partial x_{n}} \end{pmatrix}_{\vec{a}}
    \end{align}
\end{defi}

%----------------------------------------------------------------------------------------
\subsection{Camps vectorials diferenciables}
Un camp vectorial $\vec{f}(\vec{x}) = (f_{1}(\vec{x}), f_{2}(\vec{x}), \dots, f_{m}(\vec{x}))$ és diferenciable al punt $\vec{x} = \vec{a}$ si les seves components ho són, és a dir, si 
\begin{align}
    \vec{f}(\vec{x}) = \vec{f}(\vec{a}) + (\vec{J}^{\vec{f}})_{\vec{a}} (\vec{x} - \vec{a}) + \vec{o}[\| \vec{x} - \vec{a} \|]
\end{align}

\begin{thm}[Condició suficient per a la diferenciabilitat]
    Si les components de la matriu jacobiana ($\partial f_{i} / \partial x_{j})$ existeixen en algun entorn del punt $\vec{a}$ i són contínues en aquest punt $\Rightarrow \vec{f}(\vec{x})$ és diferenciable en el punt $\vec{a}$.
\end{thm}

\subsubsection*{Diferencial d'un camp vectorial diferenciable}
\begin{defi}[Diferencial]
    És el «pla tangent» $\vec{y} = \vec{f}(\vec{a}) + (\vec{J}^{\vec{f}})_{\vec{a}} (\vec{x} - \vec{a})$ reexpressat en coordenades locals de $\mathbb{R}^{n+m}$ $(\dif \vec{x}, \dif \vec{y})$ que tenen l'origen en el punt de contacte $(\vec{a}, \vec{f}(\vec{a}))$ d'aquest pla amb el gràfic de la funció:
    \begin{align}
        (\dif \vec{y}) = (\vec{J}^{\vec{f}})_{\vec{a}} (\dif \vec{x})
    \end{align}
    o en forma matricial
    \begin{align}
        \begin{pmatrix} \dif y_{1} \\ \vdots \\ \dif y_{m} \end{pmatrix} = \begin{pmatrix} \displaystyle \frac{\partial f_{1}}{\partial x_{1}} & \dots & \displaystyle \frac{\partial f_{m}}{\partial x_{n}} \\ \vdots & \ddots & \vdots \\ \displaystyle \frac{\partial f_{1}}{\partial x_{1}} & \dots & \displaystyle \frac{\partial f_{m}}{\partial x_{n}} \end{pmatrix}_{\vec{a}} \begin{pmatrix} \dif x_{1} \\ \vdots \\ \dif x_{n} \end{pmatrix}
    \end{align}
\end{defi}
\begin{defi}[Jacobià]
    Quan $m = n$, la matriu jacobiana de $\vec{f}(\vec{x})$ és quadrada. El determinant s'anomena jacobià de $\vec{f}$ en el punt $\vec{a}$, i s'expressa
    \begin{align}
        \left( \frac{\partial(f_{1}, \dots, f_{n})}{\partial (x_{1}, \dots, x_{n})} \right)_{\vec{a}} = \det (\vec{J}^{\vec{f}})_{\vec{a}}
    \end{align}
    Utilitzarem el terme jacobià també en el cas $m \neq n$ per referir-nos a determinants de menors de matrius jacobianes. Per exemple, si $\vec{f}(\vec{x}) = (f_{1}(x,y,z),f_{2}(x,y,z))$, podem parlar del jacobià ($\partial (f_{1}, f_{2}) / \partial (x,y))$.
\end{defi}

%----------------------------------------------------------------------------------------
\subsection{Regla de la cadena $l-n-m$}
Sigui $\vec{x} = \vec{g}(\vec{u})$ una funció $\mathbb{R}^{l} \to \mathbb{R}^{n}$ diferenciable en el punt $\vec{u} = \vec{a}$; sigui $\vec{y}=\vec{f}(\vec{x})$ una funció $\mathbb{R}^{n} \to \mathbb{R}^{m}$ diferenciable en el punt $\vec{x} = \vec{g}(\vec{u})$; sigui $\vec{F}(\vec{u}) \equiv (\vec{f} \circ \vec{g})(\vec{u})$. Llavors $\vec{F}(\vec{u})$ és diferenciable en el punt $\vec{u} = \vec{a}$ i es compleix
\begin{align}
    (\vec{J}^{\vec{F}})_{\vec{a}} = (\vec{J}^{\vec{f}})_{\vec{g}(\vec{a})} (\vec{J}^{\vec{g}})_{\vec{a}}
\end{align}
o, també
\begin{align}
    \left( \frac{\partial F_{i}}{\partial u_{j}} \right)_{\vec{a}} = \sum_{k=1}^{n}  \left( \frac{\partial f_{i}}{\partial x_{k}} \right)_{\vec{g}(\vec{a})}  \left( \frac{\partial x_{k}}{\partial u_{j}} \right)_{\vec{a}}
\end{align}

%----------------------------------------------------------------------------------------
\subsection{Funció inversa}
\begin{defi}
    Si $\vec{f}(\vec{x})$ és un camp vectorial, anomenem funció inversa de $\vec{f}$ a la funció $\vec{f}^{-1}$ que compleix $\vec{f} \circ \vec{f}^{-1} = \vec{f}^{-1} \circ \vec{f} = \vec{I}$ (si existeix), on $\vec{I}$ és la funció identitat. Per tal que una funció tingui inversa és necessari i suficient que sigui injectiva.
\end{defi}

\subsubsection*{Matriu jacobiana de la funció inversa}
\begin{thm}
    Si $\vec{f}$ i $\vec{f}^{-1}$ són diferenciables, la matriu jacobiana de $\vec{f}^{-1}$ és la inversa de la matriu jacobiana de $\vec{f}$.
\end{thm}
\begin{cor}
    Si $\vec{f}$ és diferenciable i té inversa diferenciable, el jacobià de $\vec{f}$ ha de ser $\neq 0$.
\end{cor}
\begin{thm}[de la funció inversa]
    Sigui $\vec{f}(\vec{x})$ un camp vectorial definit a $D \subseteq \mathbb{R}^{n}$ amb valors a $\mathbb{R}^{n}$. Si es compleix
    \begin{enumerate}[i)]
        \item $\vec{f}(\vec{x})$ és de classe $C^{1}_{D}$.
        \item $\displaystyle \frac{\partial (f_{1}, \dots, f_{n})}{\partial (x_{1}, \dots, x_{n})} \neq 0$.
    \end{enumerate}
    Llavors, 
    \begin{enumerate}[i)]
        \item Hi ha un entorn $\varepsilon (\vec{x}_{0})$ de $\vec{x}_{0}$ en què $\vec{f}$ té inversa $\vec{f}^{-1}$.
        \item $\vec{f}^{-1}$ és de classe $C^{1}$ en la imatge de $\varepsilon (\vec{x}_{0})$.
    \end{enumerate}
\end{thm}

%----------------------------------------------------------------------------------------
\subsection{Funcions implícites}
\begin{thm}[de la funció implícita]
    Sigui $\vec{F}(\vec{x}) = (F_{1}(\vec{x}), \dots, F_{n}(\vec{x}))$ un camp vectorial definit a $D \subseteq \mathbb{R}^{n}$ amb valors a $\mathbb{R}^{m}$, amb $m<n$. Si es compleix
    \begin{enumerate}[i)]
        \item $\vec{F}(\vec{x})$ és de classe $C^{1}_{D}$ (les derivades parcials són contínues).
        \item $\vec{F}(\vec{x}_{0}) = \vec{0}$. 
        \item $\displaystyle \left( \frac{\partial (f_{1}, \dots, f_{m})}{\partial (x_{1}, \dots, x_{m})} \right)_{\vec{x_{0}}} \neq 0$.
    \end{enumerate}
    Llavors, hi ha un entorn de $(x_{0_{m+1}}, \dots, x_{0_{n}}) \in \mathbb{R}^{n-m}$ en el qual existeixen $m$ funcions de classe $C^{1}$ úniques $g_{i} (x_{m+1}, \dots, x_{n})$, $i=1,\dots, m$, tals que 
    \begin{align}
        \vec{F}(g_{1}(\dots), \dots, g_{m}(\dots), x_{m+1}, \dots, x_{n}) \equiv \vec{0}
    \end{align}
\end{thm}
\begin{cor}
    $\vec{F}(\vec{x}) = \vec{0}$ defineix implícitament $m$ funcions $x_{i} = g_{i} (x_{m+1}, \dots, x_{n})$, $i=1,\dots, m$ en algun entorn de $(x_{0_{m+1}}, \dots, x_{0_{n}})$.
\end{cor}
En general, aplicant la regla de la cadena, podem arribar a la següent expressió
\begin{align}
    \frac{\partial g_{i}}{\partial x_{k}} = - \left( \frac{\displaystyle \frac{\partial (F_{1}, \dots , F_{m})}{\partial (x_{1}, \dots , x_{k} , \dots , x_{m})}}{\displaystyle \frac{\partial (F_{1}, \dots , F_{m})}{\partial (x_{1}, \dots , x_{m})}} \right)
\end{align}
amb $i = 1, 2, \dots, m$ i $k = m+1, m+2, \dots, n$; on la columna que correspon a $x_{k}$ del jacobià és la columna $i$. 

És a dir, al jacobià del numerador se substitueix la columna de la variable que volem aïllar per la columna de la variable respecte la qual derivem parcialment.

\begin{example}
    Sigui $f(x,y) = x^{2}y^{2}+6x^{x}y + 5y^{3} + 3x^{2} - 12 = 0$. Es pot afirmar que $y = g(x)$ està definida implícitament?
    \begin{enumerate}[i)]
        \item $f(x,y)$ és de classe $C^{\infty}$.
        \item $\displaystyle \frac{\partial f}{\partial y} \neq 0 = 2x^{2}y + 6x^{2} + 15 y^{2}$. 
        \item $f(x_{0}, y_{0}) = 0$, per a algun $x_{0}, y_{0}$.
    \end{enumerate}
    Així doncs, es pot aïllar $y$ en funció de $x$.
    \begin{align*}
        \frac{\partial g}{\partial x} = - \frac{\displaystyle \frac{\partial f}{\partial x}}{\displaystyle \frac{\partial f}{\partial y}} = - \frac{2xy^{2} + 2xy + 6x}{2x^{2}y + 6x^{2} + 15 y^{2}}
    \end{align*}
\end{example}

%----------------------------------------------------------------------------------------
\subsection{Extrems condicionats: multiplicadors de Lagrange}
Sigui $f(x_{1}, \dots , x_{n})$ un camp escalar $\mathbb{R}^{n} \to \mathbb{R}$ de classe $C^{1}$. Considerem el problema de trobar els màxims i mínims de $f(\vec{x})$, on les variables $\vec{x}$ estan condicionades a satisfer $F_{i} (\vec{x}) = 0$, $i = 1, \dots, m(<n)$.
\begin{thm}[dels multiplicadors de Lagrange]
    Sigui $f(x_{1}, \dots , x_{n})$ un camp escalar de classe $C^{1}$. Si es compleix
    \begin{enumerate}[i)]
        \item $F_{i}(x_{1}, \dots, x_{n})$, $i = 1, \dots, m(<n)$ són de classe $C^{1}$.
        \item $S$ és el conjunt de punts que satisfan $F_{i}(\vec{x}) = 0$, $i = 1, \dots , m$ i $\partial (F_{1}, \dots, F_{m}) / \partial (x_{1}, \dots, x_{m}) \neq 0$.
        \item $f(\vec{x})$ té un màxim o mínim relatiu a $\vec{x}_{0} \in S$.
    \end{enumerate}
    Llavors $(\vnabla f)_{x_{0}}$ és combinació lineal dels $(\vnabla F_{i})_{\vec{x}_{0}}$. És a dir existeixen $m$ números reals $\lambda_{1}, \dots, \lambda_{m}$ (multiplicadors de Lagrange) tals que 
    \begin{align}
        (\vnabla f)_{x_{0}} = \lambda_{1} (\vnabla F_{1})_{\vec{x}_{0}} + \dots + \lambda_{m} (\vnabla F_{m})_{\vec{x}_{0}}
    \end{align}
\end{thm}

%----------------------------------------------------------------------------------------
\subsection{Divergència, rotacional i laplaciana}
\subsubsection*{Divergència d'un camp vectorial $\mathbb{R}^{n} \to \mathbb{R}^{n}$}
\begin{defi}[Divergència]
    Sigui $\vec{f}(\vec{x})$ un camp vectorial definit a $D \subseteq \mathbb{R}^{n}$, amb valors a $\mathbb{R}^{n}$. Si existeixen les derivades parcials de $\vec{f}$, definim la divergència de $\vec{f}$ com el camp escalar
    \begin{align}
        \operatorname{div} \vec{f} \equiv \vnabla \cdot \vec{f} = \begin{pmatrix} \displaystyle \frac{\partial}{\partial x_{1}} & \dots & \displaystyle \frac{\partial}{\partial x_{n}} \end{pmatrix} \begin{pmatrix} f_{1} \\ \vdots \\ f_{n} \end{pmatrix}
    \end{align}
\end{defi}
Propietats: si $\vec{f}(\vec{x})$ i $\vec{g}(\vec{x})$ són camps vectorials $\mathbb{R}^{n} \to \mathbb{R}^{n}$ i $\phi (\vec{x})$ és un camp escalar de $\mathbb{R}^{n}$,
\begin{enumerate}[i)]
    \item $\vnabla \cdot (\vec{f} + \vec{g}) = \vnabla \cdot \vec{f} + \vnabla \cdot \vec{g}$.
    \item $\vnabla \cdot (\phi \vec{f}) = (\vnabla \phi) \cdot \vec{f} + \phi (\vnabla \cdot \vec{f})$.
\end{enumerate}

\subsubsection*{Rotacional d'un camp vectorial $\mathbb{R}^{3} \to \mathbb{R}^{3}$}
\begin{defi}[Rotacional]
    Sigui $\vec{f}(\vec{x})$ un camp vectorial definit a $D \subseteq \mathbb{R}^{3}$, amb valors a $\mathbb{R}^{3}$. Si existeixen les derivades parcials de $\vec{f}$, definim la divergència de $\vec{f}$ com el camp vectorial
    \begin{align}
        \operatorname{rot} \vec{f} \equiv \vnabla \times \vec{f} = \begin{vmatrix} \hat{e}_{1} & \hat{e}_{2} & \hat{e}_{3} \\ \displaystyle \frac{\partial}{\partial x_{1}} & \displaystyle \frac{\partial}{\partial x_{2}} & \displaystyle \frac{\partial}{\partial x_{3}} \\ f_{1} & f_{2} & f_{3} \end{vmatrix}
    \end{align}
\end{defi}
Propietats: si $\vec{f}(\vec{x})$ i $\vec{g}(\vec{x})$ són camps vectorials $\mathbb{R}^{3} \to \mathbb{R}^{3}$ i $\phi (\vec{x})$ és un camp escalar de $\mathbb{R}^{3}$,
\begin{enumerate}[i)]
    \item $\vnabla \times (\vec{f} + \vec{g}) = \vnabla \times \vec{f} + \vnabla \times \vec{g}$.
    \item $\vnabla \times (\phi \vec{f}) = (\vnabla \phi) \times \vec{f} + \phi (\vnabla \times \vec{f})$.
    \item Si $\phi (\vec{x})$ és de classe $C^{2}$: $\vnabla \times (\vnabla \phi) = \vec{0}$.
    \item Si $\vec{f}(\vec{x})$ és de classe $C^{2}$: $\vnabla \cdot (\vnabla \times \vec{f}) = 0$. 
\end{enumerate}

\subsubsection*{Laplaciana}
\begin{defi}[Laplaciana]
    Si $\phi (\vec{x})$ és un camp escalar definit a $D \subseteq \mathbb{R}^{n}$, definim la laplaciana de $\phi$ com el camp escalar
    \begin{align}
        \nabla^{2} \phi \equiv \operatorname{div} (\operatorname{grad} \phi ) = \frac{\partial^{2} \phi}{\partial x_{1}^{2}} + \frac{\partial^{2} \phi}{\partial x_{2}^{2}} + \dots + \frac{\partial^{2} \phi}{\partial x_{n}^{2}}= \vnabla \cdot (\vnabla \phi) = (\vnabla \cdot \vnabla) \phi 
    \end{align}
    i si $\vec{f}(\vec{x})$ és un camp vectorial definit a $D \subseteq \mathbb{R}^{n}$, amb valors a $\mathbb{R}^{m}$, definim
    \begin{align}
        \nabla^{2} \vec{f} \equiv \left( \nabla^{2} f_{1}, \nabla^{2} f_{2}, \dots, \nabla^{2} f_{m} \right)
    \end{align}
\end{defi}

	%----------------------------------------------------------------------------------------
%    INTEGRALS DE LÍNIA
%----------------------------------------------------------------------------------------
\section{Integrals de línia}
En aquest capítol estendrem el concepte d'integral de funcions d'una variable, $\displaystyle \int_{a}^{b} f(x) \diff x$ (inicialment definit per a funcions fitades en un interval finit $[a,b]$), a quan l'interval d'integració no és $[a,b]$ de $\mathbb{R}$, sinó un arc de corba de $\mathbb{R}^{n}$. Aquesta nova extensió ens porta al concepte d'integral de línia o integral curvilínia.

\subsection{Integral de línia d'un camp vectorial}
Sigui
\begin{itemize}
    \item $\vec{f}(\vec{x}) = (f_{1}(\vec{x}), \dots, f_{n}(\vec{x}))$ un camp vectorial definit a $D \subseteq \mathbb{R}^{n}$, amb valors a $\mathbb{R}^{n}$.
    \item $\mathcal{C}$ una corba $\vec{x}(u)$, $a\leq u \leq b$, de classe $C^{1}_{[a,b]}$ a trossos, continguda a $D$, d'extrems $\vec{a}$ i $\vec{b}$, amb $\vec{x}(a) = \vec{a}$ i $\vec{x}(b) = \vec{b}$.
    \item $\vec{f}(\vec{x})$ fitat sobre els punts de la corba $\mathcal{C}$.
\end{itemize}
\begin{defi}
    Definim la integral de línia del camp vectorial $\vec{f}(\vec{x})$ al llarg de la corba $\mathcal{C}$ com
    \begin{align}
        \int_{\mathcal{C}} \vec{f}(\vec{x}) \cdot \dif \vec{x} \equiv \int_{a}^{b} \vec{f}[\vec{x}(u)] \cdot \vec{x}'(u) \diff u = \int_{a}^{b} \vec{f}[\vec{x}(u)] \cdot \dif \vec{x}(u)
    \end{align}
\end{defi}
La notació que s'acostuma a fer servir és 
\begin{align}
    \int_\mathcal{C} \vec{f}(\vec{x}) \cdot \dif \vec{x} = \int_{\vec{a}}^{\vec{b}} \vec{f}(\vec{x}) \cdot \dif \vec{x}
\end{align}
però cal tenir present que la integral depèn del camí d'integració $\mathcal{C}$ i no únicament dels seus extrems $\vec{a}$ i $\vec{b}$.

Si la corba és tancada (és a dir, si $\vec{a} = \vec{b}$), escriurem
\begin{align}
    \oint_\mathcal{C} \vec{f} (\vec{x}) \cdot \dif \vec{x}
\end{align}

\subsubsection*{Propietats de les integrals de línia}
\begin{enumerate}[i)]
    \item Linealitat:
        \subitem $\displaystyle \int_{\mathcal{C}} (\vec{f} + \vec{g}) \cdot \dif \vec{x} = \int_{\mathcal{C}} \vec{f} \cdot \dif \vec{x} + \int_{\mathcal{C}} \vec{g} \cdot \dif \vec{x}$.
        \subitem $\displaystyle \int_{\mathcal{C}} (\lambda \vec{f}) \cdot \dif \vec{x} = \lambda \int_{\mathcal{C}} \vec{f} \cdot \dif \vec{x}, \quad \forall \lambda \in \mathbb{R}$.
    \item Additivitat del camí d'integració:
        \subitem $\int_{\mathcal{C}} \vec{f} \cdot \dif \vec{x} = \int_{\mathcal{C}_{1}} \vec{f} \cdot \dif \vec{x} + \int_{\mathcal{C}_{2}} \vec{f} \cdot \dif \vec{x}$, $\quad$ on $\begin{cases}\mathcal{C}: &\{ \vec{x}(u) \mid u \in [a,b] \} \\ \mathcal{C}_{1}: &\{ \vec{x}(u) \mid u \in [a,c] \}\\ \mathcal{C}_{2}: &\{ \vec{x}(u) \mid u \in [c,b] \} \end{cases}$
    \item Canvi de sentit:
        \subitem $\displaystyle \int_{\underset{\to}{\mathcal{C}}} \vec{f} \cdot \dif \vec{x} = - \int_{\underset{\leftarrow}{\mathcal{C}}} \vec{f} \cdot \dif \vec{x}$.
\end{enumerate}

\subsubsection*{Invariància sota reparametritzacions}
Sigui
\begin{itemize}
    \item $\vec{x}(u)$, $a\leq u \leq b$, una corba $\mathcal{C}$ de classe $C^{1}_{[a,b]}$, d'extrems $\vec{a}$ i $\vec{b}$.
    \item $u = u(v)$ una funció de classe $C^{1}_{[c,d]}$ amb $u'(v) > 0$, és a dir, $u(v)$ és estrictament creixent a $[c,d]$ i tenim, per tant, $u(c) = a$ i $u(d) = 4$.
\end{itemize}
La funció $\vec{y}(v) \equiv \vec{x}[u(v)]$ descriu el mateix camí $\mathcal{C}$ i en el mateix sentit, ja que $\vec{y}(c) = \vec{x}[u(c)] = \vec{x}(a) = \vec{a}, \quad \vec{y}(d) = \vec{x}[u(d)] = \vec{x}(b) = \vec{b}$.

Direm que $\vec{y}(v) \equiv \vec{x}[u(v)]$ és una reparametrització del camí $\mathcal{C}$ originalment definit per $\vec{x}(u)$. Aquesta reparametrització del camí $\mathcal{C}$ no modifica el valor de la integral.

%----------------------------------------------------------------------------------------
\subsection{Integral de línia d'un camp escalar respecte el paràmetre arc}
Sigui
\begin{itemize}
    \item $\phi(\vec{x})$ un camp escalar definit a $D \subseteq \mathbb{R}^{n}$, amb valors a $\mathbb{R}$.
    \item $\mathcal{C}$ una corba $\vec{x}(u)$, $a\leq u \leq b$, de classe $C^{1}_{[a,b]}$ a trossos, continguda a $D$, d'extrems $\vec{a}$ i $\vec{b}$, amb $\vec{x}(a) = \vec{a}$ i $\vec{x}(b) = \vec{b}$. Com que $\vec{x}(u)$ és de classe $C^{1}$, la corba $\mathcal{C}$ és rectificable i tenim el paràmetre arc $s(u) = \int_{a}^{u} \| \vec{x}'(u) \| \diff u$, la derivada del qual és $s'(u) = \| \vec{x}'(u) \|$.
    \item $\phi(\vec{x})$ fitat sobre els punts de la corba $\mathcal{C}$.
\end{itemize}
\begin{defi}
    Definim la integral de línia del camp escalar $\phi(\vec{x})$ al llarg de la corba $\mathcal{C}$ com
    \begin{align}
        \int_{\mathcal{C}} \phi(\vec{x}) \diff s \equiv \int_{a}^{b} \phi[\vec{x}(u)] s'(u) \diff u
    \end{align}
\end{defi}
La relació entre $\int_{\mathcal{C}} \phi(\vec{x}) \diff s$ i $\int_{\mathcal{C}} \vec{f}(\vec{x}) \cdot \dif \vec{x}$ ve donada pel següent teorema.
\begin{thm}
    Si $\phi(\vec{x})$ és la component de $\vec{f}(\vec{x})$ en la direcció tangent a la corba $\mathcal{C}$, llavors
    \begin{align}
        \int_{\mathcal{C}} \phi(\vec{x}) \diff s = \int_{\mathcal{C}} \vec{f}(\vec{x}) \cdot \dif \vec{x}
    \end{align}
\end{thm}

%----------------------------------------------------------------------------------------
\subsection{Integrals de línia independents del camí}
\subsubsection*{Conjunts connexos i conjunts convexos}
Sigui un conjunt $S \subseteq \mathbb{R}^{n}$
\begin{itemize}
    \item $S$ és connex si $\forall \vec{a}, \vec{b} \in S, \quad \exists$ una corba contínua $\vec{x}(u)$, d'extrems $\vec{a}$ i $\vec{b}$, continguda a $S$.
    \item $S$ és convex si $\forall \vec{a}, \vec{b} \in S$, si la recta que uneix $\vec{a}$ i $\vec{b}$ està continguda a $S$.
\end{itemize}
Evidentment, $S$ convex $\Rightarrow S$ connex.

\subsubsection*{Integrals de línia independents del camí}
Per als camps en què les integrals no depenen del camí $\mathcal{C}$ (és a dir que només depenen dels seus extrems), les següents afirmacions són equivalents:
\begin{itemize}
    \item $\displaystyle \int_{\vec{a}}^{\vec{b}} \vec{f} \cdot \dif \vec{x}$ és independent del camí que uneix $\vec{a}$ i $\vec{b}$ ($\forall \vec{a}, \vec{b} \in D$.
    \item $\displaystyle \oint_{\mathcal{C}} \vec{f} \cdot \dif \vec{x} = 0$ per a qualsevol camí $\mathcal{C}$ tancat de $D$.
\end{itemize}
O en altres paraules
\begin{align}
    \int_{\underset{\to}{\mathcal{C}_{1}}} = \int_{\underset{\to}{\mathcal{C}_{2}}} \Leftrightarrow \int_{\underset{\to}{\mathcal{C}_{1}}} = - \int_{\underset{\leftarrow}{\mathcal{C}_{2}}} \Leftrightarrow \oint_{\underset{\to}{\mathcal{C}_{1}} + \underset{\leftarrow}{\mathcal{C}_{2}}} = 0
\end{align}

\subsubsection*{Condició per a la independència del camí d'integració}
\begin{thm}[Generalització del 2n teorema fonamental del càlcul]
    Sigui $\vec{f}(\vec{x})$ un camp vectorial continu definit en un conjunt obert connex $D \subseteq \mathbb{R}^{n}$, amb valors a $\mathbb{R}^{n}$.
    
    Si $\vec{f}(\vec{x})$ és el gradient d'un camp escalar $\phi(\vec{x})$ definit a $D$, és a dir $\vec{f}(\vec{x}) = \vnabla \phi (\vec{x})$, llavors $\forall \vec{a}, \vec{b} \in D$ es compleix
    \begin{align}
        \int_{\vec{a}}^{\vec{b}} \vec{f}(\vec{x}) \cdot \dif \vec{x} = \phi(\vec{b}) - \phi(\vec{a})
    \end{align}
    La integral no depèn, per tant, del camí que uneix $\vec{a}$ i $\vec{b}$.
\end{thm}
\begin{thm}[Generalització del 1r teorema fonamental del càlcul]
    Sigui $\vec{f}(\vec{x})$ un camp vectorial continu definit en un conjunt obert connex $D \subseteq \mathbb{R}^{n}$, amb valors a $\mathbb{R}^{n}$.
    
    Si $\forall \vec{a}, \vec{b} \in D$ la integral $\displaystyle \int_{\vec{a}}^{\vec{b}} \vec{f}(\vec{x}) \cdot \dif \vec{x}$ és independent del camí que uneix $\vec{a}$ i $\vec{b}$, llavors $\forall \vec{x} \in D$, 
    \begin{align}
        \vec{f}(\vec{x}) \text{ és el gradient de } \phi(\vec{x}) = \int_{\vec{a}}^{\vec{x}} \vec{f}(\vec{x}) \cdot \dif \vec{x}
    \end{align}
\end{thm}
\begin{cor}
Els teoremes anteriors estableixen que si $\vec{f}(\vec{x})$ és continu en un conjunt obert connex, les següents afirmacions són equivalents
\begin{itemize}
    \item $\vec{f}(\vec{x})$ és un gradient.
    \item $\displaystyle \int_{\vec{a}}^{\vec{b}} \vec{f} \cdot \dif \vec{x}$ és independent del camí.
    \item $\displaystyle \oint_{\mathcal{C}} \vec{f} \cdot \dif \vec{x} = 0$ per a qualsevol camí tancat.
\end{itemize}
\end{cor}

\subsubsection*{Condicions per què un camp vectorial de classe $C^{1}$ sigui un gradient}
\begin{thm}[Condició necessària]
    Si $\vec{f}(\vec{x})$ (de classe $C^{1}_{D}$) és el gradient de $\phi(\vec{x})$ a $D \subseteq \mathbb{R}^{n}$, llavors
    \begin{align}
        \frac{\partial f_{i}}{\partial x_{j}} = \frac{\partial f_{j}}{\partial x_{i}}, \quad \forall \vec{x} \in D
    \end{align}
\end{thm}
Notem que a $\mathbb{R}^{3}$ la condició anterior equival a $\vnabla \times \vec{f} = \vec{0}$.

\begin{thm}[Condició suficient]
    Sigui $\vec{f}(\vec{x})$ un camp vectorial de classe $C^{1}_{D}$, on $D$ és un obert convex de $\mathbb{R}^{n}$. Llavors
    \begin{align}
        \text{Si } \frac{\partial f_{i}}{\partial x_{j}} = \frac{\partial f_{j}}{\partial x_{i}} \Rightarrow \vec{f}(\vec{x}) \text{ és un gradient.}
    \end{align}
\end{thm}


	%----------------------------------------------------------------------------------------
%    INTEGRALS MÚLTIPLES
%----------------------------------------------------------------------------------------
\section{Integrals múltiples}
En aquest capítol estendrem el concepte d'integral a regions d'integració $n$-dimensionals. Més concretament, si $f(\vec{x})$ és un camp escalar de $\mathbb{R}^{n}$, donarem sentit a $\displaystyle \int_{\mathcal{S}}f$, on $\mathcal{S}$ és una regió de $\mathbb{R}^{n}$. Ho denotarem
\begin{align*}
	\int_{\mathcal{S}} f \equiv \iint \overset{(n)}{\dots} \int_{\mathcal{S}} f(x_{1}, \dots, x_{n}) \diff x_{1} \dots \dif x_{n}
\end{align*}
Considerarem, primer, el cas en què $\mathcal{S}$ és una regió «rectangular», per estendre-ho a regions més generals. Detallarem el cas en què $\mathcal{S}$ és de $\mathbb{R}^{2}$. La generalització a més dimensions és òbvia.

%----------------------------------------------------------------------------------------
\subsection{Integral d'un camp escalar sobre un «interval rectangular»}
La integració en intervals rectangulars és conceptualment idèntica al cas de les funcions d'una variable.
\begin{defi}[Interval rectangular tancat de $\mathbb{R}^{n}$]
	Un interval rectangular tancat $I$ de $\mathbb{R}^{n}$ és el producte cartesià de $n$ intervals tancats de $\mathbb{R}$:
	\begin{align}
		I = [a_{1}, b_{1}] \times [a_{2}, b_{2}] \times \dots [a_{n}, b_{n}] = \{ \vec{x} \mid x_{i} \in [a_{i}, b_{i}] \}
	\end{align}
\end{defi}
\begin{defi}[Mesura d'un interval rectangular]
	Si $I$ és el «rectangle $n$-dimensional» $[a_{1}, b_{1}] \times [a_{2}, b_{2}] \times \dots [a_{n}, b_{n}]$ de $\mathbb{R}^{n}$, s'anomena mesura de $I$, a $\mathbb{R}^{n}$, al producte
	\begin{align}
		\mu(I) \equiv (b_{1} - a_{1}) (b_{2} - a_{2}) \dots (b_{n} - a_{n})
	\end{align}
	Cal notar que a $\mathbb{R}^{n}$ la mesura d'un rectangle $r$-dimensional és 0 si $r<n$.
\end{defi}

\subsubsection*{Particions d'un interval rectangular}
\begin{defi}
	Sigui $I = [a_{1}, b_{1}] \times [a_{2}, b_{2}]	$ un interval rectangular de $\mathbb{R}^{2}$.

	Si $\Pi_{1}$ és una partició de $[a_{1}, b_{1}]$: $a_{1} = x_{0} < \dots < x_{k_{1}} = b_{1}$; i $\Pi_{2}$ és una partició de $[a_{2}, b_{2}]$: $a_{2} = y_{0} < \dots < y_{k_{2}} = b_{2}$.

	Llavors, $\Pi = \Pi_{1} \times \Pi_{2}$ és una partició del rectangle $I$ que el descompon en $k_{1}k_{2}$ subintervals rectangulars que denotem per $I_{ij}$, la mesura dels quals és $\mu(I_{ij}) = \Delta x_{i} \Delta y_{j}$.

	El concepte es generalitza trivialment a intervals rectangulars de dimensió superior.
\end{defi}
Una partició $\Pi'$ és més fina que $\Pi$ si $\Pi'$ s'obté afegint punts a $\Pi$ (o equivalentment, si $\Pi' \supset \Pi$).

En general, donades dues particions $\Pi_{a}$ i $\Pi_{b}$ del rectangle $I$, no es pot afirmar quina és més fina que l'altra, però sempre es pot construir una partició $\Pi_{ab}$ més fina que $\Pi_{a}$ i que $\Pi_{b}$ alhora.

\subsubsection*{Sumer superiors i inferiors}
Sigui $f(x,y)$ un camp escalar definit i fitat en un interval rectangular tancat $I$ de $\mathbb{R}^{2}$. Sigui $\Pi$ una partició de $I$.
\begin{defi}[Suma superior]
	\begin{align}
		S(f,\Pi) \equiv \sum_{ij} M_{ij} \mu(I_{ij}) = \sum_{ij} M_{ij} \Delta x_{i} \Delta y_{j}
	\end{align}
	on $M_{ij}$ és el suprem de $f(x,y)$ al subinterval $I_{ij}$.
\end{defi}
\begin{defi}[Suma inferior]
	\begin{align}
		s(f,\Pi) \equiv \sum_{ij} m_{ij} \mu(I_{ij}) = \sum_{ij} m_{ij} \Delta x_{i} \Delta y_{j}
	\end{align}
	on $m_{ij}$ és l'ínfim de $f(x,y)$ al subinterval $I_{ij}$.
\end{defi}
Propietats:
\begin{itemize}
	\item El conjunt de les sumes superiors de $f$ és fitat inferiorment. El seu ínfim s'anomena integral superior de $f$ sobre l'interval $I$ i es denota $\displaystyle \overline{\int_{I}} f$.
	\item El conjunt de les sumes inferiors de $f$ és fitat superiorment. El seu suprem s'anomena integral inferior de $f$ sobre l'interval $I$ i es denota $\displaystyle \underline{\int_{I}} f$.
\end{itemize}
Evidentment, es compleix
\begin{align}
	\underline{\int_{I}} f \leq \overline{\int_{I}} f
\end{align}

\subsubsection*{Funcions integrables}
El camp escalar $f(x,y)$ és integrable a l'interval rectangular tancat de $I$ de $\mathbb{R}^{2}$ si $\displaystyle \underline{\int_{I}} f = \overline{\int_{I}} f$. En aquest cas, aquest valor comú s'anomena integral de $f(x,y)$ sobre l'interval rectangular i tancat $I$, i es denota per $\displaystyle \int_{I} f$.

Aquesta definició d'integral és general per a camps escalars de $\mathbb{R}^{n}$. A $\mathbb{R}^{n}$ escriurem
\begin{align}
	\int_{I} f = \iint \cdot^{(n)} f(x_{1}, \dots x_{n}) \diff x_{1} \dots \dif x_{n} \quad \text{(integral $n$-múltiple)}
\end{align}
\begin{thm}[1r criteri d'integrabilitat]
	$f(\vec{x})$ és integrable a l'interval rectangular tancat $I \Leftrightarrow \forall \varepsilon >0, \quad \exists \Pi \mid S(f, \Pi) - s(f, \Pi) < \varepsilon$.
\end{thm}
\begin{cor}
	Si $f(\vec{x})$ és contínua a l'interval rectangular tancat $I \Rightarrow f(\vec{x})$ és integrable a $I$.
\end{cor}

\subsubsection*{Sumes de Riemann}
\begin{defi}
	Sigui $f(\vec{x})$ un camp escalar de $\mathbb{R}^{2}$, definit i fitat en l'interval rectangular tancat $I$; i $\Pi$ una partició que divideix $I$ en subintervals $I_{ij}$. Escollim un punt $\vec{t}_{ij}$ en cadascun dels subintervals $I_{ij}$.
	% Comprovar que no quedi massa recarregat, sinó fer tots sigui per enumeració o simiars.

	S'anomena suma de Riemann de $f(\vec{x})$, associada a la partició $\Pi$ i als punts $\vec{t}_{ij}$, a
	\begin{align}
		S_{R} (f, \Pi, \vec{t}_{ij}) \equiv \sum_{ij} f(\vec{t}_{ij}) \mu (I_{ij})
	\end{align}
\end{defi}
Direm que les sumes de Riemann de $f$, a l'interval $I$, tenen límit si $\forall \varepsilon, \quad \exists \Pi_{0} \mid \forall \Pi \supset \Pi_{0}$, $|S_{R}(f, \Pi) - A| < \varepsilon \quad \forall$ elecció dels $\vec{t}_{ij}$.

\begin{thm}[2n criteri d'integrabilitat]
	$f(\vec{x})$ és integrable $\Leftrightarrow$ les sumes de Riemann de $f$, a $I$, tenen límit (en aquest cas, el límit és $\displaystyle \int_{I} f$).
\end{thm}
Propietats de $\displaystyle \int_{I} f$:
\begin{enumerate}[i)]
	\item $\displaystyle \int_{I} (\lambda f + \eta g) = \lambda \int_{I} f + \eta \int_{I} g, \quad \forall \mu, \eta \in \mathbb{R}$.
	\item $\displaystyle \int_{I \cup J} f = \int_{I} f + \int_{J} f$ si $I$ i $J$ són dos intervals rectangulars amb un costat comú.
	\item Si $f$ i $g$ són integrables i $\displaystyle f \geq g \Rightarrow \int_{I} f \geq \int_{I} g$.
\end{enumerate}

\subsubsection*{Càlcul d'una integral doble per integració unidimensional successiva}
\begin{thm}
	Sigui $f(x,y)$ definida i fitada a l'interval rectangular $I = [a_{1}, b_{1}] \times [a_{2}, b_{2}]$ de $\mathbb{R}^{2}$.

	Si es dóna que
	\begin{enumerate}[i)]
		\item $f(x,y)$ és integrable a $I$.
		\item $\displaystyle \exists A(y) = \int_{a_{1}}^{b_{1}} f(x,y) \diff x, \quad \forall y \in [a_{2}, b_{2}]$.
		\item $A(y)$ és integrable a $[a_{2}, b_{2}]$.
	\end{enumerate}
	Llavors, es compleix que
	\begin{align}
		\int_{a_{2}}^{b_{2}} A(y) \diff y = \iint_{I} f(x,y) \diff x \diff y = \int_{a_{2}}^{b_{2}} \left[ \int_{a_{1}}^{b_{1}} f(x,y) \diff x \right] \diff y
	\end{align}
\end{thm}
\newpage
Notem que:
\begin{itemize}
	\item Si el teorema és aplicable, l'ordre de les integrals és l'indicat i no és permutable, llevat que es compleixi que $\displaystyle \int_{a_{2}}^{b_{2}} f(x,y) \diff y$ sigui, també integrable (respecte $x$) a $[a_{1}, b_{1}]$. En aquest cas tindríem també
		\begin{align}
		\iint_{I} f(x,y) \diff x \diff y = \int_{a_{1}}^{b_{1}} \left[ \int_{a_{2}}^{b_{2}} f(x,y) \diff y \right] \diff x
		\end{align}
	\item Si $f(x,y)$ és contínua a $I$, el teorema és aplicable, ja que $A(y)$ és, també, contínua (i, per tant, integrable).
\end{itemize}
\begin{cor}
	Així doncs, la integral doble ($n$-múltiple) d'un camp escalar continu, sobre un interval rectangular, es pot obtenir sempre per integració successiva (i, en aquest cas, en qualsevol ordre) respecte cadascuna de les variables.
\end{cor}

\subsubsection*{Integrabilitat de funcions amb discontinuïtats}
Sigui $f(\vec{x})$ un camp escalar, definit i fitat a l'interval rectangular $I$ de $\mathbb{R}^{n}$; i $D_{I}^{f}$ el conjunt de les discontinuïtats de $f$ a $I$.
% Comprovar que no quedi massa recarregat, sinó fer tots sigui per enumeració o simiars.

Llavors, direm que $D_{I}^{f}$ té contingut nul si $\forall \varepsilon > 0$ es pot cobrir $D_{I}^{f}$ amb un nombre finit d'intervals rectangulars de mesura total $\mu < \varepsilon$.
\begin{thm}
	Sigui $f(\vec{x})$ definida i fitada a l'interval rectangular $I$ i sigui $D^{f}_{I}$ el conjunt de les discontinuïtats de $f$ a $I$. Llavors, si $D^{f}_{I}$ té contingut nul $\Rightarrow f(\vec{x})$ és integrable a $I$.
\end{thm}

%----------------------------------------------------------------------------------------
\subsection{Integració sobre regions més generals}
Veurem ara com estendre el concepte d'integral per incloure regions d'integració que no siguin intervals rectangulars.

Sigui $\mathcal{S}$ una regió fitada de $\mathbb{R}^{n}$; $f(\vec{x})$ un camp escalar, definit i fitat a $\mathcal{S}$; i $I$ un interval rectangular tancat e $\mathbb{R}^{n}$ que contingui $\mathcal{S}$. Llavors, definim, a $I$, la funció
% Comprovar que no quedi massa recarregat, sinó fer tots sigui per enumeració o simiars.
\begin{align}
	\tilde{f} (\vec{x}) = \begin{cases} f(\vec{x}), &\text{si } \vec{x} \in \mathcal{S} \\ 0, &\text{si } \vec{x} \in I \backslash \mathcal{S} \end{cases}
\end{align}
Llavors, per definició
\begin{align}
	\iint \overset{(n)}{\dots} \int_{\mathcal{S}} f(\vec{x}) \diff x_{1} \dots \dif x_{n} = \iint \overset{(n)}{\dots} \int_{I} \tilde{f}(\vec{x}) \diff x_{1} \dots \dif x_{n}
\end{align}
A $\mathbb{R}^{2}$ considerarem únicament regions $\mathcal{S}$ que anomenarem de tipus I i de tipus II.

\subsubsection*{Integrals dobles en regions de tipus I i de tipus II}
\begin{defi}[Regions de tipus I (projectables--$x$)]
	S'anomenen així les regions $\mathcal{S}$ limitades per
	\begin{align*}
		a \leq x \leq b, \quad \varphi_{1} (x) \leq y \leq \varphi_{2}(x)
	\end{align*}
	on $\varphi_{1}(x)$ i $\varphi_{2}(x)$ són contínues a $[a,b]$.

	El que caracteritza aquestes regions és el fet que els segments verticals, d'extrems $\varphi_{1}(x)$ i $\varphi_{2}(x)$, (amb $a \leq x \leq b$) estan totalment dins $\mathcal{S}$.
\end{defi}
\begin{thm}
    Si $f(x,y)$ és contínua a $\mathcal{S}$ (de tipus I) $\displaystyle \Rightarrow$
    \begin{align*}
        \iint_{\mathcal{S}} f(x,y) \diff x \diff y = \int_{a}^{b} \left[ \int_{\varphi_{1}(x)}^{\varphi_{2}(x)} f(x,y) \diff y \right] \diff x
    \end{align*}
% Consideraria utilitzar un entorn align* segons si queda bé o no.
\end{thm}

\begin{defi}[Regions de tipus II (projectables--$y$)]
	S'anomenen així les regions $\mathcal{S}$ limitades per
	\begin{align*}
		c \leq y \leq d, \quad \psi_{1} (y) \leq x \leq \psi_{2}(y)
	\end{align*}
	on $\psi_{1}(y)$ i $\psi_{2}(y)$ són contínues a $[c,d]$.

	El que caracteritza aquestes regions és el fet que els segments horitzontals, d'extrems $\psi_{1}(y)$ i $\psi_{2}(y)$, (amb $c \leq y \leq d$) estan totalment dins $\mathcal{S}$.
\end{defi}
\begin{thm}
	Si $f(x,y)$ és contínua a $\mathcal{S}$ (de tipus II) $\displaystyle \Rightarrow$
	\begin{align*}
        \iint_{\mathcal{S}} f(x,y) \diff x \diff y = \int_{c}^{d} \left[ \int_{\psi_{1}(y)}^{\psi_{2}(y)} f(x,y) \diff x \right] \diff y
    \end{align*}
% Consideraria utilitzar un entorn align* segons si queda bé o no.
\end{thm}

\subsubsection*{Integrals dobles en altres regions}
\begin{itemize}
	\item Una regió $\mathcal{S}$ pot ser, alhora, de tipus I i II (e.g., si $\mathcal{S}$ és convex). Llavors l'ordre d'integració és irrellevant (però la dificultat pot ser diferent).
	\item També pot succeir que una regió $\mathcal{S}$ no sigui de tipus I ni del II. No obstant, en la majoria de casos pràctics, $\mathcal{S}$ es pot descompondre em subregions de tipus I i/o II. La integració total és, llavors, la suma de les integracions sobre cadascuna d'aquestes subregions.
\end{itemize}

\subsubsection*{Integrals en més dimensions}
Hem vist que les integrals dobles es poden calcular per integració successiva de les variables (en l'ordre adient) quan la regió és de tipus I o de tipus II.

Això es pot generalitzar a integrals $n$-múltiples. Vegem-ho en el cas de les integrals triples. En aquest cas tindrem tres tipus de regions similars a les de tipus II i I de les integrals dobles.
\begin{defi}[Regions projectables--$xy$]
	S'anomenen així les regions d'integració $\mathcal{S} \subseteq \mathbb{R}^{3}$, definides per
	\begin{align*}
		(x,y) \in Q, \quad \phi_{1}(x,y) \leq z \leq \phi_{2}(x,y)
	\end{align*}
	on $Q$ és una regió del pla $x-y$, i $\phi_{1}(x,y)$ i $\phi_{2}(x,y)$ són contínues a $Q$.

	El que caracteritza aquestes regions és el fet que els segments verticals (paral·lels a l'eix $z$) d'extrems $\phi_{1}(x,y)$ i $\phi_{2}(x,y)$ estan totalment continguts dins $\mathcal{S}$, $\forall (x,y) \in Q$. Q és la projecció de $\mathcal{S}$ sobre el pla $x-y$. Argumentant com en el cas de les integrals dobles, tenim
	\begin{align}
		\iiint_{\mathcal{S}} f(x,y,z) \diff x \diff y \diff z = \iint_{Q}\left[ \int_{\phi_{1}(x,y)}^{\phi_{2}(x,y)} f(x,y,z) \diff z \right] \diff x \diff y
	\end{align}
\end{defi}
De forma similar, es defineixen a $\mathbb{R}^{3}$ regions projectables--$yz$ i projectables--$zx$. Cal notar que si $\mathcal{S}$ és convex, és simultàniament dels tres tipus i les seves projeccions sobre els plans $x-y$, $y-z$ i $z-x$ són simultàniament de tipus I i II. En aquest cas podem integrar successivament cada variable en qualsevol ordre, amb els límits d'integració adients.

%----------------------------------------------------------------------------------------
\subsection{Teorema de Green}
\begin{thm}[de Green]
	Siguin
	\begin{enumerate}[i)]
		\item $P(x,y)$ i $Q(x,y)$ dos camps escalars de classe $C^{1}_{D}$, on $D \subseteq \mathbb{R}^{2}$.
		\item $\mathcal{C}$ una corba $\vec{x}(t)$ de classe $C^{1}_{[a,b]}$ a trossos, tancada (i.e., $\vec{x}(a) = \vec{x}(b)$) i simple (i.e., no es talla a si mateixa: $\vec{x}(t) \neq \vec{x}(t')$, si $t \neq t'$).
		\item $\mathcal{S}$ la regió formada per $\mathcal{C}$ i el seu interior.
	\end{enumerate}
	Llavors, si $\mathcal{S}$ és simultàniament de tipus I i tipus II, es compleix
	\begin{align}
		\iint_{\mathcal{S}} \left( \frac{\partial Q}{\partial x} - \frac{\partial P}{\partial y} \right) \diff x \diff y = \oint_{\mathcal{C}} (P \diff x + Q \diff y)
	\end{align}
\end{thm}
Notem que la igualtat anterior és equivalent a les igualtats
\begin{align}
	\iint_{\mathcal{S}} \left( \frac{\partial Q}{\partial x} \right) \diff x \diff y = \oint_{\mathcal{C}} Q \diff y \quad \text{i} \quad \iint_{\mathcal{S}} \left( - \frac{\partial P}{\partial y} \right) \diff x \diff y = \oint_{\mathcal{C}} P \diff x
\end{align}
El teorema de Green és específic de $\mathbb{R}^{2}$. Relaciona, sota determinades condicions, una integral doble sobre una regió de $\mathbb{R}^{2}$ amb una integral de línia sobre el contorn d'aquesta regió. Però és. de fet, un cas particular d'un teorema més general que relaciona integrals sobre regions $n$-dimensionals amb integrals sobre les fronteres $(n-1)$-dimensionals d'aquestes regions. A $\mathbb{R}^{3}$ hi ha dos casos particulars més d'aquest teorema general: el teorema d'Stokes i el teorema de Gauss (respectivament, teoremes \ref{thm:stokes} i \ref{thm:gauss}).

\subsubsection*{Generalitzacions}
\begin{itemize}
	\item Tot i que el teorema de Green requereix que $\mathcal{S}$ sigui simultàniament de tipus I i II, també es compleix encara que no es compleixi aquest requisit si es pot descompondre en un nombre finit de subregions que siguin simultàniament de tipus I i II, ja que al sumar les contribucions de cadascuna d'aquestes subregions, les integrals de línies internes es cancel·len.
	\item Si $\mathcal{S}$ té un nombre finit de «forats» el teorema de Green també es compleix, amb integrals de línia sobre els contorns fetes sobre el sentit contrari. En definitiva, el contorn es recorre de forma que l'interior de $\mathcal{S}$ quedi sempre a l'esquerra.
\end{itemize}

%----------------------------------------------------------------------------------------
\subsection{Canvi de variables en una integral múltiple}
\begin{thm}[del canvi de variables]
	Si es dóna que
	\begin{enumerate}[i)]
		\item $f(\vec{x})$ és una funció $\mathbb{R}^{n} \to \mathbb{R}$, contínua a $\mathcal{S} \subseteq \mathbb{R}^{n}$.
		\item $\vec{x} = \vec{g}(\vec{t})$ és un canvi de variables $\mathbb{R}^{n} \to \mathbb{R}^{n}$, de classe $C^{1}_{T}$, on $\mathcal{S} = \vec{g}(T)$.
		\item $\displaystyle \left(\frac{\partial (g_{1}, \dots, g_{n})}{\partial (t_{1}, \dots, t_{n})} \right)_{T} \neq 0$.
	\end{enumerate}
	Llavors, es compleix
	\begin{align}
	\iint \overset{(n)}{\dots} \int_{\mathcal{S}} f(\vec{x}) \diff x_{1} \dots \dif x_{n} = \iint \overset{(n)}{\dots} \int_{T} f(\vec{g}(\vec{t})) \left| \frac{\partial (g_{1}, \dots, g_{n})}{\partial (t_{1}, \dots, t_{n})} \right| \diff t_{1} \dots \dif t_{n}
	\end{align}
\end{thm}

	%-------------------------------------------------------------------------------
%          INTEGRALS DE SUPERFÍCIE
%-------------------------------------------------------------------------------
\section{Integrals de superfície}
En aquest capítol estendrem el concepte d'integral doble quan la regió no és de $\mathbb{R}^{2}$ sinó que és una superfície de $\mathbb{R}^{3}$. Aquesta extensió ens porta al concepte d'integral de superfície.

En molts sentits, les integrals de superfície són, respecte les integrals dobles, el que les integrals de línia són respecte les integrals simples.

%-------------------------------------------------------------------------------
\subsection{Superfícies a $\mathbb{R}^{3}$}
En general, una superfície de $\mathbb{R}^{3}$ es pot descriure de tres formes:
\begin{itemize}
	\item Implícita: conjunt de punts que satisfan l'equació $F(x,y,z) = 0$, on $F$ és una funció contínua.
	\item Explícita: quan és possible aïllar una de les variables de l'equació anterior, per exemple $z = f(x,y)$.
	\item Paramètrica: expressant les coordenades dels punts com funcions contínues de dos paràmetres
\end{itemize}
\begin{align*}
	\vec{r}(u,v) = (X(u,v), Y(u,v), Z(u,v))
\end{align*}
Les funcions $X(u,v)$, $Y(u,v)$ i $Z(u,v)$ generen els punts de la superfície quan els paràmetres $u$ i $v$ es mouen en una regió $T$ de $\mathbb{R}^{2}$:
\begin{align*}
	\mathcal{S} \equiv \vec{r}(T)
\end{align*}
Direm que la superfície $\mathcal{S}$ és simple si l'aplicació $T \to \mathcal{S}$ és un a un.

Notem no que forma explícita és un cas particular de la paramètrica en què els paràmetres són $x$ i $y$.
\begin{align*}
	\vec{r}(x,y) = (x,y,f(x,y))
\end{align*}

\subsubsection*{Producte vectorial fonamental}
\begin{defi}
	Anomenem producte vectorial fonamental de la superfície $\mathcal{S}$ al vector
	\begin{align}
		\frac{\partial \vec{r}}{\partial u} \times \frac{\partial \vec{r}}{\partial v} \equiv \left( \frac{\partial (Y, Z)}{\partial (u,v)}, \frac{\partial (Z, X)}{\partial (u,v)}, \frac{\partial (X, Y)}{\partial (u,v)} \right)
	\end{align}
	El producte vectorial fonamental és ortogonal al pla tangent a $\mathcal{S}$ en cada punt. Quan ens movem dins la regió paramètrica $T$, l'orientació del pla tangent a $\mathcal{S}$ varia d'un punt a una altra. Aquesta variació és contínua si les funcions $X(u,v)$, $Y(u,v)$ i $Z(u,v)$ són de classe $C^{1}_{T}$.
\end{defi}
Si expressem $\mathcal{S}$ en forma explícita, $\vec{r}(x,y) = (x,y,f(x,y))$, el producte vectorial fonamental s'expressa
\begin{align*}
	\frac{\partial \vec{r}}{\partial u} \times \frac{\partial \vec{r}}{\partial v} = \left( -\frac{\partial f}{\partial x}, -\frac{\partial f}{\partial y}, 1 \right)
\end{align*}
Com veurem, a les integrals de superfície, el producte vectorial fonamental juga un paper similar al del factor $\vec{r}'(t)$ a les integrals de línia.

\subsubsection*{Àrea d'una superfície $\mathcal{S}$}
\begin{defi}
	El mòdul del producte vectorial fonamental és el factor de proporcionalitat entre les àrees del pla $u-v$ i les corresponents àrees de la superfície $\mathcal{S}$. Llavors, si $\mathcal{S}$ és de classe $C^{1}$, definim l'àrea de $\mathcal{S}$
	\begin{align}
		a(\mathcal{S}) \equiv \iint_{T} \left\| \frac{\partial \vec{r}}{\partial u} \times \frac{\partial \vec{r}}{\partial v} \right\| \diff u \diff v = \iint_{\mathcal{S}} \dif \mathcal{S}
	\end{align}
\end{defi}
Si expressem $\mathcal{S}$ en forma explícita, $z=f(x,y)$, la igualtat anterior esdevé
\begin{align*}
	a(\mathcal{S}) = \iint_{T} \sqrt{1 + \left( \frac{\partial f}{\partial x} \right)^{2} + \left( \frac{\partial f}{\partial y} \right)^{2} } \diff x \diff y
\end{align*}

%-------------------------------------------------------------------------------
\subsection{Integrals sobre una superfície $\mathcal{S}$ de $\mathbb{R}^{3}$}
\subsubsection*{Integral d'un camp escalar de $\mathbb{R}^{3}$ sobre una superfície $\mathcal{S}$}
\begin{defi}[Integral de $f$ sobre $\mathcal{S}$]
	Sigui
	\begin{itemize}
		\item $f(x,y,z)$ un camp escalar de $\mathbb{R}^{3}$, continu a trossos.
		\item $\mathcal{S}$ una superfície regular de classe $C^{1}$, d'equacions paramètriques $x=X(u,v)$, $y=Y(u,v)$ i $z=Z(u,v)$.
	\end{itemize}
	Llavors, tenim
	\begin{align}
		\iint_{\mathcal{S}} f \diff \mathcal{S} \equiv \iint_{T} f(X(u,v), Y(u,v), Z(u,v)) \left\| \frac{\partial \vec{r}}{\partial u} \times \frac{\partial \vec{r}}{\partial v} \right\| \diff u \diff v
	\end{align}
\end{defi}
En particular l'àrea de $\mathcal{S}$ és la integral sobre $\mathcal{S}$ de la funció constant $f(x,y,z) \equiv 1$.

\subsubsection*{Integral d'un camp vectorial $\mathbb{R}^{3} \to \mathbb{R}^{3}$ sobre una superfície $\mathcal{S}$}
\begin{defi}[Integral de $\vec{f}$ sobre $\mathcal{S}$]
	Sigui
	\begin{itemize}
		\item $\vec{f}(x,y,z) = (P(x,y,z), Q(x,y,z), R(x,y,z))$ un camp vectorial $\mathbb{R}^{3} \to \mathbb{R}^{3}$, continu a trossos.
		\item $\mathcal{S}$ una superfície regular de classe $C^{1}$, d'equacions paramètriques $x=X(u,v)$, $y=Y(u,v)$ i $z=Z(u,v)$.
	\end{itemize}
	Llavors, tenim
	\begin{align}
		\iint_{\mathcal{S}} \vec{f} \cdot \dif \vec{\mathcal{S}} \equiv \iint_{\mathcal{S}} (P,Q,R) \cdot (\dif y \wedge \dif z, \dif z \wedge \dif x, \dif x \wedge \dif z) 
	\end{align}
\end{defi}

\subsubsection*{Invariància sota reparametritzacions}
Sigui $\mathcal{S}$ és una superfície regular $\vec{r}(u,v) = (X(u,v), Y(u,v), Z(u,v))$, de classe $C^{1}_{T})$, i fem el canvi de paràmetres
\begin{align*}
	\begin{cases} u & = U(s,t) \\ v & = V(s,t) \end{cases} \quad \text{amb } \frac{\partial (U,V)}{\partial (s,t)} \neq 0
\end{align*}
on $U$ i $V$ són funcions de classe $C^{1}_{W}$ que generen els punts $(u,v)$ de $T$ quan $(s,t)$ es mou a $W$.

Tenim doncs, dues parametritzacions de la superfície $\mathcal{S}$
\begin{align*}
	\vec{r} = \vec{r}(u,v) \leftrightarrow \vec{r} \vec{R}(s,t) = \vec{r}( U(s,t), V(s,t))
\end{align*}
Es pot veure que les reparametritzacions de classe $C^{1}$ i jacobià no nul no afecten les integrals de superfície:
\begin{align}
\begin{aligned}
	\iint_{\mathcal{S}} f \diff \mathcal{S} & = \iint_{T} f(\vec{r}(u,v)) \left\| \frac{\partial \vec{r}}{\partial u} \times \frac{\partial \vec{r}}{\partial v} \right\| \diff u \diff v \\
	& = \iint_{W} f(\vec{R}(s,t)) \left\| \frac{\partial \vec{r}}{\partial u} \times \frac{\partial \vec{r}}{\partial v} \right\| \left| \frac{\partial (U,V)}{\partial (s,t)} \right| \diff s \diff t \\
	& = \iint_{W} f(\vec{R}(s,t)) \left\| \frac{\partial \vec{R}}{\partial s} \times \frac{\partial \vec{R}}{\partial t} \right\| \diff s \diff t
\end{aligned}
\end{align}
En el cas de $\displaystyle \iint_{\mathcal{S}} \vec{f} \cdot \dif \vec{\mathcal{S}}$ el raonament és similar.
%-------------------------------------------------------------------------------
\subsection{Els teoremes d'Stokes i de Gauss}
Es tracta de dos teoremes de $\mathbb{R}^{3}$, anàlegs al teorema de Green de $\mathbb{R}^{2}$. El d'Stokes relaciona una integral de superfície amb una integral de línia sobre el contorn d'aquesta superfície. El de Gauss relaciona una integral triple sobre una regió de $\mathbb{R}^{3}$ amb una integral de superfície sobre la frontera d'aquesta regió.
\begin{thm}[d'Stokes]\label{thm:stokes}
	Si
	\begin{enumerate}[i)]
		\item $\mathcal{S}$ és una superfície regular de $\mathbb{R}^{3}$ d'equacions paramètriques $\vec{r}(u,v)$, on $(u,v) \in T$.
		\item La regió $T$ del pla $u-v$ està limitada per una corba $\Gamma$ tancada, simple, i de classe $C^{1}$ a trossos.
		\item $\vec{r}(u,v) = (X(u,v), Y(u,v), Z(u,v))$ és de classe $C^{2}_{D}$, on $D$ és un obert i conté $T \cup \Gamma$.
		\item $\mathcal{C}$ és la imatge per $\vec{r}(u,v)$ de $\Gamma$.
		\item $P(x,y,z)$, $Q(x,y,z)$ i $R(x,y,z)$ són tres camps escalars de classe $C^{1}_{\mathcal{S}}$
	\end{enumerate}
	Llavors, tenim
	\begin{align}
	\begin{gathered}
		\iint_{\mathcal{S}} \left[ \left( \frac{\partial R}{\partial y} - \frac{\partial Q}{\partial z} \right) \diff y \wedge \dif z + \left( \frac{\partial P}{\partial z} - \frac{\partial R}{\partial Q} \right) \diff z \wedge \dif x + \left( \frac{\partial Q}{\partial x} - \frac{\partial P}{\partial y} \right) \diff x \wedge \dif y \right] \\
		= \oint_{\mathcal{C}} \left[ P \diff x + Q \diff y + R \diff z \right]
	\end{gathered}
	\end{align}
\end{thm}
Introduint el camp vectorial $\vec{f}(\vec{x}) \equiv (P(x,y,z), Q(x,y,z), R(x,y,z)$, podem expressar el teorema d'Stokes de la forma següent:
\begin{align}
	\iint_{\mathcal{S}} (\vnabla \times \vec{f}) \cdot \dif \vec{\mathcal{S}} = \oint_{\mathcal{C}} \vec{f}(\vec{x}) \cdot \dif \vec{x}
\end{align}

% V o \mathcal{V} a tot el teorema ( o fins i tot \mathcal{v})
\begin{thm}[de Gauss]\label{thm:gauss}
	Si
	\begin{enumerate}
		\item $\mathcal{V}$ és una regió sòlida (l'equivalent en 3 dimensions de regió connexa).
		\item  La regió $\mathcal{V}$ està limitada per una superfície $\mathcal{S}$ tancada, regular de classe $C^{1}$ a trossos, i «orientable» (i.e., que té dues «cares», una interior i una exterior).
		\item $P(x,y,z)$, $Q(x,y,z)$ i $R(x,y,z)$ són tres camps escalars de classe $C^{1}_{D}$, on $D$ és un obert que conté $\mathcal{V} \cup \mathcal{S}$.
	\end{enumerate}
	Llavors, tenim
	\begin{align}
	\begin{gathered}
		\iiint_{\mathcal{V}} \left( \frac{\partial P}{\partial x} + \frac{\partial Q}{\partial y} + \frac{\partial R}{\partial z} \right) \diff x \diff y \diff z \\
		= \iint_{\mathcal{S}} \left[ P \diff y \wedge \dif z + Q \diff z \wedge \dif x + R \diff x \wedge \dif y \right]
	\end{gathered}
	\end{align}
\end{thm}
Introduint el camp vectorial $\vec{f}(\vec{x}) \equiv (P(x,y,z), Q(x,y,z), R(x,y,z)$, i la notació $\dif \mathcal{V} = \dif x \diff y \diff z$, i recordant que $\dif \vec{\mathcal{S}} = (\dif y \wedge \dif z , \dif z \wedge \dif x, \dif x \wedge \dif y)$, podem expressar el teorema de Gauss de la forma següent:
\begin{align}
	\iiint_{\mathcal{V}} (\vnabla \cdot \vec{f}) \diff \mathcal{V} = \iint_{\mathcal{S}} \vec{f} \cdot \dif \vec{\mathcal{S}}
\end{align}


%\section*{Demostracions}
%\printproofs

\end{document}

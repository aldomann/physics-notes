%----------------------------------------------------------------------------------------
%    FUNCIONS A R^N
%----------------------------------------------------------------------------------------
\section{Funcions a $\mathbb{R}^{n}$, límits i continuïtat}
\subsection{Funcions a $\mathbb{R}^{n}$}
Sigui $D \subseteq \mathbb{R}^{n}$. Una funció és una aplicació de $D$ sobre $\mathbb{R}^{n}$: $\vec{x} \mapsto \vec{f}(\vec{x})$.
    \begin{defi}
    Són funcions amb valors a $\mathbb{R}$.
    \begin{itemize}
        \item $n=m=1$: funcions escalars d'una variable real, $y = f(x)$.
        \item $n>1, m=1$: camps escalars, $y = f(\vec{x})$.
    \end{itemize}
\end{defi}
\begin{defi}
    Són funcions amb valors a $\mathbb{R}^{m}$.
    \begin{itemize}
        \item $n=1, m>1$: funcions vectorials d'una variable real, $\vec{y} = \vec{f}(x)$.
        \item $n>1, m>1$: camps vectorials, $\vec{y} = \vec{f}(\vec{x})$.
    \end{itemize}
\end{defi}

%----------------------------------------------------------------------------------------
\subsection{Límit d'una funció}
Sigui $\vec{f}(\vec{x})$ una funció definida a $D \subseteq \mathbb{R}^{n}$, amb valors a $\mathbb{R}^{m}$ i sigui $\vec{a}$ un punt d'acumulació de $D$. Llavors, direm que
\begin{align}
    \lim_{\vec{x} \to \vec{a}} \vec{f}(\vec{x}) = \vec{l} \text{ si } \forall \varepsilon > 0, \exists \delta > 0 \mid \text{si }d(\vec{x}, \vec{a}) < \delta \Rightarrow d(\vec{f}(\vec{x}), \vec{l}) < \varepsilon
\end{align}
Propietats dels límits de funcions: si $\lim\limits_{\vec{x} \to \vec{a}} \vec{f}(\vec{x}) = \vec{A}$ i $\lim\limits_{\vec{x} \to \vec{a}} \vec{g}(\vec{x}) = \vec{B}$
\begin{enumerate}[i)]
    \item $\lim\limits_{\vec{x} \to \vec{a}} (\vec{f}(\vec{x}) + \vec{g}(\vec{x})) = \vec{A} + \vec{B}$.
    \item $\lim\limits_{\vec{x} \to \vec{a}} ( \lambda \vec{f}(\vec{x})) = \lambda \vec{A}$.
    \item $\lim\limits_{\vec{x} \to \vec{a}} ( \vec{f}(\vec{x}) \cdot \vec{g}(\vec{x}))= \vec{A} \cdot \vec{B}$.
    \item $\lim\limits_{\vec{x} \to \vec{a}} \| \vec{f}(\vec{x}) \| = \| \vec{A} \|$.
\end{enumerate}

%----------------------------------------------------------------------------------------
\subsection{Límits direccionals}
\begin{defi}
    Els límits d'una funció de $n$ variables en un punt $\vec{a}$ són els límits d'aquesta funció quan $\vec{x} \to \vec{a}$ seguint una trajectòria rectilínia.
\end{defi}

\subsubsection*{Límits direccionals de camps escalars}
Si $f(\vec{x})$ és un camp escalar i $\vec{u}$ és un vector de $\mathbb{R}^{n}$, definim el límit de $f(\vec{x})$ quan $\vec{x} \to \vec{a}$ en la direcció $\vec{u}$ com
\begin{align}
    \lim_{\vec{x} \to \vec{a}} f(\vec{x}) \equiv \lim_{\lambda \to 0^{+}} f(\vec{a} + \lambda \vec{u})
\end{align}
Si $\exists \lim\limits_{\vec{x} \to \vec{a}} f(\vec{x}) \Rightarrow \exists$ els límits direccionals de $f(\vec{x})$ i coincideixen en el punt $\vec{a}$. El recíproc no és cert.

\subsubsection*{Límits direccionals de camps vectorials}
Tal com passa als camps escalars, si $\exists \lim\limits_{\vec{x} \to \vec{a}} \vec{f}(\vec{x}) \Rightarrow \exists$ els límits direccionals de $\vec{f}(\vec{x})$ i coincideixen en el punt $\vec{a}$. El recíproc no és cert.

%----------------------------------------------------------------------------------------
\subsection{Continuïtat}
L'existència o no del $\lim\limits_{\vec{x} \to \vec{a}} ( \vec{f}(\vec{x})$ , així com el seu propi valor, depèn dels valors de $\vec{f}(\vec{x})$ al voltant del punt $\vec{a}$ i no del seu valor en el propi punt. De la comparació del límit amb el valor de la funció en surt el concepte de continuïtat:
\begin{align}
    \vec{f}(\vec{x}) \text{ és contínua en el punt } \vec{a} \text{ si } \lim\limits_{\vec{x} \to \vec{a}} \vec{f}(\vec{x}) = \vec{f}(\vec{a})
\end{align}
Propietats de les funcions contínues:
\begin{enumerate}[i)]
    \item Si $\vec{f}(\vec{x})$ i $\vec{g}(\vec{x})$ són contínues $\Rightarrow \vec{f}(\vec{x}) + \vec{g}(\vec{x})$, $\vec{f}(\vec{x}) \cdot \vec{g}(\vec{x})$ i $\vec{f}(\vec{x}) / \vec{g}(\vec{x})$ (si $\vec{g}(\vec{x}) \neq 0)$ són contínues.
    \item Si $\vec{f}(\vec{x})$ és contínua en $\vec{x} = \vec{a}$ i $\vec{g}(\vec{y})$ és contínua en $\vec{y} = \vec{f}(\vec{x}) \Rightarrow \vec{g}(\vec{f}(\vec{x}))$ és contínua en el punt $\vec{a}$.
    \item Una funció pot ser no contínua en el punt $\vec{a}$ i, en canvi, ser-ho a cadascuna de les variables separadament.
\end{enumerate}

\subsubsection*{Continuïtat uniforme en un domini}
\begin{defi}
    $\vec{f}(\vec{x})$ és uniformement contínua a $D$ si
    \begin{align}
        \forall \varepsilon > 0, \exists \delta > 0 \mid \text{si } \vec{x}, \vec{x}' \in D \text{ i } d(\vec{x}, \vec{x}') < \delta \Rightarrow d(\vec{f}(\vec{x}), \vec{f}(\vec{x})') < \varepsilon
    \end{align}
\end{defi}
\begin{thm}
    Si $\vec{f}(\vec{x})$ és contínua en un compacte $D \Rightarrow \vec{f}(\vec{x})$ és uniformement contínua en $D$.
\end{thm}
%-------------------------------------------------------------------------------
%          INTEGRALS DE SUPERFÍCIE
%-------------------------------------------------------------------------------
\section{Integrals de superfície}
En aquest capítol estendrem el concepte d'integral doble quan la regió no és de $\mathbb{R}^{2}$ sinó que és una superfície de $\mathbb{R}^{3}$. Aquesta extensió ens porta al concepte d'integral de superfície.

En molts sentits, les integrals de superfície són, respecte les integrals dobles, el que les integrals de línia són respecte les integrals simples.

%-------------------------------------------------------------------------------
\subsection{Superfícies a $\mathbb{R}^{3}$}
En general, una superfície de $\mathbb{R}^{3}$ es pot descriure de tres formes:
\begin{itemize}
	\item Implícita: conjunt de punts que satisfan l'equació $F(x,y,z) = 0$, on $F$ és una funció contínua.
	\item Explícita: quan és possible aïllar una de les variables de l'equació anterior, per exemple $z = f(x,y)$.
	\item Paramètrica: expressant les coordenades dels punts com funcions contínues de dos paràmetres
\end{itemize}
\begin{align*}
	\vec{r}(u,v) = (X(u,v), Y(u,v), Z(u,v))
\end{align*}
Les funcions $X(u,v)$, $Y(u,v)$ i $Z(u,v)$ generen els punts de la superfície quan els paràmetres $u$ i $v$ es mouen en una regió $T$ de $\mathbb{R}^{2}$:
\begin{align*}
	\mathcal{S} \equiv \vec{r}(T)
\end{align*}
Direm que la superfície $\mathcal{S}$ és simple si l'aplicació $T \to \mathcal{S}$ és un a un.

Notem no que forma explícita és un cas particular de la paramètrica en què els paràmetres són $x$ i $y$.
\begin{align*}
	\vec{r}(x,y) = (x,y,f(x,y))
\end{align*}

\subsubsection*{Producte vectorial fonamental}
\begin{defi}
	Anomenem producte vectorial fonamental de la superfície $\mathcal{S}$ al vector
	\begin{align}
		\frac{\partial \vec{r}}{\partial u} \times \frac{\partial \vec{r}}{\partial v} \equiv \left( \frac{\partial (Y, Z)}{\partial (u,v)}, \frac{\partial (Z, X)}{\partial (u,v)}, \frac{\partial (X, Y)}{\partial (u,v)} \right)
	\end{align}
	El producte vectorial fonamental és ortogonal al pla tangent a $\mathcal{S}$ en cada punt. Quan ens movem dins la regió paramètrica $T$, l'orientació del pla tangent a $\mathcal{S}$ varia d'un punt a una altra. Aquesta variació és contínua si les funcions $X(u,v)$, $Y(u,v)$ i $Z(u,v)$ són de classe $C^{1}_{T}$.
\end{defi}
Si expressem $\mathcal{S}$ en forma explícita, $\vec{r}(x,y) = (x,y,f(x,y))$, el producte vectorial fonamental s'expressa
\begin{align*}
	\frac{\partial \vec{r}}{\partial u} \times \frac{\partial \vec{r}}{\partial v} = \left( -\frac{\partial f}{\partial x}, -\frac{\partial f}{\partial y}, 1 \right)
\end{align*}
Com veurem, a les integrals de superfície, el producte vectorial fonamental juga un paper similar al del factor $\vec{r}'(t)$ a les integrals de línia.

\subsubsection*{Àrea d'una superfície $\mathcal{S}$}
\begin{defi}
	El mòdul del producte vectorial fonamental és el factor de proporcionalitat entre les àrees del pla $u-v$ i les corresponents àrees de la superfície $\mathcal{S}$. Llavors, si $\mathcal{S}$ és de classe $C^{1}$, definim l'àrea de $\mathcal{S}$
	\begin{align}
		a(\mathcal{S}) \equiv \iint_{T} \left\| \frac{\partial \vec{r}}{\partial u} \times \frac{\partial \vec{r}}{\partial v} \right\| \diff u \diff v = \iint_{\mathcal{S}} \dif \mathcal{S}
	\end{align}
\end{defi}
Si expressem $\mathcal{S}$ en forma explícita, $z=f(x,y)$, la igualtat anterior esdevé
\begin{align*}
	a(\mathcal{S}) = \iint_{T} \sqrt{1 + \left( \frac{\partial f}{\partial x} \right)^{2} + \left( \frac{\partial f}{\partial y} \right)^{2} } \diff x \diff y
\end{align*}

%-------------------------------------------------------------------------------
\subsection{Integrals sobre una superfície $\mathcal{S}$ de $\mathbb{R}^{3}$}
\subsubsection*{Integral d'un camp escalar de $\mathbb{R}^{3}$ sobre una superfície $\mathcal{S}$}
\begin{defi}[Integral de $f$ sobre $\mathcal{S}$]
	Sigui
	\begin{itemize}
		\item $f(x,y,z)$ un camp escalar de $\mathbb{R}^{3}$, continu a trossos.
		\item $\mathcal{S}$ una superfície regular de classe $C^{1}$, d'equacions paramètriques $x=X(u,v)$, $y=Y(u,v)$ i $z=Z(u,v)$.
	\end{itemize}
	Llavors, tenim
	\begin{align}
		\iint_{\mathcal{S}} f \diff \mathcal{S} \equiv \iint_{T} f(X(u,v), Y(u,v), Z(u,v)) \left\| \frac{\partial \vec{r}}{\partial u} \times \frac{\partial \vec{r}}{\partial v} \right\| \diff u \diff v
	\end{align}
\end{defi}
En particular l'àrea de $\mathcal{S}$ és la integral sobre $\mathcal{S}$ de la funció constant $f(x,y,z) \equiv 1$.

\subsubsection*{Integral d'un camp vectorial $\mathbb{R}^{3} \to \mathbb{R}^{3}$ sobre una superfície $\mathcal{S}$}
\begin{defi}[Integral de $\vec{f}$ sobre $\mathcal{S}$]
	Sigui
	\begin{itemize}
		\item $\vec{f}(x,y,z) = (P(x,y,z), Q(x,y,z), R(x,y,z))$ un camp vectorial $\mathbb{R}^{3} \to \mathbb{R}^{3}$, continu a trossos.
		\item $\mathcal{S}$ una superfície regular de classe $C^{1}$, d'equacions paramètriques $x=X(u,v)$, $y=Y(u,v)$ i $z=Z(u,v)$.
	\end{itemize}
	Llavors, tenim
	\begin{align}
		\iint_{\mathcal{S}} \vec{f} \cdot \dif \vec{\mathcal{S}} \equiv \iint_{\mathcal{S}} (P,Q,R) \cdot (\dif y \wedge \dif z, \dif z \wedge \dif x, \dif x \wedge \dif z) 
	\end{align}
\end{defi}

\subsubsection*{Invariància sota reparametritzacions}
Sigui $\mathcal{S}$ és una superfície regular $\vec{r}(u,v) = (X(u,v), Y(u,v), Z(u,v))$, de classe $C^{1}_{T})$, i fem el canvi de paràmetres
\begin{align*}
	\begin{cases} u & = U(s,t) \\ v & = V(s,t) \end{cases} \quad \text{amb } \frac{\partial (U,V)}{\partial (s,t)} \neq 0
\end{align*}
on $U$ i $V$ són funcions de classe $C^{1}_{W}$ que generen els punts $(u,v)$ de $T$ quan $(s,t)$ es mou a $W$.

Tenim doncs, dues parametritzacions de la superfície $\mathcal{S}$
\begin{align*}
	\vec{r} = \vec{r}(u,v) \leftrightarrow \vec{r} \vec{R}(s,t) = \vec{r}( U(s,t), V(s,t))
\end{align*}
Es pot veure que les reparametritzacions de classe $C^{1}$ i jacobià no nul no afecten les integrals de superfície:
\begin{align}
\begin{aligned}
	\iint_{\mathcal{S}} f \diff \mathcal{S} & = \iint_{T} f(\vec{r}(u,v)) \left\| \frac{\partial \vec{r}}{\partial u} \times \frac{\partial \vec{r}}{\partial v} \right\| \diff u \diff v \\
	& = \iint_{W} f(\vec{R}(s,t)) \left\| \frac{\partial \vec{r}}{\partial u} \times \frac{\partial \vec{r}}{\partial v} \right\| \left| \frac{\partial (U,V)}{\partial (s,t)} \right| \diff s \diff t \\
	& = \iint_{W} f(\vec{R}(s,t)) \left\| \frac{\partial \vec{R}}{\partial s} \times \frac{\partial \vec{R}}{\partial t} \right\| \diff s \diff t
\end{aligned}
\end{align}
En el cas de $\displaystyle \iint_{\mathcal{S}} \vec{f} \cdot \dif \vec{\mathcal{S}}$ el raonament és similar.
%-------------------------------------------------------------------------------
\subsection{Els teoremes d'Stokes i de Gauss}
Es tracta de dos teoremes de $\mathbb{R}^{3}$, anàlegs al teorema de Green de $\mathbb{R}^{2}$. El d'Stokes relaciona una integral de superfície amb una integral de línia sobre el contorn d'aquesta superfície. El de Gauss relaciona una integral triple sobre una regió de $\mathbb{R}^{3}$ amb una integral de superfície sobre la frontera d'aquesta regió.
\begin{thm}[d'Stokes]\label{thm:stokes}
	Si
	\begin{enumerate}[i)]
		\item $\mathcal{S}$ és una superfície regular de $\mathbb{R}^{3}$ d'equacions paramètriques $\vec{r}(u,v)$, on $(u,v) \in T$.
		\item La regió $T$ del pla $u-v$ està limitada per una corba $\Gamma$ tancada, simple, i de classe $C^{1}$ a trossos.
		\item $\vec{r}(u,v) = (X(u,v), Y(u,v), Z(u,v))$ és de classe $C^{2}_{D}$, on $D$ és un obert i conté $T \cup \Gamma$.
		\item $\mathcal{C}$ és la imatge per $\vec{r}(u,v)$ de $\Gamma$.
		\item $P(x,y,z)$, $Q(x,y,z)$ i $R(x,y,z)$ són tres camps escalars de classe $C^{1}_{\mathcal{S}}$
	\end{enumerate}
	Llavors, tenim
	\begin{align}
	\begin{gathered}
		\iint_{\mathcal{S}} \left[ \left( \frac{\partial R}{\partial y} - \frac{\partial Q}{\partial z} \right) \diff y \wedge \dif z + \left( \frac{\partial P}{\partial z} - \frac{\partial R}{\partial Q} \right) \diff z \wedge \dif x + \left( \frac{\partial Q}{\partial x} - \frac{\partial P}{\partial y} \right) \diff x \wedge \dif y \right] \\
		= \oint_{\mathcal{C}} \left[ P \diff x + Q \diff y + R \diff z \right]
	\end{gathered}
	\end{align}
\end{thm}
Introduint el camp vectorial $\vec{f}(\vec{x}) \equiv (P(x,y,z), Q(x,y,z), R(x,y,z)$, podem expressar el teorema d'Stokes de la forma següent:
\begin{align}
	\iint_{\mathcal{S}} (\vnabla \times \vec{f}) \cdot \dif \vec{\mathcal{S}} = \oint_{\mathcal{C}} \vec{f}(\vec{x}) \cdot \dif \vec{x}
\end{align}

% V o \mathcal{V} a tot el teorema ( o fins i tot \mathcal{v})
\begin{thm}[de Gauss]\label{thm:gauss}
	Si
	\begin{enumerate}
		\item $\mathcal{V}$ és una regió sòlida (l'equivalent en 3 dimensions de regió connexa).
		\item  La regió $\mathcal{V}$ està limitada per una superfície $\mathcal{S}$ tancada, regular de classe $C^{1}$ a trossos, i «orientable» (i.e., que té dues «cares», una interior i una exterior).
		\item $P(x,y,z)$, $Q(x,y,z)$ i $R(x,y,z)$ són tres camps escalars de classe $C^{1}_{D}$, on $D$ és un obert que conté $\mathcal{V} \cup \mathcal{S}$.
	\end{enumerate}
	Llavors, tenim
	\begin{align}
	\begin{gathered}
		\iiint_{\mathcal{V}} \left( \frac{\partial P}{\partial x} + \frac{\partial Q}{\partial y} + \frac{\partial R}{\partial z} \right) \diff x \diff y \diff z \\
		= \iint_{\mathcal{S}} \left[ P \diff y \wedge \dif z + Q \diff z \wedge \dif x + R \diff x \wedge \dif y \right]
	\end{gathered}
	\end{align}
\end{thm}
Introduint el camp vectorial $\vec{f}(\vec{x}) \equiv (P(x,y,z), Q(x,y,z), R(x,y,z)$, i la notació $\dif \mathcal{V} = \dif x \diff y \diff z$, i recordant que $\dif \vec{\mathcal{S}} = (\dif y \wedge \dif z , \dif z \wedge \dif x, \dif x \wedge \dif y)$, podem expressar el teorema de Gauss de la forma següent:
\begin{align}
	\iiint_{\mathcal{V}} (\vnabla \cdot \vec{f}) \diff \mathcal{V} = \iint_{\mathcal{S}} \vec{f} \cdot \dif \vec{\mathcal{S}}
\end{align}
